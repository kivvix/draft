\documentclass{beamer}
\usepackage[utf8]{inputenc}

\usepackage{color}
\usepackage{xcolor}
\definecolor{PLB}{rgb}{0.06,0.42,0.60}
\definecolor{PLBfonce}{rgb}{0.06,0.42,0.60}
\definecolor{PLBmoyen}{RGB}{176, 216, 232}
\definecolor{PLBpale}{rgb}{0.94,0.965,0.965}

%\usetheme{PaloAlto}
\usetheme{Madrid}
\setbeamercolor{frametitle}{bg=PLB}     %controls the color of the headline
\setbeamercolor{sidebar}{bg=PLB}        %controls the color of the sidebar
\setbeamercolor{logo}{bg=PLB!70!black}
\setbeamercolor{structure}{fg=PLB, bg=PLB!40}
%\setbeamertemplate{footline}[default]
\setbeamertemplate{enumerate items}[square]
\setbeamertemplate{itemize items}[circle]

\setbeamerfont{section number projected}{size=\large}
\setbeamertemplate{section in toc}[circle]
\setbeamertemplate{subsection in toc}[square]

\usepackage{amssymb}
\usepackage{pifont}
\usepackage{bigints}
\usepackage{mathrsfs}
\usepackage{physics}

\definecolor{mblue}{rgb}{0.18,0.21,0.67}
\definecolor{mgree}{rgb}{0.,0.6,0.}
\definecolor{dgreen}{rgb}{0.,0.6,0.}
\definecolor{violet}{RGB}{142, 68, 173}
\definecolor{bleu}{RGB}{41, 128, 185}
\definecolor{gris}{rgb}{0.5,0.5,0.5}
\newcommand{\cmark}{{\color{dgreen}\ding{52}}}
\newcommand{\xmark}{{\color{red}\ding{55}}}
\newcommand{\bmark}{{\color{orange}$\sim$}}
\newcommand{\arrow}{{\color{PLB}\ding{220}}}
\newcommand{\mbold}[1]{{\textbf{\color{PLB}#1}}}

\usepackage{soul}
\usepackage{multicol}
\usepackage{multirow}

\usepackage{ifthen}

\usepackage{amsmath}
\usepackage{cancel}
\DeclareMathOperator*{\argmax}{argmax}
\DeclareMathOperator*{\argmin}{argmin}
\DeclareMathOperator\erfi{erfi}

\usepackage[backend=bibtex, style=authoryear-comp]{biblatex}
\usepackage{filecontents}
\renewbibmacro*{cite}{%
  \iffieldundef{shorthand}
    {\ifthenelse{\ifnameundef{labelname}\OR\iffieldundef{labelyear}}
       {\usebibmacro{cite:label}%
        \setunit{\printdelim{nonameyeardelim}}}
       {\printnames{labelname}%
        \setunit{\printdelim{nameyeardelim}}}%
     \usebibmacro{cite:labeldate+extradate}%
     \setunit{\addcomma\space}%
     \usebibmacro{journal}}
    {\usebibmacro{cite:shorthand}}}
%\newcommand{\customcite}[1]{\citeauthor{#1} (\citeyear{#1})}
\newcommand{\customcite}[1]{\cite{#1}}

\bibliography{biblio}

\beamertemplatenavigationsymbolsempty

\usepackage{bm}
\newcommand{\Mvb}[1]{\boldsymbol{#1}}

%for backup slides
\newcommand{\backupbegin}{
  \newcounter{finalframe}
  \setcounter{finalframe}{\value{framenumber}}
}
\newcommand{\backupend}{
  \setcounter{framenumber}{\value{finalframe}}
}


\AtBeginSection[
  {\frame<beamer>{\frametitle{Outline}   
    \tableofcontents[currentsection,currentsection]}}%
]%
{
  \frame<beamer>{ 
    \frametitle{Outline}   
    \tableofcontents[currentsection,currentsection]}
}

\newenvironment{bframe}[1]% frame of backup (to get bakcup in title) 
{% begin code
  \begin{frame}{{\small\texttt{backup}\ } #1}
}%
{% end code
  \end{frame}
}


%\title[Workshop Bordeaux]{Hybrid model of Vlasov-Maxwell equations\\ and\\ comparison of Hamiltonian method and Lawson method}
\title[CMAP]{Comparison of high-order Eulerian methods for electron hybrid model}
\author[Josselin Massot]{A. Crestetto \inst{1} \and N. Crouseilles \inst{2,3} \and Y. Li \inst{4} \and \underline{J. Massot} \inst{3,2}}
\institute[IRMAR]{
       \inst{1} LMJL, Université de Nantes
  \and \inst{2} Inria Rennes -- Bretagne Atlantique
  \and \inst{3} IRMAR, Université de Rennes
  \and \inst{4} Max Planck Institute for Plasma Physics, Garching, Germany
}
\date{October 8, 2021}


\AtBeginDocument{\colorlet{defaultcolor}{.}}

\begin{document}
% -------------------------------------------------------------------
% --- BEGIN DOCUMENT ------------------------------------------------
% -------------------------------------------------------------------

\begin{frame}[plain]
  \titlepage
\end{frame}
\begin{frame}{Outline}
  \tableofcontents
\end{frame}

% -------------------------------------------------------------------
% --- INTRODUCTION --------------------------------------------------
% -------------------------------------------------------------------
\section{Introduction}
\begin{frame}{Vlasov-Maxwell $1dz-3dv$ model}
  Transport of electron density distribution $f=f(t,z,\Mvb{v})$, $\Mvb{B}(t,z),\Mvb{E}(t,z)\in\mathbb{R}^2$, $z\in[0,2\pi]$, $\Mvb{B}_0 = (0,0,B_0)^\top$, $\Mvb{v}\in\mathbb{R}^3$, $v_\perp=(v_x,v_y)^\top\in\mathbb{R}^2$:
  $$
    \begin{cases}
      \partial_t f + v_z\partial_z f - (\Mvb{E}+\Mvb{v}\times(\Mvb{B}+\Mvb{B}_0))\cdot\nabla_{\Mvb{v}} f = 0 \\
      \partial_t \Mvb{B} = -\partial_z\Mvb{E}\\
      \partial_t \Mvb{E} = \partial_z\Mvb{B} + \int_{\mathbb{R}^3} v_\perp f\dd{\Mvb{v}} \\
    \end{cases}
  $$
\mbold{Motivation:}
\begin{itemize}
  \item We consider an initial condition of the form $f = f_c+f_h$ with: $f_c(t=0,z,\Mvb{v})=\rho_c(t,z)\delta_{\Mvb{v}=\Mvb{u}_c(t,z)}(\Mvb{v})$
  \item We want high order methods in $(z,\Mvb{v})$ \begin{itemize}\item FFT in $z$ + WENO in $\Mvb{v}$\end{itemize}
  \item We want high order methods in time $t$ \begin{itemize}\item splitting method vs exponential integrator\end{itemize}
\end{itemize}
\end{frame}
\begin{frame}{The idea}
  \begin{itemize}
    \item Grid methods can't have an initial condition like: $f_0(z,\Mvb{v}) = \rho_{c,0}(z)\textcolor{red}{\delta_{\Mvb{v}-\Mvb{u}_c}(\Mvb{v})} + f_{h,0}(z,\Mvb{v})$
    \item Idea is to derive an hybrid model:
        \begin{itemize}
          \item \emph{Cold plasma approximation}: $\frac{T_c}{T_h}\ll 1$\arrow $\cancel{f_c(t,z,\Mvb{v})}$ \arrow $\Mvb{j}_c(t,z)$
              \begin{itemize}
                \item Fluid dynamic for cold particles (no velocity grid)
              \end{itemize}
          \item Hypothesis on hot particles: $\int_{\mathbb{R}^3} f_h(t,z,\Mvb{v})\dd{\Mvb{v}} \ll \rho_c(t,z)$
              \begin{itemize}
                \item Kinetic dynamic for hot particles 
              \end{itemize}
        \end{itemize}
  \end{itemize}

  \vfill
 
  \arrow{} Split $f=f_c+f_h$ \mbold{+} Compute momentum of $f_c$ with \emph{cold plasma approximation} \mbold{+} Linearize the model
  \begin{thebibliography}{9}
    \setbeamertemplate{bibliography item}[article]
    \bibitem{a} \customcite{Holderied:2020}
  \end{thebibliography}
\end{frame}

\begin{frame}{Linearized hybrid Vlasov-Maxwell $1dz-3dv$ model}
  The new model: a nonlinear transport in $(z,v_x,v_y,v_z)\in\Omega\times\mathbb{R}^3$ of a cold (fluid) electron density distribution (reconstruction from current variable $\Mvb{j}_c$) and a hot (kinetic) electron density distribution $f_h$:
  $$
    \begin{cases}
      \partial_t\Mvb{j}_c = \Omega_{pe}^2\Mvb{E} - J\Mvb{j}_c B_0 \\
      \partial_t\Mvb{B}   = J\partial_z\Mvb{E} \\
      \partial_t\Mvb{E}   = -J\partial_z\Mvb{B} - \Mvb{j}_c + \int_{\mathbb{R}^3} v_\perp f_h\dd{\Mvb{v}} \\
      \partial_t f_h  + v_z\partial_z f_h - \left( \Mvb{E} + \Mvb{v}\times(\Mvb{B}+\Mvb{B}_0) \right)\cdot\nabla_{\Mvb{v}} f_h = 0
    \end{cases}
  $$
  with:
  $$
    J = \begin{pmatrix}
       0 & 1 \\
      -1 & 0
    \end{pmatrix}
  $$
  %we define the Hamiltonian as :
  %$$
  %  \begin{aligned}
  %    \mathcal{H} &=
  %      \underbrace{\frac{1}{2}\int \|\Mvb{E}\|^2 \dd{z}}_{\mathcal{H}_E}
  %    + \underbrace{\frac{1}{2}\int \|\Mvb{B}\|^2 \dd{z}}_{\mathcal{H}_B}
  %    + \underbrace{\frac{1}{2}\int \frac{1}{\Omega_{pe}^2}\|\Mvb{j}_c\|^2 \dd{z}}_{\mathcal{H}_{j_c}}\\
  %    &\qquad+ \underbrace{\frac{1}{2}\int\int \|\Mvb{v}\|^2f_h \dd{\Mvb{v}}\dd{z}}_{\mathcal{H}_{f_h}}
  %  \end{aligned}
  %$$
\end{frame}

% -------------------------------------------------------------------
% --- NUMERICAL METHODS ---------------------------------------------
% -------------------------------------------------------------------
\section{Numerical methods}
\begin{frame}
  Two time integrators to compute a numerical solution of:
  $$
    \dot{u} = L(t,u) + N(t,u),\quad u(0) = u_0
  $$
  $u\in\mathbb{R}^d$, $L$ and $N$ functions $(t,u)\in\mathbb{R}_+\times\mathbb{R}^d\mapsto\mathbb{R}^d$, $d\in\mathbb{N}$.
  \vfill
  \begin{itemize}
    \item Splitting method (Lie, Strang, Suzuki)
    \item Lawson method (LRK(4,4), LDP4(3))
  \end{itemize}
  \vfill
  
  In space $z$: we use a Fourier transform (FFT).

  In velocity $\Mvb{v}$: we use WENO5 of Lagrange 5.
\end{frame}
\begin{frame}{Splitting method}
  Successive resolution of:
  $$
    \begin{aligned}
      \dot{u} &= L(t,u) \qquad & \rightarrow \tilde{u}_t = \varphi^{[L]}_t(u_0) \\
      \dot{u} &= N(t,u)        & \rightarrow \tilde{u}_t = \varphi^{[N]}_t(u_0)
    \end{aligned}
  $$
  Solution at time $t$:
  \begin{description}
    \item[\mbold{Lie:}] order 1 method composition of sub-steps:
        $\varphi_t(u_0) \approx \varphi_t^{[L]} \circ \varphi_t^{[N]}(u_0)$
    \item[\mbold{Strang:}] order 2 method:
        $\varphi_t(u_0) \approx \mathcal{S}_t(u_0) = \varphi_{t/2}^{[L]} \circ \varphi_t^{[N]} \circ \varphi_{t/2}^{[L]} (u_0)$
        \vspace{-0.25cm}
        \begin{thebibliography}{9}
          \setbeamertemplate{bibliography item}[article]
          \bibitem{b} \customcite{Strang:1968}
        \end{thebibliography}
    \item[\mbold{Suzuki:}] order 4 method, composition of 5 Strang methods:
        $$\varphi_t(u_0) \approx \mathcal{S}_{\alpha_1t}\circ\mathcal{S}_{\alpha_2t}\circ\mathcal{S}_{\alpha_3t}\circ\mathcal{S}_{\alpha_2t}\circ\mathcal{S}_{\alpha_1t}(u_0)$$
        with:
        $\alpha_1 = \alpha_2 = \frac{1}{4-\sqrt[3]{4}}$ and $\alpha_3 = \frac{1}{1-4^{\frac{2}{3}}}$
        \vspace{-0.25cm}
        \begin{thebibliography}{9}
          \setbeamertemplate{bibliography item}[article]
          \bibitem{c} \customcite{Suzuki:1990}
        \end{thebibliography}
  \end{description}

  \vspace{-0.25cm}
  \begin{thebibliography}{9}
    \setbeamertemplate{bibliography item}[article]
    \bibitem{d} \customcite{Casas:2020} \textcolor{defaultcolor}{(for some higher order methods)}
  \end{thebibliography}
\end{frame}
%-------%
\begin{frame}{Splitting method}
  {Pros \& Cons}
  \begin{description}
    \item[\cmark] Good behavior in long time
    \item[\cmark] Error in time only depends on splitting method
    \item[\cmark] Split a difficult problem into small easier sub-problems
    \item[\xmark] Numerical cost for high order method
    \item[\bmark] Needs to find a way to solve exactly \mbold{in time} each step
  \end{description}
  
\end{frame}

%-------%
\begin{frame}{Lawson method}
  $$
    \partial_t u = Lu + N(t,u)
  $$
  Change of variable: \mbold{$v = e^{-tL}u$}, we obtain a \mbold{Duhamel formula}:
  $$
    \begin{aligned}
      \dot{v}(t) &= -Le^{-tL}u(t) + e^{-tL}\underbrace{\left(Lu(t) + N(t,u)\right)}_{\dot{u}(t)} \\
                 &= e^{-tL}N(t,e^{tL}v)
    \end{aligned}
  $$
  which can be solve with a \mbold{Runge-Kutta method} in $v$, that can be rewritten in $u$, for example with Euler method:
  $$
    v(t^n+\Delta t) \approx v^{n+1} = v^n + \Delta t e^{-t^nL}N(t^n,e^{t^nL}v^n)
  $$
  or as an expression of $u$ :
  $$
    u^{n+1} = e^{\Delta t L}u^n + \Delta te^{\Delta t L}N(t^n,u^n)
  $$
  \begin{thebibliography}{9}
    \setbeamertemplate{bibliography item}[article]
    \bibitem{e} \customcite{Lawson:1967a}
    \bibitem{f} \customcite{Hochbruck:2020}
  \end{thebibliography}
\end{frame}
%-------%
\begin{frame}{Lawson method}
  {Pros \& Cons}
  \begin{description}
    \item[\cmark] Numerically efficient (order increases linearly-ish with the number of stages)
    \item[\cmark] Literature on Runge-Kutta method (embedded-RK, IMEX methods, low storage methods, \dots)
    \item[\cmark] Linear part is solved exactly
    \item[\xmark] Stability constraint (not from the linear part \cmark)
    \item[\xmark] Behavior in long time
    \item[\bmark] Needs to compute (efficiently) $e^{\tau L}$ for any $\tau=c_j\Delta t$ and $L$
  \end{description}
\end{frame}

%-------%
\begin{frame}{Main idea of adaptive time step methods (error estimate)}
  for a generic ODE $\dot{u} = f(t,u)$, adaptive time step method need 2 numerical estimations of solution $u(t^{n+1})$ of different order, $p$ and $p+1$:
  $$
    u^{n+1}_{[p]} = u(t^{n+1}) + \order{\Delta t^{p+1}},\qquad u^{n+1}_{[p+1]} = u(t^{n+1}) + \order{\Delta t^{p+2}}
  $$
  Estimate of local error:
  \vspace{-0.5cm}
  $$
    L_{[p]}^{n+1} = \left| u^{n+1}_{[p+1]} - u^{n+1}_{[p]} \right|
  $$
  \mbold{If $L^{n+1}_{[p]}>tol$:} we reject the step and start again from time $t^n$. \mbold{Else} we accept the step. \mbold{In both cases}, the optimal new time step is:
  $$
    \Delta t_\text{opt} = \sqrt[p]{\frac{tol}{L^{n+1}_{[p]}}}\Delta t^n
  $$
  In practice $u^{n+1}_{[p]}$ is computed from sub-steps of $u^{n+1}_{[p+1]}$.

  \begin{thebibliography}{9}
    \setbeamertemplate{bibliography item}[article]
    \bibitem{g} \customcite{Dormand:1978} \textcolor{defaultcolor}{(for RK method)}
    \bibitem{h} \customcite{Blanes:2019} \textcolor{defaultcolor}{(for splitting method)}
  \end{thebibliography}
  %In practice we don't want volatile time step:
  %$
  %  \Delta t^{n+1} = \max\left( 0.5\Delta t^n , \min\left( 2\Delta t^n , \Delta t_\text{opt} \right) \right)
  %$
\end{frame}

% -------------------------------------------------------------------
% --- APPLICATION FOR HYBRID VLASOV-MAXWELL MODEL -------------------
% -------------------------------------------------------------------
\section{Application for hybrid Vlasov-Maxwell model}
\begin{frame}{Linearized hybrid Vlasov-Maxwell model}
  $U = \left(\Mvb{j}_c , \Mvb{B} , \Mvb{E} , f_h \right)^\top$, $\Mvb{j}_c(t,z),\Mvb{B}(t,z),\Mvb{E}(t,z)\in\mathbb{R}^2$
  $$
    \begin{cases}
      \partial_t\Mvb{j}_c = \Omega_{pe}^2\Mvb{E} - J\Mvb{j}_c B_0 \\
      \partial_t\Mvb{B}   = J\partial_z\Mvb{E} \\
      \partial_t\Mvb{E}   = -J\partial_z\Mvb{B} - \Mvb{j}_c + \int v_\perp f_h\dd{v_\perp} \\
      \partial_t f_h  + v_z\partial_z f_h - \left( \Mvb{E} + \Mvb{v}\times(\Mvb{B}+\Mvb{B}_0) \right)\cdot\nabla_{\Mvb{v}} f_h = 0
    \end{cases}
  $$
  we define the Hamiltonian as :
  $$
    \begin{aligned}
      \mathcal{H} &=
        \underbrace{\frac{1}{2}\int \|\Mvb{E}\|^2 \dd{z}}_{\mathcal{H}_E}
      + \underbrace{\frac{1}{2}\int \|\Mvb{B}\|^2 \dd{z}}_{\mathcal{H}_B}
      + \underbrace{\frac{1}{2}\int \frac{1}{\Omega_{pe}^2}\|\Mvb{j}_c\|^2 \dd{z}}_{\mathcal{H}_{j_c}}\\
      &\qquad+ \underbrace{\frac{1}{2}\int\int \|\Mvb{v}\|^2f_h \dd{\Mvb{v}}\dd{z}}_{\mathcal{H}_{f_h}}
    \end{aligned}
  $$
  Following the Hamiltonian we built a Hamiltonian splitting.
\end{frame}
%-------%
\subsection{With splitting method}
\begin{frame}{Splitting method}
  5 subsystems $\varphi^{[E]}$, $\varphi^{[B]}$, $\varphi^{[j_c]}$, $\varphi^{[f_h]}$


  \begin{itemize}
    \item Solution with Lie splitting method:
      $$
        U^{n+1} = \varphi_{\Delta t}^{[E]}
            \circ \varphi_{\Delta t}^{[B]}
            \circ \varphi_{\Delta t}^{[j_c]}
            \circ \varphi_{\Delta t}^{[f_h]} (U^n)
      $$
    \item or Strang method:
      $$
        U^{n+1} = \varphi_{\Delta t/2}^{[E]}
            \circ \varphi_{\Delta t/2}^{[B]}
            \circ \varphi_{\Delta t/2}^{[j_c]}
            \circ \varphi_{\Delta t}^{[f_h]}
            \circ \varphi_{\Delta t/2}^{[j_c]}
            \circ \varphi_{\Delta t/2}^{[B]}
            \circ \varphi_{\Delta t/2}^{[E]} (U^n)
      $$
  \end{itemize}
\end{frame}
%-------%
\begin{frame}{Splitting method}{Example with: $\varphi^{[E]}$}
  One of sub-steps of Hamiltonian splitting:
  $$
    \varphi^{[E]}(U) =
    \begin{cases}
      \partial_t \Mvb{j}_c = \Omega_{pe}^2\Mvb{E} \\
      \partial_t \Mvb{B} = J\partial_z\Mvb{E} \\
      \partial_t \Mvb{E} = 0 \\
      \partial_t f_h = \Mvb{E}\cdot\nabla_{\Mvb{v}}f_h
    \end{cases}
    \rightarrow
    \varphi_{t}^{[E]}(U^0) = \begin{pmatrix}
      \Mvb{j}_c(0) + t\Omega_{pe}^2\Mvb{E}(0) \\
      \Mvb{B}(0) + tJ\partial_z\Mvb{E}(0) \\
      \Mvb{E}(0) \\
      f_h(0,z,\Mvb{v}+t\Mvb{E}(0),v_z)
    \end{pmatrix}
  $$
  \mbold{Numerical tools:}
  \begin{itemize}
    \item 2D interpolation with 2 Lagrange 5 interpolations to approximate $f_h(0,z,\Mvb{v}+t\Mvb{E}(0),v_z)$
  \end{itemize}
\end{frame}

%-------%

\subsection{With Lawson method}
\begin{frame}{Lawson method}
  $$\partial_tU = LU + N(t,U)$$
  with:
  $$
    \only<2>{\hspace{-2cm}}
    L = \begin{pmatrix}
      0   & -B_0 & 0          &  0          &  \Omega_{pe}^2 & 0             & 0 \\ 
      B_0 &  0   & 0          &  0          &  0             & \Omega_{pe}^2 & 0 \\
      0   &  0   & 0          &  0          &  0             & \partial_z    & 0 \\ 
      0   &  0   & 0          &  0          & -\partial_z    & 0             & 0 \\ 
      -1  &  0   & 0          & -\partial_z &  0             & 0             & 0 \\ 
      0   & -1   & \partial_z &  0          &  0             & 0             & 0 \\ 
      0   &  0   & 0          &  0          &  0             & 0             & -v_z\partial_z \\ 
    \end{pmatrix}
    ,\ 
    N:t,U\mapsto\begin{pmatrix}
      0 \\
      0 \\
      0 \\
      0 \\
      \int v_xf_h\dd{\Mvb{v}} \\
      \int v_yf_h\dd{\Mvb{v}} \\
      (\Mvb{E}+\Mvb{v}\times\Mvb{B})\cdot\nabla_{\Mvb{v}}f_h
    \end{pmatrix}
  $$
  But $e^{\tau L}$ can't be computed with symbolic computation software.
\end{frame}


\begin{frame}{How to compute $e^{\tau L}$?}
  2 solutions are proposed:
  \begin{enumerate}
    \item Remove some terms of the linear part $L$ and put them in nonlinear part $N$.
      \begin{description}
        \item[\cmark] symbolic computation to write efficient code
        \item[\xmark] add CFL stability condition
      \end{description}
    \item Approximate $e^{\tau L}$ with Taylor series or Pade approximant.
      \begin{description}
        \item[\cmark] no CFL stability from all (local) linear terms
        \item[\xmark] add error of approximation
      \end{description}
  \end{enumerate}
\end{frame}

\begin{frame}{Remove terms}
  Remove Maxwell equations from linear part $L$, and add them in nonlinear term $N$:
  $$
    L = \begin{pmatrix}
      0   & \hspace{-0.25cm}-B_0 & 0 & 0 & \hspace{-0.125cm}\Omega_{pe}^2\hspace{-0.125cm} &                  0                              & 0 \\ 
      B_0 &                  0   & 0 & 0 &                  0                              & \hspace{-0.125cm}\Omega_{pe}^2\hspace{-0.125cm} & 0 \\
      0   &                  0   & 0 & 0 &                  0                              &                  \textcolor{violet}{0}                              & 0 \\ 
      0   &                  0   & 0 & 0 &                  \textcolor{violet}{0}                              &                  0                              & 0 \\ 
      -1  &                  0   & 0 & \textcolor{violet}{0} &                  0                              &                  0                              & 0 \\ 
      0   & \hspace{-0.25cm}-1   & \textcolor{violet}{0} & 0 &                  0                              &                  0                              & 0 \\ 
      0   &                  0   & 0 & 0 &                  0                              &                  0                              & \hspace{-0.25cm}-v_z\partial_z \\ 
    \end{pmatrix}
    ,\ 
    N(t,U) = \begin{pmatrix}
      0 \\
      0 \\
      \textcolor{violet}{ \partial_zE_y} \\
      \textcolor{violet}{-\partial_zE_x} \\
      \textcolor{violet}{-\partial_zB_y} + \int v_xf_h\dd{\Mvb{v}} \\
      \textcolor{violet}{ \partial_zB_x} + \int v_yf_h\dd{\Mvb{v}} \\
      (\Mvb{E}-\Mvb{v}\times\Mvb{B})\cdot\nabla_{\Mvb{v}}f_h
    \end{pmatrix}
  $$
  \begin{description}
    \item[\cmark] $e^{\tau L}$ is exactly computed with symbolic computation
    \item[\xmark] Add a CFL stability condition in $z$ (coming from explicit resolution of \textcolor{violet}{Maxwell equations}) which can be estimated.
  \end{description}
\end{frame}

%\begin{frame}{How to estimate CFL condition}
%  \textcolor{red}{Ajouter une slide ou deux sur l'estimation de la CFL induite par Maxwell, mais en même temps ça me semble déjà long}
%\end{frame}

\begin{frame}{Approximation of $e^{\tau L}$}
  Complete linear part $L$, after Fourier transform in $z$: $\partial_z\mapsto i\kappa$
  $$
    L = \begin{pmatrix}
      0   & -B_0 & 0       &  0       &  \Omega_{pe}^2 & 0             & 0 \\ 
      B_0 &  0   & 0       &  0       &  0             & \Omega_{pe}^2 & 0 \\
      0   &  0   & 0       &  0       &  0             & i\kappa       & 0 \\ 
      0   &  0   & 0       &  0       & -i\kappa       & 0             & 0 \\ 
      -1  &  0   & 0       & -i\kappa &  0             & 0             & 0 \\ 
      0   & -1   & i\kappa &  0       &  0             & 0             & 0 \\ 
      0   &  0   & 0       &  0       &  0             & 0             & -i\kappa v_z \\ 
    \end{pmatrix}
  $$
  We have:
  $$
    \forall \kappa, \sigma({L(\kappa)})\subset \dot{\imath}\,\mathbb{R}
  $$
\end{frame}

\begin{frame}{Taylor series}
  Simplest approximation:
  $$
    T_n(\tau L) = \sum_{k=0}^n \frac{\tau^k}{k!}L^k = e^{\tau L} + \order{\tau^{n+1}}
  $$

  \begin{description}
    \item[\xmark] Bad behavior of eigenvalues
    \item[\xmark] Numerical instability in scheme
  \end{description}
  \arrow Don't keep Taylor series
\end{frame}
\begin{frame}{Eigenvalues of Taylor series}
   \begin{figure}\centering \includegraphics[height=0.7\textheight]{img/evT5} \caption{Eigenvalues of $T_5$ depend on Fourier mode $\kappa$ in $z$}\end{figure}
\end{frame}

\begin{frame}{Padé approximant}
  Best rational approximation of exponential function. Defined (for order $(p,q)$) as:
  $$
    \begin{aligned}
      h_{p,q}(M) & = \sum_{i=0}^p        \frac{\frac{p!}{(p-i)!}}{\frac{(p+q)!}{(p+q-i)!}} \frac{M^i}{i!} \\
      k_{p,q}(M) & = \sum_{j=0}^q (-1)^j \frac{\frac{q!}{(q-j)!}}{\frac{(p+q)!}{(p+q-j)!}} \frac{M^j}{j!}
    \end{aligned}
  $$

  Finally Padé approximant is:
  $$
    P_{p,q}(\tau L) = h_{p,q}(\tau L)\left( k_{p,q}(\tau L) \right)^{-1} = e^{\tau L} + \order{\tau^{p+q+1}}
  $$

  \begin{description}
    \item[\xmark] Needs matrix inversion, or some tricks:
    \begin{thebibliography}{9}
      \setbeamertemplate{bibliography item}[article]
    \bibitem{i} \customcite{Li:2011}
  \end{thebibliography}
    \item[\cmark] Best approximation for this numerical cost
    \item[\cmark] Preserve eigenvalues
  \end{description}
\end{frame}
\begin{frame}{Eigenvalues of Padé approximants}
    \only<1>{\begin{figure}\centering \includegraphics[height=0.7\textheight]{img/ev_P22} \caption{Eigenvalues of $P_{2,2}$ depend on Fourier mode $\kappa$ in $z$}\end{figure}}
    \only<2>{
      \begin{columns}
        \begin{column}{0.5\textwidth}
          \begin{figure}\centering \includegraphics[width=\textwidth]{img/ev_P21} \caption{Eigenvalues of $P_{2,1}$ depend on Fourier mode $\kappa$ in $z$}\end{figure}
        \end{column}
        \begin{column}{0.5\textwidth}
          \begin{figure}\centering \includegraphics[width=\textwidth]{img/ev_P12} \caption{Eigenvalues of $P_{1,2}$ depend on Fourier mode $\kappa$ in $z$}\end{figure}
        \end{column}
      \end{columns}
    }
    \only<3>{
      \vfill
        \cmark\ If $p=q$, no problem !
      \vfill
      \arrow Good choice !
    }
\end{frame}

\begin{frame}{Test 1}
  Simulation of $\partial_t u + a\partial_x u + b\partial_y u = 0$ (2D translation test case) and measure order.
  \only<1>{
    \begin{figure}
      \centering
      \includegraphics[height=0.6\textheight]{img/order_lrk33ref}
      \caption{Order of Lawson RK(3,3) method, and Lawson RK(3,3), $P(2,2)$ approximant method and Lawson RK(3,3) $T(4)$ series method.}
    \end{figure}
  }\only<2>{
    \begin{columns}
      \begin{column}{0.5\textwidth}
        \begin{figure}
          \centering
          \includegraphics[height=0.6\textheight]{img/order_lrk33taylor}
          \caption{Order of Lawson RK(3,3) $T(n)$ series method, $n=1,\dots,4$.}
        \end{figure}
      \end{column}
      \begin{column}{0.5\textwidth}
        \begin{figure}
          \centering
          \includegraphics[height=0.6\textheight]{img/order_lrk33pade}
          \caption{Order of Lawson RK(3,3) $P(p,q)$ approximant, $p=1,2$, $q=1,2$}
        \end{figure}
      \end{column}
    \end{columns}
  }
\end{frame}
\begin{frame}{Test 2}
  Simulation of $\partial_t u - y\partial_x u + x\partial_y u = 0$ (2D rotation) and test instability
  \begin{figure}
    \centering
    \includegraphics[width=\textwidth]{img/uf_t3p11}
    \caption{Initial condition (left), solution with Lawson RK(3,3) $T(3)$ series (middle) and Lawson RK(3,3) $P(1,1)$ approximant (right)}
  \end{figure}
\end{frame}

% -------------------------------------------------------------------
% --- NUMERICAL RESULTS ---------------------------------------------
% -------------------------------------------------------------------
\section{Numerical results}
\begin{frame}{Numerical results}
  We compare:
  \begin{itemize}
    \item Splitting method:\begin{itemize}
      \item Strang (order 2)
      \item Suzuki (order 4)
    \end{itemize}
    \item Lawson method:\begin{itemize}
      \item LRK(4,4) (order 4)
      \item LDP4(3) (adaptive time step method)
    \end{itemize}
    \item Lawson method with approximation of linear part:\begin{itemize}
      \item LRK(4,4) with Padé $(2,2)$ (order 4 + approximation of order $2+2=4$)
      \item LDP4(3) with Padé $(2,2)$ (adaptive time step method)
    \end{itemize}
  \end{itemize}

  \mbold{But:} Padé approximant implies a huge rational function (with invert of matrix), high order Lawson methods have a lot of coefficients, with 7 variables problem\dots \ \arrow \ bug source !!!
\end{frame}
\begin{frame}{Code generator}
  The main idea of code generator:
  \begin{enumerate}
    \item Write with SymPy the Lawson method with a vector $U\in\mathbb{C}^7$, an abstract matrix $L\in\mathcal{M}_7(\mathbb{C})$ and an abstract nonlinear part $N:t,U\mapsto N(t,U)\in\mathbb{C}^7$
    \item Compute $e^{\tau L}$ with our $L$ matrix, and given approximation of $\exp$
    \item Loop for each stage of Lawson method into a code template (Jinja2)
    \item Save the file, compile and run with a given configuration file
  \end{enumerate}
\end{frame}

\begin{frame}{Numerical results}
  {$N_z\times N_{v_x} \times N_{v_y} \times N_{v_z} = 27 \times 32 \times 32 \times 41$}
  \only<1>{
    \begin{figure}
      \includegraphics[height=0.75\textheight]{img/energy_a_vmhl}
      \caption{Energies evolution, $\Delta t = 0.05$}
    \end{figure}
  }
  \only<2>{
    \begin{figure}
      \includegraphics[width=\textwidth]{img/H_a_vmhl}
      \caption{Relative error on total energy, $\Delta t = 0.05$}
    \end{figure}
  }
  \only<3>{
    \begin{figure}
      \includegraphics[height=0.75\textheight]{img/energy_max}
      \caption{Energies evolution, Lawson with Taylor or Padé approximation, $\Delta t = 0.12$}
    \end{figure}
  }
  \only<4>{
    \begin{figure}
      \includegraphics[height=0.72\textheight]{img/energy_dtn}
      \caption{Energies evolution, Lawson with Taylor or Padé approximation, $\Delta t^n$}
    \end{figure}
  }
  \only<5>{
    \begin{figure}
      \includegraphics[height=0.6\textheight]{img/iterations_dtn}
      \caption{Time step evolution and estimate of local error, Lawson with Taylor or Padé approximation, $\Delta t^n$}
    \end{figure}
  }
\end{frame}
  \begin{frame}{Simulation time}
  \begin{table}[h]
    \centering
    \begin{tabular}{l|r}
      time integrator & simulation time \\
      \hline
      Lie splitting        &                $13\,\text{h}\ 25\,\text{min}\ 10\,\text{s}$ \\
      Strang splitting     &                $17\,\text{h}\ 09\,\text{min}\ 54\,\text{s}$ \\
      Suzuki splitting     &   $3\,\text{j}\ 03\,\text{h}\ 05\,\text{min}\ 24\,\text{s}$ \\
      \hline
      LRK(3,3)             &                $11\,\text{h}\ 29\,\text{min}\ 09\,\text{s}$ \\
      LRK(3,3) - $T_4$     &                $10\,\text{h}\ 53\,\text{min}\ 40\,\text{s}$ \\
      LRK(3,3) - $P_{1,1}$ &                $10\,\text{h}\ 54\,\text{min}\ 11\,\text{s}$ \\
      LRK(3,3) - $P_{2,2}$ &                $10\,\text{h}\ 55\,\text{min}\ 26\,\text{s}$ \\
      \hline
      LRK(4,4)             &                $14\,\text{h}\ 06\,\text{min}\ 15\,\text{s}$ \\
      LRK(4,4) - $T_5$     &                $14\,\text{h}\ 00\,\text{min}\ 03\,\text{s}$ \\
      LRK(4,4) - $P_{2,2}$ &                $13\,\text{h}\ 59\,\text{min}\ 59\,\text{s}$ \\
      \hline
      LDP4(3)              &                $11\,\text{h}\ 44\,\text{min}\ 04\,\text{s}$ \\
      LDP4(3) - $P_{2,2}$  &                $04\,\text{h}\ 09\,\text{min}\ 44\,\text{s}$ \\
    \end{tabular}
    \caption{Simulation time for some simulation, on mesh $N_z \times N_{v_x} \times N_{v_y} \times N_{v_z}=27\times32\times32\times41$ and time step $\Delta t = 0.05$ (initial time step for adaptive time step strategy).}
  \end{table}
\end{frame}

% -------------------------------------------------------------------
% --- CONCLUSION ----------------------------------------------------
% -------------------------------------------------------------------
\section{Conclusion}
\begin{frame}{Conclusion}
  \begin{enumerate}
    \item[\cmark] First simulations of this system using Hamiltonian splitting
    \item[\xmark] Numerical cost of splitting methods (not bad in $1dz-1dv$ but very bad in $1dz-3dv$, must be very very very bad in $3dx-3dv$)
     \item[\cmark] Numerical cost of Lawson methods
     \item[\bmark] Behavior of total energy of Lawson method (but we can use high order method easily)
     \item[\cmark] Error of approximation with Padé approximant can be lower than time integrator
  \end{enumerate}
\end{frame}
\begin{frame}{Future works}
  \begin{itemize}
    \item Compare Lawson method with Padé approximant with a PIC simulation
    \item Add $\int \Mvb{v}f_h\dd{\Mvb{v}}$ in linear part (for $1dx-1dv$ model)
  \end{itemize}
\end{frame}

% -------------------------------------------------------------------
% --- THANKS --------------------------------------------------------
% -------------------------------------------------------------------
\begin{frame}[t]
  \vfill
  {\usebeamerfont{title} Thank you for your attention}
  \vfill
\end{frame}

\appendix
\backupbegin

% -------------------------------------------------------------------
% --- BACKUP --------------------------------------------------------
% -------------------------------------------------------------------
\begin{frame}[plain]
  \vspace{0.65\textwidth}
  \hfill\footnotesize{Backup}
\end{frame}
% -------------------------------------------------------------------
\begin{bframe}{Adaptive time step method for splitting method}
  \begin{thebibliography}{9}
    \setbeamertemplate{bibliography item}[article]
    \bibitem{backup:a} \customcite{Blanes:2019} \textcolor{defaultcolor}{for Suzuki splitting method}
  \end{thebibliography}
  $$
    u^{n+1}_{[4]} = \mathcal{S}_{\Delta t}(u^n)
      = S_{\alpha_1\Delta t}
        \circ \underbrace{ S_{\alpha_2\Delta t}
        \circ \underbrace{ S_{\alpha_3\Delta t}
        \circ \underbrace{ S_{\alpha_2\Delta t}
        \circ \underbrace{ S_{\alpha_1\Delta t} (u^n). }_{u^{(1)}}
                                                       }_{u^{(2)}}
                                                       }_{u^{(3)}}
                                                       }_{u^{(4)}}
  $$
  We compute an order 3 approximation from $U^n$ and $U^{(s)}$, $s=1,2,3,4$ :
  $$
    u^{n+1}_{[3]} = -u^n + w_1(u^{(1)}+u^{(4)}) + w_2(u^{(2)}+u^{(3)})
  $$
  with:
  $$
    w_1 = \frac{g_2(1-g_2)}{g_1(g_1-1)-g_2(g_2-1)},\quad w_2 = 1-w_1, \quad \begin{aligned}g_1 &= \alpha_1\\ g_2&=\alpha_1+\alpha_2\end{aligned}
  $$
  and $L^{n}_{[3]} = \left\| u^{n+1}_{[4]} - u^{n+1}_{[3]} \right\|_2$
\end{bframe}
%-------%
\begin{bframe}{Adaptive time step method for Lawson method}
  Lawson methods are built on Runge-Kutta method, embedded Lawson method are written with an underlying embedded Runge-Kutta method.

  \begin{thebibliography}{9}
    \setbeamertemplate{bibliography item}[article]
    \bibitem{backup:b} \customcite{Dormand:1978}
  \end{thebibliography}
  With DP4(3) (Dormand-Prince method of order 4, with embedded 3 method):
    $$
    \begin{array}{c|cccccc}
      0           &             &             &             &             &                & \multirow{5}{*}{$\left.\phantom{\begin{matrix}0\\1\\2\\3\\4\end{matrix}}\right\}$Classical RK(4,4)} \\
      \frac{1}{2} & \frac{1}{2} &             &             &             &                & \\
      \frac{1}{2} & 0           & \frac{1}{2} &             &             &                & \\
      1           & 0           & 0           & 1           &             &                & \\
    \cline{1-6}
      1           & \frac{1}{6} & \frac{1}{3} & \frac{1}{3} & \frac{1}{6} &                &\\
    \cline{1-6}
                  & \frac{1}{6} & \frac{1}{3} & \frac{1}{3} & \frac{2}{30} & \frac{1}{10}  & 
    \end{array}
  $$
  We compute a 3\textsuperscript{rd} order approximation from $u^n$, $u^{(s)}$, $s=1,2,3,4$ done by the last line of Butcher tableau.

  And $L^{n}_{[3]} = \left\| u^{n+1}_{[4]} - u^{n+1}_{[3]} \right\|_2$
\end{bframe}
% -------------------------------------------------------------------
\begin{bframe}{Poisson bracket}
  For two given functionals  $\mathcal{F}$, $\mathcal{G}$ of $\Mvb{j}_c$, $\Mvb{B}$, $\Mvb{E}$, $f_h$, the Poisson bracket is given by 
  $$
    \begin{aligned}
      \{ {\cal F}, {\cal G} \}[\Mvb{j}_c, \Mvb{B}, \Mvb{E}, f_h] &=
          \frac{1}{m_e} \int_\Omega \int_{\mathbb{R}^3} f_h \Big[ \frac{\delta \mathcal{F}}{\delta f_h}, \frac{\delta \mathcal{G}}{\delta f_h} \Big]_{\Mvb{x}\Mvb{v}} \dd{\Mvb{v}}\dd{\Mvb{x}} \\
      & + \frac{q_e}{m_e\varepsilon_0} \int_\Omega \int_{\mathbb{R}^3} f_h \left( \nabla_{\Mvb{v}}\frac{\delta \mathcal{F}}{\delta f_h} \cdot \frac{\delta \mathcal{G}}{\delta\Mvb{E}} -  \nabla_{\Mvb{v}}\frac{\delta \mathcal{G}}{\delta f_h} \cdot \frac{\delta \mathcal{F}}{\delta\Mvb{E}} \right) \dd{\Mvb{v}}\dd{\Mvb{x}} \\
      & + \frac{q_e}{m^2_e} \int_\Omega \int_{\mathbb{R}^3} f_h (\Mvb{B} + \Mvb{B}_0) \cdot \left( \nabla_{\Mvb{v}}\frac{\delta \mathcal{F}}{\delta f_h} \times \nabla_{\Mvb{v}}\frac{\delta \mathcal{G}}{\delta f_h} \right) \dd{\Mvb{v}}\dd{\Mvb{x}} \\
      & + \frac{1}{\varepsilon_0} \int_\Omega  \left( \nabla \times \frac{\delta \mathcal{F}}{\delta \Mvb{E}} \cdot  \frac{\delta \mathcal{G}}{\delta \Mvb{B}} - \nabla \times \frac{\delta \mathcal{G}}{\delta \Mvb{E}} \cdot  \frac{\delta \mathcal{F}}{\delta \Mvb{B}} \right) \dd{\Mvb{x}} \\
      & +  \int_\Omega \Omega_{pe}^2 \left( \frac{\delta \mathcal{F}}{\delta \Mvb{j}_c} \cdot \frac{\delta \mathcal{G}}{\delta \Mvb{E}} -  \frac{\delta \mathcal{G}}{\delta \Mvb{j}_c} \cdot \frac{\delta \mathcal{F}}{\delta \Mvb{E}} \right)  \dd{\Mvb{x}} \\
      & + \frac{q_e \varepsilon_0}{m_e}  \int_\Omega   \Omega_{pe}^2 \Mvb{B}_0 \cdot \left( \frac{\delta \mathcal{F}}{\delta \Mvb{j}_c} \times \frac{\delta \mathcal{G}}{\delta \Mvb{j}_c} \right) \dd{\Mvb{x}}.
    \end{aligned}
  $$
\end{bframe}
\begin{bframe}{Splitting method $\varphi^{[j_c]}$}
  $$
    \varphi^{[j_c]}(U) =
    \begin{cases}
      \partial_t \Mvb{j}_c = -J\Mvb{j}B_0 \\
      \partial_t \Mvb{B} = 0 \\
      \partial_t \Mvb{E} = -\Mvb{j}_c \\
      \partial_t f_h = 0
    \end{cases}
    \rightarrow
    \varphi_{t}^{[j_c]}(U^0) = \begin{pmatrix}
      e^{-tJ}\Mvb{j}_c(0)B_0 \\
      \Mvb{B}(0) \\
      \Mvb{E}(0) - J(e^{-tJ}-I)\Mvb{j}_c(0) \\
      f_h(0)
    \end{pmatrix}
  $$
  Obtain because: $\int_0^t \exp(-sJ)\Mvb{j}_c(0)\dd{s} = J(\exp(-tJ)-I)\Mvb{j}_c(0)$, with:
  $$
    \exp(-tJ) = \begin{pmatrix}\cos(t) & -\sin(t) \\ \sin(t) & \cos(t) \end{pmatrix}
  $$
\end{bframe}
\begin{bframe}{Splitting method $\varphi^{[B]}$}
  $$
    \varphi^{[B]}(U) =
    \begin{cases}
      \partial_t \Mvb{j}_c = 0 \\
      \partial_t \Mvb{B} = 0 \\
      \partial_t \Mvb{E} = -J\partial_z\Mvb{B} \\
      \partial_t f_h = 0
    \end{cases}
    \rightarrow
    \varphi_{t}^{[B]}(U^0) = \begin{pmatrix}
      \Mvb{j}_c(0) \\
      \Mvb{B}(0) \\
      \Mvb{E}(0) - tJ\partial_z\Mvb{B}(0) \\
      f_h(0)
    \end{pmatrix}
  $$
  \mbold{Numerical tools:}
  \begin{itemize}
    \item Solve in Fourier space
  \end{itemize}
\end{bframe}
\begin{bframe}{Splitting method $\varphi^{[E]}$}
  $$
    \varphi^{[E]}(U) =
    \begin{cases}
      \partial_t \Mvb{j}_c = \Omega_{pe}^2\Mvb{E} \\
      \partial_t \Mvb{B} = J\partial_z\Mvb{E} \\
      \partial_t \Mvb{E} = 0 \\
      \partial_t f_h = \Mvb{E}\cdot\nabla_{\Mvb{v}}f_h
    \end{cases}
    \rightarrow
    \varphi_{t}^{[E]}(U^0) = \begin{pmatrix}
      \Mvb{j}_c(0) + t\Omega_{pe}^2\Mvb{E}(0) \\
      \Mvb{B}(0) + tJ\partial_z\Mvb{E}(0) \\
      \Mvb{E}(0) \\
      f_h(0,z,\Mvb{v}+t\Mvb{E}(0),v_z)
    \end{pmatrix}
  $$
  \mbold{Numerical tools:}
  \begin{itemize}
    \item 2D interpolation with 2 Lagrange 5 interpolations to approximate $f_h(0,z,\Mvb{v}+t\Mvb{E}(0),v_z)$
  \end{itemize}
\end{bframe}
\begin{bframe}{Splitting method $\varphi^{[f_h]}$}
  $$
    \varphi^{[f_h]}(U) =
    \begin{cases}
      \partial_t \Mvb{j}_c = 0 \\
      \partial_t \Mvb{B} = 0 \\
      \partial_t \Mvb{E} = \int\Mvb{v}f_h\dd{\Mvb{v}} \\
      \partial_t f_h = -v_z\partial_zf_h + (\Mvb{v}\times(\Mvb{B}+\Mvb{B}_0))\cdot\nabla_{\Mvb{v}}f_h
    \end{cases}
  $$
  This step is split again onto 3 parts.
\end{bframe}
\begin{bframe}{Splitting method $\varphi^{[f_h]}$, $f_{h,\star}$, $\star\in\{x,y\}$}
  $$
    \!\!\varphi^{[f_{h,x}]}(U)\!=\!\!
    \begin{cases}
      \partial_t \Mvb{j}_c = 0 \\
      \partial_t \Mvb{B} = 0 \\
      \partial_t E_x = \int\!v_x f_h\dd{\Mvb{v}} \\
      \partial_t E_y = 0 \\
      \partial_t f_h = -v_x B_0\partial_{v_y} f_h + v_x B_y\partial_{v_z}f_h 
    \end{cases}
    \hspace{-2.625cm}
    \rightarrow
    \varphi_t^{[f_{h,x}]}(U^0)\!=\!\!\begin{pmatrix}
      \Mvb{j}_c(0) \\
      \Mvb{B}(0) \\
      E_x(0) + t\int\!v_xf_h(0)\dd{\Mvb{v}} \\
      E_y(0) \\
      \!f_h(\!0,\!z,\!\!v_x,\!\!v_y\!\!-\!\!tv_x\!B_0,\!v_z\!\!+\!\!tB_y\!v_x\!)\!
    \end{pmatrix}
  $$
  \mbold{Numerical tools:}
  \begin{itemize}
    \item 2D interpolation with Lagrange 5 interpolation to approximate $f_h(0,z,v_x,v_y-tv_xB_0,v_z+tB_yv_x)$
  \end{itemize}
  Same thing for $\varphi^{[f_{h,y}]}$ in $v_y$ direction.
\end{bframe}
\begin{bframe}{Splitting method $\varphi^{[f_h]}$, $f_{h,z}$}
  $$
    \varphi^{[f_{h,z}]}(U) =
    \begin{cases}
      \partial_t \Mvb{j}_c = 0 \\
      \partial_t \Mvb{B} = 0 \\
      \partial_t \Mvb{E} = 0 \\
      \partial_t f_h = -v_z\partial_zf_h + (-v_zB_y\partial_{v_x}f_h + v_zB_x\partial_{v_y}f_h) 
    \end{cases}
  $$
  \mbold{Numerical tools:}
  \begin{itemize}
    \item Split \mbold{again} onto 3 parts, with change of variable $g(t,z,\Mvb{v}):=f(t,z+tv_z,\Mvb{v})$
    \item 2D interpolation with Lagrange 5 interpolation to approximate $g(0,z,v_x-\sum_k\hat{B}_{y}(0,k)\frac{1}{ik}e^{ikz}(e^{iktv_z}-1), v_y+\sum_k\hat{B}_x(0,k)\frac{1}{ik}e^{ikz}(e^{iktv_z}-1),v_z)$
     \item Revert change of variable with Fourier transform
  \end{itemize}
\end{bframe}
\begin{bframe}{Splitting method}
  For Lie method:

  $U^{n+1} = \varphi_{\Delta t}^{[j_c]}
        \circ\varphi_{\Delta t}^{[B]}
        \circ\varphi_{\Delta t}^{[E_{v_x}]}
        \circ\varphi_{\Delta t}^{[E_{v_y}]}
        \circ\varphi_{\Delta t}^{[f_{h,x,v_x}]}
        \circ\varphi_{\Delta t}^{[f_{h,x,v_z}]}
        \circ\varphi_{\Delta t}^{[f_{h,y,v_y}]}
        \circ\varphi_{\Delta t}^{[f_{h,y,v_z}]}
        \circ\varphi_{\Delta t}^{[f_{h,z,1}]}
        \circ\varphi_{\Delta t}^{[f_{h,z,2}]}
        \circ\varphi_{\Delta t}^{[f_{h,z,3}]}
        (U^n)
   $
\end{bframe}


%-------%
\backupend

\end{document}

