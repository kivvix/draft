\documentclass[a4paper, 11pt]{article}

\usepackage{mmap}
\usepackage[T1]{fontenc}
\usepackage[frenchb,french]{babel}
\usepackage[utf8]{inputenc}

\usepackage[          %
    labelalpha=true,  % 
    backend=biber,    %
    bibencoding=utf8, %
    sorting=none,     %
    hyperref=true,    %
    url=false,        %
    backref=true,     %
    backrefstyle=three]{biblatex}
\addbibresource{biblio.bib}

\usepackage{graphicx}
%\usepackage{subfig}
\usepackage{subcaption}
\usepackage{color}
\usepackage{tikz}

% figures pas plus loin que la fin de la section
\usepackage[section]{placeins}

%\usepackage[pdfborder={0 0 0 [3 3]},pdftex,unicode=true,pdfa=true]{hyperref}
\usepackage[unicode=true,pdfa=true]{hyperref}
\usepackage{bookmark}
\usepackage[         %
    top    = 2.75cm, %
    bottom = 3.50cm, %
    left   = 3.00cm, %
    right  = 2.50cm]{geometry}

\usepackage{amsmath}
\usepackage{amssymb}
\usepackage{amsthm}
\usepackage{mathabx}
\usepackage{mathrsfs}
\usepackage{wasysym}
\usepackage{textcomp}
\DeclareMathAlphabet{\mathpzc}{OT1}{pzc}{m}{it}
\usepackage{eufrak}

\usepackage{algorithm}
\usepackage{algpseudocode}
\floatname{algorithm}{Algorithme}
\renewcommand{\algorithmicprocedure} {\textbf{Proc\'edure} }
\renewcommand{\algorithmicwhile}     {\textbf{tant que}    }
\renewcommand{\algorithmicdo}        {\textbf{faire :}     }
\renewcommand{\algorithmicend}       {\textbf{fin}         }
\renewcommand{\algorithmicif}        {\textbf{si}          }
\renewcommand{\algorithmicelse}      {\textbf{sinon}       }
\renewcommand{\algorithmicthen}      {\textbf{alors}       }
\renewcommand{\algorithmicfor}       {\textbf{pour}        }
\renewcommand{\algorithmicforall}    {\textbf{pour tout}   }
\renewcommand{\algorithmicrepeat}    {\textbf{répéter}     }
\renewcommand{\algorithmicuntil}     {\textbf{jusqu'à}     }
\renewcommand{\algorithmicfunction}  {\textbf{Fonction}    }
\renewcommand{\algorithmicreturn}    {\textbf{retourner}   }

\theoremstyle{plain}\newtheorem{theorem}{Théorème}[section]
\theoremstyle{plain}\newtheorem{lemma}{Lemme}[section]
\newtheorem{pro}{Proposition}
%\theoremstyle{remark}[theorem]
\newtheorem{remark}{Remarque}
%\theoremstyle{plain}\newtheorem{ex}[theorem]{Example}
%\theoremstyle{plain}\newtheorem{exper}[theorem]{Experiment}
\newtheorem{property}{Property}
\newtheorem{hyp}{Hypothèse}[section]

\newcommand{\alertred}[1]{{\color[rgb]{0.95,0,0.05}{#1}}}
\newcommand{\sumli}{\sum\limits}
\newcommand{\tl}{\tilde}
\newcommand{\vp}{\varphi}
\newcommand{\pa}{\partial}

% source: https://isocpp.org/wiki/faq/misc-environmental-issues#latex-macros
\def\CC{{C\nolinebreak[4]\hspace{-.05em}\raisebox{.4ex}{\tiny\bf ++}}}

\newcommand{\Python}{Python}
\newcommand{\sympy}{SymPy}

\title{\textbf{Chapitre 3 : Modèle hybride linéarisé dans le cas $1dz-3dv$}}
\date{}

\newcommand{\commentaire}[2][]{%
  \ifthenelse{\equal{#1}{}}%
    {\textbf{#2}}%
  {%
  \ifthenelse{\equal{#1}{Anais}}%
    {\textcolor{blue}{#2}}%
  {\ifthenelse{\equal{#1}{Nicolas}}%
    {\textcolor{orange}{#2}}%
  {\ifthenelse{\equal{#1}{Josselin}}%
  {\textcolor{purple}{#2}}%
  {\textcolor{teal}{#2}\footnote{Commentaire rédigé par #1}%
  }}}}%
}
\newcommand{\Anais}[1]{\commentaire[Anais]{#1}}
\newcommand{\Nicolas}[1]{\commentaire[Nicolas]{#1}}
\newcommand{\Josselin}[1]{\commentaire[Josselin]{#1}}

\begin{document}

\maketitle

J'ai ajouter une commande pour des commentaires, qui sera simple à modifier pour que tout le texte apparaisse normalement (ou le supprimer complètement) au lieu d'enlever tous les \texttt{\textbackslash{}textcolor\{color\}\{text\}} :
\begin{itemize}
  \item \commentaire{Ceci est un commentaire anonyme : \texttt{\textbackslash{}commentaire\{Test\}}}
  \item \Anais{Ceci est un commentaire rédigé par Anaïs : \texttt{\textbackslash{}Anais\{Test\}}.\\C'est un alias de \texttt{\textbackslash{}commentaire[Anais]\{Test\}}}
  \item \Nicolas{Ceci est un commentaire rédigé par Nicolas : \texttt{\textbackslash{}Nicolas\{Test\}}.\\C'est un alias de \texttt{\textbackslash{}commentaire[Nicolas]\{Test\}}}
  \item \Josselin{Ceci est un commentaire rédigé par Josselin : \texttt{\textbackslash{}Josselin\{Test\}}.\\C'est un alias de \texttt{\textbackslash{}commentaire[Josselin]\{Test\}}}
  \item \commentaire[Bob]{Ceci est un commentaire rédigé par quelqu'un d'autre, son nom s'affiche alors en note en bas de page : \texttt{\textbackslash{}commentaire[Bob]\{Test\}}}
\end{itemize}


\Josselin{Dans ce chapitre, l'introduction, la présentation du modèle, la section sur les schémas numériques ainsi que les premiers résultats numériques seront une reprise (plus ou moins étoffée) de l'article. Dans l'immédiat j'avais envie d'écrire des trucs un peu plus neufs, et savoir comment l'étoffer.}

\section{Introduction}
L'objectif de ce chapitre est d'introduire et de simuler numériquement une hiérarchie de modèles permettant de décrire des systèmes de particules chargées où une population de particules chaudes int\'eragit avec un plasma ambiant plus froid. Une telle configuration physique peut par exemple \^etre 
\'etudi\'ee dans les plasmas de tokamak o\`u les particules alpha (g\'en\'er\'ees par les r\'eactions de 
fusion) interagissent avec le plasma ambiant. Un autre exemple se trouve dans la haute atmosph\`ere  o\`u les \'electrons \'energ\'etiques du vent solaire interagissent avec la magn\'etosph\`ere terrestre. 
Des modèles adapt\'es \`a ces configurations ont ainsi \'et\'e obtenus par exemple dans 
les deux contextes (voir \cite{Holderied:2019}, \cite{Chen:2016} \cite{Katoh:2007} 
\cite{Tao:2014} \cite{Tronci:2010} \cite{Tronci:2014}). 
Le modèle de départ qui servira de référence repose sur une description cinétique pour l'ensemble du plasma consid\'er\'e. On introduit alors la fonction de distribution des \'electrons  $f(t,x,v)\in\mathbb{R}_+$ solution du modèle de Vlasov-Poisson (les ions sont consid\'er\'ees immobiles, 
comme \'etant un fond neutralisant). En supposant que la population \'electronique peut \^etre 
s\'epar\'ee entre une population "froide" $f_c$ et une population d\'electrons \'energ\'etiques $f_h$,  
une première étape consiste à écrire $f$ comme la somme de ces deux fonctions de distribution 
$f=f_c+f_h$. Une seconde étape consiste à supposer que les particules froides restent proches d'un équilibre thermodynamique de temp\'erature $T_c\approx 0$ et peuvent donc être réprésentées par un modèle fluide. On obtient le modèle hybride fluide/cin\'etique où la partie fluide (linéaire) décrit la dynamique des particules froides alors que les particules chaudes sont décrites à l'aide d'un modèle cinétique.
Ce mod\`ele hybride peut encore \^etre simplifi\'e en consid\'erant des perturbations de type ondes de faible amplitude. Les termes non lin\'eaires de la partie fluide sont donc n\'eglig\'es alors que la 
partie cin\'etique reste non lin\'eaire. Le mod\`ele ainsi obtenu (voir \cite{Holderied:2019}) est le mod\`ele hybride lin\'earis\'e VHL (Vlasov Hybrid Linearized).

Du fait de la forte disparité des vitesses thermiques entre les particules froides et chaudes, le modèle cinétique est très coûteux à résoudre numériquement, notamment car le maillage en vitesse doit être choisi très fin pour capturer la vitesse thermique des particules froides. Cela implique en outre, pour les schémas numériques classiques, une condition restrictive sur le pas de temps et 
donc des simulations co\^uteuses. La dérivation de modèles simplifi\'es moins coûteux à résoudre numériquement est donc d'un grand int\'er\^et. Parmi ces mod\`eles simplifi\'es, 
nous consid\'ererons ici le mod\`ele hybride lin\'earis\'e VHL \'etudi\'e dans \cite{Holderied:2019}.  
Afin d'effectuer une \'etude comparative entre le mod\`ele VHL et le mod\`ele cin\'etique original 
et de tester les sch\'emas num\'eriques, nous nous placerons dans le cas de la dimension $1$ 
en espace et en vitesse. Ce cadre nous permettra aussi de poser les bases de l'\'etude 
du cas $1d$-$3v$ pour lequel il est beaucoup plus complexe d'effectuer ces comparaisons et ces tests. Ce type d'\'etude permettra enfin de comprendre le domaine de validité du mod\`ele VHL. 

Pour résoudre numériquement le modèle VHL, nous proposons deux méthodes. La première repose sur le fait que le modèle VHL possède une structure géométrique \cite{Morrison:2017}\cite{Tronci:2010}, \cite{Tronci:2014}. Plus pr\'ecis\'ement, le mod\`ele VHL poss\`de une structure Hamiltonienne non canonique, ce qui signifie que les \'equations peuvent \^etre 
obtenunes \`a partir d'un crochet de Poisson et d'un Hamiltonien. Cette structure garantit la préservation d'invariants, comme l'énergie totale. L'objectif est d'exploiter cette structure 
pour construire des sch\'emas num\'eriques qui poss\`edent un bon comportement en temps long. 
Le sch\'ema utilis\'e est un sch\'ema de type splitting construit \`a partir d'un \emph{splitting} du Hamiltonien. Cette approche permet de combiner astucieusement certains termes du mod\`ele et 
on est alors amené à résoudre trois sous-systèmes simples (comme dans \cite{Crouseilles:2015}, \cite{Casas:2017}, \cite{Li:2020}). Une propriété remarquable est que chacun des sous-systèmes peut être résolu exactement en temps, l'erreur en temps de la méthode provient donc uniquement de la méthode de \emph{splitting} utilisée. Des méthodes d'ordre arbitraire en temps peuvent être obtenues par composition \cite{Hairer:2006}. 
La deuxième méthode est basée sur un schéma exponentiel \cite{Hochbruck:2010}, \cite{Hochbruck:2005}, \cite{Lawson:1967a}, \cite{Isherwood:2018}, \cite{Lawson:1967}, \cite{Crouseilles:2019b}. En exploitant le fait que la partie linéaire du modèle VHL peut être résolue exactement et efficacement, on construit alors des schémas de type Lawson d'ordre élevé. Les r\'esultats du chapitre pr\'ec\'edent et de \cite{Crouseilles:2019b} sont donc repris et \'etendus au cas du syst\`eme VHL. 

Pour les deux méthodes en temps (splitting et Lawson), nous avons introduit une technique 
de pas de temps adaptatif. Pour les m\'ethodes de type Lawson, le cadre des m\'ethode 
{\it embedded} \cite{Dormand:1980}\cite{Dormand:1978}  \cite{Balac:2013b}\cite{Balac:2013a} 
permet de calculer l'erreur locale facilement. Dans le cas des m\'ethodes de splitting, 
nous utiliserons le travail r\'ecent \cite{Blanes:2019} qui propose des m\'ethodes 
de splitting {\it embedded}. Des m\'ethodes d'ordre $4(3)$ seront utilis\'ees dans le cadre de la 
comparaison (ordre $3$ et ordre $4$ pour estimer l'erreur locale). Pour l'approximation de l'espace des phases, nous avons choisi une méthode spectrale en espace et une approximation type différences finies d'ordre élevé (ordre 5 en pratique) pour la direction en vitesse. 

La premi\`ere approche (splitting Hamiltonien) comporte des similarit\'es avec les approches propos\'ees dans \cite{Kraus:2017} et \cite{Holderied:2019} ; n\'eanmoins, ces m\'ethodes reposent sur une approximation de type Particle-In-Cell de l'espace des phases alors que nous utilisons des m\'ethodes eul\'eriennes. Ainsi, 
on est plus dans l'esprit de \cite{Crouseilles:2015}, \cite{Li:2020} o\`u on effectue un splitting 
puis on discr\'etise alors que dans  \cite{Kraus:2017} et \cite{Holderied:2019}, on discr\'etise l'espace des phases puis on discr\'etise en temps.  


Afin de valider les résultats numériques, une étude approfondie des relations de dispersion est effectuée. Ces relations de dispersion sont obtenues par la résolution du modèle VHL lin\'earis\'e.  À l'aide de transformée de Fourier en espace, de transformée de Laplace en temps, il est en effet possible de déterminer très précisément la phase linéaire des simulations de modèle non linéaire ; on peut calculer le taux d'amortissement ou d'instabilité d'un équilibre perturbé \cite{Sonnendrucker:2015}, \cite{Fried:1961}, mais aussi reconstruire le mode fondamental du champ électrique. En plus de fournir des informations pour valider de manière quantitative les codes développés, cette analyse nous permet de faire le lien entre les modèles. En effet, 
en faisant tendre $T_c$ vers z\'ero dans la relation de dispersion du mod\`ele de Vlasov original, 
il est possible de retrouver la relation de dispersion du mod\`ele VHL. 

Le chapitre est organisé comme suit : nous présentons tout d'abord la hiérarchie de modèles que nous souhaitons étudier, depuis le modèle cinétique jusqu'au modèle hybride linéarisé. La structure géométrique de ce modèle est exhibée en section \ref{s:geom}. La section \ref{s:scheme} est dédiée à la présentation des méthodes numériques construites pour la résolution du modèle hybride linéarisé. Dans la section \ref{s:dispersion}, les relations de dispersion sont introduites et étudiées. Les sections \ref{s:limit} et \ref{s:compare} contiennent de nombreuses illustrations numériques. La section \ref{s:limit} se concentre sur la comparaison du modèle cinétique avec le modèle hybride linéarisé, alors que dans la section \ref{s:compare}, nous étudions les avantages et les inconvénients des deux méthodes numériques pour le modèle hybride linéarisé.


% !TEX root = ../chap3.tex

\section{Présentation du modèle}

$$
  \begin{cases}
    \partial_t j_{c,x} &= \Omega_{pe}^2E_x - j_{c,y}B_0 \\
    \partial_t j_{c,y} &= \Omega_{pe}^2E_y + j_{c,x}B_0 \\
    \partial_t B_{x}   &=  \partial_zE_y \\
    \partial_t B_{y}   &= -\partial_zE_x \\
    \partial_t E_{x}   &= -\partial_zB_y - j_{c,x} + \int v_xf_h\,\mathrm{d}v \\
    \partial_t E_{y}   &=  \partial_zB_x - j_{c,y} + \int v_yf_h\,\mathrm{d}v \\
    \partial_t f_h     &= -v_z\partial_zf_h + (E_x+v_yB_0-v_zB_y)\partial_{v_x}f_h + (E_y-v_xB_0+v_zB_x)\partial_{v_y}f_h + (v_xB_y - v_yB_x)\partial_{v_z}f_h
  \end{cases}
$$


% !TEX root = ../../main.tex

\section{Méthodes de résolution numérique en temps}
%%%%%%%%%%%%%%%%%%%%%%%%%%%%%%%%%%%%%%%%%%%%%%%%%%%%%%%%%%%%%%%%%%%%%%

Dans cette section nous allons présenter les principales méthodes numériques utilisées pour résoudre numériquement des équations dites cinétiques en temps, et plus spécifiquement le système~\eqref{eq:0:vmhl:1}-\eqref{eq:0:vmhl:4}. Une fois discrétisé en $(\vb{x},\vb{v})$, les différents systèmes que nous regardons peuvent se réduire au modèle abstrait suivant :
\begin{equation}
  \dot{u}(t) = L(t,u) + N(t,u),\quad u(t=0)=u_0
  \label{eq:0:dtu}
\end{equation}
d'inconnue $u\in\mathbb{R}^n$ et où $L$ et $N$ sont des fonctions $(t,u)\in\mathbb{R}_+\times\mathbb{R}^n\mapsto\mathbb{R}^n$, $n\in\mathbb{N}$ est le nombre de dimensions, ou d'inconnues du problème. C'est sur cette équation~\eqref{eq:0:dtu} que nous allons présenter les différentes méthodes d'intégration en temps utilisées ici.

% --------------------------------------------------------------------
\subsection{Méthode de \emph{splitting} hamiltonien}
% --------------------------------------------------------------------

Les méthodes de \emph{splitting} sont classiquement utilisées dans la résolution d'équations cinétiques (\cite{Morrison:2017,Grandgirard:2006,Tronci:2010,Tronci:2014}), elles consiste à diviser l'équation à résoudre en plusieurs parties. La construction de ces méthodes en temps se fait par concaténation des différentes étapes en formant des palindromes.

Une méthode de \emph{splitting} consiste à résoudre les deux équations suivantes successivement :
\begin{eqnarray}
    \dot{u} = L(t,u) \label{eq:0:split:1}\\
    \dot{u} = N(t,u) \label{eq:0:split:2}
\end{eqnarray}
La solution l'équation~\eqref{eq:0:dtu} au temps $t$ est $\varphi_t(u_0)$, et sera approchée par une composition de $\varphi_t^{[L]}(u_0)$ et $\varphi_t^{[N]}(u_0)$, respectivement solutions de~\eqref{eq:0:split:1} et~\eqref{eq:0:split:2}. Ainsi la méthode de Lie, \emph{splitting} d'ordre 1, consiste à approcher $\varphi_t(u_0)$ par $\varphi_t(u_0)\approx \varphi_t^{[L]} \circ \varphi_t^{[N]}(u_0)$. Si la résolution de chaque sous-système $\varphi_t^{[L]}$ et $\varphi_t^{[N]}$ est exacte, la seule erreur en temps provient du \emph{splitting}.

La résolution de chaque sous-système peut se faire sur des intervalles de temps différents (que nous noterons en indice), ainsi la méthode de Strang~\cite{Strang:1968}, \emph{splitting} d'ordre 2, s'écrit comme :
$$
  u(t) = S_{t}(u_0) = \varphi^{[L]}_{t/2}\circ\varphi^{[N]}_{t}\circ\varphi^{[L]}_{t/2}(u_0)
$$

Lorsque l'équation met en jeu plusieurs termes, comme c'est le cas pour le système~\eqref{eq:0:vmhl:1}-\eqref{eq:0:vmhl:4}, il est difficile de savoir comment choisir $L$ et $N$. L'hamiltonien du système permet de suggérer une décomposition intéressante, et de construire des méthodes appelées \emph{splitting} hamiltonien. \Josselin{voir dans \cite{Hairer:2006} s'il est nécessaire de résoudre exactement chaque sous-système pour avoir une splitting hamiltonien.}


\subsection{Méthode de type Runge-Kutta}
% --------------------------------------------------------------------

Les méthodes de type Runge-Kutta sont des méthodes d'approximation de solutions d'équations différentielles, développées dès 1901. Elles peuvent être vues comme une extension, à des ordres supérieurs, de la méthode d'Euler. Nous utiliserons ce type de méthodes pour résoudre la discrétisation en temps. Nous allons présenter ce type de méthode sur l'équation :
$$
  \dot{u} = N(t,u)
$$
où $u\in\mathbb{R}^n$, et $N:(t,u)\in\mathbb{R}_+\times\mathbb{R}^n\mapsto N(t,u)\in\mathbb{R}^n$ une fonction agissant sur $u$ et pouvant dépendre du temps $t$. Il s'agit d'un cas particulier de l'équation~\eqref{eq:0:dtu} où $L$ est la fonction nulle. Nous résumerons les méthodes par leur tableau de Butcher\cite{Butcher:2008}, qui se représentent sous la forme :
\begin{equation}  
  \begin{array}{c|c}
    \begin{matrix}
      c_1 \\
      \vdots \\
      c_s
    \end{matrix}
    &
    \begin{matrix}
      a_{11} & \cdots & a_{1s} \\
      \vdots & \ddots & \vdots \\
      a_{s1} & \cdots & a_{ss}
    \end{matrix} \\
    \hline
     & \begin{matrix} b_1 & \cdots & b_s \end{matrix} \\
  \end{array}
  \label{eq:0:butcher}
\end{equation}
et qui se lit :
$$
  \begin{aligned}
    u^{(i)} &= u^n + \Delta t \sum_{j=1}^s a_{ij} N(t^n+c_j\Delta t,u^{(j)}) \\
    u^{n+1} &= u^n + \Delta t \sum_{i=1}^s b_i N(t^n+c_i\Delta t, u^{(i)}).
  \end{aligned}
$$
où $u^n\approx u(t^n)$ avec $t^n=n\Delta t$, et où $\Delta t$ est le pas de temps.

Nous n’étudierons, pour des raisons de performances numériques, que des méthodes dites explicites, c'est-à-dire que chaque étage ne nécessite que les étages précédents pour être calculé. Dans ce cas, la matrice $(a_{ij})_{i,j}$ est triangulaire strictement inférieure. Dans le cadre de méthode explicite, il est possible de convertir la méthode, comme la méthode RK(3,3) de Shu-Osher, pour n'avoir qu'une seule évaluation de la fonction non linéaire $N$ par étage de la méthode.

Un intérêt des méthodes de type Runge-Kutta explicite est la montée en ordre. En effet celle-ci peut se faire de manière presque linéaire du nombre d'étages. À l'inverse, ces méthodes ne préserve pas l'énergie du système qu'elles résolvent, la montée en ordre est donc une nécessité pour réduire l'erreur et garantir la validité des résultats. Un autre inconvénient de ce type de résolution est l'introduction de condition de stabilité, que nous détaillerons un peu plus dans le cadre du chapitre~\ref{chap1}.

Nous bénéficions de la large littérature sur le sujet des méthodes de type Runge-Kutta, l'étude de stabilité ou de convergence (voir~\cite{Shu:2001,Butcher:2008,Gottlieb:2011,Baldauf:2008,Spiteri:2002}), ainsi que des améliorations dans des contextes spécifiques ; telles que les méthodes de Dormand-Prince permettant des stratégies de pas de temps adaptatifs (voir~\cite{Dormand:1978,Dormand:1980,,Gustafsson:1988,,Gustafsson:1994,Balac:2013,Balac:2014}), ou les méthodes de Lawson qui profite de la structure linéaire de l'équation (voir~\cite{Lawson:1967,Isherwood:2018,Hochbruck:2020}).

\subsubsection{Méthode de Lawson}
% --------------------------------------------------------------------

Les méthodes de Lawson sont une optimisation des méthodes de type Runge-Kutta à des équations ayant une partie linéaire que l'on écrit comme suit :
$$
  \dot{u}(t) = Lu(t) + N(t,u)
$$
il s'agit du cas particulier de l'équation~\eqref{eq:0:dtu} où $L$ est une matrice où un opérateur linéaire agissant sur $u$. Le principe de la méthode de Lawson est d'utiliser une formule de Duhamel sur $u$ pour résoudre exactement le terme linéaire. Ceci permet de se soustraire d'une condition de stabilité provenant du terme linéaire, et réduire l'erreur en résolvant exactement le plus de termes possibles.

Nous effectuons une formule de Duhamel en notant $v = e^{-tL}u$, ce qui nous permet de calculer :
$$
  \dot{v}(t) = -Le^{-tL}u(t) + e^{-tL}\dot{u}(t)
$$
d'où :
$$
  \dot{v}(t) = -Le^{-tL}u(t) + e^{-tL}Lu(t) + e^{-tL}N(t,u).
$$
On peut maintenant écrire l'équation sur $v$ que nous souhaitons résoudre avec une méthode de type Runge-Kutta :
$$
  \dot{v} = \tilde{N}(t,v)
$$
avec $\tilde{N}:(t,v)\in\mathbb{R}_+\times\mathbb{R}^n\mapsto e^{-tL}N(t,e^{tL}v)\in\mathbb{R}^n$. La méthode de Lawson consiste à réécrire la méthode Runge-Kutta sur $v$ en la variable $u$, où la partie linéaire est résolue exactement. La méthode de Lawson, induite par une méthode Runge-Kutta explicite décrit par le tableau de Butcher~\eqref{eq:0:butcher}, s'écrit alors :
$$
  \begin{aligned}
    u^{(i)} &= e^{c_i\Delta t L}u^n + \Delta t \sum_{j=1}^{i-1} a_{ij}e^{-(c_j-c_i)\Delta t L}N(t^n+c_j\Delta t,u^{(j)}) \\
    u^{n+1} &= e^{\Delta t L}u^n + \Delta t \sum_{i=1}^{s} b_i e^{(1-c_i)\Delta tL} N(t^n+c_i\Delta t,u^{(i)})
  \end{aligned}
$$

Comme pour une méthode Runge-Kutta classique, il est possible d'appliquer la même méthode d'optimisation de Shu-Osher pour n'avoir qu'une seule évaluation de la fonction non-linéaire $N$ par étage dans le cadre d'une méthode explicite.


\section{Méthodes de résolution numérique en espace}
%%%%%%%%%%%%%%%%%%%%%%%%%%%%%%%%%%%%%%%%%%%%%%%%%%%%%%%%%%%%%%%%%%%%%%

Nous présentons dans cette section les méthodes numériques permettant de discrétiser en espace ($\vb{x}$ ou $\vb{v}$) que nous allons utiliser pour résoudre numériquement le système~\eqref{eq:0:vmhl:1}-\eqref{eq:0:vmhl:4}.

% --------------------------------------------------------------------
\subsection{Méthode WENO}
% --------------------------------------------------------------------

La méthode WENO, pour \emph{Weighted Essentially Non-Oscillatory}, est une méthode volumes finis ou différences finies, dont l'écriture classique est d'ordre 5. Il s'agit d'une méthode \emph{upwind}, d'ordre élevé, combinée à des poids non-linéaires permettant de réduire les oscillations par de la baisse l'ordre et de la diffusion numérique. La méthode d'ordre 5 est présentée dans \cite{Liu:1994,Jiang:1996,Shu:1999,Shu:2003}. Nous la présentons ici pour une équation de transport de la forme :
$$
  \partial_t u + \partial_x f(u) = 0,\qquad u(t=0,x) = u_0(x)
$$
avec $u(t,x)$ la fonction inconnue dépendant du temps $t\geq 0$ et de l'espace $x\in\Omega$ (supposé ici périodique par commodité), et $f:u\mapsto f(u)$ une fonction agissant sur $u$. On définit une discrétisation de l'espace $x_i = i\Delta x + x_0$, $i=0,\dots,N_x$, avec $\Delta x>0$ le pas d'espace. La méthode WENO se présente comme suit :
$$
  \partial_t u_j(t) + \frac{1}{\Delta x}\left( \hat{f}_{j+\frac{1}{2}} - \hat{f}_{j-\frac{1}{2}} \right) = 0,
$$
où $u_j(t)\approx u(t,x_j)$, $j=0,\dots,N$, et où $\hat{f}_{j+\frac{1}{2}} = \hat{f}(u_{j-3},\dots,u_{j+3})$ est le flux numérique, ici présenté pour WENO5, avec $(u_{j-3},\dots,u_{j+3})$ le \emph{stencil} de la méthode, c'est-à-dire le voisinage de points nécessaire pour calculer une approximation de la dérivée en espace. Comme pour une méthode \emph{upwind}, il est nécessaire de distinguer le flux en eux parties, positive et négative :
$$
  f(u) = f^+(u) + f^-(u).
$$
Pour cela il est possible d'utiliser le flux de Lax-Friedrichs (voir~\cite{Shu:1997}). Dans les cas qui nous intéressent, $f:u\mapsto au$ est une fonction linéaire, il est donc simplement nécessaire de connaître le signe de la vitesse d'advection $a$, on note alors $a^+ = \max(a,0)$ et $a^-=\min(a,0)$ et on a $f^\pm_j=f^\pm(u_j)=a^\pm u_j$.

La méthode WENO5 consiste en 3 interpolations pondérées par des poids non-linéaires issus des approximations des dérivées successives de $f$. L'écriture des poids s'effectue comme suit dans le cas $f^-=0$ :
$$
  \begin{aligned}
    \beta_0 &= \frac{13}{12}( \underbrace{f^+_{j-2} - 2f^+_{j-1} + f^+_{j}  }_{\Delta x^2(f''_{j} + \mathcal{O}(\Delta x))}))^2   + \frac{1}{4}( \underbrace{  f^+_{j-2} - 4f^+_{j-1} + 3f^+_{j}  }_{ 2\Delta x ( f'_{j} + \mathcal{O}(\Delta x^2))})^2 \\
    \beta_1 &= \frac{13}{12}( \underbrace{f^+_{j-1} - 2f^+_{j}   + f^+_{j+1}}_{\Delta x^2(f''_{j} + \mathcal{O}(\Delta x^2))} )^2 + \frac{1}{4}( \underbrace{  f^+_{j-1} -               f^+_{j+1}}_{ 2\Delta x   f'_{j} + \mathcal{O}(\Delta x^2))})^2 \\
    \beta_2 &= \frac{13}{12}( \underbrace{f^+_{j}   - 2f^+_{j+1} + f^+_{j+2}}_{\Delta x^2(f''_{j} + \mathcal{O}(\Delta x))} )^2   + \frac{1}{4}( \underbrace{ 3f^+_{j}   - 4f^+_{j+1} +  f^+_{j+2}}_{-2\Delta x ( f'_{j} + \mathcal{O}(\Delta x^2))})^2 \\
  \end{aligned}
$$
où les coefficients $\beta_0$ sont appelés indicateurs de continuité (\emph{indicators of smoothness}). Ce qui nous permet de calculer les poids définis par :
$$
  \alpha_i = \frac{\gamma_i}{(\varepsilon + \beta_i)^2},\quad i=0,1,2
$$
où $\varepsilon$ est un paramètre numérique pour assurer la non nullité du dénominateur, il sera pris à $10^{-6}$ ; et avec $\gamma_0=\frac{1}{10}$, $\gamma_1=\frac{6}{10}$ et $\gamma_2=\frac{3}{10}$. La normalisation des poids s'effectue comme suit :
$$
  w_i = \frac{\alpha_i}{\sum_m \alpha_m},\quad i=0,1,2
$$
Nous pouvons ensuite calculer les flux numériques pour WENO5 \cite{Shu:2003}, donnés par :
$$
  \begin{aligned}
    \hat{f}_{j+\frac{1}{2}}^+   =\ & w_0\left(  \frac{2}{6}f^+_{j-2} - \frac{7}{6}f^+_{j-1} + \frac{11}{6}f^+_{j}   \right)
                                +    w_1\left( -\frac{1}{6}f^+_{j-1} + \frac{5}{6}f^+_{j}   +  \frac{2}{6}f^+_{j+1} \right) \\
                                +  & w_2\left(  \frac{2}{6}f^+_{j}   + \frac{5}{6}f^+_{j+1} -  \frac{1}{6}f^+_{j+2} \right),
  \end{aligned}
$$
La méthode WENO5 prend la forme finale :
$$
  \partial_xf(x_j) \approx \frac{1}{\Delta x}\left[ \left(\hat{f}_{j+\frac{1}{2}}^+ - \hat{f}_{j-\frac{1}{2}}^+ \right) + \left(\hat{f}_{j+\frac{1}{2}}^- - \hat{f}_{j-\frac{1}{2}}^- \right) \right].
$$

Il existe des variantes de la méthode WENO5, permettant de réduire la perte d'ordre à l'approche d'un choc, à savoir WENO-M (\cite{Henrick:2005}) ou WENO-Z (\cite{Borges:2008}). Ces variations se font sur le calcul des poids non-linéaires. Ainsi la méthode WENO-M utilise une fonction de \emph{mappage} pour équilibrer les poids et est définie par :
$$
  \begin{aligned}
    \alpha_i    &= \frac{\gamma_i}{(\epsilon + \beta_i)^2} \\
    \tilde{w}_i &= \frac{\alpha_i}{\sum_k \alpha_k} \\
    g_i         &= w_i\left( \frac{\gamma_i + \gamma_i^2 - 3w_i\gamma_i + w_i^2}{\gamma_i^2 + w_i(1-2\gamma_i)} \right) \\
    w_i         &= \frac{g_i}{\sum_k g_k}
  \end{aligned}
$$
avec le paramètre $\epsilon = 10^{-4}$. La méthode WENO-Z est quant à elle définit par :
$$
  \begin{aligned}
    \alpha_i &= \gamma_i\left( 1+ \frac{\tau_5}{\epsilon + \beta_i} \right) \\
    w_i      &= \frac{\alpha_i}{\sum_k \alpha_k}
  \end{aligned}
$$
avec les paramètres $\epsilon = 10^{-40}$ et $\tau_5 = \beta_0 - \beta_2$. Cette dernière méthode est celle qui réduit le plus la perte d'ordre à l'approche d'une discontinuité. L'étude approfondie de ces méthodes n'est pas envisagée dans ce travail car les solutions de la physique des plasmas ne présentent pas de discontinuités. Il est à noter que ces méthodes conservent la même linéarisation que le schéma WENO5 classique de Jiang et Shu~\cite{Jiang:1996}, ce qui permet d'y appliquer les résultats de stabilités obtenus dans le chapitre~\ref{chap1}.

Il est possible de monter en ordre en suivant les résultats dans~\cite{Wu:2021}, l'ordre 5 sera considéré comme suffisant dans la suite de ce travail.

L'étude de la stabilité de la méthode WENO5 couplée avec différentes méthodes de type Runge-Kutta pour la résolution en temps a été initiée dans~\cite{Wang:2007} où il a été démontré l'instabilité de la méthode couplée avec la méthode d'Euler explicite, il est nécessaire d'avoir au moins un étage supplémentaire permettant d'assurer la stabilité, ou d'utiliser une méthode d'ordre 3. Des estimations de stabilité et de conditions de stabilité ont par la suite été proposées dans \cite{Motamed:2010,Lunet:2017}. Une étude automatique de la stabilité est présentée dans~\cite{Crouseilles:2019b} qui constitue le chapitre~\ref{chap1} de ce document.

% --------------------------------------------------------------------
\subsection{Méthode semi-lagrangienne}
% --------------------------------------------------------------------

Une méthode très populaire pour la résolution numérique de l'équation de Vlasov, car les termes de transports sont linéaires, et que cette méthode d'introduit pas de contrainte de stabilité est la méthode semi-lagrangienne.

% --------------------------------------------------------------------
\subsection{Méthode pseudo-spectrale}
% --------------------------------------------------------------------

Une autre méthode souvent utilisée pour la résolution d'équation aux dérivées partielles linéaires est la méthode pseudo-spectrale qui consiste dans notre cas à effectuer une transformée de Fourier discrète et ainsi transformer une dérivée dans l'espace réel en un produit dans l'espace de Fourier.





% !TEX root = ../chap3.tex

\section{Génération automatique de code}

\Josselin{Je propose de mettre ceci sous forme d'une section ici, mais je ne sais pas trop quoi y dire. Il est compliqué de développer plus sans mettre d'extrait de code \Python{}, et je ne sais pas si cela est nécessaire ou non (rentrer plus dans les détails nécessite de parler un peu plus de l'implémentation de \sympy{}). Je présente ici la génération de code de manière globale, sans parler des problèmes de minimisation des expressions nécessaire dans le cas Padé, et je ne fais que lister les bibliothèques \Python{} que j'utilise. Sachant que ces outils ont déjà été utilisé pour la partie sans approximation de $e^{tL}$, seulement pour de l'aide à l'écriture.}

La simulation d'un système à 7 variables, 6 variables à une dimension, et 1 variable à 4 dimensions, avec une méthode de type Lawson-Runge-Kutta (LRK) d'ordre élevé, nécessite de nombreuses lignes de code dont l'écriture peut s'avérer fastidieuse. Une part importante de l'analyse ayant été réalisée à l'aide de la bibliothèque de calcul symbolique \Python{} : \sympy, il a été décidé de poursuivre son utilisation pour aider à l'écriture du code de simulation. Dans un premier temps cet usage s'est limité à une aide à l'écriture en générant chacune des 7 expressions pour chaque variable, et ce à chaque étage de la méthode LRK (3 étages pour RK(3,3), jusqu'à 5 étages pour une méthode comme DP4(3)). Des outils de méta-programmation ont été utilisés pour obtenir une génération complète du code à partir d'un squelette de code et de l'écriture du schéma LRK que l'utilisateur souhaite utiliser.

Les expressions \sympy{} sont gérer comme des arbres syntaxiques dont les feuilles sont des nombres ou des symboles. Ces derniers vont servir à représenter des variables \CC, il est donc nécessaire dans un premier temps de s'assurer que la conversion de ces symboles en chaînes de caractères assure des noms de variables valide en \CC. En effet il est fréquent d'utiliser des symboles s'exportant facilement en \LaTeX{}, or un tel symbole n'est pas utilisable de la sorte comme nom de variable, par exemple $\Delta t$ sera s'exportera par défaut en chaîne de caractères en "\texttt{\textbackslash Delta\textbackslash\ t}". Les nœuds de l'arbre syntaxique sont des fonctions, il y a alors deux cas à distinguer, soit il s'agit d'une fonction dont la représentation en \Python{} est la même qu'en \CC, auquel cas aucune opération particulière n'est nécessaire, c'est le cas par exemple des opérations arithmétiques $+$, $-$, $\times$ et $\div$ qui sont représentées par les opérateurs binaires \texttt{+}, \texttt{-}, \texttt{*} et \texttt{/} en \Python{} et \CC{} ; soit il s'agit d'une fonction dont la représentation \Python{} et \CC{} diffère, auquel cas il est nécessaire de créer une fonction \sympy{} qui aura le même nom que la fonction \CC{} associée, et de substituer le nœud de l'arbre syntaxique par cette nouvelle fonction. La conversion en chaîne de caractère de l'arbre ainsi modifié sera une expression \CC{} valide. Il est possible d'améliorer l'expression \CC{} en faisant une évaluation numérique des nombre rationnels (et potentiellement aussi irrationnels) présents, pour limiter le nombre d'opérations dans l'expression finale. Ainsi l'expression \texttt{1/3} sera substituée par \texttt{0.333333333333333}, cela permet d'éviter des interprétation de fractions comme des divisions entières par le compilateur.

Pour chaque étage de la méthode LRK, il est ainsi possible d'obtenir une expression \CC{} valide par variable. L'étape supplémentaire pour assumer que l'on est un gros fainéant est d'utiliser un moteur de \emph{template} pour insérer ces expressions dans un squelette de code qui s'adapte automatiquement au nombre d'étages de la méthode LRK, en initialisant et allouant les variables temporaires nécessaires. Ce travail est effectuer par le moteur de \emph{template} Jinja2 qui est une bibliothèque \Python{} permettant d'ajouter des opérations logiques en plus d'une simple substitution de champs dans un squelette de code préexistant. Le squelette en pseudo-code d'un étage d'une méthode LRK est donné en exemple dans l'algorithme~\ref{alg:squeltte}

\begin{algorithm}
  \caption{Squelette de l'algorithme d'un étage $s$ d'une méthode LRK}
  \label{alg:squeltte}
  \begin{algorithmic}
    \State{$\triangleright$ Calcul des variables $\hat{j}_{c,x}^{(s)}$, $\hat{j}_{c,y}^{(s)}$, $\hat{B}_{x}^{(s)}$, $\hat{B}_{y}^{(s)}$, $\hat{E}_{x}^{(s)}$ et $\hat{E}_{y}^{(s)}$}

    \For{$i=0,\dots,N_z$}
      \State $\hat{j}_{h,x,[i]} \gets \sum_{k_x,k_y,k_z} v_{k_x}\,\hat{f}_{h,[i,k_x,k_y,k_z]}^{(s-1)}\,\Delta v$
      \State $\hat{j}_{h,y,[i]} \gets \sum_{k_x,k_y,k_z} v_{k_y}\,\hat{f}_{h,[i,k_x,k_y,k_z]}^{(s-1)}\,\Delta v$
    \EndFor

    \For{$i=0,\dots,N_z$}
      \State $\hat{j}_{c,x,[i]}^{(s)} \gets \dots$ \Comment{les expressions ici sont données par \sympy}
      \State $\hat{j}_{c,y,[i]}^{(s)} \gets \dots$
      \State $\hat{B}_{x,[i]}^{(s)}   \gets \dots$
      \State $\hat{B}_{y,[i]}^{(s)}   \gets \dots$
      \State $\hat{E}_{x,[i]}^{(s)}   \gets \dots$
      \State $\hat{E}_{y,[i]}^{(s)}   \gets \dots$
    \EndFor

    \State
    \State{$\triangleright$ Calcul de la variable $\hat{f}_h^{(s)}$}

    \State $\left(f\right)_{h,[\cdot,k_x,k_y,k_z]} \gets \textrm{iFFT}_z\left( \hat{f}^{(s-1)}_{h,[\cdot,k_x,k_y,k_z]} \right)$
    \ForAll{ $(k_x,k_y,k_z)\in [\![0,N_x]\!] \times [\![0,N_y]\!] \times [\![0,N_z]\!]$ }
      \For{$i=0,\dots,N_z$}
        \State $a_{v_x} \gets E_{x,[i]} + v_{k_y}B_0 + v_{k_z}B_{y,[i]}$
        \State $a_{v_y} \gets E_{y,[i]} + v_{k_x}B_0 + v_{k_z}B_{x,[i]}$
        \State $a_{v_z} \gets v_{k_x}B_{y,[i]} + v_{k_y}B_{x,[i]}$
        \State $\begin{aligned}\partial_vf_{h,[i,k_x,k_y,k_z]} \gets &\text{WENO}(a_{v_x},f_{h,[i,k_x-3:k_x+3,k_y,k_z]})+\text{WENO}(a_{v_y},f_{h,[i,k_x,k_y-3:k_y+3,k_z]})\\
       &+ \text{WENO}(a_{v_z},f_{h,[i,k_x,k_y,k_z-3:k_z+3]})\end{aligned}$
      \EndFor
    \EndFor

    \ForAll{ $(k_x,k_y,k_z)\in [\![0,N_x]\!] \times [\![0,N_y]\!] \times [\![0,N_z]\!]$ }
      \State $\left(\widehat{\partial_vf}\right)_i \gets \text{FFT}_z(\partial_vf_{\cdot,k_x,k_y,k_z})$
      \For{$i=0,\dots,N_z$}
        \State $\hat{f}_h^{(s)} \gets \dots$ \Comment{l'expression ici est donnée par \sympy}
      \EndFor
    \EndFor
  \end{algorithmic}
\end{algorithm}

La mise en place de l'opération de filtrage dans le pseudo-code~\ref{alg:squeltte} nécessite seulement de modifier le calcul des variables de courants chauds $\left(\hat{j}_{h,x}\right)_i$, $\left(\hat{j}_{h,y}\right)_i$ et des vitesses d'advection $a_{v_x}$, $a_{v_y}$ et $a_{v_z}$ :
$$
  \begin{aligned}
    \hat{j}_{h,x,[i]} &\gets \sum_{k_1,k_2,k_z} ( w_1\cos(B_0\tau^{n,s}) - w_2\sin(B_0\tau^{n,s}) ) \hat{g}_{[i,k_1,k_2,k_z]} \Delta w\Delta v_z\\
    \hat{j}_{h,y,[i]} &\gets \sum_{k_1,k_2,k_z} ( w_1\sin(B_0\tau^{n,s}) + w_2\cos(B_0\tau^{n,s}) ) \hat{g}_{[i,k_1,k_2,k_z]} \Delta w\Delta v_z\\
    a_{v_x} &\gets E_{x,[i]}\cos(B_0\tau^{n,s}) + E_{y,[i]}\sin(B_0\tau^{n,s}) + v_zB_{x,[i]}\sin(B_0\tau^{n,s}) - v_zB_{y,[i]}\cos(B_0\tau^{n,s})\\
    a_{v_y} &\gets -E_{x,[i]}\sin(B_0\tau^{n,s}) + E_{y,[i]}\cos(B_0\tau^{n,s}) + v_zB_{x,[i]}\cos(B_0\tau^{n,s}) + v_zB_{y,[i]}\sin(B_0\tau^{n,s})\\
    a_{v_z} &\gets -B_{x,[i]}(w_1\sin(B_0\tau^{n,s}) + w_2\cos(B_0\tau^{n,s})) + B_{y,[i]}(w_1\cos(B_0\tau^{n,s}) - w_2\sin(B_0\tau^{n,s}) )\\
  \end{aligned}
$$
où $\tau^{n,s}=t^n+c_s\Delta t$.

\paragraph{Nota Bene :} La bibliothèque \sympy{} contient des fonctions permettant la génération de code en C ou Fortran, mais le fonctionnement de celles-ci s'adapte mal à une intégration dans une boucle d'un code déjà existant. De plus les fonctions ainsi générés ne fonctionnent pas avec un code contenant des \emph{template} \CC, pour changer éventuellement de type pour de possibles optimisations. Elles ne prennent en paramètre que des valeurs par copie ou par pointeur, ce qui limite leur usage avec des structures de données évoluées proposées par les librairies \CC. Il serait envisageable d'utiliser certains des mécanismes présents dans ces fonctions pour améliorer la génération de code proposé ci-dessus, en utilisant un parcours d'arbre syntaxique pour construire un \emph{Abstract Syntax Tree} (AST) permettant la génération dans n'importe quel langage d'une expression. \Josselin{Et derniers points, ces fonctions sont très mal documentées (je les ai découverte alors que je générais déjà les lignes de code pour le Lawson-RK(3,3) et que celui-ci tournait bien), et elles laissent des 1/2, 1/3 etc. qui peuvent valoir 0 selon les options de compilation ou les compilateurs.}



% !TEX root = ../chap3.tex

\section{Résultats numériques}

\subsection{Calcul de stabilité avec les équations de Maxwell}




% !TEX root = ../../main.tex

\section{Approximation de la partie linéaire}
\label{s3:approx}

L'obtention, à l'aide d'un logiciel de calcul formel, de l'exponentielle de la partie linéaire n'est pas toujours envisageable. Il est possible de recourir à une méthode d'approximation pour obtenir une formulation formelle de celle-ci qui sera possible d'utiliser pour l'écriture du code de simulation. On s'intéressera dans cette section à la partie linéaire $L$ définie par :
$$
  L = \begin{pmatrix}
    0   & -B_0 & 0          &  0          &  \Omega_{pe}^2 & 0             & 0 \\
    B_0 &  0   & 0          &  0          &  0             & \Omega_{pe}^2 & 0 \\
    0   &  0   & 0          &  0          &  0             & \partial_z    & 0 \\
    0   &  0   & 0          &  0          & -\partial_z    & 0             & 0 \\
   -1   &  0   & 0          & -\partial_z &  0             & 0             & 0 \\
    0   & -1   & \partial_z &  0          &  0             & 0             & 0 \\
    0   &  0   & 0          &  0          &  0             & 0             & -v_z\partial_z \\
  \end{pmatrix}
$$
où on rappelle que $\Omega_{pe}=2$. Cette matrice est de la forme :
$$
  L = \begin{pmatrix}
    A & 0 \\
    0 & -v_z\partial_z
  \end{pmatrix}
$$
matrice diagonale par blocs, dont seul le bloc $A$ pose problème pour calculer formellement l'exponentielle. Ainsi on s'intéressera surtout à la sous-matrice $A$ obtenue après une transformée de Fourier en $z$ du système :
$$
  A = \begin{pmatrix}
    0 & -1 & 0  &  0  &  \Omega_{pe}^2  & 0             \\
    1 &  0 & 0  &  0  &  0              & \Omega_{pe}^2 \\
    0 &  0 & 0  &  0  &  0              & ik            \\
    0 &  0 & 0  &  0  & -ik             & 0             \\
   -1 &  0 & 0  & -ik &  0              & 0             \\
    0 & -1 & ik &  0  &  0              & 0             \\
  \end{pmatrix}
$$
Par abus de notation, nous noterons $A_0$, la matrice $A$ pour $k=0$, ce qui revient à une partie linéaire sans les équations de Maxwell, ceci sera utile lors de la comparaison des résultats entre les méthodes.

La matrice $A$ est telle que toutes ses valeurs propres sont imaginaires pures, $\texttt{sp}(A)\subset i\mathbb{R}$. De plus, $A$ est presque anti-hermitienne, on peut trouver une matrice de passage $\Omega$ permettant de transformer notre matrice $A$ en une matrice anti-hermitienne :
$$
  \Omega = 
  \begin{pmatrix}
    \dmat[0]{\Omega_{pe}^{-1/2},\Omega_{pe}^{-1/2},\Omega_{pe}^{1/2},\Omega_{pe}^{1/2},\Omega_{pe}^{1/2},\Omega_{pe}^{1/2}}
  \end{pmatrix},
$$
où $\Omega$ est obtenu avec les mêmes stratégies qu'un symétriseur. On a $H=\Omega A \Omega^{-1}$ anti-hermitienne. Le spectre de $H$ est imaginaire pur, et $H$ est diagonalisable dans une base orthonormale, c'est-à-dire :
$$
  \exists Q \in GL_6(\mathbb{C}) / H = QDQ^{-1}, \|Q\|=1
$$
où $D$ est une matrice diagonale des valeurs propres de $H$. Ce qui nous permet de déterminer que les valeurs propres de l'exponentielle de $H$ sont sur le cercle unitaire, en effet :
$$
  \forall t\in\mathbb{R}, e^{tH} = Qe^{tD}Q^{-1}.
$$
Et on a, de même pour la matrice $A$, des valeurs propres distribuées sur le cercle unité, avec la matrice de passage $\Omega Q$, comme l'illustre la figure~\ref{fig:evexpAk}, où les valeurs propres de $A$ sont calculées numériquement pour différents modes de Fourier.

\begin{figure}
  \centering
  \includegraphics[width=0.5\textwidth]{\localPath/figures/evexpAk.png}
  \caption{Valeurs propres de $e^{A}$ pour différentes fréquences $\kappa\in[\![0,15]\!]$. Les valeurs propres des modes de Fourier négatifs, sont obtenues par parité.}
  \label{fig:evexpAk}
\end{figure}

\subsection{Troncature de la série de Taylor}
%--------------------------------------------------------------------

Un manière classique de définir la fonction exponentielle est par sa série de Taylor, ainsi on définit $e^{tA}$ par :
$$
  e^{tA} = \sum_{n=0}^\infty \frac{t^nA^n}{n!}.
$$
Une troncature d'ordre suffisamment élevé permet d'obtenir une approximation de l'exponentielle $e^{tA}$ à un ordre plus élevé que la méthode LRK($s$,$n$) où elle sera utilisé garanti que l'erreur de troncature reste inférieur à $n$, l'ordre de la méthode en temps. On définit la troncature de la série de Taylor à l'ordre $m$ par :
$$
  T_m(A) = \sum_{k=0}^m \frac{A^k}{k!}.
$$

Nous savons que les valeurs propres de $e^{A}$ sont sur le cercle unité, préserver cette propriété permet d'assurer la stabilité du schéma. Nous regardons donc les valeurs propres de $T_5(A)$, pour les modes de Fourier $\kappa\in[\![0,15]\!]$ sur la figure~\ref{fig:ev:T5}. On remarque que le module des valeurs propres à 1 n'est pas préservé.

\begin{figure}
  \centering
  \includegraphics[width=0.5\textwidth]{\localPath/figures/ev_T5.png}
  \caption{Valeurs propres de $T_5(A)$ pour différentes fréquences $\kappa\in[\![0,15]\!]$. Les valeurs propres des modes de Fourier négatifs, sont obtenues par parité.}
  \label{fig:ev:T5}
\end{figure}

On peut regarder, pour différentes valeurs de $m$, l'erreur de troncature faite en norme matricielle pour 2 modes de Fourier $\kappa=2$ et $15$. C'est ce que l'on trace sur la figure~\ref{fig:taylor:error}.

% \begin{figure}
%   \begin{subfigure}{.5\textwidth}
%     \centering
%     \includegraphics[width=\textwidth]{\localPath/figures/approx_evA0T5.png}
%     \caption{Les valeurs propres de $e^{A_0}$ et de $T_5(A_0)$}
%   \end{subfigure}
%   \begin{subfigure}{.5\textwidth}
%     \centering
%     \includegraphics[width=\textwidth]{\localPath/figures/approx_evAkT5.png}
%     \caption{Les valeurs propres de $e^{A}$ et de $T_5(A)$ pour différentes valeurs de $k\in[\![0,15]\!]$, par symétrie on obtient aussi celles pour $k<0$}
%   \end{subfigure}
%   \caption{Valeurs propres de $e^{A}$ et de $T_5(A)$ pour $k=0$ (sans les équations de Maxwell) à gauche, et pour différentes valeurs de $k\in[\![0,15]\!]$ à droite.}
% \end{figure}

\begin{figure}
  \begin{subfigure}{.5\textwidth}
    \centering
    \includegraphics[width=\textwidth]{\localPath/figures/approx_errortA2T.png}
    \caption{L'erreur absolue locale $\|e^{tA}-T_p(tA)\|$ pour le mode de Fourier $k=2$}
  \end{subfigure}
  \begin{subfigure}{.5\textwidth}
    \centering
    \includegraphics[width=\textwidth]{\localPath/figures/approx_errortA15T.png}
    \caption{L'erreur absolue locale $\|e^{tA}-T_p(tA)\|$ pour le mode de Fourier $k=15$}
  \end{subfigure}
  \caption{Erreur absolue locale $\|e^{tA}-T_p(tA)\|$ pour deux modes de Fourier $k=2$ à gauche et $k=15$ à droite.}
  \label{fig:taylor:error}
\end{figure}

\subsection{Approximant de Padé}
%--------------------------------------------------------------------

Pour approcher une fonction, au lieu d'utiliser un polynôme comme dans le cadre des séries des développement limités, il est possible de construire une fraction rationnelle. L'approximant de Padé de la fonction exponentielle est la meilleure approximation de la fonction exponentielle par une fraction rationnelle et est définie par :
$$
  \begin{aligned}
    h_{p,q}(x) &= \sum_{i=0}^p \frac{\frac{p!}{(p-i)!}}{\frac{(p+q)!}{(p+q-i)!}}\frac{x^i}{i!} \\
    k_{p,q}(x) &= \sum_{j=0}^q (-1)^j \frac{\frac{q!}{(q-j)!}}{\frac{(p+q)!}{(p+q-j)!}} \frac{x^j}{j!}
  \end{aligned}
$$

$$
  P_{p,q}(x) = \frac{h_{p,q}(x)}{k_{p,q}(x)} \approx e^x
$$

Pour utiliser cet approximant de Padé, qui est une fraction rationnelle, avec des matrices il faut utiliser la définition suivante :

$$
  e^M \approx P_{p,q}(M) = h_{p,q}(M)\cdot\left(k_{p,q}(M)\right)^{-1}
$$

L'introduction d'une fraction rationnelle permet de préserver certaines propriétés comme le fait que $P_{p,p}(-z) = \frac{1}{P_{p,p}(z)}$. Nous regardons les valeurs propres de $P_{2,2}(A)$, pour les modes de Fourier $\kappa\in[\![0,15]\!]$ sur la figure~\ref{fig:ev:P22}, et l'on remarque que, contrairement à la troncature de la série de Taylor, le module des valeurs propres est 1.

\begin{figure}
  \centering
  \includegraphics[width=0.5\textwidth]{\localPath/figures/ev_P22.png}
  \caption{Valeurs propres de $P_{2,2}(A)$ pour différentes fréquences $\kappa\in[\![0,15]\!]$. Les valeurs propres des modes de Fourier négatifs, sont obtenues par parité.}
  \label{fig:ev:P22}
\end{figure}

% \begin{figure}
%   \begin{subfigure}{.5\textwidth}
%     \centering
%     \includegraphics[width=\textwidth]{\localPath/figures/approx_evA0P.png}
%     \caption{Les valeurs propres de $e^{A_0}$ et de $P_{n,n}(A_0)$}
%   \end{subfigure}
%   \begin{subfigure}{.5\textwidth}
%     \centering
%     \includegraphics[width=\textwidth]{\localPath/figures/approx_evAkP22.png}
%     \caption{Les valeurs propres de $e^{A}$ et de $P_{2,2}(A)$ pour différentes valeurs de $k\in[\![0,15]\!]$, par symétrie on obtient aussi celles pour $k<0$}
%   \end{subfigure}
%   \caption{Valeurs propres de $e^{A}$ et de $P_{n,n}(A)$ pour $k=0$ (sans les équations de Maxwell) à gauche, et pour différentes valeurs de $k\in[\![0,15]\!]$ à droite.}
%   \label{fig:evAP22}
% \end{figure}

\begin{figure}
  \begin{subfigure}{.5\textwidth}
    \centering
    \includegraphics[width=\textwidth]{\localPath/figures/approx_errortA2P.png}
    \caption{L'erreur absolue locale $\|e^{tA}-T_p(tA)\|$ pour le mode de Fourier $k=2$}
  \end{subfigure}
  \begin{subfigure}{.5\textwidth}
    \centering
    \includegraphics[width=\textwidth]{\localPath/figures/approx_errortA15P.png}
    \caption{L'erreur absolue locale $\|e^{tA}-T_p(tA)\|$ pour le mode de Fourier $k=15$}
  \end{subfigure}
  \caption{Erreur absolue locale $\|e^{tA}-T_p(tA)\|$ pour deux modes de Fourier $k=2$ à gauche et $k=15$ à droite.}
\end{figure}

\begin{figure}
  \centering
  \includegraphics[width=0.75\textwidth]{\localPath/figures/approx_errortAkP22.png}
  \caption{L'erreur absolue locale $\|e^{tA}-T_5(tA)\|$ pour différents mode de Fourier}
\end{figure}


\subsection{Test sur un modèle jouet}
%--------------------------------------------------------------------

Nous souhaitons mettre en place cette stratégie d'approximation de la partie linéaire d'une méthode de Lawson sur une première équation qui nous servira de \emph{toy model} :
$$
  \partial_t u = a\partial_x u + b\partial_y u
$$
où $u(t,x,y)$ est une fonction à valeurs réelles, $(x,y)\in[-2,2]\times[-2,2]$, $t\geq0$, $a,b\in\mathbb{R}$ donnés, et $u(t=0,x,y)=u_0(x,y)$ donné. On sait que la solution au temps $t$ est donnée par :
$$
  u(t,x,y) = u_0(x-at,y-bt).
$$
On souhaite résoudre cette équation par une méthode spectrale dans la direction $y$ (partie linéaire de la méthode de Lawson), une méthode WENO5 dans la direction $x$ (partie non-linéaire), et une méthode LRK($s$,$n$) en temps (méthode de Lawson d'ordre $n$ à $s$ étages). Nous obtenons donc l'équation :
$$
  \partial_t \hat{u} = L\hat{u} + N(\hat{u}),\quad u(t=0,x,y) = u_0(x,y)
$$
avec $L = i\kappa b$ et, $N:\hat{u}\mapsto\widehat{a\partial_xu}$. À partir des résultats du chapitre~\ref{chap1}, on peut calculer l'erreur du schéma en linéarisant la partie non-linéaire :
$$
  u^{n+1} = e^{\Delta tL}\sum_{i=0}^{n}\frac{\Delta t^iN^i}{i!}u^n = e^{\Delta t (L+N)}u^n + \order{\Delta t^{n+1}}
$$
ce qui nous permet de dire que la partie linéaire est résolue exactement, et que la partie non-linéaire est d'ordre 3. Il est également possible de vouloir approcher l'exponentielle de la partie linéaire $e^{\Delta tL}$ avec une troncature de la série de Taylor ou un approximant de Padé :
$$
  \begin{aligned}
    u^{n+1} =& T_m(\Delta tL)\sum_{i=0}^{n}\frac{\Delta t^iN^i}{i!}u^n     & & \text{pour la série de Taylor d'ordre $m$} \\
    u^{n+1} =& P_{p,q}(\Delta tL)\sum_{i=0}^{n}\frac{\Delta t^iN^i}{i!}u^n & & \text{pour l'approximant de Padé d'ordre $(p,q)$}
  \end{aligned}
$$
avec $T_m(\Delta tL) = e^{\Delta tL}+\order{\Delta t^m}$ et $P_{p,q}(\Delta tL) = e^{\Delta tL}+\order{\Delta t^{p+q}}$. Ce qui nous permet d'écrire :
$$
  u^{n+1} = e^{\Delta t (L+N)}u^n + \order{\Delta t^r} + \order{\Delta t^{n+1}}
$$
avec $r=m$ pour la troncature de la série de Taylor, ou $r=p+q$ pour l'approximant de Padé. Ainsi l'erreur en temps du schéma, approché par une méthode d'interpolation de la partie linéaire, est une contribution de deux termes, et l'ordre en temps de la méthode est $\min(r,n)$.

Il est possible de retrouver ce résultat numériquement en comparant les différentes méthodes et en mesurant l'erreur effectuée pour différentes valeurs de pas de temps $\Delta t$ et ainsi mesure l'ordre en temps de la méthode. Pour cela on choisit le schéma LRK(3,3) induit par la méthode SSP RK(3,3) de Shu-Osher, on se munit d'une discrétisation de $243$ points par direction, de plusieurs valeurs de pas de temps $\Delta t\in[0.00158,0.02370]$. La simulation s'effectue jusqu'au temps final $T_f=0.07111$ avec les vitesses $a=1.0$ et $b=0.75$. On regarde l'erreur, par rapport à notre solution de référence en norme 1 :
$$
  e_1 = \| u(t,x,y) - u_0(x-at,y-bt) \|_1 \approx \sum_{i,j}|u^n_{i,j}-u_0(x_i-at,y_j-bt)|\Delta x\Delta y,
$$
ce qui nous permet de tracer, sur la figure~\ref{fig:lrk33ref} l'erreur pour la méthode de référence avec un calcul exact de l'exponentielle de la partie linéaire, ainsi qu'avec une troncature de la série de Taylor d'ordre 4, ordre supérieur à celui de la méthode de Lawson, et un approximant de Padé d'ordre $(2,2)$ (équivalent à un ordre 4), aussi supérieur à l'ordre de la méthode de Lawson, on retrouve bien dans ces cas là l'ordre 3 de la méthode LRK(3,3). La figure~\ref{fig:lrk33taylor} représente l'ordre pour différentes troncatures de la série de Taylor, on remarque que pour des ordres inférieurs à l'ordre de la méthode de Lawson, on mesure l'ordre de la méthode de Taylor. La figure~\ref{fig:lrk33pade} quant à elle, représente l'ordre pour différents approximants de Padé, on retrouve les mêmes résultats avec une constante d'erreur plus faible.

\begin{figure}
  \centering
  \begin{subfigure}{.45\textwidth}
    \centering
    \includegraphics[width=\textwidth]{\localPath/figures/order_lrk33ref.png}
    \caption{Ordre en temps de la méthode LRK(3,3) associée à exponentielle, la série de Taylor d'ordre 4, et l'approximant Padé d'ordre $(2,2)$.}
    \label{fig:lrk33ref}
  \end{subfigure}
  \begin{subfigure}{.45\textwidth}
    \centering
    \includegraphics[width=\textwidth]{\localPath/figures/order_lrk33taylor.png}
    \caption{Ordre en temps de la méthode LRK(3,3) associée à la série de Taylor d'ordre 1 à 4.\\ }
    \label{fig:lrk33taylor}
  \end{subfigure}
  \begin{subfigure}{.45\textwidth}
    \centering
    \includegraphics[width=\textwidth]{\localPath/figures/order_lrk33pade.png}
    \caption{Ordre en temps de la méthode LRK(3,3) associée à l'approximant Padé d'ordre $(1,1)$, $(2,1)$, $(1,2)$, et $(2,2)$.}
    \label{fig:lrk33pade}
  \end{subfigure}
  \caption{Ordre en temps de la méthode LRK(3,3) où l'exponentielle de la partie linéaire est approchée par différentes méthodes, série de Taylor ou approximant de Padé de différents ordre. La solution de référence à gauche, le test avec différentes séries de Taylor à droite, et avec différents approximant de Padé en bas.}
  \label{fig:3:order}
\end{figure}


\section{Les sections manquantes sur de l'optimisation de code}
\Josselin{Je n'ai toujours pas touché à l'optimisation avec OpenMP. J'ai trouvé des références pour faire en sorte que ça se passe bien avec FFTW, donc ça ne devrait pas prendre trop de temps (je l'espère).}

\Josselin{On avait aussi évoqué avec Nicolas de tester de la \emph{mixed precision} c'est-à-dire de ne travailler qu'avec des \texttt{float} sur $f_h$ et des \texttt{double} sur les autres variables, le problème (pour que ce soit performant) est d'avoir des algorithmes efficaces pour faire en sorte qu'une somme de \texttt{float} donne un \texttt{double}, j'ai cru voir des choses dans ce sens là dans la bibliothèque \CC{} que j'utilise : Boost.}

\Josselin{N'est pas du tout évoqué une utilisation de MPI, les multiples FFT (algorithme non local) ne sont pas favorables à une telle mise en place. Pas de portage sur GPU n'est pas évoqué non plus. On remarque que dans l'algorithme~\ref{alg:squeltte} sur un étage que la partie sur $f_h$ peut être traitée indépendamment des champs spatiaux, donc potentiellement sur GPU (qui gèrent par défaut des \texttt{float} et non des \texttt{double}) pendant que les champs sont calculés sur CPU, permettant des sorties de monitoring plus faciles des grandeurs intégrales. Bref c'est une ouverture possible.}

\section{Références bibliographiques}

\printbibliography

\appendix
% annexe
%% !TEX root = ../../main.tex

\section{Résultats sur les relations de dispersion}
\label{a:dispersion}

Cette annexe est dédiée aux démonstrations des propriétés énoncées dans la section \ref{s:dispersion} sur les relations de dispersion.

Nous démontrons tout d'abord le lemme \ref{lemma:doubleracine}, qui concerne la symétrie des racines de $D(k,\omega)$, et dont l'énoncé est rappelé ci-dessous.
\begin{lemma}
  Si $f^{(0)}(v)$ (respectivement $f_h^{(0)}(v)$) est une fonction paire, alors pour $D(k,\omega)$ défini par (\ref{eq:D}) (respectivement (\ref{eq:relD_H})) nous avons $D(k,\omega_r+i\omega_i) = 0 \Leftrightarrow D(k,-\omega_r+i\omega_i)=0$.
\end{lemma}

\begin{proof}
  Nous le vérifions dans le cas cinétique, les calculs étant similaires dans le cas hybride. Avec la définition (\ref{eq:D}) de $D(k,\omega)$, nous avons 
  \begin{eqnarray*}
    &&D(k,\omega_r+i\omega_i)=0\\
    &\Leftrightarrow&
    \Re\left(\frac{1}{k^2}\int_\gamma\frac{\partial_vf^0}{v-\frac{\omega_r+i\omega_i}{k}}dv\right)=1,~\Im\left(\frac{1}{k^2}\int_\gamma\frac{\partial_vf^0}{v-\frac{\omega_r+i\omega_i}{k}}dv\right)=0. 
  \end{eqnarray*}
  Distinguons les parties réelles et imaginaires :
  \begin{eqnarray*}
    \int_\gamma\frac{\partial_vf^0(v)}{v-\frac{\omega_r+i\omega_i}{k}}dv=\int_\gamma\frac{\partial_vf^0(v)}{\left(v-\frac{\omega_r+i\omega_i}{k}\right)\left(v-\frac{\omega_r-i\omega_i}{k}\right)}\left(v-\frac{\omega_r-i\omega_i}{k}\right)dv\\
    =\int_\gamma\frac{\partial_vf^0(v)}{\left(v-\frac{\omega_r}{k}\right)^2+\left(\frac{\omega_i}{k}\right)^2}\left(v-\frac{\omega_r}{k}\right)dv+i\int_\gamma\frac{\partial_vf^0(v)}{\left(v-\frac{\omega_r}{k}\right)^2+\left(\frac{\omega_i}{k}\right)^2}\frac{\omega_i}{k}dv.
  \end{eqnarray*}
  Maintenant, considérons $\omega=-\omega_r+i\omega_i$ et rappelons qu'on a supposé que $f^0(v)$ était une fonction paire. Nous obtenons
  \begin{eqnarray*}
    \int_\gamma\frac{\partial_vf^0(v)}{v-\frac{-\omega_r+i\omega_i}{k}}dv~~~~~~~~~~~~~~~~~~~~~~~~~~~~~~~~~~~~~~~~~~~~~~~~~~~~~~~~~~~~~~~~~~~~~~~~~~~~~~~~\\
    =\int_\gamma\frac{\partial_vf^0(v)}{\left(v+\frac{\omega_r}{k}\right)^2+\left(\frac{\omega_i}{k}\right)^2}\left(v+\frac{\omega_r}{k}\right)dv+i\int_\gamma\frac{\partial_vf^0(v)}{\left(v+\frac{\omega_r}{k}\right)^2+\left(\frac{\omega_i}{k}\right)^2}\frac{\omega_i}{k}dv~~~~\\
    =-\int_\gamma\frac{\partial_{v}f^0(-v)}{\left(-v+\frac{\omega_r}{k}\right)^2+\left(\frac{\omega_i}{k}\right)^2}\left(v-\frac{\omega_r}{k}\right)dv+i\int_\gamma\frac{\partial_vf^0(-v)}{\left(-v+\frac{\omega_r}{k}\right)^2+\left(\frac{\omega_i}{k}\right)^2}\frac{\omega_i}{k}dv\\
    =\int_\gamma\frac{\partial_vf^0(v)}{\left(v-\frac{\omega_r}{k}\right)^2+\left(\frac{\omega_i}{k}\right)^2}\left(v-\frac{\omega_r}{k}\right)dv-i\int_\gamma\frac{\partial_vf^0(v)}{\left(v-\frac{\omega_r}{k}\right)^2+\left(\frac{\omega_i}{k}\right)^2}\frac{\omega_i}{k}dv~~~~.
  \end{eqnarray*}
  D'où
  \begin{eqnarray*}
    \Re\left(\frac{1}{k^2}\int_\gamma\frac{\partial_vf^0}{v-\frac{\omega_r+i\omega_i}{k}}dv\right)=1,~\Im\left(\frac{1}{k^2}\int_\gamma\frac{\partial_vf^0}{v-\frac{\omega_r+i\omega_i}{k}}dv\right)=0\\
    \Leftrightarrow
    \Re\left(\frac{1}{k^2}\int_\gamma\frac{\partial_vf^0}{v-\frac{-\omega_r+i\omega_i}{k}}dv\right)=1,~\Im\left(\frac{1}{k^2}\int_\gamma\frac{\partial_vf^0}{v-\frac{-\omega_r+i\omega_i}{k}}dv\right)=0
  \end{eqnarray*}
  et
  $$
    D(k,\omega_r+i\omega_i)=0\Leftrightarrow D(k,-\omega_r+i\omega_i)=0.
  $$
\end{proof}

Nous allons maintenant démontrer les lemmes \ref{lemma:Z0}, \ref{lemma:Z+} et \ref{lemma:Z-}, dont les énoncés sont rappelés ci-dessous, qui donnent des propriétés de la fonction de Fried-Conte (\ref{eq:Zfct}).

\begin{lemma}
  La fonction $Z_\alpha^0(\omega):\omega\in\mathbb{C}\mapsto Z\left(\alpha\omega\right)\in\mathbb{C}$, avec $\alpha\in\mathbb{R}$ fixé, est telle que : $Z_\alpha^0(-\bar{\omega}) = -\overline{Z_\alpha^0(\omega)}$.
\end{lemma}
 
\begin{proof}
  Par définition de la fonction de Fried-Conte, et avec la notation $\omega=\omega_r+i\omega_i$, nous avons
  \begin{eqnarray*}
    Z(\alpha(\omega_r+i\omega_i))=\frac{1}{\sqrt{\pi}}\int_\gamma\frac{e^{-z^2}}{z-\alpha(\omega_r+i\omega_i)}dz=\frac{1}{\sqrt{\pi}}\int_\gamma\frac{e^{-z^2}(z-\alpha\omega_r+i\alpha\omega_i)}{(z-\alpha\omega_r)^2+(\alpha\omega_i)^2}dz
  \end{eqnarray*}
  d'où
  \begin{eqnarray*}
    \Re\left(Z(\alpha(\omega_r+i\omega_i))\right)=\frac{1}{\sqrt{\pi}}\int_\gamma\frac{e^{-z^2}(z-\alpha\omega_r)}{(z-\alpha\omega_r)^2+(\alpha\omega_i)^2}dz\\
    \Im\left(Z(\alpha(\omega_r+i\omega_i))\right)=\frac{1}{\sqrt{\pi}}\int_\gamma\frac{e^{-z^2}\alpha\omega_i}{(z-\alpha\omega_r)^2+(\alpha\omega_i)^2}dz.
  \end{eqnarray*}

  Maintenant, $-\overline{\omega}=-\omega_r+i\omega_i$, implique
  \begin{eqnarray*}
    Z(\alpha(-\omega_r+i\omega_i))&=&\frac{1}{\sqrt{\pi}}\int_\gamma\frac{e^{-z^2}}{z-\alpha(-\omega_r+i\omega_i)}dz\\
    &=&\frac{1}{\sqrt{\pi}}\int_\gamma\frac{e^{-z^2}(z+\alpha\omega_r+i\alpha\omega_i)}{(z+\alpha\omega_r)^2+(\alpha\omega_i)^2}dz\\
    &=&\frac{1}{\sqrt{\pi}}\int_\gamma\frac{e^{-z^2}(-z+\alpha\omega_r+i\alpha\omega_i)}{(-z+\alpha\omega_r)^2+(\alpha\omega_i)^2}dz\\
    &=&-\frac{1}{\sqrt{\pi}}\int_\gamma\frac{e^{-z^2}(z-\alpha\omega_r)}{(z-\alpha\omega_r)^2+(\alpha\omega_i)^2}dz+i\frac{1}{\sqrt{\pi}}\int_\gamma\frac{e^{-z^2}\alpha\omega_i}{(z-\alpha\omega_r)^2+(\alpha\omega_i)^2}dz
  \end{eqnarray*}
  d'où
  \begin{eqnarray*}
    \Re\left(Z(\alpha(-\omega_r+i\omega_i))\right)=-\Re\left(Z(\alpha(\omega_r+i\omega_i))\right)\\
    \Im\left(Z(\alpha(-\omega_r+i\omega_i))\right)=\Im\left(Z(\alpha(\omega_r+i\omega_i))\right),
  \end{eqnarray*}
  ce qui termine la preuve.
\end{proof}


\begin{lemma}
  La fonction $Z_{\alpha,\beta}^+(\omega):\omega\in\mathbb{C}\mapsto Z\left(\alpha\omega-\beta\right)+Z\left(\alpha\omega+\beta\right)\in\mathbb{C}$, avec $\alpha\in\mathbb{R}$, $\beta\in\mathbb{R}$ fixés, est telle que : $Z_{\alpha,\beta}^+\left(-\overline{\omega}\right)=-\overline{Z_{\alpha,\beta}^+(\omega)}$.
\end{lemma}
  
\begin{proof}
  Nous avons par définition de la fonction de Fried-Conte
  \begin{eqnarray*}
    &&Z(\alpha\omega-\beta)+Z(\alpha\omega+\beta)=\frac{1}{\sqrt{\pi}}\int_\gamma\frac{e^{-z^2}}{z-\alpha\omega+\beta}+\frac{e^{-z^2}}{z-\alpha\omega-\beta}dz\\
    &=&\frac{1}{\sqrt{\pi}}\int_\gamma\frac{e^{-z^2}(z-\alpha\omega-\beta)+e^{-z^2}(z-\alpha\omega+\beta)}{(z-\alpha\omega)^2-\beta^2}dz\\
    &=&\frac{2}{\sqrt{\pi}}\int_\gamma\frac{e^{-z^2}(z-\alpha\omega)}{(z-\alpha\omega)^2-\beta^2}dz.
  \end{eqnarray*}
  Maintenant, avec la notation $\omega=\omega_r+i\omega_i$, nous avons
  \begin{eqnarray*}
    &&Z(\alpha(\omega_r+i\omega_i)-\beta)+Z(\alpha(\omega_r+i\omega_i)+\beta)=\frac{2}{\sqrt{\pi}}\int_\gamma\frac{e^{-z^2}(z-\alpha\omega_r-i\alpha\omega_i)}{(z-\alpha\omega_r-i\alpha\omega_i)^2-\beta^2}dz\\
    &=&\frac{2}{\sqrt{\pi}}\int_\gamma\frac{e^{-z^2}(z-\alpha\omega_r-i\alpha\omega_i)}{(z-\alpha\omega_r)^2-(\alpha\omega_i)^2-\beta^2-2i\alpha\omega_i(z-\alpha\omega_r)}dz\\
    &=&\frac{2}{\sqrt{\pi}}\int_\gamma\frac{e^{-z^2}(z-\alpha\omega_r-i\alpha\omega_i)\left((z-\alpha\omega_r)^2-(\alpha\omega_i)^2-\beta^2+2i\alpha\omega_i(z-\omega_r)\right)}{\left((z-\alpha\omega_r)^2-(\alpha\omega_i)^2-\beta^2\right)^2+4\left(\alpha\omega_i\right)^2(z-\alpha\omega_r)^2}dz\\
    &=&\frac{2}{\sqrt{\pi}}\int_\gamma\frac{e^{-z^2}\left((z-\alpha\omega_r)\left((z-\alpha\omega_r)^2-(\alpha\omega_i)^2-\beta^2\right)+2(\alpha\omega_i)^2(z-\alpha\omega_r)\right)}{\left((z-\alpha\omega_r)^2-(\alpha\omega_i)^2-\beta^2\right)^2+4\left(\alpha\omega_i\right)^2(z-\alpha\omega_r)^2}dz\\
    &+&i\frac{2}{\sqrt{\pi}}\int_\gamma\frac{e^{-z^2}\left(2\alpha\omega_i(z-\alpha\omega_r)^2-\alpha\omega_i\left((z-\alpha\omega_r)^2-(\alpha\omega_i)^2-\beta^2\right)\right)}{\left((z-\alpha\omega_r)^2-(\alpha\omega_i)^2-\beta^2\right)^2+4\left(\alpha\omega_i\right)^2(z-\alpha\omega_r)^2}dz\\
    &=&\frac{2}{\sqrt{\pi}}\int_\gamma\frac{e^{-z^2}(z-\alpha\omega_r)\left((z-\alpha\omega_r)^2+(\alpha\omega_i)^2-\beta^2\right)}{\left((z-\alpha\omega_r)^2-(\alpha\omega_i)^2-\beta^2\right)^2+4\left(\alpha\omega_i\right)^2(z-\alpha\omega_r)^2}dz\\
    &+&i\frac{2}{\sqrt{\pi}}\int_\gamma\frac{e^{-z^2}\alpha\omega_i\left((z-\alpha\omega_r)^2+(\alpha\omega_i)^2+\beta^2\right)}{\left((z-\alpha\omega_r)^2-(\alpha\omega_i)^2-\beta^2\right)^2+4\left(\alpha\omega_i\right)^2(z-\alpha\omega_r)^2}dz.
  \end{eqnarray*}
  Par ailleurs, en considérant $-\overline{\omega}=-\omega_r+i\omega_i$, nous avons
  \begin{eqnarray*}
    &&Z(\alpha(-\omega_r+i\omega_i)-\beta)+Z(\alpha(-\omega_r+i\omega_i)+\beta)\\
    &=&\frac{2}{\sqrt{\pi}}\int_\gamma\frac{e^{-z^2}(z+\alpha\omega_r)\left((z+\alpha\omega_r)^2+(\alpha\omega_i)^2-\beta^2\right)}{\left((z+\alpha\omega_r)^2-(\alpha\omega_i)^2-\beta^2\right)^2+4\left(\alpha\omega_i\right)^2(z+\alpha\omega_r)^2}dz\\
    &+&i\frac{2}{\sqrt{\pi}}\int_\gamma\frac{e^{-z^2}\alpha\omega_i\left((z+\alpha\omega_r)^2+(\alpha\omega_i)^2+\beta^2\right)}{\left((z+\alpha\omega_r)^2-(\alpha\omega_i)^2-\beta^2\right)^2+4\left(\alpha\omega_i\right)^2(z+\alpha\omega_r)^2}dz.
  \end{eqnarray*}
  La seule fonction impaire en $z$ est $(z+\alpha\omega_r)$, qui apparaît dans la partie réelle, ainsi
  \begin{eqnarray*}
    &&Z(\alpha(-\omega_r+i\omega_i)-\beta)+Z(\alpha(-\omega_r+i\omega_i)+\beta)\\
    &=&-\frac{2}{\sqrt{\pi}}\int_\gamma\frac{e^{-z^2}(z-\alpha\omega_r)\left((z-\alpha\omega_r)^2+(\alpha\omega_i)^2-\beta^2\right)}{\left((z-\alpha\omega_r)^2-(\alpha\omega_i)^2-\beta^2\right)^2+4\left(\alpha\omega_i\right)^2(z-\alpha\omega_r)^2}dz\\
    &+&i\frac{2}{\sqrt{\pi}}\int_\gamma\frac{e^{-z^2}\alpha\omega_i\left((z-\alpha\omega_r)^2+(\alpha\omega_i)^2+\beta^2\right)}{\left((z-\alpha\omega_r)^2-(\alpha\omega_i)^2-\beta^2\right)^2+4\left(\alpha\omega_i\right)^2(z-\alpha\omega_r)^2}dz.
  \end{eqnarray*}
  L'identification des parties réelles et imaginaires de $Z(\alpha\omega-\beta)+Z(\alpha\omega+\beta)$ et $Z(-\alpha\overline{\omega}-\beta)+Z(-\alpha\overline{\omega}+\beta)$ achève la preuve.
\end{proof}


\begin{lemma}
  La fonction $Z_{\alpha,\beta}^-(\omega):\omega\in\mathbb{C}\mapsto Z\left(\alpha\omega-\beta\right)-Z\left(\alpha\omega+\beta\right)\in\mathbb{C}$, avec $\alpha\in\mathbb{R}$, $\beta\in\mathbb{R}$ fixés, est telle que : $Z_{\alpha,\beta}^-\left(-\overline{\omega}\right)=\overline{Z_{\alpha,\beta}^-(\omega)}$.
\end{lemma}

\begin{proof}
  Nous avons par définition de la fonction de Fried-Conte
  \begin{eqnarray*}
    &&Z(\alpha\omega-\beta)-Z(\alpha\omega+\beta)=\frac{1}{\sqrt{\pi}}\int_\gamma\frac{e^{-z^2}}{z-\alpha\omega+\beta}-\frac{e^{-z^2}}{z-\alpha\omega-\beta}dz\\
    &=&\frac{1}{\sqrt{\pi}}\int_\gamma\frac{e^{-z^2}(z-\alpha\omega-\beta)-e^{-z^2}(z-\alpha\omega+\beta)}{(z-\alpha\omega)^2-\beta^2}dz\\
    &=&-\frac{2}{\sqrt{\pi}}\int_\gamma\frac{e^{-z^2}\beta}{(z-\alpha\omega)^2-\beta^2}dz.
  \end{eqnarray*}
  Maintenant, avec la notation $\omega=\omega_r+i\omega_i$, nous avons
  \begin{eqnarray*}
    &&Z(\alpha(\omega_r+i\omega_i)-\beta)-Z(\alpha(\omega_r+i\omega_i)+\beta)=-\frac{2}{\sqrt{\pi}}\int_\gamma\frac{e^{-z^2}\beta}{(z-\alpha\omega_r-i\alpha\omega_i)^2-\beta^2}dz\\
    &=&-\frac{2}{\sqrt{\pi}}\int_\gamma\frac{e^{-z^2}\beta}{(z-\alpha\omega_r)^2-(\alpha\omega_i)^2-\beta^2-2i\alpha\omega_i(z-\alpha\omega_r)}dz\\
    &=&-\frac{2}{\sqrt{\pi}}\int_\gamma\frac{e^{-z^2}\beta\left((z-\alpha\omega_r)^2-(\alpha\omega_i)^2-\beta^2+2i\alpha\omega_i(z-\alpha\omega_r)\right)}{\left((z-\alpha\omega_r)^2-(\alpha\omega_i)^2-\beta^2\right)^2+4\left(\alpha\omega_i\right)^2(z-\alpha\omega_r)^2}dz
  \end{eqnarray*}
  Par ailleurs, avec $-\overline{\omega}=-\omega_r+i\omega_i$, nous avons
  \begin{eqnarray*}
    &&Z(\alpha(-\omega_r+i\omega_i)-\beta)-Z(\alpha(-\omega_r+i\omega_i)+\beta)\\
    &=&-\frac{2}{\sqrt{\pi}}\int_\gamma\frac{e^{-z^2}\beta\left((z+\alpha\omega_r)^2-(\alpha\omega_i)^2-\beta^2+2i\alpha\omega_i(z+\alpha\omega_r)\right)}{\left((z+\alpha\omega_r)^2-(\alpha\omega_i)^2-\beta^2\right)^2+4\left(\alpha\omega_i\right)^2(z+\alpha\omega_r)^2}dz
  \end{eqnarray*}
  La seule fonction impaire en $z$ est $(z+\alpha\omega_r)$, apparaissant dans la partie imaginaire, d'où
  \begin{eqnarray*}
    &&Z(\alpha(-\omega_r+i\omega_i)-\beta)-Z(\alpha(-\omega_r+i\omega_i)+\beta)\\
    &=&-\frac{2}{\sqrt{\pi}}\int_\gamma\frac{e^{-z^2}\beta\left((z-\alpha\omega_r)^2-(\alpha\omega_i)^2-\beta^2-2i\alpha\omega_i(z-\alpha\omega_r)\right)}{\left((z-\alpha\omega_r)^2-(\alpha\omega_i)^2-\beta^2\right)^2+4\left(\alpha\omega_i\right)^2(z-\alpha\omega_r)^2}dz
  \end{eqnarray*}
  L'identification des parties réelles et imaginaires de $Z(\alpha\omega-\beta)-Z(\alpha\omega+\beta)$ et $Z(-\alpha\overline{\omega}-\beta)-Z(-\alpha\overline{\omega}+\beta)$ achève la preuve.
\end{proof}

Nous pouvons enfin démontrer les lemmes \ref{lemme:hypcashyb} et \ref{lemme:hypcascin} concernant la vérification de l'hypothèse \ref{hyp:sym}, qui conduit à l'expression (\ref{eq:Etk}) du mode fondamental du champ électrique linéarisé puis à l'approximation (\ref{eq:enelec}) de l'énergie électrique linéarisée. Ces lemmes sont rappelés ci-dessous.

D'une part, nous rappelons le résultat \ref{lemme:hypcascin} dans le cas cinétique.
\begin{lemma}
  Pour $\frac{\partial D(k,\omega)}{\partial\omega}$ donnée par~\eqref{eq:3bumpderD} et $N(k,\omega)$ par~\eqref{eq:N_3bump}, l'hypothèse~\ref{hyp:sym} est satisfaite.
\end{lemma}

\begin{proof}
  En utilisant (\ref{eq:3bumpderD}) et les lemmes \ref{lemma:Z0}, \ref{lemma:Z+}, \ref{lemma:Z-} avec $\delta=\frac{1}{\sqrt{2T_c}k}$, $\eta=\frac{1}{\sqrt{2}k}$ et $\beta=\frac{v_0}{\sqrt{2}}$, nous avons
  \begin{eqnarray*}
    \frac{\partial D}{\partial \omega}(k,\omega)&=&\frac{1}{\sqrt{2}k^3}\left[\frac{1-\alpha}{T_c\sqrt{T_c}}\left(\left(1-\frac{\omega^2}{T_ck^2}\right)Z_\delta^0\left(\omega\right)-2\frac{\omega}{\sqrt{2T_c}k}\right)\right.\nonumber\\
    &&~~~~~~~~~~~~~+\frac{\alpha}{2}\left(\left(1-\left(\frac{\omega}{k}-v_0\right)^2\right)Z\left(\frac{1}{\sqrt{2}}\left(\frac{\omega}{k}-v_0\right)\right)\right.\nonumber\\
    &&~~~~~~~~~~~~~~~~~~+\left.\left.\left(1-\left(\frac{\omega}{k}+v_0\right)^2\right)Z\left(\frac{1}{\sqrt{2}}\left(\frac{\omega}{k}+v_0\right)\right)\right.\right.\nonumber\\
    &&~~~~~~~~~~~~~\left.\left.-\frac{2}{\sqrt{2}}\left(\frac{\omega}{k}-v_0\right)-\frac{2}{\sqrt{2}}\left(\frac{\omega}{k}+v_0\right)\right)\right]\nonumber\\
    &=&\frac{1}{\sqrt{2}k^3}\left[\frac{1-\alpha}{T_c\sqrt{T_c}}\left(\left(1-\frac{\omega^2}{T_ck^2}\right)Z_\delta^0\left(\omega\right)-2\frac{\omega}{\sqrt{2T_c}k}\right)-2\sqrt{2}\frac{\omega}{k}\right.\nonumber\\
    &&~~~~~~~~~~~~~\left.+\frac{\alpha}{2}\left((1-v_0^2)Z_{\eta,\beta}^+\left(\omega\right)-\frac{\omega^2}{k^2}Z_{\eta,\beta}^+\left(\omega\right)+2v_0\frac{\omega}{k}Z_{\eta,\beta}^-\left(\omega\right)\right)\right]
  \end{eqnarray*}
  et
  \begin{eqnarray*}
    \frac{\partial D}{\partial \omega}(k,-\overline{\omega})&=&\frac{1}{\sqrt{2}k^3}\left[\frac{1-\alpha}{T_c\sqrt{T_c}}\left(\left(1-\frac{(-\overline{\omega})^2}{T_ck^2}\right)Z_\delta^0\left(-\overline{\omega}\right)+2\frac{\overline{\omega}}{\sqrt{2T_c}k}\right)+2\sqrt{2}\frac{\overline{\omega}}{k}\right.\nonumber\\
    &&~~~~~~~~~~~~~\left.+\frac{\alpha}{2}\left((1-v_0^2)Z_{\eta,\beta}^+\left(-\overline{\omega}\right)-\frac{(-\overline{\omega})^2}{k^2}Z_{\eta,\beta}^+\left(-\overline{\omega}\right)-2v_0\frac{\overline{\omega}}{k}Z_{\eta,\beta}^-\left(-\overline{\omega}\right)\right)\right]\\
    &=&\frac{1}{\sqrt{2}k^3}\left[\frac{1-\alpha}{T_c\sqrt{T_c}}\left(-\left(1-\frac{\overline{\omega^2}}{T_ck^2}\right)\overline{Z_\delta^0\left(\omega\right)}+2\frac{\overline{\omega}}{\sqrt{2T_c}k}\right)+2\sqrt{2}\frac{\overline{\omega}}{k}\right.\nonumber\\
    &&~~~~~~~~~~~~~\left.+\frac{\alpha}{2}\left(-(1-v_0^2)\overline{Z_{\eta,\beta}^+\left(\omega\right)}+\frac{\overline{\omega^2}}{k^2}\overline{Z_{\eta,\beta}^+\left(\omega\right)}-2v_0\frac{\overline{\omega}}{k}\overline{Z_{\eta,\beta}^-\left(\omega\right)}\right)\right]\nonumber\\
    &=&-\overline{\frac{\partial D}{\partial \omega}(k,\omega)}.
  \end{eqnarray*}

  Maintenant, en utilisant (\ref{eq:N_3bump}) et le lemme \ref{lemma:Z+} avec $\eta=\frac{1}{\sqrt{2}k}$ et $\beta=\frac{v_0}{\sqrt{2}}$, nous avons
  \begin{eqnarray*}
    N(k,\omega)&=&-\frac{\hat{g}(k)}{k^2}\left(\frac{\alpha}{2\sqrt{2}}Z_{\eta,\beta}^+\left(\omega\right)\right)
  \end{eqnarray*}
  et
  \begin{eqnarray*}
    N(k,-\overline{\omega})&=&-\frac{\hat{g}(k)}{k^2}\left(\frac{\alpha}{2\sqrt{2}}Z_{\eta,\beta}^+\left(-\overline{\omega}\right)\right)\\
    &=&-\frac{\hat{g}(k)}{k^2}\left(-\frac{\alpha}{2\sqrt{2}}\overline{Z_{\eta,\beta}^+\left(\omega\right)}\right)=-\overline{N(k,\omega)}.
  \end{eqnarray*}

  Ainsi, nous obtenons 
  \begin{eqnarray*}
    \frac{N(k,-\overline{\omega})}{\frac{\partial D}{\partial \omega}(k,-\overline{\omega})}=\overline{\left(\frac{N(k,\omega)}{\frac{\partial D}{\partial \omega}(k,\omega)}\right)}.
  \end{eqnarray*}
  Autrement dit, $\frac{N(k,\omega)}{\frac{\partial D}{\partial \omega}(k,\omega)}=re^{i\phi}$ si et seulement si $\frac{N(k,-\overline{\omega})}{\frac{\partial D}{\partial \omega}(k,-\overline{\omega})}=re^{-i\phi}$.
\end{proof}

D'autre part, nous rappelons le résultat \ref{lemme:hypcashyb} dans le cas hybride.
\begin{lemma}
  Sous l'hypothèse $\hat{u}(t=0,k)=0$, pour $\frac{\partial D(k,\omega)}{\partial\omega}$ donnée par~\eqref{eq:hchybderD} et $N(k,\omega)$ par~\eqref{eq:N_hchyb}, l'hypothèse~\ref{hyp:sym} est satisfaite.
\end{lemma}

\begin{proof}
  Regardons d'abord $\frac{\partial D(k,\omega)}{\partial\omega}$. Les termes en facteur de $\alpha$ (venant de la partie chaude cinétique) se comportent comme dans la preuve du lemme \ref{lemme:hypcashyb} (voir la preuve ci-dessus). Les termes en facteur de $1-\alpha$ sont tels que
  $$
    \frac{1}{(-\overline{\omega})^3}=-\frac{1}{\overline{\omega^3}}=-\overline{\frac{1}{\omega^3}}.
  $$

  Nous en déduisons $\frac{\partial D}{\partial \omega}(k,-\overline{\omega})=-\overline{\frac{\partial D}{\partial \omega}(k,\omega)}$.

  Regardons ensuite $N(k,\omega)$. Sous l'hypothèse $\hat{u}(t=0,k)=0$ et avec les notations $\eta=\frac{1}{\sqrt{2}k}$ et $\beta=\frac{v_0}{\sqrt{2}}$, nous avons
  \begin{eqnarray*}
    N(k,-\overline{\omega})&=&-\frac{1}{-i\overline{\omega}}\hat{E}(t=0,k)-\frac{\hat{g}(k)}{ k^2}\left[\alpha\frac{k}{-\overline{\omega}}+\frac{\alpha}{2\sqrt{2}}Z_{\eta,\beta}^+\left(-\overline{\omega}\right)\right].
  \end{eqnarray*}
  %Maintenant, pour $E(t=0,x)$ donné par l'équation de Poisson, nous avons 
  %\begin{eqnarray*}
  %\partial_xE(t=0,x)&=&\rho_c(t=0,x)+\int f^h(t=0,x,v)dv-1\\
  %&=&\left(1-\alpha\right)\left(1+\varepsilon\cos\left(\frac{2\pi}{L}x\right)\right)+\alpha\left(1+\varepsilon\cos\left(\frac{2\pi}{L}x\right)\right)-1\\
  %&=&\varepsilon\cos\left(\frac{2\pi}{L}x\right)
  %\end{eqnarray*}
  %Donc le champ électrique initial s'écrit
  %$$E(t=0,x)=\frac{\varepsilon L}{2\pi}\sin\left(\frac{2\pi}{L}x\right),~~~\int_0^LE(t=0,x)dx=0,$$
  %et 
  %$$E^1(t=0,x)=\frac{L}{2\pi}\sin\left(\frac{2\pi}{L}x\right).$$
  %Sa transformée de Fourier, pour $k=\frac{2\pi}{L}n,~n\in\mathbb{Z}$, s'écrit
  %\begin{eqnarray*}
  %\hat{E}(t=0,k)&=&\frac{1}{2\pi}\int_0^L\sin\left(\frac{2\pi}{L}x\right)e^{-\frac{2i\pi n}{L}x}dx\\
  %&=&\frac{1}{4i\pi}\int_0^Le^{\frac{2i\pi}{L}(1-n)x}-e^{-\frac{2i\pi}{L}(1+n)x}dx.
  %\end{eqnarray*}
  %Si $k\neq \frac{2\pi}{L}$ et $k\neq -\frac{2\pi}{L}$, $\hat{E}(t=0,k)=0$. If $n=1$, ou de manière équivalente $k=\frac{2\pi}{L}$, nous avons
  %\begin{eqnarray*}
  %\hat{E}(t=0,k)&=&\frac{1}{4i\pi}\left[x+\frac{L}{4i\pi}e^{-\frac{2i\pi}{L}2x}\right]_0^L=-i\frac{L}{4\pi}=-\frac{i}{2k}.
  %\end{eqnarray*}
  %Si $n=-1$, ou de manière équivalente $k=-\frac{2\pi}{L}$, nous avons
  %\begin{eqnarray*}
  %\hat{E}(t=0,k)&=&\frac{1}{4i\pi}\left[\frac{L}{4i\pi}e^{\frac{2i\pi}{L}2x}-x\right]_0^L=i\frac{ L}{4\pi}=-\frac{i}{2k}.
  %\end{eqnarray*}
  %Ainsi
  %\begin{equation}
  %\hat{E}\left(t=0,k\right)=-\frac{i}{2k},~k\in\left\{-\frac{2\pi}{L},\frac{2\pi}{L}\right\},~~~\hat{E}(k)=0,~k\notin\left\{-\frac{2\pi}{L},\frac{2\pi}{L}\right\},
  %\label{eq:Ek}\end{equation}
  Nous rappelons que $\hat{E}(t=0,k)$ est un imaginaire pur (éventuellement nul) donné par (\ref{eq:Ekbis}). Ceci implique
  \begin{eqnarray*}
    N(k,-\overline{\omega})&=&\frac{1}{i\overline{\omega}}\hat{E}(t=0,k)+\frac{\hat{g}(k)}{ k^2}\left[\alpha\frac{k}{\overline{\omega}}+\frac{\alpha}{2\sqrt{2}}\overline{Z_{\eta,\beta}^+\left(\omega\right)}\right]\\
    &=&\overline{\frac{1}{i\omega}\hat{E}(t=0,k)}+\frac{\hat{g}(k)}{ k^2}\left[\alpha\overline{\frac{k}{\omega}}+\frac{\alpha}{2\sqrt{2}}\overline{Z_{\eta,\beta}^+\left(\omega\right)}\right]=-\overline{N(k,\omega)}.
  \end{eqnarray*}
  La preuve est terminée.
\end{proof}


\end{document}
