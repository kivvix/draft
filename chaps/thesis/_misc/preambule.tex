\usepackage{cmap}
%\usepackage[T1]{fontenc}
\usepackage[utf8]{inputenc}

\geometry{vmargin=4.0cm}

\usepackage{graphicx}
%\usepackage{subfig}
\usepackage{subcaption}
\usepackage{color}
\usepackage{tikz}

% figures pas plus loin que la fin de la section
\usepackage[section]{placeins}

% insert additional whitespace between the main text and the foonote rule rather than over-stretching the inter-paragraph glue
\raggedbottom
\usepackage[bottom]{footmisc}

% make a bibliography for each chapter
%\usepackage{chapterbib}

%\usepackage[pdfborder={0 0 0 [3 3]},pdftex,unicode=true,pdfa=true]{hyperref}
%\usepackage[unicode=true,pdfa=true]{hyperref}
\usepackage{bookmark}
\usepackage{appendix}

\usepackage{placeins} % to force floating element placement with \FloatBarrier

\usepackage{physics} % please, read: http://mirrors.ibiblio.org/CTAN/macros/latex/contrib/physics/physics.pdf
\usepackage{amsmath}
\usepackage{amssymb}
\usepackage{amsthm}
%\usepackage{mathabx}
\usepackage{mathrsfs}
\usepackage{wasysym}
\usepackage{textcomp}
\DeclareMathAlphabet{\mathpzc}{OT1}{pzc}{m}{it}
\usepackage{eufrak}

% for widebar
\makeatletter
\let\save@mathaccent\mathaccent
\newcommand*\if@single[3]{%
  \setbox0\hbox{${\mathaccent"0362{#1}}^H$}%
  \setbox2\hbox{${\mathaccent"0362{\kern0pt#1}}^H$}%
  \ifdim\ht0=\ht2 #3\else #2\fi
  }
%The bar will be moved to the right by a half of \macc@kerna, which is computed by amsmath:
\newcommand*\rel@kern[1]{\kern#1\dimexpr\macc@kerna}
%If there's a superscript following the bar, then no negative kern may follow the bar;
%an additional {} makes sure that the superscript is high enough in this case:
\newcommand*\widebar[1]{\@ifnextchar^{{\wide@bar{#1}{0}}}{\wide@bar{#1}{1}}}
%Use a separate algorithm for single symbols:
\newcommand*\wide@bar[2]{\if@single{#1}{\wide@bar@{#1}{#2}{1}}{\wide@bar@{#1}{#2}{2}}}
\newcommand*\wide@bar@[3]{%
  \begingroup
  \def\mathaccent##1##2{%
%Enable nesting of accents:
    \let\mathaccent\save@mathaccent
%If there's more than a single symbol, use the first character instead (see below):
    \if#32 \let\macc@nucleus\first@char \fi
%Determine the italic correction:
    \setbox\z@\hbox{$\macc@style{\macc@nucleus}_{}$}%
    \setbox\tw@\hbox{$\macc@style{\macc@nucleus}{}_{}$}%
    \dimen@\wd\tw@
    \advance\dimen@-\wd\z@
%Now \dimen@ is the italic correction of the symbol.
    \divide\dimen@ 3
    \@tempdima\wd\tw@
    \advance\@tempdima-\scriptspace
%Now \@tempdima is the width of the symbol.
    \divide\@tempdima 10
    \advance\dimen@-\@tempdima
%Now \dimen@ = (italic correction / 3) - (Breite / 10)
    \ifdim\dimen@>\z@ \dimen@0pt\fi
%The bar will be shortened in the case \dimen@<0 !
    \rel@kern{0.6}\kern-\dimen@
    \if#31
      \overline{\rel@kern{-0.6}\kern\dimen@\macc@nucleus\rel@kern{0.4}\kern\dimen@}%
      \advance\dimen@0.4\dimexpr\macc@kerna
%Place the combined final kern (-\dimen@) if it is >0 or if a superscript follows:
      \let\final@kern#2%
      \ifdim\dimen@<\z@ \let\final@kern1\fi
      \if\final@kern1 \kern-\dimen@\fi
    \else
      \overline{\rel@kern{-0.6}\kern\dimen@#1}%
    \fi
  }%
  \macc@depth\@ne
  \let\math@bgroup\@empty \let\math@egroup\macc@set@skewchar
  \mathsurround\z@ \frozen@everymath{\mathgroup\macc@group\relax}%
  \macc@set@skewchar\relax
  \let\mathaccentV\macc@nested@a
%The following initialises \macc@kerna and calls \mathaccent:
  \if#31
    \macc@nested@a\relax111{#1}%
  \else
%If the argument consists of more than one symbol, and if the first token is
%a letter, use that letter for the computations:
    \def\gobble@till@marker##1\endmarker{}%
    \futurelet\first@char\gobble@till@marker#1\endmarker
    \ifcat\noexpand\first@char A\else
      \def\first@char{}%
    \fi
    \macc@nested@a\relax111{\first@char}%
  \fi
  \endgroup
}
\makeatother

\theoremstyle{plain}\newtheorem{theorem}{Théorème}[section]
\theoremstyle{plain}\newtheorem{lemma}{Lemme}[section]
\newtheorem{pro}{Proposition}[chapter]
%\theoremstyle{remark}[theorem]
\newtheorem{remark}{Remarque}[chapter]
%\theoremstyle{plain}\newtheorem{ex}[theorem]{Example}
%\theoremstyle{plain}\newtheorem{exper}[theorem]{Experiment}
\newtheorem{property}{Property}[chapter]
\newtheorem{hyp}{Hypothèse}[chapter]

\usepackage{scalerel,stackengine}
\stackMath
\newcommand\Widehat[1]{%
\savestack{\tmpbox}{\stretchto{%
  \scaleto{%
    \scalerel*[\widthof{\ensuremath{#1}}]{\kern-.6pt\bigwedge\kern-.6pt}%
    {\rule[-\textheight/2]{1ex}{1.2\textheight}}%WIDTH-LIMITED BIG WEDGE
  }{\textheight}% 
}{0.6ex}}%
\stackon[1pt]{#1}{\tmpbox}%
}
\parskip 1ex

% for \ceil \floot
\usepackage{mathtools}
\DeclarePairedDelimiter{\ceil}{\lceil}{\rceil}
\DeclarePairedDelimiter{\floor}{\lfloor}{\rfloor}


\newcommand\ie{\emph{i.e.}~}

\usepackage{algpseudocode}
\floatname{algorithm}{Algorithme}
\renewcommand{\algorithmicprocedure} {\textbf{Proc\'edure} }
\renewcommand{\algorithmicwhile}     {\textbf{tant que}    }
\renewcommand{\algorithmicdo}        {\textbf{faire :}     }
\renewcommand{\algorithmicend}       {\textbf{fin}         }
\renewcommand{\algorithmicif}        {\textbf{si}          }
\renewcommand{\algorithmicelse}      {\textbf{sinon}       }
\renewcommand{\algorithmicthen}      {\textbf{alors}       }
\renewcommand{\algorithmicfor}       {\textbf{pour}        }
\renewcommand{\algorithmicforall}    {\textbf{pour tout}   }
\renewcommand{\algorithmicrepeat}    {\textbf{répéter}     }
\renewcommand{\algorithmicuntil}     {\textbf{jusqu'à}     }
\renewcommand{\algorithmicfunction}  {\textbf{Fonction}    }
\renewcommand{\algorithmicreturn}    {\textbf{retourner}   }

\usepackage{refcount}
\newcommand{\setlineref}[1]{
  \setcounterref{ALG@line}{#1}\addtocounter{ALG@line}{-1}
}

% source: https://isocpp.org/wiki/faq/misc-environmental-issues#latex-macros
\def\CC{{C\nolinebreak[4]\hspace{-.05em}\raisebox{.4ex}{\tiny\bf ++}}}

\newcommand{\Python}{Python}
\newcommand{\sympy}{SymPy}

%%%%%%%%%%%% COMMENTAIRES ! À essayer de virer au fur et à mesure
\newcommand{\commentaire}[2][]{%
  \ifthenelse{\equal{#1}{}}%
    {\textbf{#2}}%
  {%
  \ifthenelse{\equal{#1}{Anais}}%
    {\textcolor{blue}{#2}}%
  {\ifthenelse{\equal{#1}{Nicolas}}%
    {\textcolor{orange}{#2}}%
  {\ifthenelse{\equal{#1}{Josselin}}%
  {\textcolor{purple}{#2}}%
  {\textcolor{teal}{#2}\footnote{Commentaire rédigé par #1}%
  }}}}%
}
\newcommand{\Anais}[1]{\commentaire[Anais]{#1}}
\newcommand{\Nicolas}[1]{\commentaire[Nicolas]{#1}}
\newcommand{\Josselin}[1]{\commentaire[Josselin]{#1}}



