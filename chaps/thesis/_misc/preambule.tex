\usepackage{cmap}
%\usepackage[T1]{fontenc}
\usepackage[utf8]{inputenc}

\geometry{vmargin=4.0cm}

\usepackage{graphicx}
%\usepackage{subfig}
\usepackage{subcaption}
\usepackage{color}
\usepackage{tikz}

% figures pas plus loin que la fin de la section
\usepackage[section]{placeins}

% insert additional whitespace between the main text and the foonote rule rather than over-stretching the inter-paragraph glue
\raggedbottom
\usepackage[bottom]{footmisc}

% make a bibliography for each chapter
%\usepackage{chapterbib}

%\usepackage[pdfborder={0 0 0 [3 3]},pdftex,unicode=true,pdfa=true]{hyperref}
%\usepackage[unicode=true,pdfa=true]{hyperref}
\usepackage{bookmark}
\usepackage{appendix}

\usepackage{placeins} % to force floating element placement with \FloatBarrier

\usepackage{physics} % please, read: http://mirrors.ibiblio.org/CTAN/macros/latex/contrib/physics/physics.pdf
\usepackage{amsmath}
\usepackage{amssymb}
\usepackage{amsthm}
%\usepackage{mathabx}
\usepackage{mathrsfs}
\usepackage{wasysym}
\usepackage{textcomp}
\DeclareMathAlphabet{\mathpzc}{OT1}{pzc}{m}{it}
\usepackage{eufrak}

\theoremstyle{plain}\newtheorem{theorem}{Théorème}[section]
\theoremstyle{plain}\newtheorem{lemma}{Lemme}[section]
\newtheorem{pro}{Proposition}[chapter]
%\theoremstyle{remark}[theorem]
\newtheorem{remark}{Remarque}[chapter]
%\theoremstyle{plain}\newtheorem{ex}[theorem]{Example}
%\theoremstyle{plain}\newtheorem{exper}[theorem]{Experiment}
\newtheorem{property}{Property}[chapter]
\newtheorem{hyp}{Hypothèse}[chapter]

\usepackage{scalerel,stackengine}
\stackMath
\newcommand\Widehat[1]{%
\savestack{\tmpbox}{\stretchto{%
  \scaleto{%
    \scalerel*[\widthof{\ensuremath{#1}}]{\kern-.6pt\bigwedge\kern-.6pt}%
    {\rule[-\textheight/2]{1ex}{1.2\textheight}}%WIDTH-LIMITED BIG WEDGE
  }{\textheight}% 
}{0.6ex}}%
\stackon[1pt]{#1}{\tmpbox}%
}
\parskip 1ex

\newcommand\ie{\emph{i.e.}~}

\usepackage{algpseudocode}
\floatname{algorithm}{Algorithme}
\renewcommand{\algorithmicprocedure} {\textbf{Proc\'edure} }
\renewcommand{\algorithmicwhile}     {\textbf{tant que}    }
\renewcommand{\algorithmicdo}        {\textbf{faire :}     }
\renewcommand{\algorithmicend}       {\textbf{fin}         }
\renewcommand{\algorithmicif}        {\textbf{si}          }
\renewcommand{\algorithmicelse}      {\textbf{sinon}       }
\renewcommand{\algorithmicthen}      {\textbf{alors}       }
\renewcommand{\algorithmicfor}       {\textbf{pour}        }
\renewcommand{\algorithmicforall}    {\textbf{pour tout}   }
\renewcommand{\algorithmicrepeat}    {\textbf{répéter}     }
\renewcommand{\algorithmicuntil}     {\textbf{jusqu'à}     }
\renewcommand{\algorithmicfunction}  {\textbf{Fonction}    }
\renewcommand{\algorithmicreturn}    {\textbf{retourner}   }

% source: https://isocpp.org/wiki/faq/misc-environmental-issues#latex-macros
\def\CC{{C\nolinebreak[4]\hspace{-.05em}\raisebox{.4ex}{\tiny\bf ++}}}

\newcommand{\Python}{Python}
\newcommand{\sympy}{SymPy}

%%%%%%%%%%%% COMMENTAIRES ! À essayer de virer au fur et à mesure
\newcommand{\commentaire}[2][]{%
  \ifthenelse{\equal{#1}{}}%
    {\textbf{#2}}%
  {%
  \ifthenelse{\equal{#1}{Anais}}%
    {\textcolor{blue}{#2}}%
  {\ifthenelse{\equal{#1}{Nicolas}}%
    {\textcolor{orange}{#2}}%
  {\ifthenelse{\equal{#1}{Josselin}}%
  {\textcolor{purple}{#2}}%
  {\textcolor{teal}{#2}\footnote{Commentaire rédigé par #1}%
  }}}}%
}
\newcommand{\Anais}[1]{\commentaire[Anais]{#1}}
\newcommand{\Nicolas}[1]{\commentaire[Nicolas]{#1}}
\newcommand{\Josselin}[1]{\commentaire[Josselin]{#1}}



