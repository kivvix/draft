% !TEX root = ../chap3.tex

\section{Présentation du modèle}

\begin{align}
  \label{eq:VHM:jx}
  &\pdv{j_{c,x}}{t} = \Omega_{pe}^2 { E}_x - { j}_{c,y}B_0,\\
  \label{eq:VHM:jy}
  &\pdv{j_{c,y}}{t} = \Omega_{pe}^2 { E}_y + { j}_{c,x}B_0,\\
  \label{eq:VHM:Bx}
  &\pdv{B_{x  }}{t} = \partial_z E_y,\\
  \label{eq:VHM:By}
  &\pdv{B_{y  }}{t} = -\partial_z E_x,\\
  \label{eq:VHM:Ex}
  &\pdv{E_{x  }}{t} =-\partial_z B_y - {j}_{c,x} + \int v_x f_h \dd{\vb{v}},\\
  \label{eq:VHM:Ey}
  &\pdv{E_{y  }}{t} =\partial_z B_x - {j}_{c,y} + \int v_y f_h \dd{\vb{v}}, \\
  \label{eq:VHM:fh}
  &\pdv{f_{h  }}{t} +  v_z\partial_z f_h - \left(E_x + v_y B_0 - v_zB_y\right)\partial_{v_x} f_h\\
  &\hspace{2cm}- \left(E_y - v_x B_0 + v_z B_x\right)\partial_{v_y} f_h- \left(v_x B_y- v_y B_x\right)\partial_{v_z} f_h = 0.\nonumber
\end{align}
Cela nous permet de définir l'hamiltonien :
\begin{equation}
  \mathcal{H} = {}
      \underbrace{\frac{1}{2}\int_{\mathbb{R}}(E_x^2+E_y^2) \dd{z}}_{\mathcal{H}_E}
    + \underbrace{\frac{1}{2}\int_{\mathbb{R}}(B_x^2+B_y^2) \dd{z}}_{\mathcal{H}_B}
    + \underbrace{\frac{1}{2}\int_{\mathbb{R}}\frac{1}{\Omega_{pe}^2}(j_{c,x}^2+j_{c,y}^2) \dd{z}}_{\mathcal{H}_{j_c}}
    + \underbrace{\frac{1}{2}\int_{\mathbb{R}}\int_{\mathbb{R}^2}|\vb{v}|^2 f_h \dd{\vb{v}}\dd{z}}_{\mathcal{H}_{f_h}}
  \label{eq:ham1dz3dv}
\end{equation}

On définit le crouchet suivant, définit pour deux fonctionnelles $\mathcal{F}$ et $\mathcal{G}$ :
\begin{equation}
  \begin{aligned}
    \{ \mathcal{F},\mathcal{G} \}[ j_{c,x}, j_{c,y}, B_x, B_y, E_x, E_y, f_h] &=
                       \int_{\mathbb{R}}\int_{\mathbb{R}^3} f_h \left( \partial_z\fdv{\mathcal{F}}{f_h}\partial_{v_z}\fdv{\mathcal{G}}{f_h} - \partial_{v_z}\fdv{\mathcal{F}}{f_h}\partial_z\fdv{\mathcal{G}}{f_h} \right) \dd{\vb{v}}\dd{z} \\
      &\hspace{-4cm} + \int_{\mathbb{R}}\int_{\mathbb{R}^3} f_h \left(
            \partial_{v_x}\fdv{\mathcal{F}}{f_h}\fdv{\mathcal{G}}{E_x} + \partial_{v_y}\fdv{\mathcal{F}}{f_h}\fdv{\mathcal{G}}{E_y}
          - \partial_{v_x}\fdv{\mathcal{G}}{f_h}\fdv{\mathcal{F}}{E_x} - \partial_{v_y}\fdv{\mathcal{G}}{f_h}\fdv{\mathcal{F}}{E_y}
          \right) \dd{\vb{v}}\dd{z} \\
      &\hspace{-4cm} + \int_{\mathbb{R}}\int_{\mathbb{R}^3} f_h(\vb{B}+\vb{B}_0)\cdot\left( \nabla_{\vb{v}}\fdv{\mathcal{F}}{f_h}\times\nabla_{\vb{v}}\fdv{\mathcal{G}}{f_h} \right) \dd{\vb{v}}\dd{z} \\
      &\hspace{-4cm} + \int_{\mathbb{R}} \left(
          - \partial_z\fdv{\mathcal{F}}{E_y}\fdv{\mathcal{G}}{B_x} + \partial_z\fdv{\mathcal{F}}{E_x}\fdv{\mathcal{G}}{B_y}
          + \partial_z\fdv{\mathcal{G}}{E_y}\fdv{\mathcal{F}}{B_x} - \partial_z\fdv{\mathcal{G}}{E_x}\fdv{\mathcal{F}}{B_y}
          \right) \dd{z} \\
      &\hspace{-4cm} + \int_{\mathbb{R}} \Omega_{pe}^2 \left(
            \fdv{\mathcal{F}}{j_{c,x}}\fdv{\mathcal{G}}{E_x} + \fdv{\mathcal{F}}{j_{c,y}}\fdv{\mathcal{G}}{E_y}
          - \fdv{\mathcal{G}}{j_{c,x}}\fdv{\mathcal{F}}{E_x} - \fdv{\mathcal{G}}{j_{c,y}}\fdv{\mathcal{F}}{E_y}
          \right) \dd{z} \\
      &\hspace{-4cm} + \int_{\mathbb{R}} \Omega_{pe}^2B_0 \left(
            \fdv{\mathcal{F}}{j_{c,x}}\fdv{\mathcal{G}}{j_{c,y}}
          - \fdv{\mathcal{F}}{j_{c,y}}\fdv{\mathcal{G}}{j_{c,x}}
          \right) \dd{z}.
  \end{aligned}
  \label{bracket}
\end{equation}
Cela nous permet de réécrire le système~\eqref{eq:VHM:jx}-\eqref{eq:VHM:fh} comme 
$$
  \partial_t U = \{ U,\mathcal{H} \}
$$
où $U(t,z,\vb{v}) = ( j_{c,\perp}(t,z) , B_\perp(t,z) , E_\perp(t,z) , f)h(t,z,\vb{v}) )^\top$ où $\mathcal{H}$ est donné par~\eqref{eq:ham1dz3dv} et $j_{c,\perp}=(j_{c,x},j_{c,y})^\top$, $B_{\perp}=(B_{x},B_{y})^\top$ et $E_{\perp}=(E_{x},E_{y})^\top$. Par la suite, nous utiliserons également les notations $v_\perp = (v_x,v_y)^\top\in\mathbb{R}^2$ ainsi que $\vb{v} = (v_\perp,v_z)^\top\in\mathbb{R}^3$ et nous définissions la matrice symplectique 
$$
  J = \begin{pmatrix}
     0 & 1 \\
    -1 & 0
  \end{pmatrix}.
$$
