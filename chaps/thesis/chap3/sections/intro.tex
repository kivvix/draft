% !TEX root = ../chap3.tex

\section{Introduction}

L'objectif de ce chapitre est d'étendre les stratégies développées pour la résolution d'un modèle hybride linéarisé, présentées dans le chapitre~\ref{chap2}, au cas $1dz-3dv$ permettant de prendre en compte, entre autre, les effets du champ magnétique sur la dynamtique des particules. Le modèle que nous étudions ici s'applique au cas où la distribution de particules est composée de deux populations, une première ayant une vitesse thermique faible, et considéréees comme \emph{froide}, dont on approximera la dynamique comme celle d'un fluide ; une seconde population ayant une vitesse thermique plus élevé, considérées comme \emph{chaude}, mais contrairement au chapitre~\ref{chap2}, celle-ci n'a pas besoin d'être distribuée selon une bi-maxwellienne. Ce type de modèle peut être utilisé pour modéliser des particules d'un plasma dans un tokamak, ou des particules du vent solaire interagissant avec la magnétosphère terrestre. Dans ces contextes les particules se déplacent de manière hélicoïdale selon les lignes du champ extéreur $\vb{B}_0 = (0,0,B_0)^\top$, et seul le déplacement dans cette direction sera étudié ici, pour le déplacement hélicoïdale complet \Josselin{il faudrait ajouter ici une référence que je n'ai pas car je ne connais pas trop ces modèles, de mémoire c'est ce que faisait Xiaofei}. Cela justifie la réduction du modèle à une seule dimension spatiale, ce qui permet de réduire le coût de calcul et ainsi tester différentes stratégies de simulation. Les deux méthodes de simulation que nous nous proposons de comparer sont les générations des deux stragégies développées dans le chapitre~\ref{chap2}, une méthode de ~\emph{splitting} et une méthode de Lawson.
