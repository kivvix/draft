% !TEX root = ../chap3.tex

\section{Introduction}

L'objectif de ce chapitre est d'étendre les stratégies développées pour la résolution d'un modèle hybride linéarisé, présentées dans le chapitre~\ref{chap2}, au cas $1dz-3dv$ permettant de prendre en compte, entre autre, les effets du champ magnétique sur la dynamtique des particules. Nous étudierons dans ce chapitre un modèle hybride, ce qui suppose que la distribution de particules est composée de deux populations, une première ayant une vitesse thermique faible, et considérée comme \emph{froide}, dont on approximera la dynamique comme celle d'un fluide ; une seconde population ayant une vitesse thermique plus élevée, considérée comme \emph{chaude}, mais contrairement au chapitre~\ref{chap2}, celle-ci n'a pas besoin d'être distribuée selon une bi-maxwellienne. Ce type de modèle peut être utilisé pour modéliser des particules d'un plasma dans un tokamak, ou des particules du vent solaire interagissant avec la magnétosphère terrestre. Dans ces contextes les particules se déplacent de manière hélicoïdale selon les lignes du champ extérieur $\vb{B}_0 = (0,0,B_0)^\top$, et seul le déplacement dans cette direction sera étudié ici, ce qui explique le nom de la seule variable d'espace considérée : $z$. Pour le déplacement hélicoïdale complet \Josselin{il faudrait ajouter ici une référence que je n'ai pas car je ne connais pas trop ces modèles, de mémoire c'est ce que faisait Xiaofei}.

La prise en compte des effets du champ magnétique dans le modèle nous mène à considérer un système à 7 inconnues $(j_{c,x},j_{c,y},B_x,B_y,E_x,E_y,f_h)$, ce qui induit un coût de calcul bien plus important avec les méthodes eulériennes que nous étudions. Pour diminuer ce coût de calcul nous souhaitons diminuer le nombre d'itérations tout en assurant la stabilité des méthodes considérées ; cela se fait en augmentant le pas de temps $\Delta t$ et en estimant les contraintes de stabilité. L'utilisation de méthodes d'ordre élevés en temps, en espace et en vitesse, permet de réduire l'erreur, capturer la dynamique non-linéaire du système, avec peu de points de discrétisation.

Nous souhaitons dans ce chapitre comparer deux méthodes d'ordre élevé en temps et dans l'espace des phrases, sur un cas à 4 dimensions, plus proches d'applications physiques, il s'agit d'une généralisation des deux stratégies développées dans le chapitre~\ref{chap2}, une méthode de \emph{splitting} et une méthode de Lawson. La contrainte du nombre de dimensions ne permet pas de raffiner le maillage\footnote{Pour une discrétisation de $128$ points par direction, la grille contient $128^4$ points, soit, représentées avec des réels à virgule flottante à double précision (64 bits), un espace mémoire de 2Go pour la seule variable $f_h$ ; une méthode de Lawson d'ordre 4 nécessite la sauvegarde des étages intermédiaires, donc $4\times 2\textrm{Go}$ minimum. L'utilisation de transformées de Fourier impose de travailler avec des nombres complexes, ce qui nécessite de doubler l'utilisation mémoire pour la même précision.}, ce qui ne permet pas l'accès à une solution de référence, nous devrons nous contenter de regarder les invariants (comme l'énergie totale), et de comparer les résultats aux taux d'instabilités fournis par les relations de dispersion.

Pour résoudre les problèmes que pose ce problème nous présenterons tout d'abord le modèle hybride $1dz-3dv$, puis nous détaillerons les schémas numériques que nous considérons. La méthode de \emph{splitting} hamiltonien contient 7 étapes, rendant de fait l'ordre élevé (méthode de Suzuki) trop coûteux pour approfondir son étude, nous nous concentrerons donc sur des méthodes de \emph{splitting} d'ordre plus faible. Pour la méthode de Lawson, il est envisageable d'effectuer un filtrage du terme induit pas le champ magnétique externe $\vb{B}_0$, permettant d'augmenter le pas de temps stabilisant la méthode. La partie linéaire du système introduit une matrice dont il n'est pas possible de calculer exactement son exponentielle, pourtant nécessaire pour la mise en œuvre d'une méthode de Lawson perforante. Nous étudierons un premier cas où nous ne profitons pas pleinement de toute la partie linéaire du système, introduisant ainsi une condition de stabilité ; puis nous proposerons une méthode permettant, à l'aide de méthodes approchées comme la troncature de la série de Taylor ou les approximants de Padé, de calculer une approximation de l'exponentielle de toute la partie linéaire, permettant ainsi de se soustraire à une condition de stabilité. \Josselin{Mettre les références vers les sections.}
