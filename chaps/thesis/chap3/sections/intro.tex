% !TEX root = ../chap3.tex

\section{Introduction}

L'objectif de ce chapitre est d'étendre les stratégies développées pour la résolution d'un modèle hybride linéarisé, présentées dans le chapitre~\ref{chap2}, au cas $1dz-3dv$ permettant de prendre en compte, entre autre, les effets du champ magnétique sur la dynamique des particules. Nous étudierons dans ce chapitre un modèle hybride, ce qui suppose que la distribution de particules est composée de deux populations, une première ayant une vitesse thermique faible, et considérée comme \emph{froide}, dont on approximera la dynamique comme celle d'un fluide ; une seconde population ayant une vitesse thermique plus élevée, considérée comme \emph{chaude}, mais contrairement au chapitre~\ref{chap2}, celle-ci n'a pas besoin d'être distribuée selon une bi-maxwellienne. Ce type de modèle peut être utilisé pour modéliser des particules d'un plasma dans un tokamak, ou des particules du vent solaire interagissant avec la magnétosphère terrestre. Dans ces contextes les particules se déplacent de manière hélicoïdale selon les lignes du champ extérieur $\vb{B}_0 = (0,0,B_0)^\top$, et seul le déplacement dans cette direction sera étudié ici, ce qui explique le nom de la seule variable d'espace considérée : $z$.

La prise en compte des effets du champ magnétique dans le modèle nous mène à considérer un système à 7 inconnues $(j_{c,x},j_{c,y},B_x,B_y,E_x,E_y,f_h)$, ce qui induit un coût de calcul bien plus important avec les méthodes eulériennes que nous étudions. Pour diminuer ce coût de calcul nous souhaitons diminuer le nombre d'itérations tout en assurant la stabilité des méthodes considérées ; cela se fait en augmentant le pas de temps $\Delta t$ et en estimant les contraintes de stabilité. L'utilisation de méthodes d'ordre élevé en temps, en espace et en vitesse, permet de réduire l'erreur, capturer la dynamique non-linéaire du système, avec peu de points de discrétisation.

Nous souhaitons dans ce chapitre comparer deux méthodes d'ordre élevé en temps et dans l'espace des phases, sur un cas à 4 dimensions, plus proches d'applications physiques, il s'agit d'une généralisation des deux stratégies développées dans le chapitre~\ref{chap2}, une méthode de \emph{splitting} et une méthode de Lawson. La contrainte du nombre de dimensions ne permet pas de raffiner le maillage\footnote{Pour une discrétisation de $128$ points par direction, la grille contient $128^4$ points, soit, représentés avec des réels à virgule flottante à double précision (64 bits), un espace mémoire de 2Go pour la seule variable $f_h$ ; une méthode de Lawson d'ordre 4 nécessite la sauvegarde des étages intermédiaires, donc $4\times 2\textrm{Go}$ minimum. L'utilisation de transformées de Fourier impose de travailler avec des nombres complexes, ce qui nécessite de doubler l'utilisation mémoire pour la même précision.}, ce qui ne permet pas l'accès à une solution de référence, nous devrons nous contenter de regarder les invariants (comme l'énergie totale), et de comparer les résultats aux taux d'instabilités fournis par les relations de dispersion.

Pour résoudre les problèmes que pose la résolution numérique de cette équation nous présenterons tout d'abord le modèle hybride $1dz-3dv$ dans la section~\ref{s:3:model}, puis nous détaillerons les schémas numériques que nous considérons au cours de la section~\ref{sec:3:scheme}. La complexité du modèle implique plusieurs difficultés par rapport au cas $1dx-1dv$ du chapitre précédent. La méthode de \emph{splitting} hamiltonien contient 7 étapes, faisant de la méthode de Strang (d'ordre 2) une méthode à 15 étapes ; la méthode de Suzuki (d'ordre 4) comptabilise 71 étapes, rendant de fait l'ordre élevé trop coûteux pour approfondir l'étude des méthodes de \emph{splitting}. Pour ce qui concerne la méthode de Lawson, plusieurs termes linéaires apparaissent rendant difficile, voire impossible, le calcul de l'exponentielle nécessaire pour la construction du schéma. Il est envisageable d'effectuer un filtrage du terme induit pas le champ magnétique externe $\vb{B}_0$, permettant d'augmenter le pas de temps stabilisant la méthode. Les équations de Maxwell ne peuvent être introduites dans la partie linéaire simplement à cause du calcul de l'exponentielle de matrice que cela implique. Dans un premier temps nous introduirons ce terme dans la partie non-linéaire du schéma de Lawson, induisant les résultats numériques de la section~\ref{sec:3:num}, dans un second temps nous proposerons une méthode permettant, à l'aide de la troncature de la série de Taylor ou des approximants de Padé, de calculer une approximation de l'exponentielle de toute la partie linéaire, permettant ainsi de se soustraire à une condition de stabilité, comme nous le verrons dans la section~\ref{s3:approx} pour l'introduction de ces approximations et la section~\ref{s:3:num_m} pour la présentation de ces nouveaux résultats numériques. Une section supplémentaire,~\ref{s:3:info}, présentera les outils informatiques développés pour permettre l'écriture du code de simulation à 4 dimensions, ainsi que les performances de celui-ci.