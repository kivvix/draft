% !TEX root = ../../main.tex

\section{Conclusion}
% -------------------------------------------------------------------

Ce chapitre est une extension de ce qui a pu être effectué dans le chapitre~\ref{chap2} au cas $1dz-3dv$ mais a nécessité de nombreux nouveaux développements, entre autre à cause du passage de 3 à 7 inconnues. Nous avons profité de l'analyse faite dans le chapitre précédent pour étudier le modèle hybride en dimensions supérieures et ses deux méthodes de résolution. La première méthode de résolution, méthode de \emph{splitting}, profite de la structure hamiltonienne du système qui n'avait pas été exploitée jusque là dans la littérature, mais le nombre d'étapes supplémentaires ne rend pas la méthode viable d'un point de vue numérique. La seconde méthode de résolution est la méthode de Lawson, et celle-ci permet une augmentation du coût numérique linéaire par rapport à l'ordre en temps choisi. Plusieurs travaux ont permis d'améliorer la méthode de Lawson à ce contexte. Il est possible de se défaire d'une contrainte de stabilité à l'aide d'un filtrage. Ensuite il n'est pas possible de calculer formellement l'exponentielle de la partie linéaire, deux options ont été proposées : tout d'abord mettre certains termes dans la partie linéaire, engendrant une condition de stabilité restrictive ; ensuite effectuer une approximation de l'exponentielle à l'aide d'approximations telles que les méthodes de Taylor ou Padé, permettant de lever les conditions de stabilités en espace. Ainsi nous avons pu développer une méthode de pas de temps adaptatif proposant de grands pas de temps dans la phase linéaire sans nuire à la stabilité de la méthode.

Des comparaisons avec le code PIC (Particle-In-Cell) développé au NMPP Garching sur la base des travaux~\cite{Holderied:2020}sont envisagées. La méthode utilisée dans ce code est basée sur le formalisme GEMPIC qui exploite la structure hamiltonienne du modèle hybride pour construire une méthode PIC géométrique (voir~\cite{Kraus:2017}). L’approximation en temps utilise un splitting hamiltonien qui implique que les équations de Maxwell sont résolues de manière explicite entrainant donc une contrainte de stabilité. Dans notre approche, nous n’avons pas cette contrainte grâce à l’utilisation d’approximants de Padé. Il serait donc intéressant de transférer les techniques développées ici au système d’EDO obtenu avec le formalisme GEMPIC pour potentiellement permettre d’utiliser des pas de temps plus grands. Un des points importants sera de gérer des matrices de taille importante (typiquement la taille est liée au nombre de particules $N_p\approx 10^5, 10^7$) et la question de la viabilité de l’approche par approximant de Padé dans ce contexte est liée à cette gestion de grande matrice. Plusieurs travaux existent néanmoins~\cite{Li:2011} qui permettent d'éviter l'inversion de matrice, étape potentiellement compliquée si celle-ci est de grande taille.

\vfill

\begin{otherlanguage}{english}
Acknowledgment: Experiments presented in this section were carried out using the PlaFRIM experimental testbed, supported by Inria, CNRS (LABRI and IMB), Université de Bordeaux, Bordeaux INP and Conseil Régional d’Aquitaine (see \url{https://www.plafrim.fr/}).
\end{otherlanguage}
