% !TEX root = ../../main.tex

\section{Conclusion}
% -------------------------------------------------------------------

Ce chapitre est une extension de ce qui a pu être effectué dans le chapitre~\ref{chap2} au cas $1dz-3dv$. Nous avons profité de l'analyse faite dans le chapitre précédent pour étudier le modèle hybride en dimensions supérieures et ses deux méthodes de résolution. La première méthode de résolution, méthode de \emph{splitting}, profite de la structure hamiltonienne du système qui n'avait pas été exploitée jusque là dans la littérature, mais le nombre d'étapes supplémentaires ne rend pas la méthode viable d'un point de vue numérique. En effet, la seconde méthode de résolution est la méthode de Lawson, et celle-ci permet une augmentation du coup numérique linéaire par rapport à l'ordre en temps choisi.

Nous avons pu enrichir la méthode de Lawson d'une méthode d'approximation de l'exponentielle de la partie linéaire à l'aide d'approximants de Padé. Cette stratégie innovante permet profiter de plus de termes linéaires lorsque l'exponentielle de la partie linéaire n'est pas calculable formellement. La condition de stabilité est ainsi réduite à quelques termes non-linéaires et permet dans la partie linéaire du problème de profiter de grands pas de temps via une méthode de pas de temps adaptatif.

Des comparaisons avec le code PIC (\emph{Particles In Cells}) développé au NMPP Garching sur la base des travaux~\cite{Holderied:2020} sont envisagées. La méthode exposée est basée sur du GEMPIC~\cite{Kraus:2017}, associé à un \emph{splitting} hamiltonien, cela implique que les équations de Maxwell sont résolues de manière explicite entrainant donc une contrainte de stabilité que nous avons pu contournée via l'utilisation d'approximants de Padé. Il est envisageable de profiter de la structure géométrique de GEMPIC avec les méthodes de Lawson. Les méthodes PIC entrainent des matrices dont la taille est liée au nombre de particules $N_p\approx 10^5,10^7$, cela pose la question de la viabilité d'un approximant de Padé dans ce contexte, qui nécessite l'inversion d'une matrice.

\vfill

\begin{otherlanguage}{english}
Acknowledgment: Experiments presented in this section were carried out using the PlaFRIM experimental testbed, supported by Inria, CNRS (LABRI and IMB), Université de Bordeaux, Bordeaux INP and Conseil Régional d’Aquitaine (see \url{https://www.plafrim.fr/}).
\end{otherlanguage}
