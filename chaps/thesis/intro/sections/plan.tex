% !TEX root = ../../main.tex

\section{Contenu du document}
%%%%%%%%%%%%%%%%%%%%%%%%%%%%%%%%%%%%%%%%%%%%%%%%%%%%%%%%%%%%%%%%%%%%%%

Le fil conducteur de ces travaux est la résolution numérique d'ordre élevé du système de Vlasov-Maxwell, et la mise en place de méthodes exponentielles pour résoudre des équations cinétiques.

Dans un premier chapitre (écrit en anglais) qui a fait l'objet d'une publication dans \emph{Journal of Computational Physics} en collaboration avec Nicolas Crouseilles\footnote{Univ Rennes, Inria Bretagne Atlantique (MINGuS) \& ENS Rennes} et Lukas Einkemmer\footnote{Department of Mathematics, Univeristy of Innsbruck, Austria}, est étudié une classe de schémas dits exponentiels pour des équations cinétiques de type Vlasov. L'idée de ces schémas repose sur la formule de Duhamel où la partie linéaire de l'équation est résolue exactement en temps alors que la partie non-linéaire est approchée par une méthode de type Runge-Kutta. Mêmes si ces méthodes sont bien connues, c'est la première fois qu'elles sont utilisées pour les équations cinétiques. Ces méthodes s'avèrent très adaptées à ce contexte puisqu'elles permettent de se défaire d'une contrainte de stabilité induite par la partie linéaire (souvent la plus restrictive). Une étude précise de la stabilité de plusieurs méthodes exponentielles a été effectuée, permettant de montrer que les schémas de Lawson étaient plus adaptées que les schémas exponentiels classiques. L'étude de stabilité et l'estimation numérique de condition de stabilité a été automatisée et a mené au développement d'un \emph{package} \Python{} ainsi qu'une mise en ligne des résultats\footnote{\url{http://jmassot.perso.math.cnrs.fr/ponio.html}}.

Le deuxième et troisième chapitre font l'objet d'un article soumis en 2021 au \emph{Journal of Computational Physics} en collaboration avec Anaïs Crestetto\footnote{Université de Nantes, laboratoire de Mathématiques Jean Leray}, Nicolas Crouseilles\footnote{Univ Rennes, Inria Bretagne Atlantique (MINGuS) \& ENS Rennes} et Yingzhe Li\footnote{Max Planck Institute, Institut Für Plasmaphysik, Germany}. Nous nous proposons ici de développer plusieurs aspects n'étant pas dans l'article soumis. L'objectif de ces deux chapitres est l'étude numérique d'un modèle hybride fluide-cinétique introduit dans la communauté physique des plasmas~\cite{Holderied:2020}. Ce modèle hybride repose sur l'hypothèse que le système étudié est composé de deux populations d'électrons, une \emph{chaude} et une plus \emph{froide}, ce qui introduit des restrictions fortes sur certains paramètres numériques (voir la figure~\ref{fig:intro:distrib}). Ainsi, la population dite \emph{froide} est approchée par une fonction de Dirac dans l'espace des vitesses dont les paramètres sont solutions d'un modèle fluide d'Euler sans pression. La population des particules \emph{chaudes} est décrite par une équation de Vlasov, et le couplage s'effectue au travers des équations de Maxwell décrivant la dynamique des champs électromagnétiques. Cette thèse s'intéresse alors à la résolution numérique de ce système par deux méthodes :
\begin{itemize}
  \item Une première reposant sur la structure géométrique de ce modèle hybride, permettant de construire un \emph{splitting} temporel dont toutes les sous-étapes peuvent être résolues exactement en temps.
  \item Une seconde utilise les résultats du premier chapitre puisqu'elle repose sur des méthodes de Lawson.
\end{itemize}

Ainsi le chapitre~\ref{chap2} étudie et valide numériquement la pertinence du modèle hybride par rapport au modèle cinétique dans la limite asymptotique du ratio des températures qui rend vers zéro en dimension réduite (1 dimension d'espace et 1 dimension de vitesse). Cette validation s'effectuera par une étude approfondie des relations de dispersion. S'en suivra une comparaison des deux méthodes de résolution pour le modèle hybride. On remarquera déjà que les méthodes de \emph{splitting}, très populaires, s'avèrent relativement coûteuse lors de leur montée en ordre, et que la méthode de pas de temps adaptatif associée est moins performante que son équivalent en méthode de Lawson.

Le chapitre~\ref{chap3} s'intéresse à un cadre plus complexe et pertinent d'un point de vue physique (1 dimension d'espace et 3 dimensions de vitesse). Le modèle fait intervenir une méthode de \emph{splitting} à 4 étapes rendant très difficile son utilisation à l'ordre élevé. Le coût de calcul de la méthode de Lawson, augmentant linéairement avec l'ordre en temps, elle devient une très bonne alternative en pratique. Le calcul du flot de la partie linéaire du modèle hybride n'est pas accessible à l'aide de logiciels de calcul formel, rendant compliqué la mise en place d'une méthode de Lawson efficace dans ce contexte. Une première stratégie a consisté à transférer des termes de la partie linéaire à la partie non-linéaire, entrainant une contrainte de stabilité ; enfin une seconde approche a été d'explorer des méthodes telles que les approximants de Padé permettant de lever toute contrainte de stabilité provenant d'un terme linéaire. Pour permettre l'implémentation rapide de différentes méthodes, des outils de métaprogrammation ont été mis en place, permettant la génération automatique de code.
