% !TEX root = ../main.tex

La simulation numérique fut introduite dès l’émergence de l’informatique pour enrichir les connaissances scientifiques dans des contextes où l’expérimentation est trop contraignante voire impossible. La simulation peut aussi avoir un intérêt prédictif pour dimensionner un problème physique (simulation de tokamak avant leur construction dans le projet ITER) ou pour tester un modèle et le confronter aux futures observations (simulation de nébuleuses ou d’étoiles). La simulation peut être vue comme une retranscription informatique de modèles mathématiques, censés représenter des phénomènes physiques. La simulation numérique doit être représentative de la réalité. Ainsi, dans des modèles où la solution exacte est souvent hors de portée, il est nécessaire de vérifier que la transcription numérique conserve certaines propriétés mathématiques du modèle (conservation de certaines quantités physiques comme la masse ou l’énergie totale par exemple).

Un enjeu majeur de la modélisation et de la simulation est de maintenir un équilibre entre les approximations au niveau du modèle, qui permettent d’accélérer le temps de traitement et la précision des résultats.
