\section{Dérivation du modèle de Vlasov-Maxwell hybride linéarisé}

\subsection{Dérivation du modèle de Vlasov-Maxwell hybride}
\label{ssec:0:vmh}

On s'intéresse ici à démontrer la proposition~\ref{pro:0:vmh}.
\begin{proof}
  Pour écrire le modèle hybride, on distingue deux populations de particles dans le modèle de Vlasov-Maxwell~\eqref{eq:0:vlasov}-\eqref{eq:0:poisson}, une population de particules froides $f_c$ donc la vitesse thermique est faible, et une seconde population de particules chaudes $f_h$ dont la vitesse thermique est grande. Ces populations sont supposées indépendantes, l'équation de Vlasov devient :
  $$
    \begin{aligned}
      \pdv{f_c}{t} &+ \vb{v}\cdot\nabla_{\vb{x}} f_c + \frac{q_e}{m_e}\left( \vb{E} + \vb{v}\times(\vb{B}+\vb{B}_0) \right)\cdot\nabla_{\vb{v}} f_c = 0 \\
      \pdv{f_h}{t} &+ \vb{v}\cdot\nabla_{\vb{x}} f_h + \frac{q_e}{m_e}\left( \vb{E} + \vb{v}\times(\vb{B}+\vb{B}_0) \right)\cdot\nabla_{\vb{v}} f_h = 0
    \end{aligned}
  $$
  et l'équation de Maxwell-Ampère~\eqref{eq:0:maxwellE} et de Maxwell-Gauss~\eqref{eq:0:poisson} deviennent :
  $$
    \begin{aligned}
      \frac{1}{c^2}\pdv{\vb{E}}{t} &= \nabla_{\vb{x}}\times \vb{B} - \mu q_e\int \vb{v} f_c\dd{\vb{v}} - \mu q_e\int \vb{v} f_h\dd{\vb{v}} \\
      \nabla_{\vb{x}}\cdot\vb{E} &= \frac{1}{\varepsilon_0}\left( q_i\rho_i + q_e\int f_c\dd{\vb{v}} + q_e\int f_h\dd{\vb{v}} \right)
    \end{aligned}
  $$
  On introduit la densité $\rho_c=\rho_c(t,\vb{x})$ et la vitesse moyenne $\vb{u}_c=\vb{u}_c(t,\vb{x})$ des particules froides :
  $$
    \begin{pmatrix}
      \rho_c \\
      \rho_c \vb{u}_c
    \end{pmatrix}
    =
    \int_{\mathbb{R}^3} \begin{pmatrix}
      1 \\
      \vb{v}
    \end{pmatrix} f_c\dd{\vb{v}}
  $$
  Maintenant on calcule les moments de l'équation de Vlasov sur les particules chaudes, c'est-à-dire que l'on multiplie l'équation par $(1,\vb{v})^\top$ puis on intègre en $\vb{v}$ :
  $$
    \begin{aligned}
    \partial_t\int_{\mathbb{R}^3} \begin{pmatrix}
      1 \\
      \vb{v}
    \end{pmatrix}f_c
    &+
    \int_{\mathbb{R}^3}\begin{pmatrix}
      1 \\
      \vb{v}
    \end{pmatrix}(\vb{v}\cdot\nabla_{\vb{x}}f_c)\dd{\vb{v}} \\
    &+
    \frac{q_e}{m_e}\int_{\mathbb{R}^3}\begin{pmatrix}
      1 \\
      \vb{v}
    \end{pmatrix} (\vb{E}\cdot\nabla_{\vb{v}}f_c)\dd{\vb{v}}
    +
    \frac{q_e}{m_e}\int_{\mathbb{R}^3}\begin{pmatrix}
      1 \\
      \vb{v}
    \end{pmatrix}(\vb{v}\times(\vb{B}+\vb{B}_0) \cdot\nabla_{\vb{v}}f_c)\dd{\vb{v}}
    = 0.
    \end{aligned}
  $$
  On réécrit cela comme deux équations, une par moment calculé :
  $$
    \begin{aligned}
      &\partial_t\rho_c + \nabla_{\vb{x}}\cdot(\rho_c\vb{u}_c) = 0 \\
      &\partial_t\rho_c\vb{u}_c + \int\vb{v}(\vb{v}\cdot\nabla_{\vb{x}}f_c)\dd{\vb{v}} + \frac{q_e}{m_e}\int\vb{v}(\vb{E}\cdot\nabla_{\vb{v}}f_c)\dd{\vb{v}} \\
      &\phantom{\partial_t\rho_c\vb{u}_c + \int\vb{v}(\vb{v}\cdot\nabla_{\vb{x}}f_c)\dd{\vb{v}}}
        + \frac{q_e}{m_e}\int\vb{v}\left((\vb{v}\times(\vb{B}+\vb{B}_0))\cdot\nabla_{\vb{v}}f_c\right)\dd{\vb{v}} = 0
    \end{aligned}
  $$
  Cette deuxième équation contient plusieurs termes que l'on peut traiter individuellement en les regardant par composante :
  $$
    \begin{aligned}
      \left( \int\vb{v}(\vb{v}\cdot\nabla_{\vb{x}}f_c)\dd{\vb{v}} \right)_i
          &= \int v_i \sum_{j=1}^3v_j\partial_{x_j}f_c\dd{v_i} \\
          &= \sum_{j=1}^3\partial_{x_j}\int v_iv_jf_x\dd{v_i} \\
          &= \sum_{j=1}^3\partial_{x_j}\int (\vb{v}\otimes\vb{v})_{ij}f_x\dd{v_i} \\
          &= \left( \nabla_{\vb{x}}\cdot\int \vb{v}\otimes\vb{v} f_c\dd{\vb{v}} \right)_i \\
          &= \left( \nabla_{\vb{x}}\cdot(\rho_c\vb{u}_c\otimes\vb{u}_c) \right)_i
    \end{aligned}
  $$
  où on note $\otimes$ le produit tensoriel : $(\alpha\otimes\beta)_{i,j} = \alpha_i\beta_j$. Le terme suivant :
  $$
    \begin{aligned}
      \left( \frac{q_e}{m_e}\int\vb{v}(\vb{E}\cdot\nabla_{\vb{v}}f_c)\dd{\vb{v}} \right)_i
          &= \frac{q_e}{m_e}\int v_i\sum_{j=1}^3 E_j\partial_{v_j}f_c\dd{v_i} \\
          &= -\frac{q_e}{m_e}\sum_{j=1}^3 e_j\int\partial_{v_j}(v_i)f_c\dd{v_i} \\
          &= -\frac{q_e}{m_e}E_i\int f_c\dd{v_i} \\
          &= -\frac{q_e}{m_e}(\rho_c E)_i.
    \end{aligned}
  $$
  Et enfin le troisième terme :
  $$
    \begin{aligned}
      \left( \frac{q_e}{m_e}\int\vb{v}\left((\vb{v}\times(\vb{B}+\vb{B}_0))\cdot\nabla_{\vb{v}}f_c\right)\dd{\vb{v}} \right)_i
          &= \frac{q_e}{m_e}\int v_i\sum_{j=1}^3\left(\vb{v}\times(\vb{B}+\vb{B}_0)\right)_j\partial_{v_j}f_c\dd{v_i} \\
          &= -\frac{q_e}{m_e}\sum_{j=1}^3\int\partial_{v_j}(v_i)\left(\vb{v}\times(\vb{B}+\vb{B}_0)\right)_jf_c\dd{v_i} \\
          &= -\frac{q_e}{m_e}\int\left(\vb{v}\times(\vb{B}+\vb{B}_0)\right)_i f_c\dd{v_i} \\
          &= \left[ -\frac{q_e}{m_e}\left(\int\vb{v}f_c\dd{\vb{v}}\right)\times(\vb{B}+\vb{B}_0) \right]_i \\
          &= \left[ -\frac{q_e}{m_e}(\rho_c\vb{u}_c)\times(\vb{B}+\vb{B}_0) \right]_i.
    \end{aligned}
  $$
  On peut ainsi réécrire les moments comme :
  $$
    \begin{aligned}
      \partial_t\rho_c + \nabla_{\vb{x}}\cdot(\rho_c\vb{u}_c) &= 0 \\
      \partial_t(\rho_c\vb{u}_c) + \nabla_{\vb{x}}\cdot(\rho_c\vb{u}_c\otimes\vb{u}_c) - \frac{q}{m}\left( \rho_c\vb{E} + (\rho_c\vb{u}_c)\times(\vb{B}+\vb{B}_0) \right) &=0.
    \end{aligned}
  $$

  On définit maintenant le courant induit par les particules froides $\vb{j}_c=\vb{j}_c(t,\vb{x})$ comme étant une renormalisation du courant $\rho_c\vb{u}_c$ :
  $$
    \vb{j}_c = q_e(\rho_c\vb{u}_c)
  $$
  ce qui nous permet de réécrire l'équation du deuxième moment de $f_c$ comme :
  $$
    \partial_t\vb{j}_c + \nabla_{\vb{x}}\cdot\frac{\vb{j}_c\otimes\vb{j}_c}{q_e\rho_c} = \frac{q_e}{m_e}\left( q_e\rho_c\vb{E} + \vb{j}_c\times(\vb{B}+\vb{B}_0) \right)
  $$

  On écrit ainsi les équations de Vlasov-Maxwell hybride :
  \begin{align}
      \partial_t f_h &+ \vb{v}\cdot\nabla_{\vb{x}} f_h + \frac{q_e}{m_e}\left( \vb{E} + \vb{v}\times(\vb{B}+\vb{B}_0) \right)\cdot\nabla_{\vb{v}} f_h = 0 \\
      \partial_t\rho_c &+ \frac{1}{q_e}\nabla_{\vb{x}}\cdot(\vb{j}_c) = 0 \\
      \partial_t\vb{j}_c &+ \nabla_{\vb{x}}\cdot\frac{\vb{j}_c\otimes\vb{j}_c}{q_e\rho_c} - \frac{q_e}{m}\left( q_e\rho_c\vb{E} + \vb{j}_c\times(\vb{B}+\vb{B}_0) \right) = 0 \\
      \pdv{\vb{B}}{t} &= - \nabla_{\vb{x}}\times\vb{E} \\
      \frac{1}{c^2}\partial_t\vb{E} &= \nabla_{\vb{x}}\times \vb{B} - \mu\vb{j}_c - \mu q_e\int \vb{v} f_h\dd{\vb{v}} \\
      \nabla_{\vb{x}}\cdot\vb{E} &= \frac{1}{\varepsilon_0}\left( q_i\rho_i + q_e\rho_c + q_e\int f_h\dd{\vb{v}} \right)
  \end{align}
\end{proof}

\subsection{Dérivation du modèle de Vlasov-Maxwell hybride linéarisé}
\label{ssec:0:vmhl}

On s'intéresse ici à démontrer la proposition~\ref{pro:0:vmhl}.
\begin{proof}
  On souhaite linéariser le système~\eqref{eq:0:vmh:1}-\eqref{eq:0:vmh:6} autour de l'état d'équilibre $\left(\rho_c^{(0)},0,0,0,f_h^{(0)}(\vb{v})\right)$, avec $f_h^{(0)}(\vb{v})$ une fonction telle que $\int\vb{v}f_h^{(0)}\dd{\vb{v}} = 0$. En réutilisant la réécriture des inconnues~\eqref{eq:0:linear}, on peut réécrire le système~\eqref{eq:0:vmh:1}-\eqref{eq:0:vmh:6} sous la forme :
  \begin{align}
    \label{eq:0:vmhleps:1}
      \partial_t f_h &+ \vb{v}\cdot\nabla_{\vb{x}} f_h + \frac{q_e}{m_e}\left( \vb{E} + \vb{v}\times(\vb{B}+\vb{B}_0) \right)\cdot\nabla_{\vb{v}} f_h = 0 \\
    \label{eq:0:vmhleps:2}
      \varepsilon\partial_t\rho_c^{(1)} &+ \frac{\varepsilon}{q_e}\nabla_{\vb{x}}\cdot(\vb{j}_c^{(1)}) = 0 \\
    \label{eq:0:vmhleps:3}
      \varepsilon\partial_t\vb{j}_c^{(1)} &+ \varepsilon^2\nabla_{\vb{x}}\cdot\frac{\vb{j}_c^{(1)}\otimes\vb{j}_c^{(1)}}{q_e(\rho_c^{(0)}+\varepsilon\rho_c^{(1)})} - \frac{q_e}{m_e}\left( q_e(\rho_c^{(0)}+\varepsilon\rho_c^{(1)})\varepsilon\vb{E} + \varepsilon\vb{j}_c^{(1)}\times(\varepsilon\vb{B}^{(1)}+\vb{B}_0) \right) = 0 \\
    \label{eq:0:vmhleps:4}
      \varepsilon\partial_t\vb{B}^{(1)} &= - \varepsilon\nabla_{\vb{x}}\times\vb{E}^{(1)} \\
    \label{eq:0:vmhleps:5}
      \frac{\varepsilon}{c^2}\partial_t\vb{E}^{(1)} &= \varepsilon\nabla_{\vb{x}}\times \vb{B}^{(1)} - \varepsilon\mu\vb{j}_c^{(1)} - \varepsilon\mu q_e\int \vb{v} f_h^{(1)}\dd{\vb{v}} \\
    \label{eq:0:vmhleps:6}
      \varepsilon\nabla_{\vb{x}}\cdot\vb{E}^{(1)} &= \frac{1}{\varepsilon_0}\left( q_i\rho_i + q_e\rho_c + q_e\int f_h\dd{\vb{v}} \right)
  \end{align}
  Nous souhaitons négliger les termes non-linéaires d'ordre $\order{\varepsilon^2}$. Dans le système~\eqref{eq:0:vmhleps:1}-\eqref{eq:0:vmhleps:5} On remarque alors que~\eqref{eq:0:vmhleps:2} est la seule équation faisant intervenir $\rho_c^{(1)}$, et cette variable peut être recalculer à partir de~\eqref{eq:0:vmhleps:6} si besoin. L'équation~\eqref{eq:0:vmhleps:2} pourra donc être omise du système par la suite. Par abus de notation, pour la lisibilité, on retirera les indices de variables linéarisées lorsqu'il n'y a pas d'ambiguïté, par conséquent les variables $\vb{j}_c$, $\vb{E}$, $\vb{B}$ qui apparaissent par la suite, donc d'ordre $\varepsilon$ et correspondent à leurs perturbations par rapport à un équilibre nul. On introduit aussi la fréquence de plasma des particules froides $\Omega_{pe}^2 = \frac{q^2\rho_c^{(0)}}{\varepsilon_0m_e}$. On peut alors réécrire le système~\eqref{eq:0:vmhleps:1}-\eqref{eq:0:vmhleps:5} comme :
  \begin{align}
      \partial_t f_h &+ \vb{v}\cdot\nabla_{\vb{x}} f_h + \frac{q_e}{m_e}\left( \vb{E} + \vb{v}\times(\vb{B}+\vb{B}_0) \right)\cdot\nabla_{\vb{v}} f_h = 0 \\
      \partial_t j_c &= \varepsilon_0\Omega_{pe}^2\vb{E} + \frac{q}{m_e}\vb{j}_c\times\vb{B}_0 \\
      \partial_t\vb{B} &= -\nabla_{\vb{x}}\times\vb{E} \\
      \frac{1}{c^2}\partial_t\vb{E} &= \nabla_{\vb{x}}\times\vb{B} - \mu_0\vb{j}_c - \mu_0q_e\int\vb{v}f_h\dd{\vb{v}}
  \end{align}
\end{proof}

\Josselin{Est-ce qu'on ajoute ici l'adimensionnement du système~\eqref{eq:0:vmhl:1}-\eqref{eq:0:vmhl:4} ou un lien vers l'annexe où cela sera fait ?}


