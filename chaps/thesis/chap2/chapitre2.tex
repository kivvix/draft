% !TEX root = ../main.tex
\renewcommand{\localPath}{chap2}

\chapter{Modèle hybride linéarisé dans~le~cas~$1dx-1dv$}
\label{chap2}

Ce chapitre est la première partie de l'étude du modèle hybride linéarisé~\eqref{eq:0:vmhl:1}-\eqref{eq:0:vmhl:4} dans le cadre restreint $1dx-1dv$. Cela permet de mettre en place certains outils d'analyse, et d'utiliser les résultats du chapitre précédent dans le contexte d'un modèle hybride linéarisé. Ce travail collaboratif avec Anaïs Crestetto\footnote{Université de Nantes, laboratoire de Mathématiques Jean Leray}, Nicolas Crouseilles\footnote{Univ Rennes, Inria Bretagne Atlantique (MINGuS) \& ENS Rennes} et Yingzhe Li\footnote{Max Planck Institute, Institut Für Plasmaphysik, Germany} a mené à un article \emph{Comparison of high-order Eulerian methods for electron hybrid model} soumis dans \emph{Journal of Computational Physics} en 2021.

%% section 1
\section{Introduction}
L'objectif de ce chapitre est d'introduire et de simuler numériquement une hiérarchie de modèles permettant de décrire des systèmes de particules chargées où une population de particules chaudes int\'eragit avec un plasma ambiant plus froid. Une telle configuration physique peut par exemple \^etre 
\'etudi\'ee dans les plasmas de tokamak o\`u les particules alpha (g\'en\'er\'ees par les r\'eactions de 
fusion) interagissent avec le plasma ambiant. Un autre exemple se trouve dans la haute atmosph\`ere  o\`u les \'electrons \'energ\'etiques du vent solaire interagissent avec la magn\'etosph\`ere terrestre. 
Des modèles adapt\'es \`a ces configurations ont ainsi \'et\'e obtenus par exemple dans 
les deux contextes (voir \cite{Holderied:2019}, \cite{Chen:2016} \cite{Katoh:2007} 
\cite{Tao:2014} \cite{Tronci:2010} \cite{Tronci:2014}). 
Le modèle de départ qui servira de référence repose sur une description cinétique pour l'ensemble du plasma consid\'er\'e. On introduit alors la fonction de distribution des \'electrons  $f(t,x,v)\in\mathbb{R}_+$ solution du modèle de Vlasov-Poisson (les ions sont consid\'er\'ees immobiles, 
comme \'etant un fond neutralisant). En supposant que la population \'electronique peut \^etre 
s\'epar\'ee entre une population "froide" $f_c$ et une population d\'electrons \'energ\'etiques $f_h$,  
une première étape consiste à écrire $f$ comme la somme de ces deux fonctions de distribution 
$f=f_c+f_h$. Une seconde étape consiste à supposer que les particules froides restent proches d'un équilibre thermodynamique de temp\'erature $T_c\approx 0$ et peuvent donc être réprésentées par un modèle fluide. On obtient le modèle hybride fluide/cin\'etique où la partie fluide (linéaire) décrit la dynamique des particules froides alors que les particules chaudes sont décrites à l'aide d'un modèle cinétique.
Ce mod\`ele hybride peut encore \^etre simplifi\'e en consid\'erant des perturbations de type ondes de faible amplitude. Les termes non lin\'eaires de la partie fluide sont donc n\'eglig\'es alors que la 
partie cin\'etique reste non lin\'eaire. Le mod\`ele ainsi obtenu (voir \cite{Holderied:2019}) est le mod\`ele hybride lin\'earis\'e VHL (Vlasov Hybrid Linearized).

Du fait de la forte disparité des vitesses thermiques entre les particules froides et chaudes, le modèle cinétique est très coûteux à résoudre numériquement, notamment car le maillage en vitesse doit être choisi très fin pour capturer la vitesse thermique des particules froides. Cela implique en outre, pour les schémas numériques classiques, une condition restrictive sur le pas de temps et 
donc des simulations co\^uteuses. La dérivation de modèles simplifi\'es moins coûteux à résoudre numériquement est donc d'un grand int\'er\^et. Parmi ces mod\`eles simplifi\'es, 
nous consid\'ererons ici le mod\`ele hybride lin\'earis\'e VHL \'etudi\'e dans \cite{Holderied:2019}.  
Afin d'effectuer une \'etude comparative entre le mod\`ele VHL et le mod\`ele cin\'etique original 
et de tester les sch\'emas num\'eriques, nous nous placerons dans le cas de la dimension $1$ 
en espace et en vitesse. Ce cadre nous permettra aussi de poser les bases de l'\'etude 
du cas $1d$-$3v$ pour lequel il est beaucoup plus complexe d'effectuer ces comparaisons et ces tests. Ce type d'\'etude permettra enfin de comprendre le domaine de validité du mod\`ele VHL. 

Pour résoudre numériquement le modèle VHL, nous proposons deux méthodes. La première repose sur le fait que le modèle VHL possède une structure géométrique \cite{Morrison:2017}\cite{Tronci:2010}, \cite{Tronci:2014}. Plus pr\'ecis\'ement, le mod\`ele VHL poss\`de une structure Hamiltonienne non canonique, ce qui signifie que les \'equations peuvent \^etre 
obtenunes \`a partir d'un crochet de Poisson et d'un Hamiltonien. Cette structure garantit la préservation d'invariants, comme l'énergie totale. L'objectif est d'exploiter cette structure 
pour construire des sch\'emas num\'eriques qui poss\`edent un bon comportement en temps long. 
Le sch\'ema utilis\'e est un sch\'ema de type splitting construit \`a partir d'un \emph{splitting} du Hamiltonien. Cette approche permet de combiner astucieusement certains termes du mod\`ele et 
on est alors amené à résoudre trois sous-systèmes simples (comme dans \cite{Crouseilles:2015}, \cite{Casas:2017}, \cite{Li:2020}). Une propriété remarquable est que chacun des sous-systèmes peut être résolu exactement en temps, l'erreur en temps de la méthode provient donc uniquement de la méthode de \emph{splitting} utilisée. Des méthodes d'ordre arbitraire en temps peuvent être obtenues par composition \cite{Hairer:2006}. 
La deuxième méthode est basée sur un schéma exponentiel \cite{Hochbruck:2010}, \cite{Hochbruck:2005}, \cite{Lawson:1967a}, \cite{Isherwood:2018}, \cite{Lawson:1967}, \cite{Crouseilles:2019b}. En exploitant le fait que la partie linéaire du modèle VHL peut être résolue exactement et efficacement, on construit alors des schémas de type Lawson d'ordre élevé. Les r\'esultats du chapitre pr\'ec\'edent et de \cite{Crouseilles:2019b} sont donc repris et \'etendus au cas du syst\`eme VHL. 

Pour les deux méthodes en temps (splitting et Lawson), nous avons introduit une technique 
de pas de temps adaptatif. Pour les m\'ethodes de type Lawson, le cadre des m\'ethode 
{\it embedded} \cite{Dormand:1980}\cite{Dormand:1978}  \cite{Balac:2013b}\cite{Balac:2013a} 
permet de calculer l'erreur locale facilement. Dans le cas des m\'ethodes de splitting, 
nous utiliserons le travail r\'ecent \cite{Blanes:2019} qui propose des m\'ethodes 
de splitting {\it embedded}. Des m\'ethodes d'ordre $4(3)$ seront utilis\'ees dans le cadre de la 
comparaison (ordre $3$ et ordre $4$ pour estimer l'erreur locale). Pour l'approximation de l'espace des phases, nous avons choisi une méthode spectrale en espace et une approximation type différences finies d'ordre élevé (ordre 5 en pratique) pour la direction en vitesse. 

La premi\`ere approche (splitting Hamiltonien) comporte des similarit\'es avec les approches propos\'ees dans \cite{Kraus:2017} et \cite{Holderied:2019} ; n\'eanmoins, ces m\'ethodes reposent sur une approximation de type Particle-In-Cell de l'espace des phases alors que nous utilisons des m\'ethodes eul\'eriennes. Ainsi, 
on est plus dans l'esprit de \cite{Crouseilles:2015}, \cite{Li:2020} o\`u on effectue un splitting 
puis on discr\'etise alors que dans  \cite{Kraus:2017} et \cite{Holderied:2019}, on discr\'etise l'espace des phases puis on discr\'etise en temps.  


Afin de valider les résultats numériques, une étude approfondie des relations de dispersion est effectuée. Ces relations de dispersion sont obtenues par la résolution du modèle VHL lin\'earis\'e.  À l'aide de transformée de Fourier en espace, de transformée de Laplace en temps, il est en effet possible de déterminer très précisément la phase linéaire des simulations de modèle non linéaire ; on peut calculer le taux d'amortissement ou d'instabilité d'un équilibre perturbé \cite{Sonnendrucker:2015}, \cite{Fried:1961}, mais aussi reconstruire le mode fondamental du champ électrique. En plus de fournir des informations pour valider de manière quantitative les codes développés, cette analyse nous permet de faire le lien entre les modèles. En effet, 
en faisant tendre $T_c$ vers z\'ero dans la relation de dispersion du mod\`ele de Vlasov original, 
il est possible de retrouver la relation de dispersion du mod\`ele VHL. 

Le chapitre est organisé comme suit : nous présentons tout d'abord la hiérarchie de modèles que nous souhaitons étudier, depuis le modèle cinétique jusqu'au modèle hybride linéarisé. La structure géométrique de ce modèle est exhibée en section \ref{s:geom}. La section \ref{s:scheme} est dédiée à la présentation des méthodes numériques construites pour la résolution du modèle hybride linéarisé. Dans la section \ref{s:dispersion}, les relations de dispersion sont introduites et étudiées. Les sections \ref{s:limit} et \ref{s:compare} contiennent de nombreuses illustrations numériques. La section \ref{s:limit} se concentre sur la comparaison du modèle cinétique avec le modèle hybride linéarisé, alors que dans la section \ref{s:compare}, nous étudions les avantages et les inconvénients des deux méthodes numériques pour le modèle hybride linéarisé.

%% section 2
% !TEX root = ../chap3.tex

\section{Présentation du modèle}

$$
  \begin{cases}
    \partial_t j_{c,x} &= \Omega_{pe}^2E_x - j_{c,y}B_0 \\
    \partial_t j_{c,y} &= \Omega_{pe}^2E_y + j_{c,x}B_0 \\
    \partial_t B_{x}   &=  \partial_zE_y \\
    \partial_t B_{y}   &= -\partial_zE_x \\
    \partial_t E_{x}   &= -\partial_zB_y - j_{c,x} + \int v_xf_h\,\mathrm{d}v \\
    \partial_t E_{y}   &=  \partial_zB_x - j_{c,y} + \int v_yf_h\,\mathrm{d}v \\
    \partial_t f_h     &= -v_z\partial_zf_h + (E_x+v_yB_0-v_zB_y)\partial_{v_x}f_h + (E_y-v_xB_0+v_zB_x)\partial_{v_y}f_h + (v_xB_y - v_yB_x)\partial_{v_z}f_h
  \end{cases}
$$

%% section -
%% !TEX root = ../../main.tex

\section{Structure géométrique du mod\`ele hybride lin\'earis\'e VHL}
\label{s:geom}

Dans cette partie, nous présentons la structure du modèle hybride linéarisé VHL \eqref{eq:vahl}, à savoir son Hamiltonien et son crochet de Poisson. Pour simplifier les notations, nous notons dans cette section $f=f_h$, $\rho_c=\rho_c^{(0)}$ et $u=u_c$. Cette structure permet notamment d'assurer la préservation de nombreux invariants (énergie totale et opérateurs de Casimir entre autres) mais sera à la base d'un \emph{splitting} en temps, dans l'esprit de \cite{Crouseilles:2015}, \cite{Casas:2017}, \cite{Kraus:2017} \cite{Li:2020}. Nous aurons besoin d'introduire certaines notations pour pouvoir introduire la structure.

Tout d'abord, rappelons que pour une fonctionnelle donnée ${\cal G}(f)$, la dérivée de Fréchet de la distribution $\frac{\delta {\cal G}}{\delta f}(f)$ évaluée au point $f$, est définie par 
\begin{equation}
  {\cal G}(f + \delta f) - {\cal G}(f) = \int_{\Omega\times \mathbb{R}} \frac{\delta {\cal G}}{\delta f}(f)(x, v) \delta f(x, v) \mathrm{d}x\mathrm{d}v +{\cal O}(\delta f^2), 
\end{equation}
pour toute variation régulière $\delta f$. On définit le Hamiltonien associé au modèle VHL \eqref{eq:vahl} 
\begin{eqnarray}
\label{hamiltonian_red}
  \mathcal{H} &=& \frac{1}{2}\int_{\mathbb{R}} {E}^2 \mathrm{d}{x}  +  \frac{1}{2}\int_{\mathbb{R}} \rho_{c} u^2\mathrm{d}{x} + \frac{1}{2}\int_{\mathbb{R}}\int_{\mathbb{R}} v^2 f\,\mathrm{d}x\,\mathrm{d}v, \\
              &=&  \mathcal{H}_E + \mathcal{H}_u + \mathcal{H}_f. 
\end{eqnarray}
Les trois termes correspondent respectivement à l'énergie électrique, l'énergie cinétique des particules froides et l'énergie cinétique des particules chaudes. Pour une fonctionnelle ${\cal G}(E, u, f)$, on notera $\delta {\cal G}/\delta f$, $\delta {\cal G}/\delta E$ et $\delta {\cal G}/\delta u$ les dérivées de Fréchet de ${\cal G}$ par rapport à $f, E$ et $u$ respectivement. On introduit à présent le crochet de Poisson de deux fonctionnelles ${\cal F}(E, u, f)$ et ${\cal G}(E, u, f)$
$$
  \begin{aligned}
    \{ {\cal F}, {\cal G} \}( u, E, f) &= \int_{\mathbb{R}}\int_{\mathbb{R}} f \left( \partial_x \frac{\delta {\cal F}}{\delta f}\partial_{v} \frac{\delta {\cal G}}{\delta f} - \partial_{v} \frac{\delta {\cal F}}{\delta f}\partial_{x} \frac{\delta {\cal G}}{\delta f}\right)\mathrm{d}v \mathrm{d}x \\
                         & + \int_{\mathbb{R}}  \left(  \frac{\delta {\cal F}}{\delta{ u}}  \frac{\delta {\cal G}}{\delta{ E}} - \frac{\delta {\cal F}}{\delta{ E}}  \frac{\delta {\cal G}}{\delta{u}} \right) \mathrm{d}x \\
                         & + \int_{\mathbb{R}}\int_{\mathbb{R}}  \left(  \frac{\delta {\cal F}}{\delta{ E}}  \partial_v f\frac{\delta {\cal G}}{\delta{ f}} - \frac{\delta {\cal G}}{\delta{ E}} \partial_v f \frac{\delta {\cal F}}{\delta{f}} \right) \mathrm{d}{ v}\mathrm{d}x \\
  \end{aligned}
$$
Avec cette notation, le modèle hybride linéarisé~\eqref{eq:vahl} se réécrit alors, avec  $U=(u, E, f)$ et $\mathcal{H}$ donn\'e par \eqref{hamiltonian_red}
\begin{equation}
\label{ham_form}
  \partial_t U = \{ U, \mathcal{H} \}. 
\end{equation}

Dans la suite, on vérifie que la réécriture \eqref{ham_form} est bien équivalente au modèle VHL. 
Pour cela, on a besoin des relations suivantes 
$$
  \frac{\delta \mathcal{H}}{\delta f} = \frac{v^2}{2}, \;\; \frac{\delta \mathcal{H}}{\delta u} = \rho_c u, \;\; \frac{\delta \mathcal{H}}{\delta E} = E. 
$$
De plus, par abus de notation, on notera la fonctionnelle associ\'ee \`a la fonction comme suit (par exemple pour $u$) 
$u(t, z)=\int_{\mathbb{R}} u(t, x)\delta(x-z) \mathrm{d}x$, de sorte que $ \frac{\delta u}{\delta u} = \delta(x-z)$. Pour $f$, 
on notera $f(t, x, w)=\int_{\mathbb{R}} f(t, z, v)\delta(x-z)\delta(w-v) \mathrm{d}x\mathrm{d}v$, de sorte que $ \frac{\delta f}{\delta f} = \delta(x-z)\delta(w-v)$. 

\medskip 

\noindent $\bullet$ On calcule dans un premier temps $\{ u, \mathcal{H} \}$ 
\begin{eqnarray*}
\partial_t u(t, z) = \{ u, \mathcal{H} \} &=& 0+ \int_{\mathbb{R}}   \delta(x-z) E(t,x) \mathrm{d}x  + 0 = E(t,z)
\end{eqnarray*}
\noindent $\bullet$ Puis on considère $\{ E, \mathcal{H} \}$ 
\begin{eqnarray*}
\partial_t E(t, z) = \{ E, \mathcal{H} \} &=& 0 -  \int_{\mathbb{R}}  \delta(x-z) \rho_c u \mathrm{d}x  +  \int_{\mathbb{R}}\int_{\mathbb{R}}  \left(  \delta(x-z)  \partial_v f \frac{v^2}{2}  \right) \mathrm{d}{ v}\mathrm{d}x \nonumber\\
&=& - \rho_c u(t, z) - \int_{\mathbb{R}} f(t, z, v) v\mathrm{d}{ v} 
\end{eqnarray*}
\noindent $\bullet$  Finalement,  $\{ f, \mathcal{H} \}$ donne 
\begin{eqnarray*}
\partial_t f(t, z,w) = \{ f, \mathcal{H} \} &=&  \int_{\mathbb{R}}\int_{\mathbb{R}} f \left( \partial_x (\delta(x-z)\delta(w-v)) \partial_v \frac{v^2}{2}   \right)\mathrm{d}v \mathrm{d}x \nonumber\\
&& + 0  -  \int_{\mathbb{R}}\int_{\mathbb{R}}  \left( E  \partial_v f \right) \delta(x-z)  \delta(w-v)    \mathrm{d}{ v}\mathrm{d}x \nonumber\\
&=& (- v\partial_x f - E\partial_v f)(t, z, w). 
\end{eqnarray*}

Enfin, on peut vérifier  que le crochet de Poisson satisfait les propriétés suivantes 
\begin{itemize}
\item anti-symétrie: $\{ F, G \} = -\{ G, F \}$ 
\item bilinéarité: $\{ F + G, H \} = \{ F, H \}+\{ G, H \}$ 
\item identité de Jacobi : $\{\{ F, G \}, H\} + \{\{G,H\}, F \} + \{\{H, F\}, G \} = 0$.  
\end{itemize}
Les deux premières propriétés sont  simples alors que la dernière est habituellement 
plus compliquée. On utilise les calculs de \cite{Li:2020}, \cite{Morrison:2012}. 

%% section 3
% !TEX root = ../../main.tex

\section{Méthodes de résolution numérique en temps}
%%%%%%%%%%%%%%%%%%%%%%%%%%%%%%%%%%%%%%%%%%%%%%%%%%%%%%%%%%%%%%%%%%%%%%

Dans cette section nous allons présenter les principales méthodes numériques utilisées pour résoudre numériquement des équations dites cinétiques en temps, et plus spécifiquement le système~\eqref{eq:0:vmhl:1}-\eqref{eq:0:vmhl:4}. Une fois discrétisé en $(\vb{x},\vb{v})$, les différents systèmes que nous regardons peuvent se réduire au modèle abstrait suivant :
\begin{equation}
  \dot{u}(t) = L(t,u) + N(t,u),\quad u(t=0)=u_0
  \label{eq:0:dtu}
\end{equation}
d'inconnue $u\in\mathbb{R}^n$ et où $L$ et $N$ sont des fonctions $(t,u)\in\mathbb{R}_+\times\mathbb{R}^n\mapsto\mathbb{R}^n$, $n\in\mathbb{N}$ est le nombre de dimensions, ou d'inconnues du problème. C'est sur cette équation~\eqref{eq:0:dtu} que nous allons présenter les différentes méthodes d'intégration en temps utilisées ici.

% --------------------------------------------------------------------
\subsection{Méthode de \emph{splitting} hamiltonien}
% --------------------------------------------------------------------

Les méthodes de \emph{splitting} sont classiquement utilisées dans la résolution d'équations cinétiques (\cite{Morrison:2017,Grandgirard:2006,Tronci:2010,Tronci:2014}), elles consiste à diviser l'équation à résoudre en plusieurs parties. La construction de ces méthodes en temps se fait par concaténation des différentes étapes en formant des palindromes.

Une méthode de \emph{splitting} consiste à résoudre les deux équations suivantes successivement :
\begin{eqnarray}
    \dot{u} = L(t,u) \label{eq:0:split:1}\\
    \dot{u} = N(t,u) \label{eq:0:split:2}
\end{eqnarray}
La solution l'équation~\eqref{eq:0:dtu} au temps $t$ est $\varphi_t(u_0)$, et sera approchée par une composition de $\varphi_t^{[L]}(u_0)$ et $\varphi_t^{[N]}(u_0)$, respectivement solutions de~\eqref{eq:0:split:1} et~\eqref{eq:0:split:2}. Ainsi la méthode de Lie, \emph{splitting} d'ordre 1, consiste à approcher $\varphi_t(u_0)$ par $\varphi_t(u_0)\approx \varphi_t^{[L]} \circ \varphi_t^{[N]}(u_0)$. Si la résolution de chaque sous-système $\varphi_t^{[L]}$ et $\varphi_t^{[N]}$ est exacte, la seule erreur en temps provient du \emph{splitting}.

La résolution de chaque sous-système peut se faire sur des intervalles de temps différents (que nous noterons en indice), ainsi la méthode de Strang~\cite{Strang:1968}, \emph{splitting} d'ordre 2, s'écrit comme :
$$
  u(t) = S_{t}(u_0) = \varphi^{[L]}_{t/2}\circ\varphi^{[N]}_{t}\circ\varphi^{[L]}_{t/2}(u_0)
$$

Lorsque l'équation met en jeu plusieurs termes, comme c'est le cas pour le système~\eqref{eq:0:vmhl:1}-\eqref{eq:0:vmhl:4}, il est difficile de savoir comment choisir $L$ et $N$. L'hamiltonien du système permet de suggérer une décomposition intéressante, et de construire des méthodes appelées \emph{splitting} hamiltonien. \Josselin{voir dans \cite{Hairer:2006} s'il est nécessaire de résoudre exactement chaque sous-système pour avoir une splitting hamiltonien.}


\subsection{Méthode de type Runge-Kutta}
% --------------------------------------------------------------------

Les méthodes de type Runge-Kutta sont des méthodes d'approximation de solutions d'équations différentielles, développées dès 1901. Elles peuvent être vues comme une extension, à des ordres supérieurs, de la méthode d'Euler. Nous utiliserons ce type de méthodes pour résoudre la discrétisation en temps. Nous allons présenter ce type de méthode sur l'équation :
$$
  \dot{u} = N(t,u)
$$
où $u\in\mathbb{R}^n$, et $N:(t,u)\in\mathbb{R}_+\times\mathbb{R}^n\mapsto N(t,u)\in\mathbb{R}^n$ une fonction agissant sur $u$ et pouvant dépendre du temps $t$. Il s'agit d'un cas particulier de l'équation~\eqref{eq:0:dtu} où $L$ est la fonction nulle. Nous résumerons les méthodes par leur tableau de Butcher\cite{Butcher:2008}, qui se représentent sous la forme :
\begin{equation}  
  \begin{array}{c|c}
    \begin{matrix}
      c_1 \\
      \vdots \\
      c_s
    \end{matrix}
    &
    \begin{matrix}
      a_{11} & \cdots & a_{1s} \\
      \vdots & \ddots & \vdots \\
      a_{s1} & \cdots & a_{ss}
    \end{matrix} \\
    \hline
     & \begin{matrix} b_1 & \cdots & b_s \end{matrix} \\
  \end{array}
  \label{eq:0:butcher}
\end{equation}
et qui se lit :
$$
  \begin{aligned}
    u^{(i)} &= u^n + \Delta t \sum_{j=1}^s a_{ij} N(t^n+c_j\Delta t,u^{(j)}) \\
    u^{n+1} &= u^n + \Delta t \sum_{i=1}^s b_i N(t^n+c_i\Delta t, u^{(i)}).
  \end{aligned}
$$
où $u^n\approx u(t^n)$ avec $t^n=n\Delta t$, et où $\Delta t$ est le pas de temps.

Nous n’étudierons, pour des raisons de performances numériques, que des méthodes dites explicites, c'est-à-dire que chaque étage ne nécessite que les étages précédents pour être calculé. Dans ce cas, la matrice $(a_{ij})_{i,j}$ est triangulaire strictement inférieure. Dans le cadre de méthode explicite, il est possible de convertir la méthode, comme la méthode RK(3,3) de Shu-Osher, pour n'avoir qu'une seule évaluation de la fonction non linéaire $N$ par étage de la méthode.

Un intérêt des méthodes de type Runge-Kutta explicite est la montée en ordre. En effet celle-ci peut se faire de manière presque linéaire du nombre d'étages. À l'inverse, ces méthodes ne préserve pas l'énergie du système qu'elles résolvent, la montée en ordre est donc une nécessité pour réduire l'erreur et garantir la validité des résultats. Un autre inconvénient de ce type de résolution est l'introduction de condition de stabilité, que nous détaillerons un peu plus dans le cadre du chapitre~\ref{chap1}.

Nous bénéficions de la large littérature sur le sujet des méthodes de type Runge-Kutta, l'étude de stabilité ou de convergence (voir~\cite{Shu:2001,Butcher:2008,Gottlieb:2011,Baldauf:2008,Spiteri:2002}), ainsi que des améliorations dans des contextes spécifiques ; telles que les méthodes de Dormand-Prince permettant des stratégies de pas de temps adaptatifs (voir~\cite{Dormand:1978,Dormand:1980,,Gustafsson:1988,,Gustafsson:1994,Balac:2013,Balac:2014}), ou les méthodes de Lawson qui profite de la structure linéaire de l'équation (voir~\cite{Lawson:1967,Isherwood:2018,Hochbruck:2020}).

\subsubsection{Méthode de Lawson}
% --------------------------------------------------------------------

Les méthodes de Lawson sont une optimisation des méthodes de type Runge-Kutta à des équations ayant une partie linéaire que l'on écrit comme suit :
$$
  \dot{u}(t) = Lu(t) + N(t,u)
$$
il s'agit du cas particulier de l'équation~\eqref{eq:0:dtu} où $L$ est une matrice où un opérateur linéaire agissant sur $u$. Le principe de la méthode de Lawson est d'utiliser une formule de Duhamel sur $u$ pour résoudre exactement le terme linéaire. Ceci permet de se soustraire d'une condition de stabilité provenant du terme linéaire, et réduire l'erreur en résolvant exactement le plus de termes possibles.

Nous effectuons une formule de Duhamel en notant $v = e^{-tL}u$, ce qui nous permet de calculer :
$$
  \dot{v}(t) = -Le^{-tL}u(t) + e^{-tL}\dot{u}(t)
$$
d'où :
$$
  \dot{v}(t) = -Le^{-tL}u(t) + e^{-tL}Lu(t) + e^{-tL}N(t,u).
$$
On peut maintenant écrire l'équation sur $v$ que nous souhaitons résoudre avec une méthode de type Runge-Kutta :
$$
  \dot{v} = \tilde{N}(t,v)
$$
avec $\tilde{N}:(t,v)\in\mathbb{R}_+\times\mathbb{R}^n\mapsto e^{-tL}N(t,e^{tL}v)\in\mathbb{R}^n$. La méthode de Lawson consiste à réécrire la méthode Runge-Kutta sur $v$ en la variable $u$, où la partie linéaire est résolue exactement. La méthode de Lawson, induite par une méthode Runge-Kutta explicite décrit par le tableau de Butcher~\eqref{eq:0:butcher}, s'écrit alors :
$$
  \begin{aligned}
    u^{(i)} &= e^{c_i\Delta t L}u^n + \Delta t \sum_{j=1}^{i-1} a_{ij}e^{-(c_j-c_i)\Delta t L}N(t^n+c_j\Delta t,u^{(j)}) \\
    u^{n+1} &= e^{\Delta t L}u^n + \Delta t \sum_{i=1}^{s} b_i e^{(1-c_i)\Delta tL} N(t^n+c_i\Delta t,u^{(i)})
  \end{aligned}
$$

Comme pour une méthode Runge-Kutta classique, il est possible d'appliquer la même méthode d'optimisation de Shu-Osher pour n'avoir qu'une seule évaluation de la fonction non-linéaire $N$ par étage dans le cadre d'une méthode explicite.


\section{Méthodes de résolution numérique en espace}
%%%%%%%%%%%%%%%%%%%%%%%%%%%%%%%%%%%%%%%%%%%%%%%%%%%%%%%%%%%%%%%%%%%%%%

Nous présentons dans cette section les méthodes numériques permettant de discrétiser en espace ($\vb{x}$ ou $\vb{v}$) que nous allons utiliser pour résoudre numériquement le système~\eqref{eq:0:vmhl:1}-\eqref{eq:0:vmhl:4}.

% --------------------------------------------------------------------
\subsection{Méthode WENO}
% --------------------------------------------------------------------

La méthode WENO, pour \emph{Weighted Essentially Non-Oscillatory}, est une méthode volumes finis ou différences finies, dont l'écriture classique est d'ordre 5. Il s'agit d'une méthode \emph{upwind}, d'ordre élevé, combinée à des poids non-linéaires permettant de réduire les oscillations par de la baisse l'ordre et de la diffusion numérique. La méthode d'ordre 5 est présentée dans \cite{Liu:1994,Jiang:1996,Shu:1999,Shu:2003}. Nous la présentons ici pour une équation de transport de la forme :
$$
  \partial_t u + \partial_x f(u) = 0,\qquad u(t=0,x) = u_0(x)
$$
avec $u(t,x)$ la fonction inconnue dépendant du temps $t\geq 0$ et de l'espace $x\in\Omega$ (supposé ici périodique par commodité), et $f:u\mapsto f(u)$ une fonction agissant sur $u$. On définit une discrétisation de l'espace $x_i = i\Delta x + x_0$, $i=0,\dots,N_x$, avec $\Delta x>0$ le pas d'espace. La méthode WENO se présente comme suit :
$$
  \partial_t u_j(t) + \frac{1}{\Delta x}\left( \hat{f}_{j+\frac{1}{2}} - \hat{f}_{j-\frac{1}{2}} \right) = 0,
$$
où $u_j(t)\approx u(t,x_j)$, $j=0,\dots,N$, et où $\hat{f}_{j+\frac{1}{2}} = \hat{f}(u_{j-3},\dots,u_{j+3})$ est le flux numérique, ici présenté pour WENO5, avec $(u_{j-3},\dots,u_{j+3})$ le \emph{stencil} de la méthode, c'est-à-dire le voisinage de points nécessaire pour calculer une approximation de la dérivée en espace. Comme pour une méthode \emph{upwind}, il est nécessaire de distinguer le flux en eux parties, positive et négative :
$$
  f(u) = f^+(u) + f^-(u).
$$
Pour cela il est possible d'utiliser le flux de Lax-Friedrichs (voir~\cite{Shu:1997}). Dans les cas qui nous intéressent, $f:u\mapsto au$ est une fonction linéaire, il est donc simplement nécessaire de connaître le signe de la vitesse d'advection $a$, on note alors $a^+ = \max(a,0)$ et $a^-=\min(a,0)$ et on a $f^\pm_j=f^\pm(u_j)=a^\pm u_j$.

La méthode WENO5 consiste en 3 interpolations pondérées par des poids non-linéaires issus des approximations des dérivées successives de $f$. L'écriture des poids s'effectue comme suit dans le cas $f^-=0$ :
$$
  \begin{aligned}
    \beta_0 &= \frac{13}{12}( \underbrace{f^+_{j-2} - 2f^+_{j-1} + f^+_{j}  }_{\Delta x^2(f''_{j} + \mathcal{O}(\Delta x))}))^2   + \frac{1}{4}( \underbrace{  f^+_{j-2} - 4f^+_{j-1} + 3f^+_{j}  }_{ 2\Delta x ( f'_{j} + \mathcal{O}(\Delta x^2))})^2 \\
    \beta_1 &= \frac{13}{12}( \underbrace{f^+_{j-1} - 2f^+_{j}   + f^+_{j+1}}_{\Delta x^2(f''_{j} + \mathcal{O}(\Delta x^2))} )^2 + \frac{1}{4}( \underbrace{  f^+_{j-1} -               f^+_{j+1}}_{ 2\Delta x   f'_{j} + \mathcal{O}(\Delta x^2))})^2 \\
    \beta_2 &= \frac{13}{12}( \underbrace{f^+_{j}   - 2f^+_{j+1} + f^+_{j+2}}_{\Delta x^2(f''_{j} + \mathcal{O}(\Delta x))} )^2   + \frac{1}{4}( \underbrace{ 3f^+_{j}   - 4f^+_{j+1} +  f^+_{j+2}}_{-2\Delta x ( f'_{j} + \mathcal{O}(\Delta x^2))})^2 \\
  \end{aligned}
$$
où les coefficients $\beta_0$ sont appelés indicateurs de continuité (\emph{indicators of smoothness}). Ce qui nous permet de calculer les poids définis par :
$$
  \alpha_i = \frac{\gamma_i}{(\varepsilon + \beta_i)^2},\quad i=0,1,2
$$
où $\varepsilon$ est un paramètre numérique pour assurer la non nullité du dénominateur, il sera pris à $10^{-6}$ ; et avec $\gamma_0=\frac{1}{10}$, $\gamma_1=\frac{6}{10}$ et $\gamma_2=\frac{3}{10}$. La normalisation des poids s'effectue comme suit :
$$
  w_i = \frac{\alpha_i}{\sum_m \alpha_m},\quad i=0,1,2
$$
Nous pouvons ensuite calculer les flux numériques pour WENO5 \cite{Shu:2003}, donnés par :
$$
  \begin{aligned}
    \hat{f}_{j+\frac{1}{2}}^+   =\ & w_0\left(  \frac{2}{6}f^+_{j-2} - \frac{7}{6}f^+_{j-1} + \frac{11}{6}f^+_{j}   \right)
                                +    w_1\left( -\frac{1}{6}f^+_{j-1} + \frac{5}{6}f^+_{j}   +  \frac{2}{6}f^+_{j+1} \right) \\
                                +  & w_2\left(  \frac{2}{6}f^+_{j}   + \frac{5}{6}f^+_{j+1} -  \frac{1}{6}f^+_{j+2} \right),
  \end{aligned}
$$
La méthode WENO5 prend la forme finale :
$$
  \partial_xf(x_j) \approx \frac{1}{\Delta x}\left[ \left(\hat{f}_{j+\frac{1}{2}}^+ - \hat{f}_{j-\frac{1}{2}}^+ \right) + \left(\hat{f}_{j+\frac{1}{2}}^- - \hat{f}_{j-\frac{1}{2}}^- \right) \right].
$$

Il existe des variantes de la méthode WENO5, permettant de réduire la perte d'ordre à l'approche d'un choc, à savoir WENO-M (\cite{Henrick:2005}) ou WENO-Z (\cite{Borges:2008}). Ces variations se font sur le calcul des poids non-linéaires. Ainsi la méthode WENO-M utilise une fonction de \emph{mappage} pour équilibrer les poids et est définie par :
$$
  \begin{aligned}
    \alpha_i    &= \frac{\gamma_i}{(\epsilon + \beta_i)^2} \\
    \tilde{w}_i &= \frac{\alpha_i}{\sum_k \alpha_k} \\
    g_i         &= w_i\left( \frac{\gamma_i + \gamma_i^2 - 3w_i\gamma_i + w_i^2}{\gamma_i^2 + w_i(1-2\gamma_i)} \right) \\
    w_i         &= \frac{g_i}{\sum_k g_k}
  \end{aligned}
$$
avec le paramètre $\epsilon = 10^{-4}$. La méthode WENO-Z est quant à elle définit par :
$$
  \begin{aligned}
    \alpha_i &= \gamma_i\left( 1+ \frac{\tau_5}{\epsilon + \beta_i} \right) \\
    w_i      &= \frac{\alpha_i}{\sum_k \alpha_k}
  \end{aligned}
$$
avec les paramètres $\epsilon = 10^{-40}$ et $\tau_5 = \beta_0 - \beta_2$. Cette dernière méthode est celle qui réduit le plus la perte d'ordre à l'approche d'une discontinuité. L'étude approfondie de ces méthodes n'est pas envisagée dans ce travail car les solutions de la physique des plasmas ne présentent pas de discontinuités. Il est à noter que ces méthodes conservent la même linéarisation que le schéma WENO5 classique de Jiang et Shu~\cite{Jiang:1996}, ce qui permet d'y appliquer les résultats de stabilités obtenus dans le chapitre~\ref{chap1}.

Il est possible de monter en ordre en suivant les résultats dans~\cite{Wu:2021}, l'ordre 5 sera considéré comme suffisant dans la suite de ce travail.

L'étude de la stabilité de la méthode WENO5 couplée avec différentes méthodes de type Runge-Kutta pour la résolution en temps a été initiée dans~\cite{Wang:2007} où il a été démontré l'instabilité de la méthode couplée avec la méthode d'Euler explicite, il est nécessaire d'avoir au moins un étage supplémentaire permettant d'assurer la stabilité, ou d'utiliser une méthode d'ordre 3. Des estimations de stabilité et de conditions de stabilité ont par la suite été proposées dans \cite{Motamed:2010,Lunet:2017}. Une étude automatique de la stabilité est présentée dans~\cite{Crouseilles:2019b} qui constitue le chapitre~\ref{chap1} de ce document.

% --------------------------------------------------------------------
\subsection{Méthode semi-lagrangienne}
% --------------------------------------------------------------------

Une méthode très populaire pour la résolution numérique de l'équation de Vlasov, car les termes de transports sont linéaires, et que cette méthode d'introduit pas de contrainte de stabilité est la méthode semi-lagrangienne.

% --------------------------------------------------------------------
\subsection{Méthode pseudo-spectrale}
% --------------------------------------------------------------------

Une autre méthode souvent utilisée pour la résolution d'équation aux dérivées partielles linéaires est la méthode pseudo-spectrale qui consiste dans notre cas à effectuer une transformée de Fourier discrète et ainsi transformer une dérivée dans l'espace réel en un produit dans l'espace de Fourier.




%% section 4
% !TEX root = ../../main.tex

\section{Relations de dispersion}
\label{s:dispersion}

Cette section est dédiée à l'étude des relations de dispersion relatives aux modèles cinétique~\eqref{eq:vlasov}-\eqref{eq:poisson} et hybride linéarisé~\eqref{eq:vahl}. Il s'agit d'effectuer une linéarisation complète (c'est-à-dire aussi des particules chaudes) du modèle étudié puis d'exprimer le mode fondamental du champ électrique linéarisé. Cela permet d'obtenir une très bonne approximation de la phase linéaire de l'énergie électrique. Cette approche, complètement indépendante des schémas numériques utilisés pour résoudre le modèle de départ, sera utilisée comme outil de validation des codes présentés dans la section~\ref{s:scheme}.

Nous allons présenter les relations de dispersion de nos deux modèles, puis nous expliquerons comment reconstruire l'approximation linéaire de l'énergie électrique. Enfin, nous détaillerons les calculs des relations de dispersion pour le cas test qui nous intéressera dans les simulations numériques (sections~\ref{s:limit} et~\ref{s:compare}).

%----------
\subsection{Relations de dispersion dans le cas cinétique}
%----------

Nous nous intéressons d'abord aux relations de dispersion du modèle cinétique de Vlasov-Poisson~\eqref{eq:vlasov}-\eqref{eq:poisson}, en nous appuyant sur \cite{Sonnendrucker:2015}. Pour obtenir les relations de dispersion, il est nécessaire de linéariser le système autour d'un équilibre, pour cela rappelons les équations de Vlasov-Poisson~\eqref{eq:vlasov}-\eqref{eq:poisson} :
\begin{equation}
  \begin{cases}
    \partial_t f + v\partial_xf + E\partial_vf = 0 \\
    \partial_x E = \int_\mathbb{R} f\,\mathrm{d}v - 1 \\
    f(t=0, x, v)=f^0(x,v)
  \end{cases}
  \label{eq:vp}
\end{equation}
Dans un premier temps, nous nous intéressons à la linérarisation de ce modèle cinétique autour d'un état d'équilibre donné par $\left(f(t,x,v)\right)_{eq} = f^{(0)}(v)$ et $\left(E(t,x)\right)_{eq} = 0$, on considère le développement suivant :
\begin{equation}
  \begin{cases}
    f(t,x,v) = f^{(0)}(v) + \varepsilon f^{(1)}(t,x,v) + \mathcal{O}(\varepsilon^2) \\
    E(t,x) = 0 + \varepsilon E^{(1)}(t,x) + \mathcal{O}(\varepsilon^2)
  \end{cases}
  \label{eq:expansions}
\end{equation}
La densité de particules est définie par $\rho_0 = \rho_{0,c}+\rho_{0,h} = \int f^{(0)}\,\mathrm{d}v$. On injecte~\eqref{eq:expansions} dans~\eqref{eq:vp} pour obtenir :
$$
  \begin{cases}
    \varepsilon\partial_t f^{(1)} + v\varepsilon\partial_x f^{(1)} + \varepsilon E^{(1)}\left(\partial_v f^{(0)}+\varepsilon\partial_v f^{(1)}\right)=\mathcal{O}(\varepsilon^2) \\
    \varepsilon\partial_x E^{(1)} = \int f^{(0)} + \varepsilon\int f^{(1)} - 1 + \mathcal{O}(\varepsilon^2)
  \end{cases}
$$
ce qui nous permet d'obtenir, en négligeant les termes d'ordre $\varepsilon^2$, le système de Vlasov-Poisson linéarisé :
\begin{equation}
  \begin{cases}
    \partial_t f^{(1)} + v\partial_x f^{(1)} + E^{(1)}\partial_v f^{(0)} = 0 \\
    \partial_x E^{(1)} = \int f^{(1)}\,\mathrm{d}v
  \end{cases}
  \label{eq:systVPlin}
\end{equation}

Pour un état d'équilibre connu $f^{(0)}(v)$, habituellement une distribution gaussienne, les inconnues de~\eqref{eq:systVPlin} sont $f^{(1)}(t,x,v)$ et $E^{(1)}(t,x)$.

Nous souhaitons dériver l'expression générale de la relation de dispersion associée au modèle cinétique linéarisé~\eqref{eq:systVPlin}. Afin de simplifier la lecture, nous supprimons l'index $(1)$ sur nos inconnues $f^{(1)}$ et $E^{(1)}$. Nous supposons que le fonctions $f^{(1)}$ et $E^{(1)}$ sont $L$-périodiques en $x$ dans le domaine $\Omega=[0,L]$, nous allons, successivement, appliquer une transformée de Fourier en $x$ et et une transformée de Laplace en $t$ sur le système~\eqref{eq:systVPlin}.

Tout d'abord, nous effectuons une transformée de Fourier en $x$, définie pour une fonction $f(x)$ comme :
$$
  \hat{f}(k) = \frac{1}{L}\int_0^L f(x)e^{-ikx}\,\mathrm{d}x\,,\quad k=\frac{2\pi}{L}n, n\in\mathbb{Z}
$$
Nous obtenons :
\begin{equation}
  \begin{cases}
    \partial_t \hat{f} + ikv\hat{f} + \hat{E}\partial_v f^{(0)} = 0 \\
    ik\hat{E} = \int \hat{f}(t,k,v) dv
  \end{cases}
  \label{eq:fourier}
\end{equation}
Maintenant, nous utilisons la transformée de Laplace définie pour une fonction $f(t)$ par :
$$
  \tilde{f}(\omega) = \int_0^{+\infty} f(t)e^{i\omega t}\,\mathrm{d}t
$$
et, si elle est définie, la transformée de Laplace inverse est donnée par :
$$
  f(t) = \frac{1}{2i\pi}\int_{u-i\infty}^{u+i\infty} \tilde{f}(\omega)e^{-i\omega t}\,\mathrm{d}\omega
$$
Appliquons la transformée de Laplace à la première équation du système~\eqref{eq:fourier} :
$$
  \int_0^{+\infty}\partial_t\hat{f}(t)e^{i\omega t}\,\mathrm{d}t
  + \int_0^{+\infty}ikv\hat{f}(t)e^{i\omega t}\,\mathrm{d}t
  + \int_0^{+\infty}\hat{E}(t)\partial_v f^{(0)}e^{i\omega t}\,\mathrm{d}t
  = 0
$$
et en utilisant une intégration par partie dans la première intégrale nous obtenons :
$$
  \begin{aligned}
    -\hat{f}(t=0,k,v) - i\omega\int_0^{+\infty} \hat{f}(t)e^{i\omega t}\,\mathrm{d}t + ikv\int_0^{+\infty} \hat{f}(t)e^{i\omega t} \mathrm{d}t \\
    +\partial_vf^0\int_0^{+\infty}\hat{E}(t)e^{i\omega t}\,\mathrm{d}t=0
  \end{aligned}
$$
et donc :
\begin{equation}
  (ikv-i\omega)\tilde{\hat{f}}(\omega,k,v) + \partial_vf^0\tilde{\hat{E}}(\omega,k) = \hat{f}_0(k,v),
  \label{eq:fourierlaplace_f}
\end{equation}
où $\hat{f}_0(k,v) = \hat{f}(t=0,k,v)$ correspond à la condition initiale. En appliquant maintenant la transformée de Laplace à la seconde équation de~\eqref{eq:fourier} nous obtenons :
$$
  \int_0^{+\infty}ik\hat{E}(t,k)e^{i\omega t}\,\mathrm{d}t = \int_0^{+\infty}\int_{-\infty}^{+\infty}\hat{f}(t,k,v)\,\mathrm{d}v\,e^{i\omega t}\,\mathrm{d}t
$$
ce qui nous donne :
\begin{equation}
  \tilde{\hat{E}}(\omega,k)=-\frac{i}{k}\int_{-\infty}^{+\infty}\tilde{\hat{f}}(\omega,k,v)dv
  \label{eq:fourierlaplace_E}
\end{equation}
Maintenant, nous souhaitons injecter l'équation~\eqref{eq:fourierlaplace_f} dans~\eqref{eq:fourierlaplace_E}. Nous devons prêter attention aux pôles $\omega = kv$. En fait, si $\Im(\omega)>0$ et pour une fonction analytique $g(v)$, alors l'intégrale $\int_{-\infty}^{+\infty}\frac{g(v)}{ikv-i\omega}\,\mathrm{d}v$ est analytique. Lorsque $\Im(\omega) \leq 0$, nous devons construire un prolongement analytique et remplacer l'intégrale par $\int_\gamma \frac{g(v)}{ikv-i\omega}\,\mathrm{d}v$ avec $\gamma$ un contour ouvert parallèle à l'axe réel à l'infini et qui passe en-dessous du pôle $\omega = kv$ (voir \cite{Sonnendrucker:2015}). Par la suite, nous utiliserons la notation $\gamma$ soit pour l'axe réel $]-\infty,+\infty[$ quand $\Im(\omega)>0$, ou pour un chemin ouvert bien choisi lorsque $\Im(\omega)\leq 0$. 

Avec cette notation, le résultat de l'injection de~\eqref{eq:fourierlaplace_f} dans~\eqref{eq:fourierlaplace_E} nous donne :
$$
  \begin{aligned}
    \tilde{\hat{E}}
    & = -\frac{i}{k}\int_\gamma \frac{1}{ikv-i\omega}\left( \hat{f}_0(k,v) - \partial_vf^{(0)}\tilde{\hat{E}}(\omega,k) \right)\,\mathrm{d}v \\
    & = -\frac{1}{k}\int_\gamma \frac{\hat{f}_0(k,v)}{kv-\omega}\,\mathrm{d}v + \frac{1}{k}\int_\gamma \frac{\partial_v f^{(0)}\tilde{\hat{E}}(\omega,k)}{kv-\omega}\,\mathrm{d}v \\
    & = - \frac{1}{k^2}\int_\gamma \frac{\hat{f}_0(k,v)}{v-\frac{\omega}{k}}\,\mathrm{d}v + \frac{1}{k^2}\tilde{\hat{E}}(\omega,k)\int_\gamma \frac{\partial_v f^{(0)}}{v-\frac{\omega}{k}}\,\mathrm{d}v
  \end{aligned}
$$
donc :
$$
  \left( 1 - \frac{1}{k^2}\int_\gamma \frac{\partial_v f^{(0)}}{v-\frac{\omega}{k}}\,\mathrm{d}v \right) \tilde{\hat{E}}(\omega,k) = -\frac{1}{k^2}\int_\gamma \frac{\hat{f}_0(k,v)}{v-\frac{\omega}{k}}\,\mathrm{d}v
$$
En introduisant :
\begin{equation}
  D(k,\omega) = 1 - \frac{1}{k^2}\int_\gamma \frac{\partial_v f^{(0)}}{v-\frac{\omega}{k}}\,\mathrm{d}v
  \label{eq:D}
\end{equation}
et
\begin{equation}
  N(k,\omega) = -\frac{1}{k^2}\int_\gamma\frac{\hat{f}_0(k,v)}{v-\frac{\omega}{k}}dv
  \label{eq:N}
\end{equation}
nous pouvons définir $\tilde{\hat{E}}(\omega,k)$ comme :
$$
  \tilde{\hat{E}}(\omega,k) = \frac{N(k,\omega)}{D(k,\omega)}
$$
L'équation~\eqref{eq:D} est appelée relation de dispersion du modèle cinétique.

%----------
\subsection{Relations de dispersion dans le cas hybride}
%----------

Intéressons-nous dans cette section à dériver l'expression générale de la relation de dispersion associée au modèle hybride linéarisé~\eqref{eq:vahl}. Pour cela, nous allons repartir du modèle hybride non-linéaire~\eqref{eq:hyb_nonlin} et linéariser à la fois les équations fluides et l'équation cinétique. On injecte le développement~\eqref{eq:lin_variables} dans~\eqref{eq:hyb_nonlin}. Les mêmes calculs que dans la section~\ref{s:models} et l'approximation en $\varepsilon^2$ y compris dans l'équation cinétique sur $f_h$ conduisent au modèle
$$
  \begin{cases}
    \partial_tu_c^{(1)}=E^{(1)}\\
    \partial_tE^{(1)}=-\rho_c^{(0)}u_c^{(1)}-\int v f_h^{(1)}\,\mathrm{d}v\\
    \partial_t f_h^{(1)}+ v\partial_xf_h^{(1)}+ E^{(1)}\partial_vf_h^{(0)}=0
  \end{cases}
$$ 
d'inconnues $E^{(1)}$, $u_c^{(1)}$ et $f_h^{(1)}$ que nous noterons dans la suite, respectivement, $E$, $u_c$ et $f_h$, solutions du système hybride linéarisé dans toutes les inconnues
\begin{equation}
  \begin{cases}
    \partial_t u_c = E \\
    \partial_t E = -\rho_c^{(0)}u_c - \int vf_h\,\mathrm{d}v \\
    \partial_t f_h + v\partial_x f_h + E\partial_v f_h^{(0)} = 0
  \end{cases}
\label{eq:systHlin}
\end{equation}
Nous insistons sur le fait que le modèle~\eqref{eq:systHlin} correspond à une linéarisation de la partie cinétique (ou chaude) du modèle hybride~\eqref{eq:vahl} que nous avons étudié précédemment. Par la suite, nous supposerons que la densité de particules froides $\rho_c^{(0)}$ est une constante (en temps $t$ et espace $x$) et que la fonction $f_h^{(0)}$ est une fonction paire en $v$ et ne dépend que de cette variable. Nous supposons que les fonctions $f_h$, $E$ et $u_c$ sont $L$-périodiques en $x$ sur le domaine spatial $\Omega = [0,L]$ et nous appliquons la transformée de Fourier en $x$ puis une transformée de Laplace en $t$.

Tout d'abord, nous appliquons la transformée de Fourier en $x$ :
\begin{equation}
  \begin{cases}
    \partial_t \hat{u}_c = \hat{E} \\
    \partial_t\hat{E}=-\rho_c^{(0)}\hat{u}_c-\int \hat{f}_h dv \\
    \partial_t \hat{f}_h+ikv\hat{f}_h+ \hat{E}\partial_v f_h^{(0)} = 0
  \end{cases}
  \label{eq:fourierH}
\end{equation}
Alors, nous multiplions par $e^{i\omega t}$ et nous intégrons en temps. L'équation sur $\hat{u}_c$ nous donne :
\begin{eqnarray}
  \int_0^{+\infty} \partial_t \hat{u}_c e^{i\omega t} \,\mathrm{d}t &=& \int_0^{+\infty} \hat{E}e^{i\omega t}\,\mathrm{d}t\nonumber\\
  -\int_0^{+\infty}i\omega  \hat{u}_c e^{i\omega t} \,\mathrm{d}t-\hat{u}_c(t=0,k) &=& \int_0^{+\infty} \hat{E}e^{i\omega t}\,\mathrm{d}t\nonumber\\
  -i\omega  \tilde{\hat{u}}_c(\omega,k)-\hat{u}_c(t=0,k) &=& \tilde{\hat{E}}(\omega,k)\nonumber\\
  \tilde{\hat{u}}_c(\omega,k)+\frac{1}{i\omega}\tilde{\hat{E}}(\omega,k) &=& -\frac{1}{i\omega}\hat{u}_c(t=0,k)
  \label{eq:LF_u}
\end{eqnarray}
Les mêmes opérations sur l'équation sur $\hat{E}$ nous donnent :
\begin{eqnarray}
  -i\omega  \tilde{\hat{E}}(\omega,k)-\hat{E}(t=0,k)&=&-\rho_c^{(0)}\tilde{\hat{u}}_c(\omega,k)-\int_{-\infty}^{+\infty} v\tilde{\hat{f}}_h(\omega,k)dv
  \label{eq:LF_E}
\end{eqnarray}
Tandis que l'équation sur $\hat{f}_h$ nous donne :
\begin{eqnarray}
  -i\omega  \tilde{\hat{f}}_h(\omega,k,v)-\hat{f}_h(t=0,k,v)+ikv\tilde{\hat{f}}_h(\omega,k,v)+\tilde{\hat{E}}(\omega,k)\partial_vf_h^{(0)}(v)=0\nonumber\\
  \tilde{\hat{f}}_h(\omega,k,v)\left(ikv-i\omega\right)=\hat{f}_h(t=0,k,v)-\tilde{\hat{E}}(\omega,k)\partial_vf_h^{(0)}(v)\nonumber\\
  \tilde{\hat{f}}_h(\omega,k,v)=-\frac{i}{k}\frac{\hat{f}_h(t=0,k,v)}{v-\frac{\omega}{k}}+\frac{i}{k}\frac{\tilde{\hat{E}}(\omega,k)\partial_vf_h^{(0)}(v)}{v-\frac{\omega}{k}}.
  \label{eq:LF_f}
\end{eqnarray}
Nous injectons l'expression~\eqref{eq:LF_f} dans~\eqref{eq:LF_E} :
$$
  \begin{aligned}
    -i\omega  \tilde{\hat{E}}(\omega,k)-\hat{E}(t=0,k)=-\rho_c^{(0)}\tilde{\hat{u}}(\omega,k)+\frac{i}{k}\int_\gamma v\frac{\hat{f}_h(t=0,k,v)}{v-\frac{\omega}{k}}dv \\
    -\frac{i}{k}\int_\gamma v\frac{\tilde{\hat{E}}(\omega,k)\partial_vf_h^{(0)}(v)}{v-\frac{\omega}{k}}dv
  \end{aligned}
$$
soit :
\begin{equation}
  \begin{aligned}
    \tilde{\hat{E}}(\omega,k)\left(1-\frac{1}{\omega k}\int_\gamma v\frac{\partial_vf_h^{(0)}(v)}{v-\frac{\omega}{k}}dv\right)-\frac{\rho_c^{(0)}}{i\omega}\tilde{\hat{u}}_c(\omega,k)=-\frac{1}{i\omega}\hat{E}(t=0,k)\nonumber\\
    -\frac{1}{\omega k}\int_\gamma v\frac{\hat{f}_h(t=0,k,v)}{v-\frac{\omega}{k}}dv
  \end{aligned}
  \label{eq:LF_Ef}
\end{equation}
Nous injectons maintenant l'expression~\eqref{eq:LF_u} dans~\eqref{eq:LF_Ef} pour obtenir le problème suivant :
$$
  \begin{aligned}
    \tilde{\hat{E}}(\omega,k)\left(1-\frac{1}{\omega k}\int_\gamma v\frac{\partial_vf_h^{(0)}(v)}{v-\frac{\omega}{k}}dv\right)+\frac{\rho_c^{(0)}}{i\omega}\left(\frac{1}{i\omega}\tilde{\hat{E}}(\omega,k)+\frac{1}{i\omega}\hat{u}_c(t=0,k)\right)\\
    =-\frac{1}{i\omega}\hat{E}(t=0,k)
    -\frac{1}{\omega k}\int_\gamma v\frac{\hat{f}_h(t=0,k,v)}{v-\frac{\omega}{k}}dv
  \end{aligned}
$$
soit :
$$
  \begin{aligned}
    \tilde{\hat{E}}(\omega,k)\left(1-\frac{1}{k^2}\left(\rho_c\frac{k^2}{\omega^2}+\int_\gamma \frac{\partial_vf_h^{(0)}(v)}{v-\frac{\omega}{k}}dv\right)\right)~~~~~~~~~~~~~~~~~~~~~~~~~~~~~~~~~~~~~~~~\\
    =\frac{\rho_c^{(0)}}{\omega^2}\hat{u}_c(t=0,k)-\frac{1}{i\omega}\hat{E}(t=0,k)
    -\frac{1}{\omega k}\int_\gamma v\frac{\hat{f}_h(t=0,k,v)}{v-\frac{\omega}{k}}dv
  \end{aligned}
$$
Nous introduisons :
\begin{equation}
  D(k,\omega) = 1-\frac{1}{k^2}\left( \rho_c^{(0)}\frac{k^2}{\omega^2}+\int_\gamma \frac{\partial_vf_h^{(0)}(v)}{v-\frac{\omega}{k}}dv\right )
  \label{eq:relD_H}
\end{equation}
et
\begin{equation}
  N(k,\omega) = \frac{\rho_c^{(0)}}{\omega^2}\hat{u}_c(t=0,k)-\frac{1}{i\omega}\hat{E}(t=0,k) 
    -\frac{1}{\omega k}\int_\gamma v\frac{\hat{f}_h(t=0,k,v)}{v-\frac{\omega}{k}}dv,
  \label{eq:relN_H}
\end{equation}
nous pouvons alors définir $\tilde{\hat{E}}(\omega,k)$ comme :
$$
  \tilde{\hat{E}}(\omega,k)=\frac{N(k,\omega)}{D(k,\omega)}
$$

\begin{remark}
  Comme nous le verrons dans la sous-section suivante, pour retrouver la pente de la partie linéaire de l'énergie électrique, il est suffisant de trouver les racines de $D(k,\omega)$, ou les pôles de $\tilde{\hat{E}}(\omega,k)$. Si seule la pente de la partie linéaire nous intéresse, un autre moyen de la retrouver est de réécrire les équations~\eqref{eq:LF_u}-\eqref{eq:LF_Ef} comme le système suivant :
  \begin{equation}
    \begin{pmatrix}
      1                       & \frac{1}{i\omega} \\
      -\frac{\rho_c}{i\omega} & 1-\frac{1}{\omega k}\int_\gamma v\frac{\partial_v f_h^{(0)}(v)}{v-\frac{\omega}{k}}\,\mathrm{d}v
    \end{pmatrix}
    \begin{pmatrix}
      \tilde{\hat{u}}_c(\omega,k) \\
      \tilde{\hat{E}}(\omega,k)
    \end{pmatrix}
    =
    \begin{pmatrix}
      -\frac{1}{i\omega}\hat{u}_c(0,k) \\
      -\frac{1}{i\omega}\hat{E}(0,k) - \frac{1}{\omega k}\int_\gamma v\frac{\hat{f}_h(0,k,v)}{v-\frac{\omega}{k}}\,\mathrm{d}v
    \end{pmatrix}
    \label{eq:systHyb}
  \end{equation}
  Le problème revient alors à trouver les racines du déterminant de ce système, qui s'écrit
%  Nous sommes alors incités à calculer le déterminant de ce système pour se faire une idée des solutions de ce système :
  $$
    \begin{aligned}
      Det(k,\omega) & = 1 - \frac{1}{\omega k}\int_\gamma v\frac{\partial_v f_h^{(0)}(v)}{v-\frac{\omega}{k}}\,\mathrm{d}v - \frac{\rho_c^{(0)}}{\omega^2} \\
                    & = 1 - \frac{1}{k^2}\left( \rho_c^{(0)}\frac{k^2}{\omega^2} + \int_\gamma \frac{\partial_v f_h^{(0)}(v)}{v-\frac{\omega}{k}}\,\mathrm{d}v \right)
    \end{aligned}
  $$
  On retrouve bien~\eqref{eq:relD_H}. La connaissance de~\eqref{eq:relN_H} nous donnera, en plus de la pente, la phase de l'énergie électrique dans sa partie linéaire.
\end{remark}

%----------
\subsection{Expression du champ électrique linéarisé}
%----------

Dans cette sous-section, nous considérons un prolongement analytique continu de $N(k,\omega)$ et $D(k,\omega)$, et nous supposons que les transformées de Laplace et de Fourier de $\tilde{\hat{E}}$ sont bien définies pour obtenir une approximation du champ électrique linéarisé.

La transformée de Laplace inverse peut être calculée à l'aide du théorème des résidus :
$$
  \hat{E}(t,k)=\frac{1}{2i\pi}\int_{u-i\infty}^{u+i\infty}\tilde{\hat{E}}(\omega,k)e^{-i\omega t}d\omega=\sum_jRes_{\omega=\omega^{k,j}}\left(\tilde{\hat{E}}(\omega,k)e^{-i\omega t}\right)
$$
où $\omega^{k,j}$ sont les pôles de $\tilde{\hat{E}}(\omega,k)$. Nous rappelons que si $\omega^{k,j}$ est un pôle simple, alors :
$$
  \begin{aligned}
    Res_{w=w^{k,j}}\left( \tilde{\hat{E}}(\omega,k)e^{-i\omega^{k,j}t} \right)
      & = \lim_{\omega\to\omega^{k,j}}\left( \omega - \omega^{k,j} \right)\tilde{\hat{E}}(\omega,k)e^{-i\omega t} \\
      & = \lim_{\omega\to\omega^{k,j}}\left( \omega - \omega^{k,j} \right)\frac{N(k,\omega)}{D(k,\omega)}e^{-i\omega t}
  \end{aligned}
$$
Maintenant, un développement de Taylor de $D(k,\omega)$ nous donne :
$$
  D(k,\omega) = \underbrace{D(k,\omega^{k,j})}_{0} + \left( \omega - \omega^{k,j} \right)\frac{\partial D}{\partial \omega}(k,\omega^{k,j}) + \mathcal{O}\left( (\omega-\omega^{k,j})^2 \right)
$$
donc, le passage à la limite nous donne :
\begin{equation}
  Res_{\omega=\omega^{k,j}}\left(\tilde{\hat{E}}(\omega,k)e^{-i\omega^{k,j}t}\right)=\frac{N(k,\omega^{k,j})}{\frac{\partial D}{\partial \omega}(k,\omega^{k,j})}e^{-i\omega^{k,j} t}.
  \label{eq:residu}
\end{equation}

\begin{remark}
  En fait, pour un $k$ fixé, on obtient une très bonne approximation de $\hat{E}(t,k)$ (excepté pour des temps courts) en considérant seulement la fréquence principale. Soient les deux racines $\omega^{k,j_0\pm}=\pm\omega_r+i\omega_i$ de $D(k,\omega)$ (où $\omega_r\in\mathbb{R}^+$, $\omega_i\in\mathbb{R}$) qui ont la plus grande partie imaginaire $\omega_i$ : pour toute autre racine $\omega^{k,j}$, on a $\Im(\omega^{k,j})<\omega_i$. Les autres pôles peuvent être négligés. En effet, nous avons :
  $$
    \begin{aligned}
       \hat{E}(t,k)&=&\sum_jC_je^{-i\omega^{k,j} t}=C_{j_0^+}e^{-i\omega^{k,j_0^+}t}+C_{j_0^-}e^{-i\omega^{k,j_0^-}t}+\sum_{j\neq j_0^\pm}C_je^{-i\omega^{k,j} t}\\
 &=&e^{\omega_it}\left(C_{j_0^+}e^{-i\omega_rt}+C_{j_0^-}e^{i\omega_rt}+\sum_{j\neq j_0^\pm}C_je^{-i\Re(\omega^{k,j}) t}e^{(\Im(\omega^{k,j})-\omega_i)t}\right)
    \end{aligned}
  $$
  et par hypothèse, $\Im(\omega^{k,j})-\omega_i < 0$ $\forall j\neq j_0^\pm$, nous pouvons conclure que la somme tend vers zéro lorsque $t\to+\infty$.
\end{remark}

\begin{lemma}
  Si $f^{(0)}(v)$ (respectivement $f_h^{(0)}(v)$) est une fonction paire, alors pour $D$ défini par~\eqref{eq:D} (respectivement~\eqref{eq:relD_H}) nous avons $D(k,\omega_r+i\omega_i) = 0 \Leftrightarrow D(k,-\omega_r+i\omega_i)=0$.
  \label{lemma:doubleracine}
\end{lemma}
\begin{proof}
  Voir en annexe~\ref{a:dispersion}.
\end{proof}

En considérant seulement les deux racines principales $\pm\omega_r + i\omega_i$ de $D(k,\omega)$, supposés pôles simples de $\tilde{\hat{E}}(\omega,k)$, nous avons l'approximation :
$$
  \hat{E}(t,k)\approx Res_{\omega=\omega_r+i\omega_i}\left(\tilde{\hat{E}}(\omega,k)e^{-i\omega t}\right)+Res_{\omega=-\omega_r+i\omega_i}\left(\tilde{\hat{E}}(\omega,k)e^{-i\omega t}\right)
$$
où les résidus sont définis par~\eqref{eq:residu}. Notons $r^\pm$ le module de $\frac{N(k,\pm\omega_r+i\omega_i)}{\frac{\partial D}{\partial \omega}(k,\pm\omega_r+i\omega_i)}$ et $\phi^\pm$ son argument, nous avons alors :
\begin{equation}
  \hat{E}(t,k)\approx r^+e^{i\phi^+}e^{-i(\omega_r+i\omega_i)t}+r^-e^{i\phi^-}e^{-i(-\omega_r+i\omega_i)t}. 
  \label{eq:Etk_sanssym}
\end{equation}
Dans plusieurs cas test classiques, nous avons une symétrie entre les racines, qui dépend de la perturbation initiale de l'équilibre. Par la suite la perturbation initiale de l'équilibre sera toujours une fonction cosinus.

\begin{hyp}
  Le module et l'argument de $\frac{N(k,\pm\omega_r+i\omega_i)}{\frac{\partial D}{\partial \omega}(k,\pm\omega_r+i\omega_i)}$  vérifient $r^+ = r^-$ (noté $r$ par la suite) et $\phi^+ = -\phi^-$ (noté simplement $\phi$).
  \label{hyp:sym}
\end{hyp}
Sous l'hypothèse~\ref{hyp:sym}, nous obtenons :
\begin{eqnarray}
  \hat{E}(t,k)\approx&& re^{i\phi}e^{-i(\omega_r+i\omega_i)t}+re^{-i\phi}e^{-i(-\omega_r+i\omega_i)t} \nonumber\\
  &=&re^{\omega_i t}\left(e^{i(\omega_r t-\phi)}+e^{-i(\omega_r t-\phi)}\right)\nonumber\\
  &=&2re^{\omega_i t}\cos\left(\omega_r t-\phi\right).
  \label{eq:Etk}
\end{eqnarray}
Maintenant, si nous considérons la définition des coefficients de Fourier, nous avons :
$$
  \hat{E}(t,k) = \frac{1}{L}\int_0^L E(t,x)e^{-ikx}\,\mathrm{d}x = \frac{1}{L}\int_0^L E(t,x)\cos(-kx)\,\mathrm{d}x + i\frac{1}{L}\int_0^L E(t,x)\sin(-kx)\,\mathrm{d}x
$$
et :
$$
  \begin{aligned}
    \hat{E}(t,-k) 
      & = \frac{1}{L}\int_0^L E(t,x)e^{ikx}\,\mathrm{d}x = \frac{1}{L}\int_0^L\cos(kx)\,\mathrm{d}x + i\frac{1}{L}\int_0^L E(t,x)\sin(kx)\,\mathrm{d}x \\
      & = \frac{1}{L}\int_0^L E(t,x)e^{ikx}\,\mathrm{d}x = \frac{1}{L}\int_0^L\cos(-kx)\,\mathrm{d}x - i\frac{1}{L}\int_0^L E(t,x)\sin(-kx)\,\mathrm{d}x \\
      & = \overline{\hat{E}(t,k)}
  \end{aligned}
$$

\begin{hyp}
  $N(k,\omega) = 0$ si $k\notin\left\{\pm\frac{2\pi}{L}\right\}$.
  \label{hyp:knuls}
\end{hyp}
Sous l'hypothèse~\ref{hyp:knuls}, avec l'approximation des coefficients de Fourier (qui sont tous réels) données par~\eqref{eq:Etk} et avec $l=\frac{2\pi}{L}$, nous obtenons l'approximation du champ électrique suivante :
$$
  \begin{aligned}
    E(t,x) &\approx \varepsilon E^{(1)}(t,x) \approx \varepsilon\left( \hat{E}(t,l)e^{ikx} + \overline{\hat{E}(t,l)}e^{-ilx} \right) \\
           &\approx 2\varepsilon \hat{E}(t,l)\cos(lx) \\
           &\approx 4\varepsilon r e^{\omega_i t}\cos(\omega_t t -\phi)\cos\left(\frac{2\pi}{L}x\right)
  \end{aligned}
$$
Ce qui nous permet d'obtenir une approximation de l'énergie électrique, définie par :
\begin{equation}
  \begin{aligned}
    \mathcal{E}(t) & := \left( \int_0^L E^2(t,x)\,\mathrm{d}x \right)^{\frac{1}{2}} \\
                   & \approx 4\varepsilon r e^{\omega_i t}\left|\cos(\omega_rt - \phi)\right|\left( \int_0^L \cos^2\left(\frac{2\pi}{L}x\right)\,\mathrm{d}x \right)^{\frac{1}{2}} \\
                   & \approx 2\sqrt{2L}\varepsilon r e^{\omega_i t}\left|\cos(\omega_rt - \phi)\right|
  \end{aligned}
\label{eq:enelec}\end{equation}
puisque :
$$
  \begin{aligned}
    \int_0^L \cos^2\left(\frac{2\pi}{L}x\right)\,\mathrm{d}x
      & = \int_0^L\frac{1}{2}\,\mathrm{d}x + \int_0^L\frac{1}{2}\cos\left(\frac{4\pi}{L}x\right)\,\mathrm{d}x \\
      & = \frac{L}{2} + \left[ \frac{L}{8\pi}\sin\left(\frac{4\pi}{L}x\right)\right]_0^L = \frac{L}{2}
  \end{aligned}
$$

\begin{remark}
  Il est possible de mener une étude similaire pour une perturbation donnée par une fonction sinus. Nous obtenons alors des résultats similaires en remplaçant dans l'approximation de $\hat{E}(t,k)$, $E(t,x)$ et $\mathcal{E}(t)$ les fonctions cosinus par des fonctions sinus.
\end{remark}

Il est à noter que ces approximations ne prennent en compte que les racines dominantes de $D(\frac{2\pi}{L},\omega)$, les deux ayant la plus grande partie imaginaire. Cette approximation devient valable en temps $t$ suffisamment long.

La partie imaginaire $\omega_i$ nous donne le comportement global des coefficients de Fourier du champ électrique, et donc de l'énergie électrique comme une fonction du temps. Nous obtenons un amortissement de l'énergie électrique si $\omega_i < 0$, ou une instabilité si $\omega_i >0$. Lorsque nous traçons l'énergie électrique en fonction du temps en échelle logarithmique, nous pouvons observer les comportements suivants :
\begin{itemize}
  \item un amortissement avec un taux $\omega_i<0$, le taux indiquant la pente globale de l'amortissement,
  \item quelques oscillations stables, suivies du développement d'une instabilité avec un taux $\omega_i>0$, jusqu'à la saturation recherchée.
\end{itemize}

%----------
\subsection{Applications}\label{ssec:disp_appl}
%----------

Dans cette sous-section, nous nous intéressons au calcul de $D(k,\omega)$ pour le modèle cinétique~\eqref{eq:D} ou hybride~\eqref{eq:relD_H}, dans le cadre des cas tests qui nous intéressent. Pour le modèle cinétique la distribution initiale est donnée par :
$$
  f_0(x,v) = \mathcal{M}_{1-\alpha,0,T_c}(v)
    + \left(
      \mathcal{M}_{^\alpha/_2,v_0,1}(v) + \mathcal{M}_{^\alpha/_2,-v_0,1}(v)
    \right)(1+\epsilon\cos(kx))
$$
avec $\alpha$ la densité de particules chaudes, centrées en $\pm v_0\in\mathbb{R}$, et les particules froides sont caractérisées par une température $T_c$, et où l'on note :
$$
  \mathcal{M}_{\rho,u,T}(v) := \frac{1}{(2\pi T)^\frac{1}{2}}\exp\left(-\frac{|v-u|^2}{2T}\right)
$$
la distribution maxwellienne de densité $\rho$, centrée en la vitesse $u$ et de température $T$. Cette distribution initiale $f_0$ nous permet de construire une condition initiale compatible pour le modèle hybride, donnée par la limite $T_c\to 0$ :
\begin{equation}
  \begin{aligned}
    f_{h,0} (x,v) & = \left(
      \mathcal{M}_{^\alpha/_2,v_0,1}(v) + \mathcal{M}_{^\alpha/_2,-v_0,1}(v)
    \right)(1+\epsilon\cos(kx)) \\
    u_{c,0} & = 0
  \end{aligned}
\label{eq:f0hdexv}\end{equation}
le champ électrique à l'instant initial $E_0$ est donné par la résolution de l'équation de Poisson avec la condition initiale :
$$
  \partial_x E_0(x) = (1-\alpha) + \int_\mathbb{R}f_{h,0}(x,v)\,\mathrm{d}v - 1
$$

Nous cherchons ensuite les racines en $\omega$ de la fonction $D(k,\omega)$ pour $k$ fixé. Celles-ci sont approchées numériquement à l'aide d'une méthode de Newton, la dérivée $\frac{\partial D}{\partial\omega}(k,\omega)$ est alors nécessaire. La racine ayant la plus grande partie imaginaire, dans la pratique nous ne conservons que celle avec une partie réelle positive, nous donne des informations sur l'évolution de l'énergie électrique au cours du temps (taux d'amortissement et taux d'instabilité en échelle logarithmique). De plus, le calcul de $N(k,\omega)$ nous permet d'obtenir plus d'informations sur le mode dominant $\hat{E}(t,k)$ donné par~\eqref{eq:Etk_sanssym} dans le cas général, ou par~\eqref{eq:Etk} sous l'hypothèse~\ref{hyp:sym}. Nous en déduisons notamment la phase des oscillations de l'énergie électrique dans sa partie linéaire.


%-------------
\subsubsection{Quelques propriétés de la fonction de dispersion du plasma}

Dans le calcul de $D(k,\omega)$ et $N(k,\omega)$ apparaît la fonction de dispersion du plasma, aussi appelée fonction de Fried-Conte~\cite{Fried:1961} :
\begin{equation}
  Z(\xi):=\frac{1}{\sqrt{\pi}} \int_\gamma \frac{e^{-z^2}}{z-\xi}\,\mathrm{d}z
  \label{eq:Zfct}
\end{equation}
On rappelle, à l'aide de~\cite{Fried:1961}, que la fonction $Z$ vérifie :
\begin{equation}
  Z'(\xi)=-2\left(1+\xi Z(\xi)\right).
  \label{eq:Zder}
\end{equation}
Nous allons maintenant établir quelques propriétés utiles pour vérifier l'hypothèse~\ref{hyp:sym} dans différents cas tests classiques.

\begin{lemma}
  La fonction $Z_\alpha^0(\omega):\omega\in\mathbb{C}\mapsto Z\left(\alpha\omega\right)\in\mathbb{C}$, avec $\alpha\in\mathbb{R}$ fixé, est telle que : $Z_\alpha^0(-\bar{\omega}) = -\overline{Z_\alpha^0(\omega)}$.
  \label{lemma:Z0}
\end{lemma}

\begin{lemma}
  La fonction $Z_{\alpha,\beta}^+(\omega):\omega\in\mathbb{C}\mapsto Z\left(\alpha\omega-\beta\right)+Z\left(\alpha\omega+\beta\right)\in\mathbb{C}$, avec $\alpha\in\mathbb{R}$, $\beta\in\mathbb{R}$ fixés, est telle que : $Z_{\alpha,\beta}^+\left(-\overline{\omega}\right)=-\overline{Z_{\alpha,\beta}^+(\omega)}$.
  \label{lemma:Z+}
\end{lemma}

\begin{lemma}
  La fonction $Z_{\alpha,\beta}^-(\omega):\omega\in\mathbb{C}\mapsto Z\left(\alpha\omega-\beta\right)-Z\left(\alpha\omega+\beta\right)\in\mathbb{C}$, avec $\alpha\in\mathbb{R}$, $\beta\in\mathbb{R}$ fixés, est telle que : $Z_{\alpha,\beta}^-\left(-\overline{\omega}\right)=\overline{Z_{\alpha,\beta}^-(\omega)}$.
  \label{lemma:Z-}
\end{lemma}

La démonstration de ces lemmes est proposée dans l'annexe~\ref{a:dispersion}.

L'introduction de la fonction $Z$ provient de la nécessité dans les relations de dispersion définies en~\eqref{eq:D}-\eqref{eq:N} et~\eqref{eq:relD_H}-\eqref{eq:relN_H} d'intégrer une distribution maxwellienne qui est une distribution gaussienne renormalisée :
$$
  \mathcal{M}_{\rho,u,T} = \frac{\rho}{\sqrt{2\pi T}}e^{-\frac{(v-u)^2}{2T}}
$$
Rappelons le résultat :
$$
  \partial_v \mathcal{M}_{\rho,u,T}(v) = -\frac{v-u}{T}\mathcal{M}_{\rho,u,T}(v)
$$
Ainsi, avant de passer à l'application de ces résultats sur le cas test qui nous intéresse, calculons une intégrale qui intervient dans le calcul de $D(k,\omega)$ :
$$
  \begin{aligned}
    \int_\gamma \frac{\partial_v\mathcal{M}_{\rho,u,T}}{v-\frac{\omega}{k}}\,\mathrm{d}v 
      & = -\frac{\rho}{\sqrt{2\pi T}T}\int_\gamma \frac{(v-\frac{\omega}{k} + \frac{\omega}{k}-u)e^{-\frac{(v-u)^2}{2T}}}{v-\frac{\omega}{k}}\,\mathrm{d}v \\
      & = -\frac{\rho}{\sqrt{2\pi T}T}\left( \int_\gamma e^{-\frac{(v-u)^2}{2T}}\,\mathrm{d}v + \left(\frac{\omega}{k}-u\right)\int_\gamma \frac{e^{-\frac{(v-u)^2}{2T}}}{v-\frac{\omega}{k}}\,\mathrm{d}v \right)
  \end{aligned}
$$
Dans la première intégrale, on utilise le changement de variable $w = \frac{v-u}{\sqrt{T}}$, $\mathrm{d}w = \frac{\mathrm{d}v}{\sqrt{T}}$, dans la seconde intégrale, nous utilisons le changement de variable suivant : $w=\frac{v-u}{\sqrt{2T}}$, $\mathrm{d}w = \frac{\mathrm{d}v}{\sqrt{2T}}$. Nous obtenons :
$$
  \begin{aligned}
    -\frac{\rho}{\sqrt{2\pi T}T}\left( \int_\gamma e^{-\frac{w^2}{2}}\sqrt{T}\,\mathrm{d}w + \left(\frac{\omega}{k}-u\right)\int_\gamma \frac{e^{-w^2}}{\sqrt{2T}w + u - \frac{\omega}{k}}\sqrt{2T}\,\mathrm{d}w \right) \\
    = \frac{\rho}{T}\left( 1 + \frac{1}{\sqrt{2\pi T}}\left(\frac{\omega}{k}-u\right)\int_\gamma \frac{e^{-w^2}}{w-\frac{1}{\sqrt{2T}}\left(\frac{\omega}{k}-u\right)}\,\mathrm{d}w \right)
  \end{aligned}
$$
et enfin nous obtenons :
\begin{equation}
  \int_\gamma \frac{\partial_v \mathcal{M}_{\rho,u,T}(v)}{v-\frac{\omega}{k}}\,\mathrm{d}v
    = -\frac{\rho}{T}\left( 1 + \frac{1}{\sqrt{2T}}\left(\frac{\omega}{k}-u\right)Z\left(\frac{\frac{\omega}{k}-u}{\sqrt{2T}}\right) \right)
    \label{eq:intforM}
\end{equation}
où $Z$ est la fonction de diffusion de plasma~\eqref{eq:Zfct}.

Le calcul de la fonction $N(k,\omega)$ demande l'évaluation d'une intégrale pour laquelle on utilise le changement de variable $w = \frac{v-u}{\sqrt{2T}}$,
$$
  \begin{aligned}
    \int_\gamma \frac{\mathcal{M}_{\rho,u,T}(v)}{v-\frac{\omega}{k}}\,\mathrm{d}v
      & = \frac{\rho}{\sqrt{2\pi T}}\int_\gamma \frac{e^{-\frac{(v-u)^2}{2T}}}{v-\frac{\omega}{k}}\,\mathrm{d}v \\
      & = \frac{\rho}{\sqrt{2\pi T}}\int_\gamma \frac{e^{-w^2}}{\sqrt{2T}w + u - \frac{\omega}{k}}\sqrt{2T}\,\mathrm{d}w \\
      & = \frac{\rho}{\sqrt{2\pi T}}\int_\gamma \frac{e^{-w^2}}{w - \frac{1}{\sqrt{2T}}\left(\frac{\omega}{k}-u\right)}\,\mathrm{d}w
  \end{aligned}
$$  
soit :
\begin{equation}
  \int_\gamma \frac{\mathcal{M}_{\rho,u,T}(v)}{v-\frac{\omega}{k}}\,\mathrm{d}v
    = \frac{\rho}{\sqrt{2T}}Z\left(\frac{\frac{\omega}{k}-u}{\sqrt{2T}}\right)
  \label{eq:NforM}
\end{equation}


%-------------
\subsubsection{Application à la modélisation hybride}

La condition initiale du cas test du modèle hybride nous donne comme état d'équilibre (état perturbé) pour les particules chaudes :
$$
  f_h^{(0)}(v) = \mathcal{M}_{^\alpha/_2,v_0,1}(v) + \mathcal{M}_{^\alpha/_2,-v_0,1}(v) = \frac{\alpha}{2\sqrt{2\pi}}\left( e^{-\frac{(v-v_0)^2}{2}} + e^{-\frac{(v+v_0)^2}{2}} \right)
$$
avec une vitesse des particules chaudes $v_0\in\mathbb{R}$ fixée et une densité de particules chaudes $\alpha\in\mathbb{R}$. Les particules froides n'étant pas perturbées, l'état d'équilibre est l'état initial caractérisé par une densité $\rho_c^{(0)}= 1-\alpha$, et une vitesse moyenne $u_c(t=0,x)=0$. L'expression~\eqref{eq:relD_H} nous donne à l'aide de~\eqref{eq:intforM} :
\begin{eqnarray}
  D(k,\omega)
    &=&1-\frac{1}{k^2}\left(\left(1-\alpha\right)\frac{k^2}{\omega^2}+\int_\gamma \frac{\partial_vf_h^{(0)}(v)}{v-\frac{\omega}{k}}dv\right)\nonumber\\
    &=&1-\frac{1}{k^2}\left[\left(1-\alpha\right)\frac{k^2}{\omega^2}-\frac{\alpha}{2}\left(1+\frac{1}{\sqrt{2}}\left(\frac{\omega}{k}-v_0\right)Z\left(\frac{1}{\sqrt{2}}\left(\frac{\omega}{k}-v_0\right)\right)\right)\right.\nonumber\\
    &&~~~~~~~~~~~~~~~~~~~\left.-\frac{\alpha}{2}\left(1+\frac{1}{\sqrt{2}}\left(\frac{\omega}{k}+v_0\right)Z\left(\frac{1}{\sqrt{2}}\left(\frac{\omega}{k}+v_0\right)\right)\right)\right].
  \label{eq:D_hchyb}
\end{eqnarray}
On dérive $D(k,\omega)$ à l'aide de~\eqref{eq:Zder} :
\begin{equation}
  \begin{aligned}
    \frac{\partial D(k,\omega)}{\partial \omega} = 2\frac{\left(1-\alpha\right)}{\omega^3}+\frac{1}{\sqrt{2}k^3}\frac{\alpha}{2}\left[\left(1-2\tilde{\omega}_-^2\right)Z\left(\tilde{\omega}_-\right)\right. \\
      \left.+\left(1-2\tilde{\omega}_+^2\right)Z\left(\tilde{\omega}_+\right)-2\tilde{\omega}_--2\tilde{\omega}_+\right]
  \end{aligned}
  \label{eq:hchybderD}
\end{equation}
où $\tilde{\omega}_\pm=\frac{1}{\sqrt{2}}\left(\frac{\omega}{k}\pm v_0\right)$.

Maintenant, remarquons que :
$$
  \hat{f}_h(t=0,k,v) = \hat{g}(k)\frac{\alpha}{2\sqrt{2\pi}}\left( e^{-\frac{(v-v_0)^2}{2}}  e^{-\frac{(v+v_0)^2}{2}}\right)\,,\quad g(x) = \cos\left(\frac{2\pi}{L}x\right)
$$
ce qui nous permet de simplifier ce calcul de $N(k,\omega)$ en utilisant~\eqref{eq:relN_H} et~\eqref{eq:NforM} :
$$
  \begin{aligned}
    N(k,\omega)
      & = \frac{(1-\alpha)}{\omega^2}\hat{u}(t=0,k) - \frac{1}{i\omega}\hat{E}(t=0,k) - \frac{\hat{g}(k)}{2\omega k}\left( \int_\gamma v\frac{\mathcal{M}_{^\alpha/_2,v_0,1}}{v-\frac{\omega}{k}}\,\mathrm{d}v  +  \int_\gamma v\frac{\mathcal{M}_{^\alpha/_2,-v_0,1}}{v-\frac{\omega}{k}}\,\mathrm{d}v \right) \\
      & = \frac{(1-\alpha)}{\omega^2}\hat{u}(t=0,k) - \frac{1}{i\omega}\hat{E}(t=0,k) - \frac{\hat{g}(k)}{2\omega k}\left( \int_\gamma\mathcal{M}_{^\alpha/_2,v_0,1}\,\mathrm{d}v + \frac{\omega}{k}\int_\gamma \frac{\mathcal{M}_{^\alpha/_2,v_0,1}}{v-\omega{k}}\,\mathrm{d}v\right. \\
      & \qquad\qquad\qquad\qquad\qquad\qquad\qquad\qquad\qquad\qquad \left. + \int_\gamma\mathcal{M}_{^\alpha/_2,-v_0,1}\,\mathrm{d}v + \frac{\omega}{k}\int_\gamma \frac{\mathcal{M}_{^\alpha/_2,-v_0,1}}{v-\omega{k}}\,\mathrm{d}v \right) \\
      & = \frac{(1-\alpha)}{\omega^2}\hat{u}(t=0,k) - \frac{1}{i\omega}\hat{E}(t=0,k) - \frac{\hat{g}(k)}{2\omega k}\left[ \alpha + \frac{\omega}{k}\frac{\alpha}{2\sqrt{2}}\left( Z\left(\frac{\frac{\omega}{k}-v_0}{\sqrt{2}}\right) \right.\right. \\
      & \qquad\qquad\qquad\qquad\qquad\qquad\qquad\qquad\qquad\qquad\qquad\qquad\quad + \left.\left.Z\left(\frac{\frac{\omega}{k}+v_0}{\sqrt{2}}\right) \right)\right]
  \end{aligned}
$$
soit finalement :
\begin{equation}
  \begin{aligned}
    N(k,\omega) =& \frac{(1-\alpha)}{\omega^2}\hat{u}(t=0,k) - \frac{1}{i\omega}\hat{E}(t=0,k) \\
      &-\frac{\hat{g}(k)}{k^2}\left[ \alpha\frac{k}{\omega} + \frac{\alpha}{2\sqrt{2}}\left( Z\left(\frac{\frac{\omega}{k}-v_0}{\sqrt{2}}\right) + Z\left(\frac{\frac{\omega}{k}+v_0}{\sqrt{2}}\right) \right) \right]
  \end{aligned}
  \label{eq:N_hchyb}
\end{equation}
où $\hat{g}(k)$ est donnée par :
\begin{equation}
  \hat{g}\left(\frac{2\pi}{L}\right) = \hat{g}\left(-\frac{2\pi}{L}\right) = \frac{1}{2}\,,\quad \hat{g}(k) = 0, k\notin\left\{-\frac{2\pi}{L} ,\frac{2\pi}{L} \right\}
\label{eq:gk}
\end{equation}

\begin{lemma}
  Sous l'hypothèse $\hat{u}(t=0,k)=0$, pour $\frac{\partial D(k,\omega)}{\partial\omega}$ donnée par~\eqref{eq:hchybderD} et $N(k,\omega)$ par~\eqref{eq:N_hchyb}, l'hypothèse~\ref{hyp:sym} est satisfaite.
  \label{lemme:hypcashyb}
\end{lemma}

La démonstration de ce lemme est effectuée dans l'annexe~\ref{a:dispersion}. Elle permet de justifier l'écriture~\eqref{eq:Etk} du mode fondamental du champ électrique linéarisé puis l'approximation~\eqref{eq:enelec} de l'énergie électrique linéarisée.

%-------------
\subsubsection{Application à la modélisation cinétique}

La densité de particules initiale de la modélisation cinétique peut se réécrire comme la somme de la densité de particules froides et de la densité de particules chaudes, avec pour les particules froides une simple distribution maxwellienne non perturbée, et pour les particules chaudes une bi-maxwellienne dont l'intégration a déjà été traitée dans le cas hybride :
\begin{equation}
  f_0(x,v) = \mathcal{M}_{1-\alpha,0,Tc}(v) + f_{h,0}(x,v)
 \label{init_f0_rel_disp} 
\end{equation}
avec $f_{h,0}(x,v)$ donnée par~\eqref{eq:f0hdexv}. L'expression de $D(k,\omega)$ s'obtient à partir de~\eqref{eq:D} et~\eqref{eq:intforM} :
\begin{equation}
  \begin{aligned}
    D(k,\omega) = 1 - \frac{1}{k^2}\left[\vphantom{\frac{}{\sqrt{}}}\right. & -\frac{1-\alpha}{T_c}\left( 1 + \frac{1}{\sqrt{2T_c}}\frac{\omega}{k}Z\left( \frac{1}{\sqrt{2T_c}}\frac{\omega}{k}\right) \right) \\
                                          & -\frac{\alpha}{2}\left( 1 + \frac{1}{\sqrt{2}}\left(\frac{\omega}{k}-v_0\right)Z\left(\frac{1}{\sqrt{2}}\left(\frac{\omega}{k}-v_0\right)\right) \right) \\
                                          & \left. -\frac{\alpha}{2}\left( 1 + \frac{1}{\sqrt{2}}\left(\frac{\omega}{k}+v_0\right)Z\left(\frac{1}{\sqrt{2}}\left(\frac{\omega}{k}+v_0\right)\right) \right)  \right]
  \end{aligned}
    \label{D_3bump}
\end{equation}
Expression que l'on peut dériver et simplifier à l'aide de~\eqref{eq:Zder} :
\begin{equation}
  \begin{aligned}
    \frac{\partial D(k,\omega)}{\partial\omega} = \frac{1}{\sqrt{2}k^3}\left[\vphantom{\frac{}{\sqrt{}}}\right.
        & \frac{1-\alpha}{\sqrt{T_c}T_c}\left( (1-2\tilde{\omega}_0^2)Z(\tilde{\omega}_0) - 2\tilde{\omega}_0 \right) \\
        & \left.\vphantom{\frac{}{\sqrt{}}} +\frac{\alpha}{2}\left( (1-2\tilde{\omega}_-^2)Z(\tilde{\omega}_-) + (1-2\tilde{\omega}_+^2)Z(\tilde{\omega}_+) - 2\tilde{\omega}_- - 2\tilde{\omega}_+ \right) \right]
  \end{aligned}
  \label{eq:3bumpderD}
\end{equation}
où $\tilde{\omega}_0 = \frac{1}{\sqrt{2T_c}}\frac{\omega}{k}$ et $\tilde{\omega}_\pm = \frac{1}{\sqrt{2}}(\frac{\omega}{k}\pm v_0)$. Maintenant, pour le calcul de $N(k,\omega)$, on remarque que l'on a :
$$
  \hat{f}(t=0,k,v) = \hat{g}(k) \frac{\alpha}{2\sqrt{2\pi}}\left( e^{-\frac{(v-v_0)^2}{2}} + e^{-\frac{(v+v_0)^2}{2}} \right) 
$$
avec la fonction $g(x)$ qui vérifie :
$$ 
  \hat{g}\left(\frac{2\pi}{L}\right) = \hat{g}\left(-\frac{2\pi}{L}\right) = \frac{1}{2}\,,\quad \hat{g}(k) = 0, k\notin\left\{-\frac{2\pi}{L} ,\frac{2\pi}{L} \right\}
$$
ce qui nous permet, en utilisant les équations~\eqref{eq:N} et~\eqref{eq:NforM} d'obtenir :
\begin{equation}
  N(k,\omega) = -\frac{\hat{g}(k)}{k^2}\frac{\alpha}{2\sqrt{2}}\left( Z\left(\frac{\frac{\omega}{k}-v_0}{\sqrt{2}}\right) + Z\left(\frac{\frac{\omega}{k}+v_0}{\sqrt{2}}\right) \right)
  \label{eq:N_3bump}
\end{equation}

Nous avons donc le lemme suivant.
\begin{lemma}
  Pour $\frac{\partial D(k,\omega)}{\partial\omega}$ donnée par~\eqref{eq:3bumpderD} et $N(k,\omega)$ par~\eqref{eq:N_3bump}, l'hypothèse~\ref{hyp:sym} est satisfaite.
  \label{lemme:hypcascin}
\end{lemma}

La démonstration de ce lemme est effectuée dans l'annexe~\ref{a:dispersion}. Elle permet de justifier l'écriture~\eqref{eq:Etk} du mode fondamental du champ électrique linéarisé puis l'approximation~\eqref{eq:enelec} de l'énergie électrique linéarisée. 

%----------
\subsubsection{Consistance des relations de dispersion}
%----------

%\commentaire[Nicolas]{
Dans les sous-sections précédentes, nous avons obtenu les relations de dispersion des modèles cinétique et VHL correspondant à la condition initiale \eqref{init_f0_rel_disp}. Une première validation va consister à vérifier que les relations de dispersion du modèle cinétique données par (\ref{D_3bump})-(\ref{eq:3bumpderD})-(\ref{eq:N_3bump}) sont consistantes, quand $T_c\to 0$, avec les relations de dispersion du modèle hybride données par (\ref{eq:D_hchyb})-(\ref{eq:hchybderD})-(\ref{eq:N_hchyb}). Pour cela, rappelons que 
$$
  Z(z)=\sqrt{\pi} \exp(-z^2) (i - erfi(z) )
$$
et qu'à la limite $z\to+\infty$, nous avons le développement asymptotique suivant 
$$
  erfi(z) = -i + \frac{\exp(z^2)}{\sqrt{\pi}} \left(\frac{1}{z} +\frac{1}{2z^3} +\frac{3}{4z^5} + \mathcal{O}\left(z^{-7}\right) \right).
$$
Ainsi, nous avons $Z(z) =  2 i \sqrt{\pi} \exp(-z^2)  - \frac{1}{z}  - \frac{1}{2z^3} - \frac{3}{4z^5}+ \mathcal{O}\left(z^{-7}\right)$ ou encore $Z(z)= - \frac{1}{z}  - \frac{1}{2z^3}- \frac{3}{4z^5}+ \mathcal{O}\left(z^{-7}\right)$ et donc
$$
  zZ(z)=-1-\frac{1}{2z^2} + \mathcal{O}\left(z^{-4}\right).
$$
Commençons par regarder la consistance en $D(k,\omega)$. Avec $z=\frac{1}{\sqrt{2T_c}}\frac{\omega}{k}$ quand $T_c\to 0$, le terme correspondant aux particules froides de \eqref{D_3bump} s'écrit
\begin{eqnarray*}
  -\frac{1-\alpha}{T_c}\left(1+\frac{1}{\sqrt{2T_c}}\frac{\omega}{k}Z\left(\frac{1}{\sqrt{2T_c}}\frac{\omega}{k}\right)\right)&=&-\frac{1-\alpha}{T_c}\left(1-1-\frac{1}{2\left(\frac{1}{\sqrt{2T_c}}\frac{\omega}{k}\right)^2}+ \mathcal{O}\left(\left(\frac{1}{\sqrt{2T_c}}\frac{\omega}{k}\right)^{-4}\right)\right)\nonumber\\
  &=& \left(1-\alpha\right)\frac{k^2}{\omega^2} + \mathcal{O}(T_c). 
\end{eqnarray*}
C'est le terme correspondant à la partie fluide (froide) de \eqref{eq:D_hchyb}. Les autres termes (venant des particules chaudes) sont les mêmes dans les deux expressions, donc $D(k,\omega)$ donné par le modèle cinétique est consistant, à la limite $T_c\to 0$, avec celui donné par le modèle hybride (avec un taux $\mathcal{O}(T_c)$). Regardons ensuite la consistance en $\frac{\partial D(k,\omega)}{\partial \omega}$. Les termes venant des particules chaudes sont les mêmes dans les modèles cinétique (\ref{eq:3bumpderD}) et hybride (\ref{eq:hchybderD}). Nous ne nous intéressons qu'aux termes venant des particules froides. De (\ref{eq:3bumpderD}), nous avons
\begin{eqnarray*}
  &&\frac{1}{\sqrt{2}k^3}\frac{1-\alpha}{T_c\sqrt{T_c}}\left(Z\left(\tilde{\omega}_0\right)-2\tilde{\omega}_0^2Z\left(\tilde{\omega}_0\right)-2\tilde{\omega}_0\right)\\
  &=&\frac{1}{\sqrt{2}k^3}\frac{1-\alpha}{T_c\sqrt{T_c}}\left(-\frac{1}{\tilde{\omega}_0}-\frac{1}{2\tilde{\omega}_0^3}+2\tilde{\omega}_0+\frac{1}{\tilde{\omega}_0}+\frac{3}{2\tilde{\omega}_0^3}-2\tilde{\omega}_0\right)+\mathcal{O}\left(\tilde{\omega}_0^{-5}\right)\\
  &=&\frac{1}{\sqrt{2}k^3}\frac{1-\alpha}{T_c\sqrt{T_c}}\frac{1}{\tilde{\omega}_0^3}+\mathcal{O}\left(\tilde{\omega}_0^{-5}\right)
\end{eqnarray*}
donc pour $\tilde{\omega}_0=\frac{1}{\sqrt{2T_c}}\frac{\omega}{k}$, nous avons 
$$
  \frac{1}{\sqrt{2}k^3}\frac{1-\alpha}{T_c\sqrt{T_c}}\frac{2T_c\sqrt{2T_c}k^3}{\omega^3}=2\frac{1-\alpha}{\omega^3}
$$
qui est le terme fluide de \eqref{eq:hchybderD}. Regardons enfin la consistance en $N(k,\omega)$. Là encore, les termes venant des particules chaudes sont les mêmes dans les modèles cinétique (\ref{eq:N_3bump}) et hybride (\ref{eq:N_hchyb}). Les termes supplémentaires dans le modèle hybride s'annulent sous l'hypothèse $\hat{u}(t=0,k)=0$ et avec $\hat{g}(k)$ donné par (\ref{eq:gk}) et $\hat{E}(t=0,k)$ obtenu à partir de l'équation de Poisson 
\begin{eqnarray*}
  \partial_xE(t=0,x)&=& \rho_c(t=0,x)+\int f^h(t=0,x,v)dv-1\\
                    &=& \left(1-\alpha\right)+\alpha\left(1+\varepsilon\cos\left(\frac{2\pi}{L}x\right)\right)-1\\
                    &=& \alpha\varepsilon\cos\left(\frac{2\pi}{L}x\right)
\end{eqnarray*}
soit
\begin{equation}
  \hat{E}\left(t=0,k\right)=-\frac{i\alpha}{2k},~k\in\left\{-\frac{2\pi}{L},\frac{2\pi}{L}\right\},~~~\hat{E}(k)=0,~k\notin\left\{-\frac{2\pi}{L},\frac{2\pi}{L}\right\}.
\label{eq:Ekbis}
\end{equation}
La consistance du modèle cinétique, à la limite $T_c\to 0$, vers le modèle hybride est établie sur les relations de dispersion.
%} 
%Dans les sous-sections suivantes, nous faisons une étude numérique de la convergence pour $T_c\to 0$ du modèle cinétique \eqref{eq:vlasov}-\eqref{eq:poisson} vers le modèle hybride linéarisé \eqref{eq:vahl}.}


%% section 5
% !TEX root = ../../main.tex

\section{Limite du modèle cinétique vers le modèle hybride}
 \label{s:limit}

Il s'agit ici d'étudier numériquement la convergence du modèle cinétique \eqref{eq:vlasov}-\eqref{eq:poisson} vers le modèle VHL \eqref{eq:vahl}, lorsque la température $T_c$ des particules froides tend vers 0. Une première étude de consistance est effectuée sur les relations de dispersion. Une seconde étude, numérique, montre la convergence de différentes quantités obtenues par les schémas proposés dans la section \ref{s:scheme}. Ces deux études complémentaires ont pour but de justifier l'utilisation de la modélisation hybride linéarisée lorsque les particules se répartissent en deux faisceaux: l'un de particules chaudes (rapides) et l'autre de particules froides (de température $T_c\ll 1$, lentes). Pour cela, le modèle cinétique \eqref{eq:vlasov}-\eqref{eq:poisson} sera initialisé avec
\begin{equation}
  f^0(x,v) = \mathcal{M}_{1-\alpha,0,T_c}(v) + (1+\epsilon\cos(kx))\left( \mathcal{M}_{^\alpha/_2,v_0,1}(v) + \mathcal{M}_{^\alpha/_2,-v_0,1}(v) \right)
\label{eq:K:init}
\end{equation}
avec $k = 0.5$, $v_0 = 3.4$, $\alpha=0.2$, $x\in[0,L]$, $L=\frac{2\pi}{k}=2\pi$, $v\in[-v_{\text{max}},v_{\text{max}}]$ avec $v_{\text{max}}=12$ et la perturbation des particules chaudes $\epsilon = 10^{-2}$. Le paramètre $T_c$ prendra différentes valeurs selon les résultats que nous souhaitons illustrer. Comme dans la sous-section \ref{ssec:disp_appl}, on a noté  $\mathcal{M}_{\rho,u,T}(v)$ la distribution maxwellienne :
$$
  \mathcal{M}_{\rho,u,T}(v) := \frac{1}{(2\pi T)^{\frac{1}{2}}}\exp\left(-\frac{\left|v-u\right|^2}{2T}\right)
$$

Pour la condition initiale des simulations avec le modèle hybride linéarisé \eqref{eq:vahl}, nous considérerons :
\begin{equation}
  \begin{aligned}
    u_c(x)   & = 0 \\
    f_h(x,v) & = (1+\epsilon\cos(kx))\left( \mathcal{M}_{^\alpha/_2,v_0,1}(v) + \mathcal{M}_{^\alpha/_2,-v_0,1}(v) \right)
  \end{aligned}
\label{eq:HL:init}
\end{equation}
où $k$, $v_0$, $\alpha$ et $\epsilon$ sont pris identiques au modèle cinétique ; le domaine en $x$ et en $v$ reste inchangé. $E(t=0,x)$ est obtenu en résolvant l'équation de Poisson sur notre condition initiale, comme indiqué dans la proposition~\ref{p:vhl_conservation} :
$$
  \partial_x E(t=0) = (1-\alpha) + \int (1+\epsilon\cos(kx))\left( \mathcal{M}_{^\alpha/_2,v_0,1}(v) + \mathcal{M}_{^\alpha/_2,-v_0,1}(v) \right)\,\mathrm{d}v - 1
$$
Avant une étude plus détaillée, nous donnons un premier aperçu des solutions des deux modèles pour le choix $T_c=0.05$. Sur la figure~\ref{fig:limit_vp}, sont tracées la condition initiale $f^0(x, v)$ du modèle cinétique (gauche), la solution numérique  $f(T_f=300, x, v)$ au temps final du modèle cinétique (milieu) et la solution numérique $f_h(T_f=300, x, v)$ des particules chaudes pour le modèle hybride ainsi que la vitesse moyenne des particules froides $u_c(T_f=300,x)$ (courbe cyan) au temps final (droite). La bande jaune correspond à la population de particules froides, absente dans la modélisation hybride. On observe une bonne corrélation des vortex dans l'espace des phases dans la population de particules froides, et une bonne reconstruction de la population de particules chaudes à partir de la vitesse moyenne $u_c$. De plus, sur la figure \ref{fig:limit_ee_Tf300} est tracée l'évolution de l'énergie électrique $\|E(t, \cdot)\|_{L_2}$ au cours du temps pour ces deux modèles avec les mêmes paramètres numériques (échelle semi-logarithmique) pour différentes valeurs de $T_c=0.05,0.1,0.2,0.4$ pour le modèle cinétique. On observe une convergence de l'énergie électrique du modèle cinétique vers le modèle hybride lorsque $T_c$ tend vers $0$.
\begin{figure}[h]
  \centering
  \includegraphics[width=\textwidth]{\localPath/figures/vp_t1.png}
  \caption{Représentation de la condition initiale du modèle cinétique à gauche et la solution obtenue au temps final $T_f=300$ avec le modèle cinétique avec $T_c = 0.05$ (au milieu) et la densité de particules chaudes obtenue avec modèle hybride linéarisé ainsi que la vitesse moyenne des particules froides (courbe cyan) (à droite).}
  \label{fig:limit_vp}
\end{figure}
\begin{figure}[h]
  \centering
  \includegraphics[width=\textwidth]{\localPath/figures/ee_t1.png}
  \caption{Énergie électrique donnée pour le modèle cinétique avec $T_c=0.05,0.1,0.2,0.4$ et le modèle hybride linéarisé.}
  \label{fig:limit_ee_Tf300}
\end{figure}
La première observation est que les résultats proches de ceux obtenus par le modèle hybride linéarisé \eqref{eq:vahl} sont très proches de ceux obtenus par le modèle cinétique \eqref{eq:vlasov}-\eqref{eq:poisson}, ce qui valide la modélisation. La perturbation des particules chaudes induit une instabilité (l'équilibre étant du type double gaussienne) qu'on voit se développer jusqu'au temps $t=75$ (voir figure \ref{fig:limit_ee_Tf300}), et deux vortex sont alors créés autour de la vitesse $v\approx 2$, au centre desquels de fines structures se développent. 

Dans la suite, nous allons approfondir cette étude en comparant les résultats obtenus aux relations de dispersion des deux modèles, puis en essayant de déterminer le domaine de validité du modèle VHL.  

%\commentaire[Anais]{(revoir un peu la rédaction de cette intro car elle ne tenait pas du tout compte de la sous-section relations de dispersion et je n'ai pas l'impression de l'avoir suffisamment modifiée)}

\FloatBarrier
%----------
\subsection{Convergence en énergie totale}
%----------

Nous nous intéresserons ici à une grandeur conservée qu'est l'énergie totale, celle-ci est la somme de l'énergie cinétique et de l'énergie électrique. Pour le modèle cinétique elle se calcule ainsi :
$$
  \mathcal{E}_K(t) = \iint_{\Omega\times\mathbb{R}} v^2 f\,\mathrm{d}x\mathrm{d}v + \int_{\Omega} E^2\,\mathrm{d}x. 
$$
Pour le modèle VHL, l'énergie cinétique comporte deux termes, un terme fluide pour les particules froides, et un terme cinétique pour les particules chaudes :
$$
  \mathcal{E}_{VHL}(t) = \int_\Omega \rho_c u_c^2\,\mathrm{d}x + \iint_{\Omega\times\mathbb{R}}v^2f_h\,\mathrm{d}x\mathrm{d}v + \int_\Omega E^2\,\mathrm{d}x
$$

\begin{pro}
  La différence en énergie totale entre le modèle cinétique et le modèle hybride linéarisé pour des conditions initiales données par~\eqref{eq:K:init} et~\eqref{eq:HL:init} converge en $(1-\alpha)T_c|\Omega|$.
\label{p:limit:convergence}
\end{pro}
\begin{proof}
  Pour le choix de $f^0$, l'énergie totale du modèle cinétique vaut :
  $$
    \begin{aligned}
      \mathcal{E}_K(t) = \mathcal{E}_K(0) &= \iint_{\Omega\times\mathbb{R}}v^2f^0(x,v)\,\mathrm{d}x\mathrm{d}v + \int_{\Omega}E^2(t=0,x)\,\mathrm{d}x \\
                                          &= \left[ (1-\alpha)T_c + \alpha v_0^2 + \alpha \right]|\Omega|
    \end{aligned}
  $$
    On remarque que lorsque $T_c\to 0$, on obtient $\lim_{T_c\to 0}\mathcal{E}_K(t) = (\alpha v_0^2 + \alpha)|\Omega|$. L'énergie totale dans le cadre du modèle hybride se calcule comme suit :
    $$
      \mathcal{E}_{HL}(t) = \int_\Omega \rho_c^{(0)}u_c^2\,\mathrm{d}x + \iint_{\Omega\times\mathbb{R}} v^2f_h\,\mathrm{d}x\mathrm{d}v + \int_\Omega E^2\,\mathrm{d}x
    $$
    ce qui nous donne, avec le choix de condition initiale $\rho_c^{(0)} = 1-\alpha$, $u_c^0 = 0$ et $f_h^0(v) = \mathcal{M}_{^\alpha/_2,v_0,1}(v) + \mathcal{M}_{^\alpha/_2,-v_0,1}(v)$, conformément à~\eqref{eq:HL:init} :
    $$
      \mathcal{E}_{HL}(t) = (\alpha v_0^2 + \alpha)|\Omega|
    $$
    qui est bien compatible avec $\lim_{T_c\to 0}\mathcal{E}_K(t)$. De plus on peut calculer :
    $$
      \mathcal{E}_K(t) - \mathcal{E}_{HL}(t) = (1-\alpha)T_c|\Omega|
    $$
    c'est-à-dire que la convergence du modèle hybride est liée à la pression $\rho_c^{(0)}T_c$ des particules froides.
\end{proof}

Pour vérifier numériquement cette proposition, nous effectuons un jeu de simulations. Le modèle cinétique de Vlasov-Poisson \eqref{eq:vlasov}-\eqref{eq:poisson} est simulé à l'aide d'une méthode en temps de type Lawson basée sur une méthode de Runge-Kutta d'ordre 4, la méthode WENO d'ordre 5 pour approcher la dérivée dans la direction $v$ et l'algorithme de FFT pour la dérivée dans la direction $x$. Il s'agit ainsi du même schéma que celui utilisé pour la fonction de distribution $f_h$ des particules chaudes du modèle hybride linéarisé. Nous choisissons la condition initiale \eqref{eq:K:init} avec $\alpha = 0.2$, $T_c \in \left\{ 0.05,0.1,0.2,0.4\right\}$, la discrétisation du domaine $\Omega = [0,4\pi]$ s'effectue avec $N_x = 135$ points, la discrétisation du domaine en vitesse $[-v_{\text{max}},v_{\text{max}}]$ nécessite de capturer la gaussienne représentant les particules froides pour différentes valeurs de $T_c$ ; nous choisissons donc d'adapter le nombre de points de discrétisation en vitesse $N_v$ à $T_c$, $N_v \in \left\{ 1431,1011,715,505 \right\}$, ceux-ci correspondant à une quinzaine de points de discrétisation pour capturer la gaussienne de température $T_c$. Nous avons une condition CFL sur le schéma WENO utilisé dans la direction $v$, nous nous assurons d'être sous cette condition quelle que soit l'évolution de $E$ en prenant $\Delta t = 0.5\Delta v$. Ce jeu de simulations s'arrête au temps 7, or le choix des différents $\Delta t$ implique des données à des temps différents ; nous choisissons d'effectuer une interpolation polynomiale de Lagrange d'ordre 5 pour exploiter les données au temps $T^\star = 6.5$. Nous obtenons ainsi l'énergie totale pour différentes températures sur la figure~\ref{fig:limit:totalenergy} ; bien que le schéma de type Runge-Kutta ne conserve pas exactement l'énergie, celle-ci est bien préservée en temps court et reste sous l'erreur machine en simple précision jusqu'au temps 50. Après l'interpolation au temps $T^\star = 6.5$ on obtient la convergence vers le modèle hybride sur la figure~\ref{fig:limit:totalenergy} (figure de droite) où l'on observe bien l'ordre 1 en température.
\begin{figure}[h]
  \centering
  \includegraphics[width=\textwidth]{\localPath/figures/order_HTc.png}
  \caption{Énergie totale avec les différents modèles en échelle semi-log (gauche) et convergence de l'énergie totale du modèle cinétique vers le modèle hybride quand $T_c$ tend vers $0$ en échelle log (droite).}
  \label{fig:limit:totalenergy}
\end{figure}

Nous effectuons le même type d'analyse sur l'énergie électrique, à partir des données des simulations précédentes, en sachant que l'on n'a pas de résultat théorique sur sa convergence. L'énergie électrique pour les différents choix de $T_c$ est représentée sur la figure~\ref{fig:limit:ee} (gauche), cette figure illustre mieux la nécessité d'effectuer une interpolation pour extraire les données. Une convergence est observée sur la figure~\ref{fig:limit:ee} (droite).
\begin{figure}[h]
  \centering
  \includegraphics[width=\textwidth]{\localPath/figures/order_eeTc.png}
  \caption{Énergie électrique avec les différents modèles en échelle semi-log (gauche) et convergence de l'énergie totale du modèle cinétique vers le modèle hybride quand $T_c$ tend vers $0$ en échelle log (droite).}
  \label{fig:limit:ee}
\end{figure}

\FloatBarrier

%----------
\subsection{Convergence en température à l'aide des relations de dispersion}
%----------

Nous étudions numériquement la convergence des racines de la relation de dispersion quant $T_c$ tend vers $0$. Pour cela, on note $D^K_{[T_c]}(\omega,k)$ la relation de dispersion du modèle cinétique \eqref{D_3bump} et $D^H(\omega,k)$ la relation de dispersion du modèle VHL \eqref{eq:D_hchyb}. Pour $k$ fixé, on note $\omega\in\mathbb{C}$ la racine de plus grande partie imaginaire. 
%Pour la relation de dispersion, nous regardons les zéros en $\omega$ de la fonction $D^K_{[T_c]}(\omega,k)$ et $D^H(\omega,k)$ à $k$ fixé par la condition initiale. Nous ne conservons que le $\omega\in\mathbb{C}$ avec la plus grande partie imaginaire. 
Cette racine est calculée numériquement à l'aide d'une méthode de Newton. On étudie maintenant la convergence des $\omega_K$ (zéro de $(D^K(\omega,k)$) vers $\omega_H$ (zéro de $(D^H(\omega,k)$). La convergence de $\omega_K(T_c)$ vers $\omega_H$ est visible sur la figure \ref{fig:omega}, où l'on représente, en échelle log-log le module de la différence des deux zéros $\Delta \omega = |\omega_K-\omega_H|$.
\begin{figure}[h!]
  \centering
  \includegraphics[width=0.8\textwidth]{\localPath/figures/omega.png}
  \caption{Convergence des zéros de la relation de dispersion cinétique vers la solution hybride}
  \label{fig:omega}
\end{figure}
On observe une convergence d'ordre $1$ en $T_c$ des zéros de la relation de dispersion, aucun argument théorique sur les fonctions holomorphes ne vient appuyer ce résultat, contrairement à ce qui a été énoncé pour l'énergie totale.

La racine de plus grande partie imaginaire permet de valider la phase linéaire du code. Cette phase linéaire peut être rendue plus longue en considérant une valeur très faible de la perturbation $\epsilon=10^{-4}$ dans les conditions initiales~\eqref{eq:K:init} et~\eqref{eq:HL:init}. Ceci va nous permettre de vérifier, non seulement le taux d'instabilité, mais aussi, grâce aux calculs de la section \ref{s:dispersion}, l'énergie électrique. Nous pouvons donc comparer pour $\alpha = 0.1$, $T_c=0.1$, $N_x=135$, $N_v=1200$, $T_f=200$ et $\Delta t = 0.5\Delta x$ ce régime linéaire sur la figure~\ref{fig:limit:ee:Tf200}\subref{fig:limit:ee:Tf200:eps10m4}. Un résultat similaire est observable pour différentes températures ainsi que sur le modèle hybride linéarisé, comme l'illustre la figure~\ref{fig:limit:ee:Tf200}\subref{fig:limit:ee:Tf200:cmp_zoom}. Les reconstructions de l'énergie électrique se font à partir des relations de dispersion, avec l'équation~\eqref{eq:enelec}.
\begin{figure}
  \centering
  \begin{subfigure}{0.8\textwidth}
    \centering
    \includegraphics[width=\textwidth]{\localPath/figures/limit_ee_Tf200_eps10m4.png}
    \caption{Énergie électrique jusqu'au temps $200$ avec un régime linéaire très long, et comparaison avec les résultats données par les relations de dispersion.}
    \label{fig:limit:ee:Tf200:eps10m4}
  \end{subfigure}
  \begin{subfigure}{0.8\textwidth}
    \centering
    \includegraphics[width=\textwidth]{\localPath/figures/limit_ee_Tf200_cmp_zoom.png}
    \caption{Énergie électrique entre les temps $198$ et $200$ pour les températures $T_c = 0.1,0.05$ et le modèle hybride, et comparaison avec les résultats des relations de dispersion.}
    \label{fig:limit:ee:Tf200:cmp_zoom}
  \end{subfigure}
  \caption{Évolution de l'énergie électrique dans une longue phase linéaire et comparaison avec les relations de dispersion.}
  \label{fig:limit:ee:Tf200}
\end{figure}

En plus du taux d'instabilité de l'énergie électrique, il est possible, à l'aide des relations de dispersion, d'obtenir une très bonne approximation de l'énergie électrique dans la phase linéaire. On peut voir que l'étude des relations de dispersion ne permet pas d'obtenir des résultats fiables en début de simulation, où d'autres modes que le mode principal sont encore visibles (modes évanescents). De même, comme on peut le voir sur la figure~\ref{fig:limit_ee_Tf300}, la phase non-linéaire où l'énergie électrique atteint une saturation mélange de nombreux modes, ce qui est incompatible avec l'étude du linéarisé. Néanmoins, même au temps $t\approx 200$, les résultats du code sont en excellent accord avec ceux obtenus grâce aux relations de dispersion  (voir figure \ref{fig:limit:ee:Tf200:cmp_zoom}). 

\FloatBarrier
%----------
\subsection{Évolution avec la densité de particules chaudes}
%----------

Nous avons validé les modèles et les relations de dispersion lorsque la température des particules froides $T_c$ tend vers $0$ ; la proposition~\ref{p:limit:convergence} nous indique que la convergence s'effectue en $(1-\alpha)T_c|\Omega|$ où $\alpha$ est la densité des particules chaudes. Nous traçons sur la figure~\ref{fig:limit:slope:alpha} l'évolution du taux d'instabilité donné par les relations de dispersion (racine de plus grande partie imaginaire) en fonction de $\alpha$ et pour différentes valeurs de $T_c$. Cette évolution est représentée pour le modèle cinétique avec différentes températures, et  pour le modèle hybride, avec comme condition initiale pour les particules chaudes :
$$
  f_h^0(x,v) = \left(\mathcal{M}_{^\alpha/_2,4,1}(v) + \mathcal{M}_{^\alpha/_2,-4,1}(v)\right)(1+\epsilon\cos\left(k x\right)), \;\; x\in [0, 4\pi]. 
$$
La condition initiale du modèle cinétique est donnée par : $f^0(x,v) = \mathcal{M}_{1-\alpha,0,T_c}(v) + f_h^0(x,v)$.
\begin{figure}[h!]
  \centering
  \includegraphics[width=0.8\textwidth]{\localPath/figures/limit_slope_alpha.png}
  \caption{Évolution de la pente du développement de l'instabilité (ou taux d'instabilité) donnée par les relations de dispersion en fonction de la densité de particules chaudes $\alpha$}
  \label{fig:limit:slope:alpha}
\end{figure}
On retrouve sur la figure~\ref{fig:limit:slope:alpha} la convergence en température du modèle cinétique vers le modèle hybride. Pour $\alpha=0$ la condition initiale se restreint aux particules froides, qui ne sont pas perturbées, il est donc normal d'obtenir une pente nulle ; pour $\alpha=1$, il n'y a que des particules chaudes et on retrouve l'instabilité double faisceau (TSI) avec le bon taux d'instabilité. On peut enfin observer que pour ce choix de $T_c$, les taux d'instabilité obtenus restent proches de ceux du modèle VHL pour $0\leq \alpha \leq 0.5$ (qui correspond à une population identique de particules chaudes et froides).

\FloatBarrier

%% section 6
% !TEX root = ../chap2.tex

\section{Comparaison des deux résolutions hybrides}
\label{s:compare}

Dans cette section on s'intéressera à la comparaison des méthodes de simulation présentées dans la section~\ref{s:scheme} pour 
approcher num\'eriquement le mod\`ele VHL. On \'etudiera en particulier les méthodes de pas de temps adaptatif associées. 
Nous utilisons dans cette section la condition initiale suivante :
$$
  \begin{aligned}
    u_c(x)   &= 0 \\
    f_h(x,v) &=  \left(\mathcal{M}_{^\alpha/_2,v_0,1}(v) +  \mathcal{M}_{^\alpha/_2,-v_0,1}(v) \right)(1 + \epsilon\cos(kx))
  \end{aligned}
$$
avec $k=0.5$, $\alpha=0.2$, $v_0 = 4$, $x\in [0,4\pi]$, $v\in[-8,8]$, et la perturbation $\epsilon = 0.1$. Le champ électrique initiale $E(t=0,x)$ est obtenu en résolvant l'équation de Poisson sur notre condition initiale, comme indiqué dans la proposition~\ref{p:vhl_conservation} :
$$
  \partial_x E(t=0) = (1-\alpha) + \int (1+\epsilon\cos(kx))\left( \mathcal{M}_{^\alpha/_2,v_0,1}(v) + \mathcal{M}_{^\alpha/_2,-v_0,1}(v) \right)\,\mathrm{d}v - 1
$$
La discrétisation du domaine s'effectue avec $N_x=81$ dans la direction $x$, et $N_v=128$ points dans la direction $v$.

Nous allons effectuer différentes configurations pour tester et comparer les deux m\'ethodes 
\begin{enumerate}
\item pas de temps fixe 
\begin{itemize}
\item  $\Delta t = 0.5\Delta v$ 
\item  $\Delta t = \sigma\frac{\Delta t}{\|E^n\|_\infty}$ où $\sigma \approx 1.433$ est la condition CFL de WENO5 avec DP4, et $\|E^n\|_\infty = \max_{t^n}(\max_i|E_i^n|) \approx 0.2$ d'après une estimation issue des résultats d'une simulation ; 
\item  $\Delta t = 1$ qui est un pas de temps choisi arbitrairement grand pour illustrer l'absence de condition CFL de la méthode de Suzuki. 
\end{itemize}
\item pas de temps adaptatif
\begin{itemize}
\item méthode présentée dans la section~\ref{ssec:dtadapt} avec une tolérance $tol = 10^{-4}$, 
\item méthode basée sur la condition de CFL en choisissant $\Delta t^n = \min\left(\sigma\frac{\Delta v}{\|E^n_i\|_\infty},2\right)$ où $\sigma \approx 1.433$, et $\|E_i^n\|_\infty = \max_i\left( |E^n_i|\right)$.
\end{itemize}
\end{enumerate}
Quelque soit la méthode de simulation choisie, nous allons regarder les estimateurs d'erreur présentées dans la section~\ref{ssec:dtadapt} :
$$
  \begin{aligned}
    L &= \left(\sum_i (\left.u_c\right.^{[4]}_i-\left.u_c\right.^{[3]}_i)^2\Delta x\right)^{\frac{1}{2}}
          + \left(\sum_i (\left.E\right.^{[4]}_i-\left.E\right.^{[3]}_i)^2\Delta x\right)^{\frac{1}{2}}
          + \left(\sum_{i,j} \left|\left.{f}_h\right.^{[4]}_{i,j}-\left.{f}_h\right.^{[3]}_{i,j}\right|^2\Delta v\Delta x\right)^{\frac{1}{2}} \\
      &= L_{u_c} + L_{E} + L_{{f}_h}
  \end{aligned}
$$
Cela permettra de comparer l'estimation de l'erreur avec une même tolérance entre la méthode DP4(3) et la méthode de Suzuki. Nous allons aussi regarder le nombre d'itération de chaque méthode de simulation, et la taille des pas de temps que la méthode de pas de temps adaptatif propose. Il est à noter que uniquement les deux méthodes de la section~\ref{ssec:dtadapt} utilisent cette estimation d'erreur et ont un critère pour rejeter une itération.

\commentaire[Nicolas]{Figure 9 : en faire 2 figure (Suzuki et DP par exemple ?). Reprendre les commentaires en fonction du coup. }
\begin{figure}[h]
  \centering
  \includegraphics[width=0.8\textwidth]{\localPath/figures/compare_ee.png}
  \caption{Évolution de l'énergie électrique avec les différentes méthodes}
  \label{fig:compare:ee}
\end{figure}

Sur la figure~\ref{fig:compare:ee} on peut voir l'évolution de l'énergie électrique au cours du temps pour les différentes simulations. Il est important de remarquer que 4 résultats sont différents de ce que prédit la relation de dispersion, la courbe rouge, qui est un résultat donné par la méthode DP4 avec un pas de temps $\Delta t=1$ hors de la condtion CFL, le schéma est instable ; la courbe jaune, dont les résultats sont donnés par la méthode DP4 avec un pas de temps $\Delta t=1.433\Delta v/\|E\|_\infty$ est l'illustration d'un schéma utilisé juste sous sa condition de CFL, le schéma est stable mais il n'est pas précis ; les courbes violettes et vertes sont issus des résultats respectivement de DP4 et Suzuki avec une méthode de pas de temps adaptatif donné par la condition de CFL, cette méthode de pas de temps adaptatif propose de très grand pas de temps dans la phase linéaire du problème où le champ électrique est faible, mais produit une erreur importante, qui se répercute sur les résultats en fin de simulation. Les autres résultats sont conformes aux relations de dispersion. 

\begin{figure}[h]
  \centering
  \includegraphics[width=0.8\textwidth]{\localPath/figures/compare_time_dtt}
  \caption{Évolution du temps au cours des itérations de la simulation}
  \label{fig:compare:time:dtt}
\end{figure}

Sur la figure~\ref{fig:compare:time:dtt} on représente l'évolution du temps courant dans la simulation en fonction des itérations. Les courbes en pointillés sont celles avec un pas de temps constant. On remarque que les simulations où le pas de temps est guidé par la condition de CFL locale sont parmi les plus performantes pour atteindre le temps final, mais comme nous avons pu le voir sur la figure précédente~\ref{fig:compare:ee}, ce sont aussi celles qui donnent des résultats avec une très mauvaise précision. Les simulations avec un pas de temps constant très grand, donnent à première vue de bon résultat avec un intégrateur en temps sans condition de CFL comme Suzuki, mais de mauvais résultats avec la méthode DP4. La méthode DP4(3) s'impose ensuite comme un très bon choix pour maintenir l'erreur sous une certaine tolérance tout en atteignant rapidement le temps final, ici $T_f=100$. Les résultats pour la méthodes de Suzuki à pas de temps adaptatif ou les simulations avec un pas de temps pris arbitrairement petit $\Delta t = 0.5\Delta v$, ont été tronqués pour mieux observer les simulations plus rapide, la méthode de Suzuki termine en 1290 itérations, et les simulations avec un pas de temps fixé à $\Delta t = 0.5\Delta v$ en 1600 itérations.
commentaire[Nicolas]{Je ne comprends pas bien la fin : pourquoi 2 m\'ethodes ont \'et\'e "tronqu\'ees" ? }

\begin{figure}[h]
  \centering
  \includegraphics[width=0.8\textwidth]{\localPath/figures/compare_dt_cfl.png}
  \caption{Évolution de la taille du pas de temps $\Delta t^n$ au cours du temps, les itérations rejetées sont notées à l'aide des carrés}
  \label{fig:compare:dt:all}
\end{figure}

Il est intéressant de regarder, pour des méthodes à pas de temps adaptatif, l'évolution de la taille du pas de temps. C'est ce qui est tracé sur la figure~\ref{fig:compare:dt:all} en fonction du temps, les itérations rejetées par un critère d'erreur n'ont pas été représentées sur cette figure. Les simulations dont le pas de temps est donné par la condition de CFL locale $\sigma \Delta v/(\max |E^n|)$ (en violet et en vert sur la figure) sont soumis à de très importantes variations dans le choix du pas de temps, de plus, pour éviter des erreurs trop importantes en début de simulation, il a été choisi de toujours prendre un pas de temps inférieur à 2, cette forte variabilité dans le choix du pas de temps implique une erreur relativement importante dans les résultats. La méthode DP4(3) propose des pas de temps autour de $0.5$ et propose des pas de temps plus important dans la phase linéaire (entre les temps 5 et 40). La méthode de Suzuki, alors qu'elle n'est soumis à aucune condition de stabilité, propose des pas de temps bien plus faible, jusqu'à proposer des pas de temps similaire au choix arbitraire $\Delta t = 0.5\Delta v$, et ce avant l'installation de la saturation de l'énergie électrique au temps 40. Sur la figure~\ref{fig:compare:dt} on représente à la fois les itérations acceptées et rejetées par le critère d'erreur, on remarque que le pas de temps des itérations rejetées reste du même ordre de grandeur que les autres itérations. Les rejets d'itérations ont toujours lieu lors des rebonds de l'énergie électrique, pour éviter trop de rejets de la sorte, il est possible de limiter les évolutions du pas de temps de l'itération suivante avec par exemple $\Delta t^{n+1} \in [0.5\Delta t^n,2\Delta t^n]$. \commentaire[Nicolas]{Dire que si on rejete moins, on va plus vite...)}
\commentaire[Nicolas]{Il faudrait parler des courbes "fail" de la figure 11 ; j'imagine que ce sont les iterations rejetees ?}


\begin{figure}[h]
  \centering
  \includegraphics[width=0.8\textwidth]{\localPath/figures/compare_dt.png}
  \caption{Évolution de la taille du pas de temps $\Delta t^n$ au cours du temps, les itérations rejetées sont notées à l'aide des carrés}
  \label{fig:compare:dt}
\end{figure}

On peut regarder l'évolution de l'erreur estimée $L$ au cours du temps de la simulation sur la figure~\ref{fig:compare:error_dtt}, ici encore les itérations rejetées par le critère sont représentées différemment. Les deux méthodes effectuent dans les itérations acceptées une erreur similaire qui semble tourner autour de $0.5 tol$. On peut également combiner les résultats de la figure~\ref{fig:compare:dt} avec ceux de la figure~\ref{fig:compare:error_dtt} pour tracer le nuage de points de l'erreur commise à chaque itération en fonction du pas de temps proposé par la méthode, ce que l'on trace sur la figure~\ref{fig:compare:error:dt}. On remarque que de manière général la méthode DP4(3) propose des pas de temps plus grand, et pour chaque méthode, ce ne sont pas les itérations avec les plus grands pas de temps qui sont rejetées. La méthode de Suzuki semble ne jamais effectuer une erreur qui excède $2tol$.

\begin{figure}[h]
  \centering
  \includegraphics[width=0.8\textwidth]{\localPath/figures/compare_error_dtt.png}
  \caption{Étude de l'erreur $L$ au cours du temps, l'erreur des itérations rejetées est notée à l'aide des carrés}
  \label{fig:compare:error_dtt}
\end{figure}

\begin{figure}[h]
  \centering
  \includegraphics[width=0.8\textwidth]{\localPath/figures/compare_error_dt.png}
  \caption{Comparaison de l'erreur commise en fonction de la taille du pas de temps}
  \label{fig:compare:error:dt}
\end{figure}

Regardons, toujours pour ces 2 méthodes, l'évolution de l'erreur au cours du temps, mais en regardant la contribution dans $L$ d\^ue à $L_{u_c}$, $L_E$ et $L_{\hat{f}_h}$. C'est ce que l'on représente dans la figure~\ref{fig:compare:error:LucEfh}. L'erreur de la méthode DP4(3) et ses différentes contributions est représentée en haut, alors que celle de la méthode de Suzuki est représentée en bas. La méthode de Lawson propose de résoudre exactement la partie linéaire du problème, dans notre cas seul $u_c$ est entièrement résolu par la partie linéaire, ce qui explique une contribution à l'erreur de l'ordre de $10^{-18}$ pour $L_{u_c}$. La partie non linéaire comprend le calcul du courant de $f_h$ dans $L_E$, et le transport dans la direction $v$ dans $L_{\hat{f}_h}$, ces 2 composantes restent élevées tout au long de la simulation. Pour la méthode de Suzuki, l'erreur provient essentiellement de $L_{\hat{f}_h}$, ce qui est  li\'e \`a l'erreur 
produite par la méthode Lagrange 5 pour la résolution du transport dans la direction $v$.

\begin{figure}[h]
  \centering
  \includegraphics[width=0.8\textwidth]{\localPath/figures/compare_error_LucEfh.png}
  \caption{Comparaison de l'erreur au cours du temps et de la contribution de chaque composante de l'erreur}
  \label{fig:compare:error:LucEfh}
\end{figure}

%\commentaire[Nicolas]{Il faut expliquer comment tu calcules l'erreur pour les methodes non-embedded. J'imagine que tu calcules l'erreur locale 
%comme avec le pas de temps adaptatif mais sans changer le pas de temps avec la formule ? 
%On peut aussi regarder l'évolution de l'erreur sur toutes les simulations effectuées, quelque soit la méthode d'intégration en temps et le choix du pas de temps. C'est ce qui est effectué sur la figure~\ref{fig:compare:error:dtc} où l'on trace l'évolution de l'erreur en échelle semi-log. 
\commentaire[Nicolas]{Quelque soit le choix du pas de temps, on peut utiliser les estimateurs d'erreur locale pour \'etudier l'\'evolution de l'erreur au cours du temps 
pour les diff\'erentes m\'ethodes. C'est l'objet de la figure~\ref{fig:compare:error:dtc} où l'on trace l'évolution de l'erreur en échelle semi-log 
pour diff\'erentes m\'ethodes ainsi que la tol\'erance $10^{-4}$.}
On remarque que l'erreur commise par la méthode de Suzuki et DP4(3) est bien plus faible que les autres méthodes, et ce d'environ deux ordres de grandeur. Il n'y a que les simulations avec un pas fixe $\Delta t = 0.5\Delta v$ qui sont également sous la tolérance  fixé \`a 
$10^{-4}$. Donc sans connaissance a priori du problème, les méthodes à pas de temps adaptatif présentées dans la section~\ref{ssec:dtadapt} sont très intéressantes. Nous pouvons aussi remarquer que la constante d'erreur en temps de la méthode de Suzuki est bien plus importante que la méthode DP4(3) en regardant pour des simulations avec un pas de temps fixe l'erreur commise (par exemple les courbes en pointillées), ceci explique pourquoi la méthode DP4(3) propose des pas de temps plus important pour la même tolérance.

\begin{figure}[h]
  \centering
  \includegraphics[width=0.8\textwidth]{\localPath/figures/compare_error_dtc.png}
  \caption{Comparaison de l'erreur obtenue à l'aide du temps de temps adaptatif et d'un pas de temps constant}
  \label{fig:compare:error:dtc}
\end{figure}



\commentaire[Nicolas]{Tracer l'energie totale pour differentes methods (pas de temps fixe Suzuki(dt=1 et dt=2) et DP4(3) + Suzuki en dt adaptatif) }
\commentaire[Nicolas]{temps de calcul par iteration et par etape de calcul. }


%% section 7
% !TEX root = ../../main.tex

\section{Conclusion}
% -------------------------------------------------------------------

Ce chapitre est une extension de ce qui a pu être effectué dans le chapitre~\ref{chap2} au cas $1dz-3dv$ mais a nécessité de nombreux nouveaux développements, entre autre à cause du passage de 3 à 7 inconnues. Nous avons profité de l'analyse faite dans le chapitre précédent pour étudier le modèle hybride en dimensions supérieures et ses deux méthodes de résolution. La première méthode de résolution, méthode de \emph{splitting}, profite de la structure hamiltonienne du système qui n'avait pas été exploitée jusque là dans la littérature, mais le nombre d'étapes supplémentaires ne rend pas la méthode viable d'un point de vue numérique. La seconde méthode de résolution est la méthode de Lawson, et celle-ci permet une augmentation du coût numérique linéaire par rapport à l'ordre en temps choisi. Plusieurs travaux ont permis d'améliorer la méthode de Lawson à ce contexte. Il est possible de se défaire d'une contrainte de stabilité à l'aide d'un filtrage. Ensuite il n'est pas possible de calculer formellement l'exponentielle de la partie linéaire, deux options ont été proposées : tout d'abord mettre certains termes dans la partie linéaire, engendrant une condition de stabilité restrictive ; ensuite effectuer une approximation de l'exponentielle à l'aide d'approximations telles que les méthodes de Taylor ou Padé, permettant de lever les conditions de stabilités en espace. Ainsi nous avons pu développer une méthode de pas de temps adaptatif proposant de grands pas de temps dans la phase linéaire sans nuire à la stabilité de la méthode.

Des comparaisons avec le code PIC (Particle-In-Cell) développé au NMPP Garching sur la base des travaux~\cite{Holderied:2020}sont envisagées. La méthode utilisée dans ce code est basée sur le formalisme GEMPIC qui exploite la structure hamiltonienne du modèle hybride pour construire une méthode PIC géométrique (voir~\cite{Kraus:2017}). L’approximation en temps utilise un splitting hamiltonien qui implique que les équations de Maxwell sont résolues de manière explicite entrainant donc une contrainte de stabilité. Dans notre approche, nous n’avons pas cette contrainte grâce à l’utilisation d’approximants de Padé. Il serait donc intéressant de transférer les techniques développées ici au système d’EDO obtenu avec le formalisme GEMPIC pour potentiellement permettre d’utiliser des pas de temps plus grands. Un des points importants sera de gérer des matrices de taille importante (typiquement la taille est liée au nombre de particules $N_p\approx 10^5, 10^7$) et la question de la viabilité de l’approche par approximant de Padé dans ce contexte est liée à cette gestion de grande matrice. Plusieurs travaux existent néanmoins~\cite{Li:2011} qui permettent d'éviter l'inversion de matrice, étape potentiellement compliquée si celle-ci est de grande taille.

\vfill

\begin{otherlanguage}{english}
Acknowledgment: Experiments presented in this section were carried out using the PlaFRIM experimental testbed, supported by Inria, CNRS (LABRI and IMB), Université de Bordeaux, Bordeaux INP and Conseil Régional d’Aquitaine (see \url{https://www.plafrim.fr/}).
\end{otherlanguage}


% annexe
\begin{subappendices}
% !TEX root = ../../main.tex

\section{Résultats sur les relations de dispersion}
\label{a:dispersion}

Cette annexe est dédiée aux démonstrations des propriétés énoncées dans la section \ref{s:dispersion} sur les relations de dispersion.

Nous démontrons tout d'abord le lemme \ref{lemma:doubleracine}, qui concerne la symétrie des racines de $D(k,\omega)$, et dont l'énoncé est rappelé ci-dessous.
\begin{lemma}
  Si $f^{(0)}(v)$ (respectivement $f_h^{(0)}(v)$) est une fonction paire, alors pour $D(k,\omega)$ défini par (\ref{eq:D}) (respectivement (\ref{eq:relD_H})) nous avons $D(k,\omega_r+i\omega_i) = 0 \Leftrightarrow D(k,-\omega_r+i\omega_i)=0$.
\end{lemma}

\begin{proof}
  Nous le vérifions dans le cas cinétique, les calculs étant similaires dans le cas hybride. Avec la définition (\ref{eq:D}) de $D(k,\omega)$, nous avons 
  \begin{eqnarray*}
    &&D(k,\omega_r+i\omega_i)=0\\
    &\Leftrightarrow&
    \Re\left(\frac{1}{k^2}\int_\gamma\frac{\partial_vf^0}{v-\frac{\omega_r+i\omega_i}{k}}dv\right)=1,~\Im\left(\frac{1}{k^2}\int_\gamma\frac{\partial_vf^0}{v-\frac{\omega_r+i\omega_i}{k}}dv\right)=0. 
  \end{eqnarray*}
  Distinguons les parties réelles et imaginaires :
  \begin{eqnarray*}
    \int_\gamma\frac{\partial_vf^0(v)}{v-\frac{\omega_r+i\omega_i}{k}}dv=\int_\gamma\frac{\partial_vf^0(v)}{\left(v-\frac{\omega_r+i\omega_i}{k}\right)\left(v-\frac{\omega_r-i\omega_i}{k}\right)}\left(v-\frac{\omega_r-i\omega_i}{k}\right)dv\\
    =\int_\gamma\frac{\partial_vf^0(v)}{\left(v-\frac{\omega_r}{k}\right)^2+\left(\frac{\omega_i}{k}\right)^2}\left(v-\frac{\omega_r}{k}\right)dv+i\int_\gamma\frac{\partial_vf^0(v)}{\left(v-\frac{\omega_r}{k}\right)^2+\left(\frac{\omega_i}{k}\right)^2}\frac{\omega_i}{k}dv.
  \end{eqnarray*}
  Maintenant, considérons $\omega=-\omega_r+i\omega_i$ et rappelons qu'on a supposé que $f^0(v)$ était une fonction paire. Nous obtenons
  \begin{eqnarray*}
    \int_\gamma\frac{\partial_vf^0(v)}{v-\frac{-\omega_r+i\omega_i}{k}}dv~~~~~~~~~~~~~~~~~~~~~~~~~~~~~~~~~~~~~~~~~~~~~~~~~~~~~~~~~~~~~~~~~~~~~~~~~~~~~~~~\\
    =\int_\gamma\frac{\partial_vf^0(v)}{\left(v+\frac{\omega_r}{k}\right)^2+\left(\frac{\omega_i}{k}\right)^2}\left(v+\frac{\omega_r}{k}\right)dv+i\int_\gamma\frac{\partial_vf^0(v)}{\left(v+\frac{\omega_r}{k}\right)^2+\left(\frac{\omega_i}{k}\right)^2}\frac{\omega_i}{k}dv~~~~\\
    =-\int_\gamma\frac{\partial_{v}f^0(-v)}{\left(-v+\frac{\omega_r}{k}\right)^2+\left(\frac{\omega_i}{k}\right)^2}\left(v-\frac{\omega_r}{k}\right)dv+i\int_\gamma\frac{\partial_vf^0(-v)}{\left(-v+\frac{\omega_r}{k}\right)^2+\left(\frac{\omega_i}{k}\right)^2}\frac{\omega_i}{k}dv\\
    =\int_\gamma\frac{\partial_vf^0(v)}{\left(v-\frac{\omega_r}{k}\right)^2+\left(\frac{\omega_i}{k}\right)^2}\left(v-\frac{\omega_r}{k}\right)dv-i\int_\gamma\frac{\partial_vf^0(v)}{\left(v-\frac{\omega_r}{k}\right)^2+\left(\frac{\omega_i}{k}\right)^2}\frac{\omega_i}{k}dv~~~~.
  \end{eqnarray*}
  D'où
  \begin{eqnarray*}
    \Re\left(\frac{1}{k^2}\int_\gamma\frac{\partial_vf^0}{v-\frac{\omega_r+i\omega_i}{k}}dv\right)=1,~\Im\left(\frac{1}{k^2}\int_\gamma\frac{\partial_vf^0}{v-\frac{\omega_r+i\omega_i}{k}}dv\right)=0\\
    \Leftrightarrow
    \Re\left(\frac{1}{k^2}\int_\gamma\frac{\partial_vf^0}{v-\frac{-\omega_r+i\omega_i}{k}}dv\right)=1,~\Im\left(\frac{1}{k^2}\int_\gamma\frac{\partial_vf^0}{v-\frac{-\omega_r+i\omega_i}{k}}dv\right)=0
  \end{eqnarray*}
  et
  $$
    D(k,\omega_r+i\omega_i)=0\Leftrightarrow D(k,-\omega_r+i\omega_i)=0.
  $$
\end{proof}

Nous allons maintenant démontrer les lemmes \ref{lemma:Z0}, \ref{lemma:Z+} et \ref{lemma:Z-}, dont les énoncés sont rappelés ci-dessous, qui donnent des propriétés de la fonction de Fried-Conte (\ref{eq:Zfct}).

\begin{lemma}
  La fonction $Z_\alpha^0(\omega):\omega\in\mathbb{C}\mapsto Z\left(\alpha\omega\right)\in\mathbb{C}$, avec $\alpha\in\mathbb{R}$ fixé, est telle que : $Z_\alpha^0(-\bar{\omega}) = -\overline{Z_\alpha^0(\omega)}$.
\end{lemma}
 
\begin{proof}
  Par définition de la fonction de Fried-Conte, et avec la notation $\omega=\omega_r+i\omega_i$, nous avons
  \begin{eqnarray*}
    Z(\alpha(\omega_r+i\omega_i))=\frac{1}{\sqrt{\pi}}\int_\gamma\frac{e^{-z^2}}{z-\alpha(\omega_r+i\omega_i)}dz=\frac{1}{\sqrt{\pi}}\int_\gamma\frac{e^{-z^2}(z-\alpha\omega_r+i\alpha\omega_i)}{(z-\alpha\omega_r)^2+(\alpha\omega_i)^2}dz
  \end{eqnarray*}
  d'où
  \begin{eqnarray*}
    \Re\left(Z(\alpha(\omega_r+i\omega_i))\right)=\frac{1}{\sqrt{\pi}}\int_\gamma\frac{e^{-z^2}(z-\alpha\omega_r)}{(z-\alpha\omega_r)^2+(\alpha\omega_i)^2}dz\\
    \Im\left(Z(\alpha(\omega_r+i\omega_i))\right)=\frac{1}{\sqrt{\pi}}\int_\gamma\frac{e^{-z^2}\alpha\omega_i}{(z-\alpha\omega_r)^2+(\alpha\omega_i)^2}dz.
  \end{eqnarray*}

  Maintenant, $-\overline{\omega}=-\omega_r+i\omega_i$, implique
  \begin{eqnarray*}
    Z(\alpha(-\omega_r+i\omega_i))&=&\frac{1}{\sqrt{\pi}}\int_\gamma\frac{e^{-z^2}}{z-\alpha(-\omega_r+i\omega_i)}dz\\
    &=&\frac{1}{\sqrt{\pi}}\int_\gamma\frac{e^{-z^2}(z+\alpha\omega_r+i\alpha\omega_i)}{(z+\alpha\omega_r)^2+(\alpha\omega_i)^2}dz\\
    &=&\frac{1}{\sqrt{\pi}}\int_\gamma\frac{e^{-z^2}(-z+\alpha\omega_r+i\alpha\omega_i)}{(-z+\alpha\omega_r)^2+(\alpha\omega_i)^2}dz\\
    &=&-\frac{1}{\sqrt{\pi}}\int_\gamma\frac{e^{-z^2}(z-\alpha\omega_r)}{(z-\alpha\omega_r)^2+(\alpha\omega_i)^2}dz+i\frac{1}{\sqrt{\pi}}\int_\gamma\frac{e^{-z^2}\alpha\omega_i}{(z-\alpha\omega_r)^2+(\alpha\omega_i)^2}dz
  \end{eqnarray*}
  d'où
  \begin{eqnarray*}
    \Re\left(Z(\alpha(-\omega_r+i\omega_i))\right)=-\Re\left(Z(\alpha(\omega_r+i\omega_i))\right)\\
    \Im\left(Z(\alpha(-\omega_r+i\omega_i))\right)=\Im\left(Z(\alpha(\omega_r+i\omega_i))\right),
  \end{eqnarray*}
  ce qui termine la preuve.
\end{proof}


\begin{lemma}
  La fonction $Z_{\alpha,\beta}^+(\omega):\omega\in\mathbb{C}\mapsto Z\left(\alpha\omega-\beta\right)+Z\left(\alpha\omega+\beta\right)\in\mathbb{C}$, avec $\alpha\in\mathbb{R}$, $\beta\in\mathbb{R}$ fixés, est telle que : $Z_{\alpha,\beta}^+\left(-\overline{\omega}\right)=-\overline{Z_{\alpha,\beta}^+(\omega)}$.
\end{lemma}
  
\begin{proof}
  Nous avons par définition de la fonction de Fried-Conte
  \begin{eqnarray*}
    &&Z(\alpha\omega-\beta)+Z(\alpha\omega+\beta)=\frac{1}{\sqrt{\pi}}\int_\gamma\frac{e^{-z^2}}{z-\alpha\omega+\beta}+\frac{e^{-z^2}}{z-\alpha\omega-\beta}dz\\
    &=&\frac{1}{\sqrt{\pi}}\int_\gamma\frac{e^{-z^2}(z-\alpha\omega-\beta)+e^{-z^2}(z-\alpha\omega+\beta)}{(z-\alpha\omega)^2-\beta^2}dz\\
    &=&\frac{2}{\sqrt{\pi}}\int_\gamma\frac{e^{-z^2}(z-\alpha\omega)}{(z-\alpha\omega)^2-\beta^2}dz.
  \end{eqnarray*}
  Maintenant, avec la notation $\omega=\omega_r+i\omega_i$, nous avons
  \begin{eqnarray*}
    &&Z(\alpha(\omega_r+i\omega_i)-\beta)+Z(\alpha(\omega_r+i\omega_i)+\beta)=\frac{2}{\sqrt{\pi}}\int_\gamma\frac{e^{-z^2}(z-\alpha\omega_r-i\alpha\omega_i)}{(z-\alpha\omega_r-i\alpha\omega_i)^2-\beta^2}dz\\
    &=&\frac{2}{\sqrt{\pi}}\int_\gamma\frac{e^{-z^2}(z-\alpha\omega_r-i\alpha\omega_i)}{(z-\alpha\omega_r)^2-(\alpha\omega_i)^2-\beta^2-2i\alpha\omega_i(z-\alpha\omega_r)}dz\\
    &=&\frac{2}{\sqrt{\pi}}\int_\gamma\frac{e^{-z^2}(z-\alpha\omega_r-i\alpha\omega_i)\left((z-\alpha\omega_r)^2-(\alpha\omega_i)^2-\beta^2+2i\alpha\omega_i(z-\omega_r)\right)}{\left((z-\alpha\omega_r)^2-(\alpha\omega_i)^2-\beta^2\right)^2+4\left(\alpha\omega_i\right)^2(z-\alpha\omega_r)^2}dz\\
    &=&\frac{2}{\sqrt{\pi}}\int_\gamma\frac{e^{-z^2}\left((z-\alpha\omega_r)\left((z-\alpha\omega_r)^2-(\alpha\omega_i)^2-\beta^2\right)+2(\alpha\omega_i)^2(z-\alpha\omega_r)\right)}{\left((z-\alpha\omega_r)^2-(\alpha\omega_i)^2-\beta^2\right)^2+4\left(\alpha\omega_i\right)^2(z-\alpha\omega_r)^2}dz\\
    &+&i\frac{2}{\sqrt{\pi}}\int_\gamma\frac{e^{-z^2}\left(2\alpha\omega_i(z-\alpha\omega_r)^2-\alpha\omega_i\left((z-\alpha\omega_r)^2-(\alpha\omega_i)^2-\beta^2\right)\right)}{\left((z-\alpha\omega_r)^2-(\alpha\omega_i)^2-\beta^2\right)^2+4\left(\alpha\omega_i\right)^2(z-\alpha\omega_r)^2}dz\\
    &=&\frac{2}{\sqrt{\pi}}\int_\gamma\frac{e^{-z^2}(z-\alpha\omega_r)\left((z-\alpha\omega_r)^2+(\alpha\omega_i)^2-\beta^2\right)}{\left((z-\alpha\omega_r)^2-(\alpha\omega_i)^2-\beta^2\right)^2+4\left(\alpha\omega_i\right)^2(z-\alpha\omega_r)^2}dz\\
    &+&i\frac{2}{\sqrt{\pi}}\int_\gamma\frac{e^{-z^2}\alpha\omega_i\left((z-\alpha\omega_r)^2+(\alpha\omega_i)^2+\beta^2\right)}{\left((z-\alpha\omega_r)^2-(\alpha\omega_i)^2-\beta^2\right)^2+4\left(\alpha\omega_i\right)^2(z-\alpha\omega_r)^2}dz.
  \end{eqnarray*}
  Par ailleurs, en considérant $-\overline{\omega}=-\omega_r+i\omega_i$, nous avons
  \begin{eqnarray*}
    &&Z(\alpha(-\omega_r+i\omega_i)-\beta)+Z(\alpha(-\omega_r+i\omega_i)+\beta)\\
    &=&\frac{2}{\sqrt{\pi}}\int_\gamma\frac{e^{-z^2}(z+\alpha\omega_r)\left((z+\alpha\omega_r)^2+(\alpha\omega_i)^2-\beta^2\right)}{\left((z+\alpha\omega_r)^2-(\alpha\omega_i)^2-\beta^2\right)^2+4\left(\alpha\omega_i\right)^2(z+\alpha\omega_r)^2}dz\\
    &+&i\frac{2}{\sqrt{\pi}}\int_\gamma\frac{e^{-z^2}\alpha\omega_i\left((z+\alpha\omega_r)^2+(\alpha\omega_i)^2+\beta^2\right)}{\left((z+\alpha\omega_r)^2-(\alpha\omega_i)^2-\beta^2\right)^2+4\left(\alpha\omega_i\right)^2(z+\alpha\omega_r)^2}dz.
  \end{eqnarray*}
  La seule fonction impaire en $z$ est $(z+\alpha\omega_r)$, qui apparaît dans la partie réelle, ainsi
  \begin{eqnarray*}
    &&Z(\alpha(-\omega_r+i\omega_i)-\beta)+Z(\alpha(-\omega_r+i\omega_i)+\beta)\\
    &=&-\frac{2}{\sqrt{\pi}}\int_\gamma\frac{e^{-z^2}(z-\alpha\omega_r)\left((z-\alpha\omega_r)^2+(\alpha\omega_i)^2-\beta^2\right)}{\left((z-\alpha\omega_r)^2-(\alpha\omega_i)^2-\beta^2\right)^2+4\left(\alpha\omega_i\right)^2(z-\alpha\omega_r)^2}dz\\
    &+&i\frac{2}{\sqrt{\pi}}\int_\gamma\frac{e^{-z^2}\alpha\omega_i\left((z-\alpha\omega_r)^2+(\alpha\omega_i)^2+\beta^2\right)}{\left((z-\alpha\omega_r)^2-(\alpha\omega_i)^2-\beta^2\right)^2+4\left(\alpha\omega_i\right)^2(z-\alpha\omega_r)^2}dz.
  \end{eqnarray*}
  L'identification des parties réelles et imaginaires de $Z(\alpha\omega-\beta)+Z(\alpha\omega+\beta)$ et $Z(-\alpha\overline{\omega}-\beta)+Z(-\alpha\overline{\omega}+\beta)$ achève la preuve.
\end{proof}


\begin{lemma}
  La fonction $Z_{\alpha,\beta}^-(\omega):\omega\in\mathbb{C}\mapsto Z\left(\alpha\omega-\beta\right)-Z\left(\alpha\omega+\beta\right)\in\mathbb{C}$, avec $\alpha\in\mathbb{R}$, $\beta\in\mathbb{R}$ fixés, est telle que : $Z_{\alpha,\beta}^-\left(-\overline{\omega}\right)=\overline{Z_{\alpha,\beta}^-(\omega)}$.
\end{lemma}

\begin{proof}
  Nous avons par définition de la fonction de Fried-Conte
  \begin{eqnarray*}
    &&Z(\alpha\omega-\beta)-Z(\alpha\omega+\beta)=\frac{1}{\sqrt{\pi}}\int_\gamma\frac{e^{-z^2}}{z-\alpha\omega+\beta}-\frac{e^{-z^2}}{z-\alpha\omega-\beta}dz\\
    &=&\frac{1}{\sqrt{\pi}}\int_\gamma\frac{e^{-z^2}(z-\alpha\omega-\beta)-e^{-z^2}(z-\alpha\omega+\beta)}{(z-\alpha\omega)^2-\beta^2}dz\\
    &=&-\frac{2}{\sqrt{\pi}}\int_\gamma\frac{e^{-z^2}\beta}{(z-\alpha\omega)^2-\beta^2}dz.
  \end{eqnarray*}
  Maintenant, avec la notation $\omega=\omega_r+i\omega_i$, nous avons
  \begin{eqnarray*}
    &&Z(\alpha(\omega_r+i\omega_i)-\beta)-Z(\alpha(\omega_r+i\omega_i)+\beta)=-\frac{2}{\sqrt{\pi}}\int_\gamma\frac{e^{-z^2}\beta}{(z-\alpha\omega_r-i\alpha\omega_i)^2-\beta^2}dz\\
    &=&-\frac{2}{\sqrt{\pi}}\int_\gamma\frac{e^{-z^2}\beta}{(z-\alpha\omega_r)^2-(\alpha\omega_i)^2-\beta^2-2i\alpha\omega_i(z-\alpha\omega_r)}dz\\
    &=&-\frac{2}{\sqrt{\pi}}\int_\gamma\frac{e^{-z^2}\beta\left((z-\alpha\omega_r)^2-(\alpha\omega_i)^2-\beta^2+2i\alpha\omega_i(z-\alpha\omega_r)\right)}{\left((z-\alpha\omega_r)^2-(\alpha\omega_i)^2-\beta^2\right)^2+4\left(\alpha\omega_i\right)^2(z-\alpha\omega_r)^2}dz
  \end{eqnarray*}
  Par ailleurs, avec $-\overline{\omega}=-\omega_r+i\omega_i$, nous avons
  \begin{eqnarray*}
    &&Z(\alpha(-\omega_r+i\omega_i)-\beta)-Z(\alpha(-\omega_r+i\omega_i)+\beta)\\
    &=&-\frac{2}{\sqrt{\pi}}\int_\gamma\frac{e^{-z^2}\beta\left((z+\alpha\omega_r)^2-(\alpha\omega_i)^2-\beta^2+2i\alpha\omega_i(z+\alpha\omega_r)\right)}{\left((z+\alpha\omega_r)^2-(\alpha\omega_i)^2-\beta^2\right)^2+4\left(\alpha\omega_i\right)^2(z+\alpha\omega_r)^2}dz
  \end{eqnarray*}
  La seule fonction impaire en $z$ est $(z+\alpha\omega_r)$, apparaissant dans la partie imaginaire, d'où
  \begin{eqnarray*}
    &&Z(\alpha(-\omega_r+i\omega_i)-\beta)-Z(\alpha(-\omega_r+i\omega_i)+\beta)\\
    &=&-\frac{2}{\sqrt{\pi}}\int_\gamma\frac{e^{-z^2}\beta\left((z-\alpha\omega_r)^2-(\alpha\omega_i)^2-\beta^2-2i\alpha\omega_i(z-\alpha\omega_r)\right)}{\left((z-\alpha\omega_r)^2-(\alpha\omega_i)^2-\beta^2\right)^2+4\left(\alpha\omega_i\right)^2(z-\alpha\omega_r)^2}dz
  \end{eqnarray*}
  L'identification des parties réelles et imaginaires de $Z(\alpha\omega-\beta)-Z(\alpha\omega+\beta)$ et $Z(-\alpha\overline{\omega}-\beta)-Z(-\alpha\overline{\omega}+\beta)$ achève la preuve.
\end{proof}

Nous pouvons enfin démontrer les lemmes \ref{lemme:hypcashyb} et \ref{lemme:hypcascin} concernant la vérification de l'hypothèse \ref{hyp:sym}, qui conduit à l'expression (\ref{eq:Etk}) du mode fondamental du champ électrique linéarisé puis à l'approximation (\ref{eq:enelec}) de l'énergie électrique linéarisée. Ces lemmes sont rappelés ci-dessous.

D'une part, nous rappelons le résultat \ref{lemme:hypcascin} dans le cas cinétique.
\begin{lemma}
  Pour $\frac{\partial D(k,\omega)}{\partial\omega}$ donnée par~\eqref{eq:3bumpderD} et $N(k,\omega)$ par~\eqref{eq:N_3bump}, l'hypothèse~\ref{hyp:sym} est satisfaite.
\end{lemma}

\begin{proof}
  En utilisant (\ref{eq:3bumpderD}) et les lemmes \ref{lemma:Z0}, \ref{lemma:Z+}, \ref{lemma:Z-} avec $\delta=\frac{1}{\sqrt{2T_c}k}$, $\eta=\frac{1}{\sqrt{2}k}$ et $\beta=\frac{v_0}{\sqrt{2}}$, nous avons
  \begin{eqnarray*}
    \frac{\partial D}{\partial \omega}(k,\omega)&=&\frac{1}{\sqrt{2}k^3}\left[\frac{1-\alpha}{T_c\sqrt{T_c}}\left(\left(1-\frac{\omega^2}{T_ck^2}\right)Z_\delta^0\left(\omega\right)-2\frac{\omega}{\sqrt{2T_c}k}\right)\right.\nonumber\\
    &&~~~~~~~~~~~~~+\frac{\alpha}{2}\left(\left(1-\left(\frac{\omega}{k}-v_0\right)^2\right)Z\left(\frac{1}{\sqrt{2}}\left(\frac{\omega}{k}-v_0\right)\right)\right.\nonumber\\
    &&~~~~~~~~~~~~~~~~~~+\left.\left.\left(1-\left(\frac{\omega}{k}+v_0\right)^2\right)Z\left(\frac{1}{\sqrt{2}}\left(\frac{\omega}{k}+v_0\right)\right)\right.\right.\nonumber\\
    &&~~~~~~~~~~~~~\left.\left.-\frac{2}{\sqrt{2}}\left(\frac{\omega}{k}-v_0\right)-\frac{2}{\sqrt{2}}\left(\frac{\omega}{k}+v_0\right)\right)\right]\nonumber\\
    &=&\frac{1}{\sqrt{2}k^3}\left[\frac{1-\alpha}{T_c\sqrt{T_c}}\left(\left(1-\frac{\omega^2}{T_ck^2}\right)Z_\delta^0\left(\omega\right)-2\frac{\omega}{\sqrt{2T_c}k}\right)-2\sqrt{2}\frac{\omega}{k}\right.\nonumber\\
    &&~~~~~~~~~~~~~\left.+\frac{\alpha}{2}\left((1-v_0^2)Z_{\eta,\beta}^+\left(\omega\right)-\frac{\omega^2}{k^2}Z_{\eta,\beta}^+\left(\omega\right)+2v_0\frac{\omega}{k}Z_{\eta,\beta}^-\left(\omega\right)\right)\right]
  \end{eqnarray*}
  et
  \begin{eqnarray*}
    \frac{\partial D}{\partial \omega}(k,-\overline{\omega})&=&\frac{1}{\sqrt{2}k^3}\left[\frac{1-\alpha}{T_c\sqrt{T_c}}\left(\left(1-\frac{(-\overline{\omega})^2}{T_ck^2}\right)Z_\delta^0\left(-\overline{\omega}\right)+2\frac{\overline{\omega}}{\sqrt{2T_c}k}\right)+2\sqrt{2}\frac{\overline{\omega}}{k}\right.\nonumber\\
    &&~~~~~~~~~~~~~\left.+\frac{\alpha}{2}\left((1-v_0^2)Z_{\eta,\beta}^+\left(-\overline{\omega}\right)-\frac{(-\overline{\omega})^2}{k^2}Z_{\eta,\beta}^+\left(-\overline{\omega}\right)-2v_0\frac{\overline{\omega}}{k}Z_{\eta,\beta}^-\left(-\overline{\omega}\right)\right)\right]\\
    &=&\frac{1}{\sqrt{2}k^3}\left[\frac{1-\alpha}{T_c\sqrt{T_c}}\left(-\left(1-\frac{\overline{\omega^2}}{T_ck^2}\right)\overline{Z_\delta^0\left(\omega\right)}+2\frac{\overline{\omega}}{\sqrt{2T_c}k}\right)+2\sqrt{2}\frac{\overline{\omega}}{k}\right.\nonumber\\
    &&~~~~~~~~~~~~~\left.+\frac{\alpha}{2}\left(-(1-v_0^2)\overline{Z_{\eta,\beta}^+\left(\omega\right)}+\frac{\overline{\omega^2}}{k^2}\overline{Z_{\eta,\beta}^+\left(\omega\right)}-2v_0\frac{\overline{\omega}}{k}\overline{Z_{\eta,\beta}^-\left(\omega\right)}\right)\right]\nonumber\\
    &=&-\overline{\frac{\partial D}{\partial \omega}(k,\omega)}.
  \end{eqnarray*}

  Maintenant, en utilisant (\ref{eq:N_3bump}) et le lemme \ref{lemma:Z+} avec $\eta=\frac{1}{\sqrt{2}k}$ et $\beta=\frac{v_0}{\sqrt{2}}$, nous avons
  \begin{eqnarray*}
    N(k,\omega)&=&-\frac{\hat{g}(k)}{k^2}\left(\frac{\alpha}{2\sqrt{2}}Z_{\eta,\beta}^+\left(\omega\right)\right)
  \end{eqnarray*}
  et
  \begin{eqnarray*}
    N(k,-\overline{\omega})&=&-\frac{\hat{g}(k)}{k^2}\left(\frac{\alpha}{2\sqrt{2}}Z_{\eta,\beta}^+\left(-\overline{\omega}\right)\right)\\
    &=&-\frac{\hat{g}(k)}{k^2}\left(-\frac{\alpha}{2\sqrt{2}}\overline{Z_{\eta,\beta}^+\left(\omega\right)}\right)=-\overline{N(k,\omega)}.
  \end{eqnarray*}

  Ainsi, nous obtenons 
  \begin{eqnarray*}
    \frac{N(k,-\overline{\omega})}{\frac{\partial D}{\partial \omega}(k,-\overline{\omega})}=\overline{\left(\frac{N(k,\omega)}{\frac{\partial D}{\partial \omega}(k,\omega)}\right)}.
  \end{eqnarray*}
  Autrement dit, $\frac{N(k,\omega)}{\frac{\partial D}{\partial \omega}(k,\omega)}=re^{i\phi}$ si et seulement si $\frac{N(k,-\overline{\omega})}{\frac{\partial D}{\partial \omega}(k,-\overline{\omega})}=re^{-i\phi}$.
\end{proof}

D'autre part, nous rappelons le résultat \ref{lemme:hypcashyb} dans le cas hybride.
\begin{lemma}
  Sous l'hypothèse $\hat{u}(t=0,k)=0$, pour $\frac{\partial D(k,\omega)}{\partial\omega}$ donnée par~\eqref{eq:hchybderD} et $N(k,\omega)$ par~\eqref{eq:N_hchyb}, l'hypothèse~\ref{hyp:sym} est satisfaite.
\end{lemma}

\begin{proof}
  Regardons d'abord $\frac{\partial D(k,\omega)}{\partial\omega}$. Les termes en facteur de $\alpha$ (venant de la partie chaude cinétique) se comportent comme dans la preuve du lemme \ref{lemme:hypcashyb} (voir la preuve ci-dessus). Les termes en facteur de $1-\alpha$ sont tels que
  $$
    \frac{1}{(-\overline{\omega})^3}=-\frac{1}{\overline{\omega^3}}=-\overline{\frac{1}{\omega^3}}.
  $$

  Nous en déduisons $\frac{\partial D}{\partial \omega}(k,-\overline{\omega})=-\overline{\frac{\partial D}{\partial \omega}(k,\omega)}$.

  Regardons ensuite $N(k,\omega)$. Sous l'hypothèse $\hat{u}(t=0,k)=0$ et avec les notations $\eta=\frac{1}{\sqrt{2}k}$ et $\beta=\frac{v_0}{\sqrt{2}}$, nous avons
  \begin{eqnarray*}
    N(k,-\overline{\omega})&=&-\frac{1}{-i\overline{\omega}}\hat{E}(t=0,k)-\frac{\hat{g}(k)}{ k^2}\left[\alpha\frac{k}{-\overline{\omega}}+\frac{\alpha}{2\sqrt{2}}Z_{\eta,\beta}^+\left(-\overline{\omega}\right)\right].
  \end{eqnarray*}
  %Maintenant, pour $E(t=0,x)$ donné par l'équation de Poisson, nous avons 
  %\begin{eqnarray*}
  %\partial_xE(t=0,x)&=&\rho_c(t=0,x)+\int f^h(t=0,x,v)dv-1\\
  %&=&\left(1-\alpha\right)\left(1+\varepsilon\cos\left(\frac{2\pi}{L}x\right)\right)+\alpha\left(1+\varepsilon\cos\left(\frac{2\pi}{L}x\right)\right)-1\\
  %&=&\varepsilon\cos\left(\frac{2\pi}{L}x\right)
  %\end{eqnarray*}
  %Donc le champ électrique initial s'écrit
  %$$E(t=0,x)=\frac{\varepsilon L}{2\pi}\sin\left(\frac{2\pi}{L}x\right),~~~\int_0^LE(t=0,x)dx=0,$$
  %et 
  %$$E^1(t=0,x)=\frac{L}{2\pi}\sin\left(\frac{2\pi}{L}x\right).$$
  %Sa transformée de Fourier, pour $k=\frac{2\pi}{L}n,~n\in\mathbb{Z}$, s'écrit
  %\begin{eqnarray*}
  %\hat{E}(t=0,k)&=&\frac{1}{2\pi}\int_0^L\sin\left(\frac{2\pi}{L}x\right)e^{-\frac{2i\pi n}{L}x}dx\\
  %&=&\frac{1}{4i\pi}\int_0^Le^{\frac{2i\pi}{L}(1-n)x}-e^{-\frac{2i\pi}{L}(1+n)x}dx.
  %\end{eqnarray*}
  %Si $k\neq \frac{2\pi}{L}$ et $k\neq -\frac{2\pi}{L}$, $\hat{E}(t=0,k)=0$. If $n=1$, ou de manière équivalente $k=\frac{2\pi}{L}$, nous avons
  %\begin{eqnarray*}
  %\hat{E}(t=0,k)&=&\frac{1}{4i\pi}\left[x+\frac{L}{4i\pi}e^{-\frac{2i\pi}{L}2x}\right]_0^L=-i\frac{L}{4\pi}=-\frac{i}{2k}.
  %\end{eqnarray*}
  %Si $n=-1$, ou de manière équivalente $k=-\frac{2\pi}{L}$, nous avons
  %\begin{eqnarray*}
  %\hat{E}(t=0,k)&=&\frac{1}{4i\pi}\left[\frac{L}{4i\pi}e^{\frac{2i\pi}{L}2x}-x\right]_0^L=i\frac{ L}{4\pi}=-\frac{i}{2k}.
  %\end{eqnarray*}
  %Ainsi
  %\begin{equation}
  %\hat{E}\left(t=0,k\right)=-\frac{i}{2k},~k\in\left\{-\frac{2\pi}{L},\frac{2\pi}{L}\right\},~~~\hat{E}(k)=0,~k\notin\left\{-\frac{2\pi}{L},\frac{2\pi}{L}\right\},
  %\label{eq:Ek}\end{equation}
  Nous rappelons que $\hat{E}(t=0,k)$ est un imaginaire pur (éventuellement nul) donné par (\ref{eq:Ekbis}). Ceci implique
  \begin{eqnarray*}
    N(k,-\overline{\omega})&=&\frac{1}{i\overline{\omega}}\hat{E}(t=0,k)+\frac{\hat{g}(k)}{ k^2}\left[\alpha\frac{k}{\overline{\omega}}+\frac{\alpha}{2\sqrt{2}}\overline{Z_{\eta,\beta}^+\left(\omega\right)}\right]\\
    &=&\overline{\frac{1}{i\omega}\hat{E}(t=0,k)}+\frac{\hat{g}(k)}{ k^2}\left[\alpha\overline{\frac{k}{\omega}}+\frac{\alpha}{2\sqrt{2}}\overline{Z_{\eta,\beta}^+\left(\omega\right)}\right]=-\overline{N(k,\omega)}.
  \end{eqnarray*}
  La preuve est terminée.
\end{proof}

\end{subappendices}

