% !TEX root = ../../main.tex

\section{Conclusion}
% -------------------------------------------------------------------

Au cours de ce chapitre nous avons pu étudier théoriquement et confirmer numériquement la convergence du modèle cinétique $1dx-1dv$ vers le modèle hybride. Nous avons décrit deux méthodes de résolutions du modèle hybride. La première méthode, méthode de \emph{splitting}, tirant parti de la structure hamiltonienne du système et assurant la conservation de certaines quantités (énergie, masse). La seconde méthode de résolution est basée sur une méthode de Lawson, et ne préserve aucune quantité particulière, mais permet une montée en ordre en temps pour un coup numérique plus faible. La comparaison des résultats s'est faites grâce à une étude fine des relations de dispersion, permettant de reconstruire le champ électrique et de déterminer le taux d'instabilité dans nos cas tests.

Un résultat supplémentaire que nous avons pu obtenir dans cette comparaison, est l'intérêt plus important de la méthode de pas de temps adaptatif basée sur la méthode de Lawson, permettant de profiter de toute la littérature sur les méthodes de type Runge-Kutta.
