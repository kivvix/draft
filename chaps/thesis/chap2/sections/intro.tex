% !TEX root = ../../main.tex

\section{Introduction}
L'objectif de ce chapitre est d'introduire et de simuler numériquement une hiérarchie de modèles permettant de décrire des systèmes de particules chargées où une population de particules chaudes interagit avec un plasma ambiant plus froid. Une telle configuration physique peut par exemple être étudiée dans les plasmas de tokamak où les particules alpha (générées par les réactions de fusion) interagissent avec le plasma ambiant. Un autre exemple se trouve dans la haute atmosphère  où les électrons énergétiques du vent solaire interagissent avec la magnétosphère terrestre. Des modèles adaptés à ces configurations ont ainsi été obtenus par exemple dans les deux contextes (voir \cite{Holderied:2019} \cite{Chen:2016} \cite{Katoh:2007} \cite{Tao:2014} \cite{Tronci:2010} \cite{Tronci:2014}). Le modèle de départ qui servira de référence repose sur une description cinétique pour l'ensemble du plasma considéré. On introduit alors la fonction de distribution des électrons  $f(t,x,v)\in\mathbb{R}_+$ solution du modèle de Vlasov-Poisson (les ions sont considérés immobiles, comme étant un fond neutralisant). En supposant que la population électronique peut être séparée entre une population "froide" $f_c$ et une population d'électrons énergétiques $f_h$, une première étape consiste à écrire $f$ comme la somme de ces deux fonctions de distribution $f=f_c+f_h$. Une seconde étape consiste à supposer que les particules froides restent proches d'un équilibre thermodynamique de température $T_c\approx 0$ et peuvent donc être représentées par un modèle fluide. On obtient le modèle hybride fluide/cinétique où la partie fluide (linéaire) décrit la dynamique des particules froides alors que les particules chaudes sont décrites à l'aide d'un modèle cinétique. Ce modèle hybride peut encore être simplifié en considérant des perturbations de type ondes de faible amplitude. Les termes non linéaires de la partie fluide sont donc négligés alors que la partie cinétique reste non linéaire. Le modèle ainsi obtenu (voir \cite{Holderied:2019}) est le modèle hybride linéarisé VHL (Vlasov Hybrid Linearized) qui sera détaillé en section~\ref{sec:2:vhl}.

Du fait de la forte disparité des vitesses thermiques entre les particules froides et chaudes, le modèle cinétique est très coûteux à résoudre numériquement, notamment car le maillage en vitesse doit être choisi très fin pour capturer la vitesse thermique des particules froides. Cela implique en outre, pour les schémas numériques classiques, une condition restrictive sur le pas de temps et donc des simulations coûteuses. La dérivation de modèles simplifiés moins coûteux à résoudre numériquement est donc d'un grand intérêt. Parmi ces modèles simplifiés, nous considérerons ici le modèle hybride linéarisé VHL étudié dans \cite{Holderied:2019}. Afin d'effectuer une étude comparative entre le modèle VHL et le modèle cinétique original et de tester les schémas numériques, nous nous placerons dans le cas de la dimension $1$ en espace et en vitesse. Ce cadre nous permettra aussi de poser les bases de l'étude du cas $1dx-3dv$ pour lequel il est beaucoup plus complexe d'effectuer ces comparaisons et ces tests. Ce type d'étude permettra enfin de comprendre le domaine de validité du modèle VHL.

Pour résoudre numériquement le modèle VHL, nous proposons deux méthodes. La première repose sur le fait que le modèle VHL possède une structure géométrique \cite{Morrison:2017}\cite{Tronci:2010} \cite{Tronci:2014}. Plus précisément, le modèle VHL possède une structure hamiltonienne non canonique, ce qui signifie que les équations peuvent être obtenues à partir d'un crochet de Poisson et d'un hamiltonien. Cette structure garantit la préservation d'invariants, comme l'énergie totale. L'objectif est d'exploiter cette structure pour construire des schémas numériques qui possèdent un bon comportement en temps long. Le schéma utilisé est un schéma de type splitting construit à partir d'un \emph{splitting} de l'hamiltonien. Cette approche permet de combiner astucieusement certains termes du modèle et on est alors amené à résoudre trois sous-systèmes simples (comme dans \cite{Crouseilles:2015}, \cite{Casas:2017}, \cite{Li:2020}). Une propriété remarquable est que chacun des sous-systèmes peut être résolu exactement en temps, l'erreur en temps de la méthode provient donc uniquement de la méthode de \emph{splitting} utilisée. Des méthodes d'ordre arbitraire en temps peuvent être obtenues par composition \cite{Hairer:2006}. La deuxième méthode est basée sur un schéma exponentiel \cite{Hochbruck:2010}, \cite{Hochbruck:2005}, \cite{Lawson:1967a}, \cite{Isherwood:2018}, \cite{Lawson:1967}, \cite{Crouseilles:2019b}. En exploitant le fait que la partie linéaire du modèle VHL peut être résolue exactement et efficacement, on construit alors des schémas de type Lawson d'ordre élevé. Les résultats du chapitre précédent et de \cite{Crouseilles:2019b} sont donc repris et étendus au cas du système VHL. 

Pour les deux méthodes en temps (splitting et Lawson), nous avons introduit une technique de pas de temps adaptatif. Pour les méthodes de type Lawson, le cadre des méthode \emph{embedded} \cite{Dormand:1980}\cite{Dormand:1978}  \cite{Balac:2013}\cite{Balac:2014} permet de calculer l'erreur locale facilement. Dans le cas des méthodes de splitting, nous utiliserons le travail récent \cite{Blanes:2019} qui propose des méthodes de splitting \emph{embedded}. Des méthodes d'ordre $4(3)$ seront utilisées dans le cadre de la comparaison (ordre $3$ et ordre $4$ pour estimer l'erreur locale). Pour l'approximation de l'espace des phases, nous avons choisi une méthode spectrale en espace et une approximation type différences finies d'ordre élevé (ordre 5 en pratique) pour la direction en vitesse. 

La première approche (splitting hamiltonien) comporte des similarités avec les approches proposées dans \cite{Kraus:2017} et \cite{Holderied:2019} ; néanmoins, ces méthodes reposent sur une approximation de type Particle-In-Cell de l'espace des phases alors que nous utilisons des méthodes eulériennes. Ainsi, on est plus dans l'esprit de \cite{Crouseilles:2015}, \cite{Li:2020} où on effectue un splitting puis on discrétise alors que dans  \cite{Kraus:2017} et \cite{Holderied:2019}, on discrétise l'espace des phases puis on discrétise en temps.  

Afin de valider les résultats numériques, une étude approfondie des relations de dispersion est effectuée. Ces relations de dispersion sont obtenues par la résolution du modèle VHL. Ce dernier est issu d'une linéarisation uniquement des équations sur les particules froides, une linéarisation complète du modèle est alors nécessaire. À l'aide de transformées de Fourier en espace, de transformées de Laplace en temps, il est en effet possible de déterminer très précisément la phase linéaire des simulations de modèles non-linéaires ; on peut calculer le taux d'amortissement ou d'instabilité d'un équilibre perturbé \cite{Sonnendrucker:2015} \cite{Fried:1961}, mais aussi reconstruire le mode fondamental du champ électrique. En plus de fournir des informations pour valider de manière quantitative les codes développés, cette analyse nous permet de faire le lien entre les modèles. En effet, en faisant tendre $T_c$ vers zéro dans la relation de dispersion du modèle de Vlasov original, il est possible de retrouver la relation de dispersion du modèle VHL. 

Le chapitre est organisé comme suit : nous présentons tout d'abord la hiérarchie de modèles que nous souhaitons étudier, depuis le modèle cinétique jusqu'au modèle hybride linéarisé. La structure géométrique de ce modèle est décrite en section \ref{s:geom}. La section \ref{s:scheme} est dédiée à la présentation des méthodes numériques construites pour la résolution du modèle hybride linéarisé. Dans la section \ref{s:dispersion}, les relations de dispersion sont introduites et étudiées. Les sections \ref{s:limit} et \ref{s:compare} contiennent de nombreuses illustrations numériques. La section \ref{s:limit} se concentre sur la comparaison du modèle cinétique avec le modèle hybride linéarisé, alors que dans la section \ref{s:compare}, nous étudions les avantages et les inconvénients des deux méthodes numériques pour le modèle hybride linéarisé.
