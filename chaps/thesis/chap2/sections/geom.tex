% !TEX root = ../../main.tex

\section{Structure géométrique du modèle hybride linéarisé VHL}
\label{s:geom}

\Josselin{Nicolas propose de faire de ceci une sous-section de la section précédente, et d'ajouter à la fin la propriété~\ref{p:vhl_conservation}.}

Dans cette partie, nous présentons la structure du modèle hybride linéarisé VHL \eqref{eq:vahl}, à savoir son hamiltonien et son crochet de Poisson. Pour simplifier les notations, nous notons dans cette section $f=f_h$, $\rho_c=\rho_c^{(0)}$ et $u=u_c$. Cette structure permet notamment d'assurer la préservation de nombreux invariants (énergie totale et opérateurs de Casimir entre autres) mais sera à la base d'un \emph{splitting} en temps, dans l'esprit de \cite{Crouseilles:2015}, \cite{Casas:2017}, \cite{Kraus:2017} \cite{Li:2020}. Nous aurons besoin d'introduire certaines notations pour pouvoir présenter la structure.

Tout d'abord, rappelons que pour une fonctionnelle donnée ${\cal G}(f)$, la dérivée de Fréchet de la distribution $\frac{\delta {\cal G}}{\delta f}(f)$ évaluée au point $f$, est définie par 
\begin{equation}
  {\cal G}(f + \delta f) - {\cal G}(f) = \int_{\Omega\times \mathbb{R}} \frac{\delta {\cal G}}{\delta f}(f)(x, v) \delta f(x, v) \mathrm{d}x\mathrm{d}v +{\cal O}(\delta f^2), 
\end{equation}
pour toute variation régulière $\delta f$. On définit le hamiltonien associé au modèle VHL \eqref{eq:vahl} 
\begin{eqnarray}
\label{hamiltonian_red}
  \mathcal{H} &=& \frac{1}{2}\int_{\mathbb{R}} {E}^2 \mathrm{d}{x}  +  \frac{1}{2}\int_{\mathbb{R}} \rho_{c} u^2\mathrm{d}{x} + \frac{1}{2}\int_{\mathbb{R}}\int_{\mathbb{R}} v^2 f\,\mathrm{d}x\,\mathrm{d}v, \\
              &=&  \mathcal{H}_E + \mathcal{H}_u + \mathcal{H}_f. 
\end{eqnarray}
Les trois termes correspondent respectivement à l'énergie électrique, l'énergie cinétique des particules froides et l'énergie cinétique des particules chaudes. Pour une fonctionnelle ${\cal G}(E, u, f)$, on notera $\delta {\cal G}/\delta f$, $\delta {\cal G}/\delta E$ et $\delta {\cal G}/\delta u$ les dérivées de Fréchet de ${\cal G}$ par rapport à $f, E$ et $u$ respectivement. On introduit à présent le crochet de Poisson de deux fonctionnelles ${\cal F}(E, u, f)$ et ${\cal G}(E, u, f)$
$$
  \begin{aligned}
    \{ {\cal F}, {\cal G} \}( u, E, f) &= \int_{\mathbb{R}}\int_{\mathbb{R}} f \left( \partial_x \frac{\delta {\cal F}}{\delta f}\partial_{v} \frac{\delta {\cal G}}{\delta f} - \partial_{v} \frac{\delta {\cal F}}{\delta f}\partial_{x} \frac{\delta {\cal G}}{\delta f}\right)\mathrm{d}v \mathrm{d}x \\
                         & + \int_{\mathbb{R}}  \left(  \frac{\delta {\cal F}}{\delta{ u}}  \frac{\delta {\cal G}}{\delta{ E}} - \frac{\delta {\cal F}}{\delta{ E}}  \frac{\delta {\cal G}}{\delta{u}} \right) \mathrm{d}x \\
                         & + \int_{\mathbb{R}}\int_{\mathbb{R}}  \left(  \frac{\delta {\cal F}}{\delta{ E}}  \partial_v f\frac{\delta {\cal G}}{\delta{ f}} - \frac{\delta {\cal G}}{\delta{ E}} \partial_v f \frac{\delta {\cal F}}{\delta{f}} \right) \mathrm{d}{ v}\mathrm{d}x \\
  \end{aligned}
$$
Avec cette notation, le modèle hybride linéarisé~\eqref{eq:vahl} se réécrit alors, avec  $U=(u, E, f)$ et $\mathcal{H}$ donné par \eqref{hamiltonian_red}
\begin{equation}
\label{ham_form}
  \partial_t U = \{ U, \mathcal{H} \}. 
\end{equation}

Dans la suite, on vérifie que la réécriture \eqref{ham_form} est bien équivalente au modèle VHL. Pour cela, on a besoin des relations suivantes 
$$
  \frac{\delta \mathcal{H}}{\delta f} = \frac{v^2}{2}, \;\; \frac{\delta \mathcal{H}}{\delta u} = \rho_c u, \;\; \frac{\delta \mathcal{H}}{\delta E} = E. 
$$
De plus, par abus de notation, on notera la fonctionnelle associée à la fonction comme suit (par exemple pour $u$) $u(t, z)=\int_{\mathbb{R}} u(t, x)\delta(x-z) \mathrm{d}x$, de sorte que $ \frac{\delta u}{\delta u} = \delta(x-z)$. Pour $f$, on notera $f(t, x, w)=\int_{\mathbb{R}} f(t, z, v)\delta(x-z)\delta(w-v) \mathrm{d}x\mathrm{d}v$, de sorte que $ \frac{\delta f}{\delta f} = \delta(x-z)\delta(w-v)$.

\begin{enumerate}
  \item[$\bullet$] On calcule dans un premier temps $\{ u, \mathcal{H} \}$ 
    \begin{eqnarray*}
    \partial_t u(t, z) = \{ u, \mathcal{H} \} &=& 0+ \int_{\mathbb{R}}   \delta(x-z) E(t,x) \mathrm{d}x  + 0 = E(t,z)
    \end{eqnarray*}
  \item[$\bullet$] Puis on considère $\{ E, \mathcal{H} \}$ 
    \begin{eqnarray*}
    \partial_t E(t, z) = \{ E, \mathcal{H} \} &=& 0 -  \int_{\mathbb{R}}  \delta(x-z) \rho_c u \mathrm{d}x  +  \int_{\mathbb{R}}\int_{\mathbb{R}}  \left(  \delta(x-z)  \partial_v f \frac{v^2}{2}  \right) \mathrm{d}{ v}\mathrm{d}x \nonumber\\
    &=& - \rho_c u(t, z) - \int_{\mathbb{R}} f(t, z, v) v\mathrm{d}{ v} 
    \end{eqnarray*}
  \item[$\bullet$]  Finalement,  $\{ f, \mathcal{H} \}$ donne 
    \begin{eqnarray*}
    \partial_t f(t, z,w) = \{ f, \mathcal{H} \} &=&  \int_{\mathbb{R}}\int_{\mathbb{R}} f \left( \partial_x (\delta(x-z)\delta(w-v)) \partial_v \frac{v^2}{2}   \right)\mathrm{d}v \mathrm{d}x \nonumber\\
    && + 0  -  \int_{\mathbb{R}}\int_{\mathbb{R}}  \left( E  \partial_v f \right) \delta(x-z)  \delta(w-v)    \mathrm{d}{ v}\mathrm{d}x \nonumber\\
    &=& (- v\partial_x f - E\partial_v f)(t, z, w). 
    \end{eqnarray*}
\end{enumerate}

Enfin, on peut vérifier  que le crochet de Poisson satisfait les propriétés suivantes 
\begin{itemize}
  \item anti-symétrie: $\{ F, G \} = -\{ G, F \}$ 
  \item bilinéarité: $\{ F + G, H \} = \{ F, H \}+\{ G, H \}$ 
  \item identité de Jacobi : $\{\{ F, G \}, H\} + \{\{G,H\}, F \} + \{\{H, F\}, G \} = 0$.  
\end{itemize}
Les deux premières propriétés sont  simples alors que la dernière est habituellement plus compliquée. On utilise les calculs de \cite{Li:2020}, \cite{Morrison:2012}. 
