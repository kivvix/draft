% !TEX root = ../main.tex

\newcommand\cchapter[1]{\chapter*{#1}\addcontentsline{toc}{chapter}{#1}\chaptermark{#1}}
\cchapter{Perspectives}

\newcommand{\parallelsum}{\mathbin{\!/\mkern-5mu/\!}}

\newcommand\csection[1]{\section*{#1}\addcontentsline{toc}{section}{#1}}
\newcommand\csubsection[1]{\subsection*{#1}\addcontentsline{toc}{subsection}{#1}}
\newcommand\csubsubsection[1]{\subsubsection*{#1}\addcontentsline{toc}{subsubsection}{#1}}

Au travers de différentes méthodes de simulations dédiées à la physique des plasmas, nous avons mis en évidence comment calculer automatiquement des conditions de stabilité et tirer profit d'une partie linéaire de l'équation ; cela autorise l'utilisation de pas de temps plus grands et permet ainsi de réduire le temps de calcul. De plus, l'utilisation des méthodes de Lawson encourage à la montée en ordre en temps, et l'utilisation d'approximations de l'exponentielle de la partie linéaire permet l'usage de la méthode dans de nombreuses situations. Tous ces développements permettent d'envisager un certain nombre de perspectives.

\csection{Analyse mathématique}

Dans le chapitre~\ref{chap1}, nous avons étudié la stabilité des méthodes de Lawson dans le cadre d'équations de transport, plus particulièrement les équations cinétiques. L'analyse de convergence dans ce cadre serait intéressante à étudier sur la base des travaux~\cite{Hochbruck:2020,Hochbruck:2010,Boscarino:2020} pour mieux comprendre les avantages de la méthode de Lawson par rapport aux méthodes exponentielles dans ce contexte. L'analyse de consistance a été effectuée mais l'analyse mathématique de la stabilité dans un cadre général nécessite des outils d'analyse fonctionnelle avancée. On peut mentionner un travail récent (voir~\cite{Boscarino:2020}) où la partie linéaire de transport est approchée par une méthode semi-lagrangienne et la partie non-linéaire est un opérateur de BGK traité implicitement. Les auteurs proposent une analyse de convergence pour l'ordre 1 en temps, qui pourrait être un bon point de départ dans notre cas.

Nous nous sommes concentrés tout au long de ce travail sur le système de Vlasov-Maxwell, où les termes linéaires peuvent être résolus à l'aide d'une transformée de Fourier. Il existe à l'inverse des termes linéaires difficiles à traiter, c'est le cas par exemple dans le contexte gyrocinétique avec un champ magnétique extérieur donné $\vb{B}(x)$ non-homogène. Ces équations font intervenir un terme de dérivée spatiale plus compliqué que dans le cadre de l'équation de Vlasov : $\nabla_{x}\cdot(v_{\parallelsum}\vb{B}f)$, qu'il n'est pas possible de traiter simplement avec une transformée de Fourier à cause du champ magnétique $\vb{B}$ ou des conditions aux bords. L'extension de nos techniques dans ce cadre plus complexe peut se faire via la décomposition suivante $\vb{B}(x)=\vb{B}(x_0)+\vb{B}(x)-\vb{B}(x_0)$ où $x_0$ est déterminé de sorte que $\|\vb{B}(x)-\vb{B}(x_0)\| \ll 1$. Le terme $\nabla_x\cdot (v_{\parallelsum} (\vb{B}(x)-\vb{B}(x_0))f)$ sera considéré dans la partie non-linéaire de la méthode de Lawson sans introduire de contrainte de stabilité trop sévère. Cette approche existe dans un cadre général dans la littérature sous le nom de schéma de Rosenbrock~\cite{Hochbruck:2006,Hochbruck:2009,Luan:2016}.

\csection{Schémas numériques}

Comme nous venons de l'évoquer, des conditions aux bords non-périodiques rendent impossible l'utilisation de transformées de Fourier. Il devient alors nécessaire d'utiliser une méthode semi-lagrangienne pour résoudre numériquement l'équation de Vlasov, méthode se devant d'être d'ordre élevé pour évaluer le terme non-linéaire aux pieds des caractéristiques $x_i - c_j\Delta v$, où $c_j$ correspond au coefficient de Butcher de la méthode Runge-Kutta choisie (voir~\cite{Boscarino:2019a}) ; cela ne nécessite plus l'utilisation de transformées de Fourier mais une interpolation supplémentaire. Les équations de Maxwell, quant à elles, peuvent être résolues à l'aide d'une méthode type éléments finis. Cette dernière méthode engendre une matrice de même taille que le maillage, qui peut questionner la possibilité de calculer formellement un approximant de Padé d'une matrice de grande taille ; une méthode ne nécessitant pas l'inversion de matrice peut alors être utilisée~\cite{Li:2011}.

L'ajout d'un terme de collisions à l'équation de Vlasov fait apparaître une condition de stabilité supplémentaire. Pour contourner ce problème, l'utilisation de méthodes implicites-explicites (IMEX), qui peuvent être incorporées au cadre des méthodes de Lawson, est une solution possible. Ainsi les termes linéaires sont traités exactement, les termes non-linéaires explicitement, et les termes collisionnels (ou dissipatifs) de manière implicite. Ces derniers demandent une stabilité sur l'axe réel négatif qui n'a pas été explorée au cours de cette thèse. Le solveur s'adapte aux particularités de chaque opérateur pour une efficacité optimale. Notons qu'il est possible d'envisager de considérer le terme non-linéaire (traité à l'aide d'une méthode WENO) de manière implicite (voir la méthode iWENO décrite dans~\cite{Gottlieb:2006}) pour réduire le nombre de contraintes de stabilité de la méthode de résolution ainsi construite.

La méthode présentée dans la conclusion du chapitre~\ref{chap2}, qui propose d'introduire le calcul du courant dans la partie linéaire d'une méthode de Lawson, permet de résoudre le système de Vlasov-Ampère, ou le modèle Vlasov-Ampère hybride linéarisé, tout en vérifiant l'équation de Poisson. L'ajout de termes dans la partie linéaire de la méthode de Lawson permet d'assurer la préservation de certaines quantités. Il devient envisageable de construire des méthodes de type Lawson qui préservent l'énergie totale sur la base de travaux récents~\cite{Mei:2021}. Le cadre GEMPIC (voir~\cite{Kormann:2021}) parait être un excellent terrain d'application à ces méthodes.

\csection{Implémentation}

Une thèse en mathématiques ne permet pas d'envisager tous les développements informatiques que l'on souhaiterait parfois effectuer.

Il serait possible de mettre en place une parallélisation via MPI ; cela peut se faire en suivant deux stratégies différentes. La première, plus classique, consiste à effectuer une décomposition de domaine, mais l'usage intensif de transformées de Fourier, algorithme non-local, nécessite de nombreux échanges de données vers un processus parent. On peut aussi remarquer que la résolution des grandeurs spatiales $(\vb{j},\vb{B},\vb{E})$ est indépendante de la résolution de la densité de particules $f_h$, il est seulement nécessaire d'échanger les quelques termes de couplages (vitesses d'advection dans la direction $\vb{v}$ pour $f_h$ et le courant des particules chaudes pour $(\vb{j},\vb{B},\vb{E})$).

Une discrétisation plus fine de l'espace des phases, plus particulièrement en vitesse $\vb{v}$, entraine l'utilisation de très grands tableaux pour représenter $f_h$ et nécessite de grands espaces en mémoire RAM. Pour le moment le code de simulation n'utilise que des nombres à virgule flottante à double précision : \texttt{double}. La mise en place d'une stratégie \emph{multiprecision} consisterait à traiter $f_h$ avec des nombres à virgule flottante à simple précision : \texttt{float} (l'usage de demi-précision : \texttt{half} est aussi envisageable), et de continuer l'usage de \texttt{double} pour les grandeurs spatiales ; cela nécessite des algorithmes de réduction performants sur $f_h$ sans perdre en précision, ce qui permet de calculer des intégrales en $\vb{v}$ tout en obtenant une valeur à double précision \texttt{double} par sommation d'éléments en simple précision \texttt{float}. L'espace mémoire se retrouve ainsi divisé par deux, et certains calculs, comme la méthode WENO, se retrouvent aussi accélérés.

L'utilisation d'arithmétique stochastique, ou d'arithmétique de Monte-Carlo~\cite{Parker:1997,Parker:1997a}, a aussi été envisagée pour vérifier la stabilité numérique des méthodes mises en œuvre, ainsi que la précision nécessaire pour les calculs sur $f_h$. L'étude prospective réalisée a montré la stabilité numérique des méthodes utilisées.
